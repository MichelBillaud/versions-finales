\section{Motivations}

\subsection{Linguistique}

La \emph{théorie des langages} est issue de problématiques
de linguistique, quand les linguistes se sont intéressés 
aux mécanismes qui permettent de construire des phrases
plus ou moins complexes, à partir de règles relativement simples.

Par exemple, la phrase 
\begin{quote}
Le chat noir dort sur le vieux banc du jardin
\end{quote}
se décompose en trois parties.

\begin{itemize}
\item le sujet ``le chat noir''
\item le verbe ``dort''
\item le complément ``sur le vieux banc du jardin''
\end{itemize}
le sujet comporte lui-même trois parties : un article, un nom, et un adjectif.

Ces règles de constructions de \emph{syntagmes}, qu'on apprend
laborieusement lorsqu'on étudie une langue étrangère, sont acquises
implicitement lors de l'apprentissage de la langue maternelle dans la
petite enfance. Ce qui laisse supposer certaines prédispositions de
l'espèce humaine.

D'où l'intérêt développé, au XX-ième siècle, pour l'étude formelle de
ces mécanismes que l'on retrouve sous des formes voisines dans les
différentes langues.

\subsection{Langages informatiques}

À la fin des années 50 sont apparus les premiers langages de
programmation. Au début, la programmation consistait à établir des
listes d'instructions machine, mais assez rapidement est apparu le
besoin d'écrire des programmes sous une forme plus ``naturelle'', dans
laquelle les éléments (fonctions, instructions, déclarations,
expressions, etc) sont combinés selon des règles précises.

Avec cela, la \emph{définition du langage}, vient une autre
préoccupation : comment écrire, à moindre frais, des
\emph{compilateurs} qui analysent le code source, et génèrent sa
traduction ?



\paragraph{Au delà de ces motivaqtions} pratiques, la \emph{ théorie des langages} s'intéresse aux propriétés
des \emph{langages formels}.


\section{Langages formels}

\subsection{Lettres, mots}

 Pour définir un langage formel on se donne un ensemble, généralement
fini, de \emph{symboles}, appelés aussi \emph{lettres}.  On appelle cet ensemble le
\emph{vocabulaire}, ou l'\emph{alphabet}.
A partir de ces lettres, on définit les \emph{mots}, qui sont simplement des séquences de \emph{lettres}.


\paragraph{Notations}
\begin{itemize}
\item On emploie souvent les lettres de la fin de l'alphabet $x, y, z$, pour désigner des lettres,
et $u,v, w$ (comme word) pour désigner des mots.
\item la longueur d'un mot $w$ est notée $\abs{w}$,
\item le mot vide, de longueur nulle est noté $\epsilon$,
% %\item on note $A^*$ le langage (infini) qui contient tous les mots sur $A$.
\end{itemize}
\paragraph{Exemples} 

\begin{itemize}
\item soit $A$ l'alphabet à trois lettres $A = \{ a, b, c \}$
\item $a$, $bc$, $aaa$ sont des mots de longueurs respectives 1, 2 et 3 sur $A$
\end{itemize}


\paragraph{La concaténation} de deux mots consiste à les mettre bout à bout. Plus formellement,
si $u = x_1 x_2 \ldots x_n$ et $v = y_1 y_2 \ldots y_p$ sont deux mots
de longueurs respectives $n$ et $p$, le mot $u.v$ (noté aussi $uv$)
est la suite $ x_1 x_2 \ldots x_n y_1 y_2 \ldots y_p$ de longueur
$n+p$.

\manicule la concaténation est-elle associative, commutative ?
A-t-elle un élément neutre ?

\subsection{Langages}

\paragraph{Définition} Un \emph{langage} est un ensemble de mots.

\paragraph{Exemples} 
\begin{itemize} 
\item le langage des mots de deux lettres au plus sur $A = \{a, b\}$
  est $$L = \{ \epsilon, a, b, c, aa, bb, cc, ab, ba, bc, cb, ac, ca
  \}$$
\item $ \{ a^n \| 4\leq n\leq 2\} = \{ aa, aaa, aaaa \}$
\end{itemize}

\subsection{Opérations sur les langages}

\paragraph{Opérations ensemblistes : } les langages sont
 des ensembles (de mots), on peut en faire
\begin{itemize}
\item l'intersection : par exemple si $L_1$ est l'ensemble des mots
  qui commencent par la lettre $a$, et $L_2$ ceux qui finissent par
  $b$, $L_1 \cap L_2$ contient les mots qui commencent $a$ \emph{et}
  finissent par $b$.
\item l'uunion, souvent notée $+$ : $L_1 + L_2$ contient les mots qui
  commencent $a$ \emph{ou} finissent par $b$ (ou les deux);
\item la différence, etc.
\end{itemize}

\paragraph{Le \emph{produit} de deux langages} est une opération 
spécifique, notée par un point : $L_1 . L_2$ est l'ensemble des mots
qui sont obtenus par concaténation d'un mot de $L_1$ avec un mot de
$L_2$.  Exemple : avec $L_1 = \{a, ab, c \}$ et $L_2 = \{ \epsilon, b
\}$
$$L_1.L_2 = \{ a, ab, c, abb, cb \}$$


\paragraph{L'élévation à une puissance  $L^n$} d'un langage $L$ 
consiste à concaténer $n$ mots de $L$.  Par exemple $\{a,ab\}^2 = \{
aa, aab, aba, abab \}$

\paragraph{l'étoile $L^n$} est l'union de tous les $L^n$, 
pour $n \geq 0$ : c'est l'ensemble des mots que l'on peut décomposer
en une suite de facteurs pris dans $L$. Il contient le mot vide, même
si celui-ci n'est pas dans $L$.

Avec une certaine logique, on note aussi $A^*$ le langage de tous les
mots sur $A$ (puisqu'ils se décomposent en lettres.


\clearpage

