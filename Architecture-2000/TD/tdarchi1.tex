\documentclass[]{article}

\usepackage{fontspec}
\usepackage[french]{babel}

\usepackage{a4wide}
\usepackage{epsf}
\renewcommand{\epsfsize}[2]{0.5#1}

\frenchspacing


\title{
TD Circuits logiques - Montages \`a diodes}

\author{Première Année \\        Département Informatique \\
        IUT ``A'' - Université Bordeaux 1}

\date{Brouillon \today}



\begin{document}
\maketitle

\section{Matériel}

\subsection{Inventaire}
\begin{itemize}
\item 1 plaque d'essai: vue de dessus (côté montage):
$$ \epsfbox{plaque.eps}$$
vue de dessous (bandes de connexions):
$$ \epsfbox{plaque2.eps}$$

\item 2 pinces crocodile
\item 4 diodes (la cathode (-) est du côté de la bague)
\item 3 LED (la broche la plus courte est la cathode)
\item 3 résistances
\item 2 interrupteurs à poussoir
\item 4 petits fils
\end{itemize}

\subsection{Vérification  des LEDs et des interrupteurs}

$$\epsfbox{essailed.eps} \hspace{1cm} \epsfbox{essaiint.eps}$$

\section{Montage porte ``ou''}

Le schéma théorique de la porte "ou":
$$\epsfbox{porteou.eps}$$
conduit au montage suivant:
$$\epsfbox{montageou.eps}$$
dans lequel RC tient lieu de résistance de rappel.

Pour des raisons de commodité, on pourra procéder ainsi:
$$\epsfbox{montageou2.eps}$$


\section{Montage porte ``et''}

Rappel de la porte ``et'':
$$\epsfbox{porteet.eps}$$
Testez le montage suivant. Dans quels cas ne fonctionne-t-il
pas ? pourquoi ?
$$\epsfbox{montageet.eps}$$


\section{Exercice}
Réalisez un montage pour l'expression $AB+C$.
\end{document}







