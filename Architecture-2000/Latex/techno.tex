\chapter{El\'ements de technologie}

Nous avons vu comment coder l'information par des
suites de valeurs binaires. Nous allons voir maintenant comment
repr\'esenter ces valeurs binaires par des signaux \'electriques
afin de les manipuler par des circuits \'electroniques.


\section{Repr\'esentations de la logique binaire}

\subsection{Repr\'esentation des signaux}


Il existe plusieurs fa\c{c}ons de coder les valeurs 0 et 1. La plus simple,
que nous adopterons dans la suite, est appel\'ee {\em logique positive} 
et consiste \`a repr\'esenter 1 par la pr\'esence
d'une tension (par exemple +5v) et 0 par une 
tension nulle.\footnote{
Une logique {\em n\'egative} aura des signaux invers\'es:
 0v repr\'esente la valeur 1,  +5v la valeur 0.}


\subsection{G\'en\'eration d'un signal}

Il est facile de fournir de tels signaux \`a un circuit: il suffit
d'un interrupteur \`a deux positions  (utilis\'e pour les va-et-vient); 
dans une position la sortie de l'interrupteur transmettra la tension +V, 
dans l'autre 0v (voir figure \ref{fig1})

\dessin{fig1}{fig1.eps}{G\'en\'eration d'un signal d'entr\'ee}
%\AFAIRE{fig1 dessin d'un interrupteur \`a deux positions}

En pratique, on utilise plutôt (figure \ref{fig2})
un interrupteur simple dont la sortie est reli\'ee
\`a une {\em r\'esistance de rappel} R assez forte. 
Lorsque l'interrupteur est ferm\'e,
la sortie est \`a +V, lorsqu'il est ouvert la r\'esistance ram\`ene 
la tension de sortie vers 0. De plus ceci évite d'avoir une entrée
``en l'air'' quand l'interrupteur est entre deux positions.

\dessin{fig2}{fig2.eps}{G\'en\'eration d'un signal d'entr\'ee (2)}
%dessin d'un interrupteur avec une r\'esistance de rappel
%\AFAIRE{fig2 dessin d'un interrupteur avec une r\'esistance de rappel}

\subsection{Observation d'un signal}

Pour observer la sortie d'un circuit on peut utiliser (figure \ref{ampoule})
une ampoule du voltage
voulu (5v = lampe de poche). 

\dessin{ampoule}{fig3.eps}{Ampoule t\'emoin}
% \AFAIRE{fig3 Ampoule en sortie}


On préfèrera généralement employer des 
diodes \'electro-luminescentes (LED = Light Emitting Diode),
qui coûtent moins cher et n\'ecessitent un courant moindre 
(voir figure \ref{temoin-led}).
0
\dessin{temoin-led}{fig4.eps}{LED t\'emoin}
%\AFAIRE{fig4 LED en sortie}


Encore mieux, pour \'eviter de trop ``tirer'' sur la sortie du circuit
observ\'e, on pourra utiliser un transistor en amplification
(voir plus loin \ref{trans-ampli}).





\section{Notions rudimentaires d'\'electronique}

Quelques notions sommaires d'\'electronique sont n\'ecessaires pour la 
compr\'ehension des circuits logiques de base. Le lecteur est suppos\'e
conna\^{\i}tre la loi d'Ohm et des rudiments d'\'electricit\'e.

\subsection{La diode}

La diode \`a semi-conducteurs est un composant \'electronique muni de deux
bornes: l'{\em anode} et la {\em cathode} (voir \ref{diode}). 
\dessin{diode}{fig7.eps}{La diode: apparence physique et symbole}
La propri\'et\'e fondamentale de
la diode est d'opposer qu'une r\'esistance tr\`es faible lorsqu'elle est 
travers\'ee par un courant de l'anode vers la cathode (sens passant), et une résistance très forte dans le sens inverse.
Voir figure \ref{diode-on-off}.
\dessin{diode-on-off}{fig6.eps}{Diodes passante (D1) et bloqu\'ee (D2)}
% Dessin de deux diodes en s\'erie avec des lampes, une dans chaque sens.
% La diode passante est allum\'e, l'autre \'eteinte.
%\AFAIRE{fig6 Diodes passante (D1) et bloqu\'ee D2}


Pour analyser les circuits logiques nous consid\'ererons que les diodes sont
parfaites, c'est-\`a-dire ayant une r\'esistance  nulle dans l'\'etat passant,
et infinie dans l'\'etat bloqu\'e.


\begin{remarque}{importante.} Lorsque nous construirons des circuits nous
aurons soin d'\'eviter de faire traverser les diodes par des courants trop 
forts (risque de claquage). Il faudra donc de les monter en s\'erie avec des
r\'esistances pour limiter le courant.
\end{remarque}

\subsection{La diode \'electro-luminescente}

Les diodes \'electro-luminescentes (LED = Light Emitting Diod, voir
fig. \ref{fig9}) émettent de la lumi\`ere quand elles sont
travers\'ees par un courant de l'ordre de 10 \`a 20 mA dans le sens
passant.  Ces diodes offrant une r\'esistance tr\`es faible, il
conviendra de les monter en série avec une {\em r\'esistance
limitatrice de courant} d'une valeur suffisante ($R=U/I=5v/10mA=500
\Omega$).

\dessin{fig9}{fig9.eps}{LED: apparence physique et sch\'ema}




\subsection{Le transistor}



Le transistor\footnote{Pour simplifier l'expos\'e nous n'\'evoquerons
que les transistors NPN.}
 est un composant \`a 3 pattes: l'\'emetteur, le collecteur et la
base (voir figure \ref{transistor}).  

\dessin{transnpn}{fig8.eps}{Transistor NPN: boitier, broches (vues de dessous), sch\'ema}
% \AFAIRE{Fig8 transistor NPN: boitier, broches sch\'ema}



Dans les montages usuels, 
le collecteur est reli\'e \`a la tension d'alimentation $+V$ au travers d'une 
r\'esistance $R_c$, l'\'emetteur \'etant reli\'e \`a la masse (0v). Le courant $I_b$
qui traverse la base permet de contr\^oler l'intensit\'e de celui qui traverse
l'\'emetteur et le collecteur.

\dessin{transistor}{fig10.eps}{Polarisation d'un transistor}
% Dessin d'un transistor NPN, avec resistance entre collecteur et +V,
%et indication des courants $I_b, I_e$ et $I_c$.


Lorsque le transistor est utilis\'e en amplification, les courants
sont li\'es par les \'equations 
$$ \begin{array}{rcl}
I_e &=& I_c + I_b \\
I_c &=& \beta . I_b + I_{ce}
\end{array} $$

Les deux constantes $\beta$ et $I_{ce}$ d\'ependent du transistor:
le gain $\beta$ est de l'ordre de 100, le courant $I_{ce}$
vaut quelques micro-amp\`eres pour les transistors au silicium,
et quelques nano-amp\`eres pour les transistors au germanium. En g\'en\'eral
on le n\'eglige pour les calculs.

Les montages logiques utilisent le mode de fonctionnement 
{\em bloqu\'e-satur\'e}, en d\'ebordant largement de la plage de valeurs o\`u les
formules ci-dessus sont valides: si on applique une tension assez forte 
(proche de +V) \`a la base, l'intensit\'e $I_b$ est maximum et le transistor
n'oppose qu'une r\'esistance tr\`es faible entre \'emetteur et collecteur:
le transistor est satur\'e. Si on applique une tension nulle \`a la base,
le courant $I_c$ sera n\'egligeable~: le transistor est alors dit bloqu\'e.

\label{trans-ampli}
\dessin{ampli-diode}{fig5.eps}{LED amplifi\'ee par un transistor}
% \AFAIRE{fig5 Diode amplifiee par transistor}
Ceci justifie le montage
 de la figure \ref{ampli-diode}~: lorsque l'entr\'ee est \`a +V, le transistor est
satur\'e,  donc il laisse passer le courant \`a travers la LED qui s'\'eclaire. Lorsque 
l'entr\'ee est au niveau bas (0v), le transistor est bloqu\'e et la LED 
reste \'eteinte.  En raison des propri\'et\'es amplificatrices du transistor, 
le courant de base n'a pas besoin d'\^etre tr\`es \'elev\'e
 (de l'ordre de $I_c/\beta$), par cons\'equent on peut mettre une 
r\'esistance limitatrice sur la base. Cette r\'esistance \'evite de trop
``tirer de courant'' du signal que l'on observe, ce qui risquerait
de perturber son fonctionnement.

\section{Portes logiques}

En assemblant ces composants on obtient des {\em portes logiques}
qui combinent les signaux.

\subsection{Porte OU}

\subsubsection{Montage \`a diodes}

Deux diodes et une r\'esistance suffisent pour r\'ealiser une {\em porte OU}
selon le sch\'ema de la figure \ref{porte-ou}.

\dessin{porte-ou}{log1.eps}{Porte logique OU \`a diodes} 
% Dessin d'une porte OU avec 2 diodes et une r\'esistance. Les entr\'ees sont 
% marqu\'ees A et B et la sortie S.


\subsubsection{Analyse du montage}


Supposons que les deux entr\'ees A et B soient port\'ees au potentiel
+V.  Les deux diodes sont dans le sens passant, et laissent donc passer le
courant.  Tout se passe alors comme si la sortie S \'etait reli\'ee \`a
la m\^eme tension +V.

Si A et B sont reli\'ees  \`a la masse (0v), les diodes ne conduisent pas 
le courant. La sortie S est donc ramen\'e \`a 0v par 
la r\'esistance de rappel.

Si une des entr\'ees est  \`a +V et l'autre  \`a la masse, la diode passante
suffit, comme dans le premier cas, \`a amener la tension +V sur S.


\subsubsection{Table de v\'erit\'e}

Si nous prenons la convention de logique positive, la tension +V correspond
\`a la valeur logique ``vrai'' (not\'ee 1) et 0v \`a ``faux'' (0). La table
de v\'erit\'e de l'op\'eration ``$+$'' ainsi obtenue est celle de 
l'op\'erateur ``OU logique'':


$$ \begin{array}{|cc|c|}
\hline
A & B & A+B \\
\hline
0 & 0 & 0 \\
0 & 1 & 1 \\
1 & 0 & 1 \\
1 & 1 & 1 \\
\hline
\end{array} $$

La sortie est \`a 1 si au moins une des entr\'ees est \`a 1.
\subsubsection{Porte OU multiple}

En reliant plusieurs entr\'ees de la m\^eme fa\c{c}on on obtient 
une porte OU multiple. Son fonctionnement se r\'esume en une phrase:
la sortie est \`a 1 si au moins une des entr\'ees est \`a 1.

Dans les sch\'emas logiques, on repr\'esente les portes logiques
par un symbole (figure \ref{log2})
\dessin{log2}{log2.eps}{Symboles des portes OU}
% \AFAIRE{Log2} 
% Symboles des portes OU \`a 2 et n entr\'ees


\subsubsection{Montage \`a transistors}

Le montage de la figure \ref{log8} remplit la m\^eme fonction. L'analyse
en est laiss\'ee au lecteur.
\dessin{log8}{log8.eps}{Porte OU \`a transistors en parall\`ele}
%\AFAIRE{Log8} 
%Porte OU \`a transistors en parall\`ele



\subsection{Porte ET}

\subsubsection{Montage \`a diodes}

Le sch\'ema (figure \ref{log3}) n'est pas sans rappeler celui de la porte OU.
\dessin{log3}{log3.eps}{Porte ET \`a diodes}


\subsubsection{Analyse du montage}

Chaque diode ne peut \^etre passante que si l'entr\'ee associ\'ee
est \`a 0. Si une des entr\'ees est \`a 0, la sortie sera donc \'egalement \`a 
0.  Par contre si les deux entr\'ees sont \`a +V, la sortie S sera \`a +V
au travers de la r\'esistance de rappel.

\subsubsection{Table de v\'erit\'e}

L'op\'eration logique ``ET'' (not\'e ``$\cdot$'') correspondant 
\`a ce montage poss\`ede donc la table de v\'erit\'e suivante:
$$ \begin{array}{|cc|c|}
\hline
A & B & A \cdot B \\
\hline
0 & 0 & 0 \\
0 & 1 & 0 \\
1 & 0 & 0 \\
1 & 1 & 1 \\
\hline
\end{array} $$

La sortie est \`a 1 si
les deux entr\'ees sont \`a 1.

On repr\'esente les portes ET (\`a deux ou plusieurs entr\'ees)
par un symbole (figure \ref{log4})
\dessin{log4}{log4.eps}{Symboles des portes ET}
%Symboles des portes ET \`a 2 et n entr\'ees


\subsubsection{Montage \`a transistors}

Le montage de la figure \ref{log7} remplit la m\^eme fonction. L'analyse
en est laiss\'ee au lecteur.
\dessin{log7}{log7.eps}{Porte ET \`a transistors en s\'erie}
% Porte ET \`a transistors en s\'erie



\subsection{Porte NON}

\subsubsection{Montage}

Dans la figure \ref{log5} on utilise un transistor en mode bloqu\'e/satur\'e.
\dessin{log5}{log5.eps}{Porte NON \`a transistor}
% \AFAIRE{Log5} 
% porte NON \`a transistor


\subsubsection{Analyse du montage}

Lorsque l'entr\'ee est au niveau haut, le transistor est saturé. La
sortie est alors reli\'ee \`a la masse par l'interm\'ediaire de la
r\'esistance interne tr\`es faible du transistor~: la sortie S est au
niveau bas.

Lorsque l'entr\'ee est au niveau bas, le transistor est bloqu\'e. La
r\'esistance de rappel ramène donc la sortie au niveau haut.

\subsubsection{Table de v\'erit\'e}

La table de l'op\'erateur NON ``$\neg$'' ainsi obtenu est
$$ \begin{array}{|c|c|}
\hline
A & \neg A \\
\hline
0& 1 \\
1& 0 \\
\hline
\end{array} $$
La sortie est \`a 1 si et seulement si l'entr\'ee est \`a 0.

On symbolise cet op\'erateur (figure \ref{log10}) par un triangle (indiquant 
par convention l'amplification) suivi d'un rond (n\'egation).
\dessin{log10}{log10.eps}{Symbole de la Porte NON}
% \AFAIRE{Log10} 
% 




\subsection{Porte NON-ET (NAND) }

\dessin{log9}{log9.eps}{Porte NAND} 
% porte NAND \`a 2 diodes + transistor


Le montage de la figure \ref{log9} r\'ealise une fonction dont la table de v\'erit\'e est:
$$ \begin{array}{|cc|c|}
\hline
A & B & S\\
\hline
0 & 0 & 1 \\
0 & 1 & 1 \\
1 & 0 & 1 \\
1 & 1 & 0 \\
\hline
\end{array} $$

On remarque ais\'ement que l'on a $S = \overline{A.B}$, et on appelle
cet op\'erateur le NON-ET (ou ``nand'').

\begin{exercice}{} Donnez un sch\'ema de cet op\'erateur utilisant 2 diodes,
une r\'esistance et un transistor.
\end{exercice}

On le repr\'esente par un symbole ``ET'' suivi du rond qui indique
la n\'egation (figure \ref{log11}).
\dessin{log11}{log11.eps}{Porte NON-ET}


La porte NAND a un int\'er\^et pratique \'evident: elle permet de reconstituer
tous les autres types de portes.
\begin{itemize}
\item La porte NON, puisque $\overline{A}= A \nand 1 $
\item La porte ET, puisque $A . B = \overline{(A \nand B)} =
 (A \nand B) \nand 1$
\item La porte OU, puisque $A + B = \overline{\overline{A}.\overline{B}}$
et donc $A+B= ((A \nand 1) \nand (B \nand  1)) \nand 1$.
\end{itemize}

\subsection{Porte NON-OU (NOR)}

La fonction ``non-ou'' (symbolis\'ee fig. \ref{log14}) est d\'efinie de la m\^eme fa\c{c}on, par l'\'equation
$$nor(A,B) = \overline{A + B}$$

\dessin{log14}{log14.eps}{Symbole de la porte NOR (non-ou)}

\begin{exercice}{} Proposez une porte NOR \`a transistors.
\end{exercice}

\subsection{Porte OU-exclusif (XOR)}

La fonction XOR ``ou-exclusif'' (fig. \ref{log15}) est souvent not\'ee
$\oplus$. On la d\'efinit par:
$$A \oplus B = \overline{A}B +A\overline{B}$$

\dessin{log15}{log15.eps}{Symbole de la porte XOR (ou-exclusif)}

\begin{exercice}{} Proposez une porte XOR \`a transistors.
\end{exercice}

\subsection{Le circuit int\'egr\'e CMOS 4011}

Les circuits int\'egr\'es logiques sont des bo\^{\i}tiers qui renferment plusieurs
portes logiques interconnect\'ees. Il existe une grande quantit\'e de
circuits int\'egr\'es logiques, renfermant des portes ET, OU, NAND, etc.
Pour des petits montages, il est \'economique d'utiliser un seul
type de circuit, comme le circuit CMOS 4011, qui contient 4 portes NAND.

\subsubsection{Brochage}

Le circuit 4011 se pr\'esente sous forme d'un bo\^{\i}tier DIL (Dual in line)
\`a 14 broches (voir figure \ref{log12}). Une encoche sur le c\^ot\'e gauche
permet de reconna\^{\i}tre le sens du circuit.
\dessin{log12}{log12.eps}{Circuit CMOS 4011 vu de dessus}
% \AFAIRE{Log12, circuit 4011} 

Les broches 14 (en haut \`a gauche) et 7 (en bas \`a droite) servent \`a 
l'alimentation du circuit 
(14: +V, 7: masse). Les autres sont les entr\'ees et sorties des
4 portes NAND:
\begin{itemize}
\item porte 1: entr\'ees 12 et 13, sortie 11;
\item porte 2: entr\'ees 8 et 9, sortie 10;
\item porte 3: entr\'ees 1 et 2, sortie 3;
\item porte 4: entr\'ees 5 et 6, sortie 4;
\end{itemize}

La figure \ref{log13} montre comment r\'ealiser une porte OU avec ce circuit.
\dessin{log13}{log13.eps}{Porte OU avec CMOS 4011}
%porte OU avec CMOS 4011

\begin{exercice}{Fonctions binaires usuelles}
Montrez comment r\'ealiser les expressions $A.B$, $A \oplus B$, $A+\overline{B}$
à l'aide d'un circuit 4011.
\end{exercice}

\begin{exercice}{Fonction majorit\'e}
R\'ealisez la fonction $maj(A,B,C)$, dont le r\'esultat est 1 si au moins deux
entr\'ees sont à 1, à l'aide d'un circuit 4011 (ou plusieurs).
\end{exercice}

\begin{exercice}{}
R\'ealisez la fonction $f(A,B,C)= \mbox{si $A=1$ alors $B.C$ sinon $B+C$}$.
\end{exercice}

\subsubsection{Conditions d'emploi des circuits CMOS}

Les circuits de type CMOS pr\'esentent certains avantages pour les
montages exp\'erimentaux:

\begin{itemize}
\item ils acceptent une tension d'alimentation entre 3 et 15 V;
\item leur consommation est tr\`es faible (de l'ordre de 0,1 mW par porte);
\item les entr\'ees des circuits CMOS ont une imp\'edance tr\`es \'elev\'ee~: on peut
relier de nombreuses entr\'ees sur la sortie d'une porte sans craindre 
de ``tirer''
trop de courant de celle-ci.
\end{itemize}

Par contre, ces circuit sont sensibles \`a l'\'electricit\'e
statique\footnote{En particulier ``effet d'antenne'' lorsqu'on
approche la main d'un circuit dont une des entrées est restée ``en
l'air''}.  De plus, la propagation des signaux de l'entr\'ee \`a la
sortie d'une porte est plus lente (20-40 ns) qu'avec d'autres familles
de circuits, comme les TTL.

\subsubsection{Emploi des circuits TTL}

Les circuits TTL (transistor-transistor-logic) sont tr\`es utilis\'es pour
les r\'ealisations professionnelles. A titre d'exemple, 
les caract\'eristiques de la s\'erie SN74
sont:
\begin{itemize}
\item tension nominale d'alimentation de $5v \pm 0.5V$, 
risque de claquage \`a partir de 7V
\item fonctionnement entre $0$ et $70^oC$
\item puissance par porte de l'ordre de 10 mW
\item courant d'entr\'ee de l'ordre de $1.5 mA$
\item temps de propagation de l'ordre de 10 \`a 20 ns.
\end{itemize} 



