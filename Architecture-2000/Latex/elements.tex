
\chapter{El\'ements d'un syst\`eme informatique}

Un ordinateur comporte:
\begin{itemize}
\item une {\em unit\'e centrale}  charg\'ee d'effectuer les traitements~; 
\item des {\em unit\'es p\'eriph\'eriques} d'entr\'ee, de sortie, de stockage.
\end{itemize}

L'unit\'e centrale est compos\'ee
d'un processeur et d'une m\'emoire centrale.
Le processeur ex\'ecute les instructions du programme qui est enregistr\'e 
dans la m\'emoire, et effectue des op\'erations
qui agissent sur les donn\'ees (\'egalement
enregistr\'ees dans la m\'emoire) ou d\'eclenchent -par le biais des
p\'eriph\'eriques- des transferts de donn\'ees entre la m\'emoire et le monde
ext\'erieur.


