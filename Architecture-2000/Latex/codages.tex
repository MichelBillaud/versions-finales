\chapter{Codages de l'information}

Les ordinateurs manipulent les informations sous forme {\em binaire}~: 
toutes les donn\'ees sont repr\'esent\'ees par
des suites d'\'el\'ements d'information ({\em bits}) qui ne peuvent
prendre que deux valeurs, appel\'ees arbitrairement $0$ et $1$ 
(ou {\em vrai} et {\em faux}).


On remarque qu'une s\'equence de 2 bits peut prendre 4 valeurs diff\'erentes
$00, 01, 10, 11$. Une s\'equence de 3 bits peut prendre 8 valeurs: $000$,
$001$,$010$,$011$,$100$,$101$,$110$,$111$. Plus g\'en\'eralement,   il
 y a $2^n$ s\'equences de $n$ bits.

\begin{exercice}{} 
Combien faut-il de bits d'information pour d\'esigner une
carte parmi les 78 d'un jeu de tarot ?
\end{exercice}

Le codage choisi d\'ependra de
la nature des informations \`a coder (caract\`eres, nombres entiers ou pas, 
positifs ou pas etc.) et les traitements que l'on d\'esire effectuer sur
ces informations.

\section{Bases de num\'eration}

\subsection{La notation positionnelle}

Dans la vie courante nous utilisons le 
{\em syst\`eme d\'ecimal positionnel} 
pour repr\'esenter les nombres~; {\em d\'ecimal} 
parce qu'il emploie dix symboles
(les chiffres 0 \`a 9), {\em positionnel} parce que la valeur que 
l'on attache \`a 
ces chiffres d\'epend de leur position dans le nombre.  

\begin{exemple}{} Dans
$4305,72$ nous accordons au 4 une valeur de milliers, au 3 des centaines, etc.
La valeur de ce nombre se lit  donc:
$$(4 \times 1000) + (3 \times 100) + (0 \times 10) + (5 \times 1) + 
(7 \time 1/10) + (2 \times 1/100)$$
ou encore
$$4 \times 10^3 + 3 \times 10^2 + 0 \times 10^1 + 5 \times 10^0 + 
7 \times 10^{-1} + 2 \times 10^{-2}$$
\end{exemple}

Ce syst\`eme positionnel est utilisable avec d'autres bases de num\'eration.
Les informaticiens emploient fr\'equemment les syst\`emes suivants:
\begin{itemize}
\item {\em binaire}, ou base 2, qui utilise seulement deux chiffres 0 et 1,
\item {\em octal}, ou base 8, (chiffres 0 \`a 7)
\item {\em hexad\'ecimal}, ou base 16, qui emploie les chiffres $0$ \`a $9$, 
puis les lettres A (qui repr\'esente la valeur $10$), B=11, \ldots F=15
\end{itemize} 

Losqu'il y a risque de confusion, on fait figurer la base de num\'eration 
en indice ou entre parenth\`eses apr\`es le nombre concern\'e.

Le principe commun de toutes ces bases est le suivant: un nombre $N$ \'ecrit 
en base $B$ est 
repr\'esent\'e par une suite de chiffres adjacents $x_n x_{n-1} \ldots x_1 x_0$.
Chacun de ces chiffres est un symbole qui a une valeur enti\`ere comprise entre
0 et $B-1$. A chaque position est attribu\'e un poids qui est une puissance de 
$B$. Le chiffre le plus \`a droite a un poids de $1 = B^0$, son voisin 
de $B=B^1$, le suivant de $B^2$ etc. La valeur du nombre $N$ s'obtient en faisant
la somme des produits des valeurs des chiffres par leur poids respectifs:
$$N = x_n \times B^N + x_{n-1} \times B^{n-1} +  \ldots + x_1 \times B^1
+ x_0 \times B^0 $$

\begin{exemple}{Expression d\'ecimale de $421(8)$ }
$$4 \times 8^2 + 2 \times 8^1 + 1 \times 8^0 
= 4 \times 64 + 2 \times 8 + 1 = \ldots (10)$$
\end{exemple}

Ceci se g\'en\'eralise facilement aux nombres ``\`a virgule'': 
les chiffres de la ``partie 
d\'ecimale'' ont alors des poids successifs de 
$B^{-1} = 1/B, B^{-2} = 1/{B^2}, \ldots$.
 
\subsection{Conversion vers le syst\`eme d\'ecimal}

La m\'ethode usuelle de conversion d\'ecoule imm\'ediatement 
de la d\'efinition donn\'ee plus haut.

\begin{exemple}{Expression d\'ecimale de $N=1001011(2)$}
On \'ecrit de droite
\`a gauche, sous chacun des chiffres de N, les puissances successives de 2.
On multiplie ensuite chaque chiffre par son poids, et on fait la somme.
$$\begin{array}{r|rrrrrrr|l}
N= &1 & 0 & 0 & 1 & 0 & 1 & 1 \\
\times poids &64 &32& 16& 8& 4 &2 &1 \\
\hline
= &64 & 0 & 0 & 8 & 0 & 2 & 1 & total =  75 (10)
\end{array}$$
\end{exemple}

\subsection{De d\'ecimal vers base quelconque}


\subsubsection{M\'ethode des restes successifs}

Diviser le nombre \`a convertir par  la valeur de la base. Le reste de
la division fournit le chiffre de droite, et recommencer avec le quotient.
S'arr\^eter \`a 0.

\begin{exemple}{Conversion de $1789 (10)$ en hexad\'ecimal}
\begin{itemize}
\item $1789 = 111 \times 16 + 13$. On pose $x_0 = D$.
\item $111 = 6 \times 16 + 15$. On pose $x_1 = F$.
\item $6 = 0 \times 16 + 6$. On pose $x_2 = 6$.
\item le r\'esultat est $6FD (16)$ 
\end{itemize}
\end{exemple}

\subsubsection{M\'ethode soustractive}


Écrire une liste des puissances successives de la base, 
de droite \`a gauche. Chercher le 
plus grand poids contenu dans le nombre. Diviser le nombre par ce poids, 
\'ecrire le quotient en dessous, et recommencer avec le reste. On met des 
0 dans les autres positions.

On utilise cette m\'ethode surtout avec le syst\`eme binaire puisque la
division ne n\'ecessite en fait qu'une soustraction (le quotient
est forc\'ement 1). 

\begin{exemple}{Conversion de $1789 (10)$ en binaire}
 
La liste des poids est
$$ \ldots 2048, 1024, 512, 256, 128, 64, 32, 16, 8, 4, 2, 1$$
\begin{itemize}
\item Comme $2048 > 1789 \geq 1024 $, on met 1 sous 1024, et on soustrait
ce poids $1789 -1024 = 765$
\item Comme $1024 > 765 \geq 512 $, on met 1 sous 512, et on soustrait
ce poids $765 - 512 = 253$
\item on met 1 sous 128, et on soustrait $253-128 = 125$
\item on met 1 sous 64, et on soustrait $125-64 = 61$
\item on met 1 sous 32, et on soustrait $61-32 = 29$
\item on met 1 sous 16, et on soustrait $29-16 = 13$
\item on met 1 sous 8, et on soustrait $13-8 = 5$
\item on met 1 sous 4, et on soustrait $5-4 = 1$
\item on met 1 sous 1, et on arr\^ete. Le r\'esultat est $110\ 1111\ 1101 (2)$.
\end{itemize}
\end{exemple}

\subsection{Conversions entre bases 2, 8 et 16}

Les nombres 8 et 16 \'etant des puissances enti\`eres de 2, 
il existe un moyen tr\`es rapide
de convertir un nombre binaire en octal ou hexad\'ecimal 
(et r\'eciproquement), puisqu'\`a
chaque chiffre octal (resp. hexad\'ecimal) correspond une ``tranche'' de
3 (resp. 4) bits dans la représentation binaire du nombre.

\begin{exemple}{Conversion de  $11011111101 (2)$ en octal et hexad\'ecimal}
On d\'ecoupe le nombre en tranches de 3 bits (mat\'erialis\'ees ici par
des espaces) en partant des unit\'es. On obtient $11\ 011\ 111\ 101 (2)$.
On lit  ensuite ce nombre en traduisant chaque tranche par un chiffre octal,
et on trouve
	$3375 (8)$.

Si on d\'ecoupe par tranches de 4 on obtient $110\ 1111\ 1101 (2)$, ce qui 
donne $6FD (16)$.
\end{exemple}

Le syst\`eme hexad\'ecimal permet de coder des séquences binaires
de façon  4 fois plus courte,  les conversions se faisant sans calcul.


\section{Arithm\'etique binaire}

Les op\'erations arithm\'etiques usuelles 
peuvent se faire dans toutes les bases, selon des proc\'ed\'es
tr\`es proches de l'arithm\'etique d\'ecimale.  Nous ne pr\'esenterons ici que
l'arithm\'etique et la multiplication binaire, les autres op\'erations (ainsi
que la g\'en\'eralisation \`a d'autres bases) est
laiss\'ee en exercice.

\subsection{Addition binaire}
Comme il n'y a que deux chiffres, la table d'addition se r\'eduit \`a peu de choses
$$\begin{array}{r|rr|}
+ &0 &1 \\
\hline
0 &0 &1 \\
1 &1 &10 \\
\hline
\end{array} $$

Pour additionner deux nombres $A$ et $B$, on les \'ecrit l'un sous l'autre, 
et on commence
l'addition par la droite. L'addition de deux chiffres 1 engendre une retenue, 
que l'on propage au rang suivant. 

\begin{exemple}{Somme de 1011101 et 1110001}
$$\begin{array}{rrrrrrrrrl}
&1 & 1 & 1 &   &   &   & 1 &   & retenues \\
&  & 1 & 0 & 1 & 1 & 1 & 0 & 1 & A \\
+&  & 1 & 1 & 1 & 0 & 0 & 0 & 1 & B \\
\hline
= &1 & 1 & 0 & 0 & 1 & 1 & 1 & 0 & Somme
\end{array} $$
\end{exemple}

% \subsection{Soustraction binaire}
%
% Similaire \`a l'op\'eration en d\'ecimal.

\subsection{Multiplication binaire}

\begin{exemple}{Produit de 110011 et 1101}
$$\begin{array}{rrrrrrrrrr}
 & & & &1&1&0&1&1 \\
\times & & & & &1&1&0&1 \\
\hline
 & & & &1&1&0&1&1 \\
 & & &0&0&0&0&0& \\
 & &1&1&0&1&1& & \\
 &1&1&0&1&1& \\
\hline
1&0&1&0&1&1&1&1&1
\end{array} $$
\end{exemple}

A chaque \'etape, les multiplications partielles par chacun des
chiffres du second nombre se ram\`enent
soit \`a recopier le premier nombre avec un d\'ecalage,  soit \`a inscrire des 0.

% \subsection{Division}
%
%Laiss\'ee en exercice.

\section{Codages des nombres entiers en binaire}

Le probl\`eme est de coder des nombres avec une s\'equence binaire de
longueur limit\'ee.  On sait d\'ej\`a qu'avec un nombre fini de bits,
on ne pourra repr\'esenter qu'un nombre fini de valeurs ($2^N$, o\`u
$N$ est le nombre de bits).

\subsection{Binaire pur}

Si les nombres \`a repr\'esenter sont toujours positifs ou nuls, on pourra
adopter le codage {\em binaire pur non-sign\'e}, qui est une application
directe de la num\'eration binaire vue ci-dessus.  

Sur un octet (8 bits) on \'ecrira
un nombre qui sera forcement compris entre 0 et $2^8-1 =  255$. 

\begin{exemple}{} Le nombre 
$35 (10)$ sera cod\'e par $0010\ 0011$. Le plus grand nombre repr\'esentable
(ici 255) sera traduit par une suite de 1\~: $1111\ 1111$.
\end{exemple}

\begin{exercice}{} Quels nombres peut on coder sur 16 bits ? Sur 32 bits ?
\end{exercice}

\subsection{Binaire sign\'e}

Si l'on veut repr\'esenter des nombres positifs ou n\'egatifs, la premi\`ere id\'ee
qui vient est de r\'eserver un bit (par exemple celui de gauche) pour le signe,
(0 pour positif, 1 pour n\'egatif)
et de coder la valeur absolue du nombre sur les bits restants.

\begin{exemple}{} Sur 8 bits, $+35 (10)$ sera cod\'e par $0010\ 0011$, et 
$-35 (10)$ par $1010\ 0011$. 
\end{exemple}

Quoique simple, ce proc\'ed\'e de codage n'est quasiment pas utilis\'e, car il
ne permet pas de faire facilement des op\'erations sur les nombres ainsi
cod\'es (pour additionner deux nombres il faut examiner leurs signes puis
additionner les valeurs absolues, ou soustraire la plus petite de la plus 
grande).

En outre, dans ce syst\`eme le nombre 0 possède
deux codages  ($+0$ et $-0$). On ne code donc que $2^N-1$ nombres 
distincts (de $-(2^{N-1}-1)$ \`a $+2^{N-1}-1)$.

\subsection{Repr\'esentation biais\'ee}

L'id\'ee est de d\'ecaler les entiers de l'intervalle choisi
en leur soustrayant la plus petite valeur de cet intervalle. Cela permet
de se ramener au codage des nombres positifs ou nuls.

\begin{exemple}{} On choisit de coder l'intervalle  $-127\ldots+128$. Le codage du 
nombre $-94$ s'obtient en lui ajoutant $127$, et en traduisant le r\'esultat
$-94+127=33$ en binaire, soit $0010\ 0001 (2)$.
\end{exemple}

Pour additionner deux nombres en représentation biaisée, il faudra additionner
leurs représentations biaisées, puis soustraire le biais.

\begin{exercice}{}
Détailler le calcul de $2+3$ en représentation biaisée sur 4 bits.
Que faut-il faire pour une soustraction ?
\end{exercice}

\subsection{Binaire compl\'ement \`a deux}

Ce proc\'ed\'e permet de coder, sur $N$ bits,
 les nombres de $-2^{N-1}$ \`a $+2^{N-1}-1$:
\begin{itemize}
\item les nombres positifs (de 0 \`a $+2^{N-1}-1$) sont \'ecrits en binaire pur 
(on remarque que le bit de gauche sera alors forc\'ement \`a 0)
\item aux nombres n\'egatifs (entre $-2^{N-1}$ et $-1$), on ajoute $2^N$.
Le nombre obtenu est alors \'ecrit en binaire pur. Comme il est compris 
entre $2^{N-1}$ et $2^N-1$, son premier bit est \`a 1.
\end{itemize}

\begin{exemple}{Codage de $-35$ sur un octet.} Le nombre \`a coder \'etant n\'egatif,
on lui ajoute $256$, et le probl\`eme se ram\`ene donc \`a coder 
$256-35 = 221$ en binaire pur, soit $1101\ 1101$.
\end{exemple}

On le voit, ce codage est bas\'e sur l'arithm\'etique modulo $2^N$. Puisque
les nombres -positifs ou n\'egatifs- sont repr\'esent\'es par leurs 
valeurs positives modulo $2^N$,
l'addition et la soustraction se font sans se pr\'eoccuper des signes~:
les circuits de calcul en seront grandement simplifi\'es.
C'est l'avantage d\'ecisif qui a conduit \`a l'adoption 
générale de ce syst\`eme.

Le d\'enomination de ``compl\'ement \`a deux'' provient 
de la remarque suivante:
pour calculer l'oppos\'e d'un nombre ainsi cod\'e, il suffit d'ajouter 1 \`a
son {\em compl\'ement \`a 1}, que l'on obtient en changeant les 1 en 0 et 
r\'eciproquement.

\begin{exemple}{Codage de -35 par compl\'ementation \`a deux.}
En binaire pur $35 (10)$ s'\'ecrit -sur un octet- $0010\ 0011$. 
Le compl\'ement \`a 1 est $1101\ 1100$.  En ajoutant 1 on trouve
$1101\ 1101$, qui est le codage attendu.
\end{exemple}

\begin{remarque}{}
Cette op\'eration est idempotente: si on compl\'emente
$1101\ 1101$ on trouve $0010\ 0010$, et en ajoutant 1 on obtient $0010\ 0011$
qui est la repr\'esentation de $+35$.  On utilise cette propri\'et\'e pour le
d\'ecodage: pour d\'ecoder un nombre \'ecrit en binaire compl\'ement \`a deux
\begin{itemize}
\item si le premier bit est \`a 0, le signe est $+$, et la valeur absolue est obtenue
par un d\'ecodage binaire non sign\'e du nombre~;
\item si le premier bit est \`a 1, le signe est $-$, et la valeur absolue
s'obtient par d\'ecodage binaire non sign\'e du compl\'ement \`a 2 de 
ce nombre.
\end{itemize}
\end{remarque}

\begin{exemple}{D\'ecodage de $1111\ 0100$.} Le signe est n\'egatif. Le
compl\'ement \`a 2 est $0000\ 1011 + 1 = 0000\  1100$ qui, traduit en d\'ecimal vaut $12$.
Le nombre cherch\'e est donc $-12$.
\end{exemple}

\section{D\'ecimal Cod\'e Binaire}

Cette notation hybride est utilis\'ee pour le codage des grands nombres
entiers ou \`a virgule fixe dans les machines destin\'ees aux applications
commerciales: chaque chiffre {\em d\'ecimal} est cod\'e par une tranche de 
4 bits.

\begin{exemple}{Codage BCD de 14285739.} En mettant des espaces pour bien voir 
les tranches: $$0001\ 0100\ 0010\ 1000\ 0101\ 0111\ 0011\ 1001$$
\end{exemple}

Les proc\'ed\'es de codages et d\'ecodages entre d\'ecimal et BCD sont tr\`es simples,
et nous verrons que les circuits d'addition ne sont pas beaucoup plus 
compliqu\'es que dans le cas du binaire pur.

\section{La repr\'esentation en virgule flottante}

La repr\'esentation en virgule flottante permet de repr\'esenter
des nombres r\'eels tr\`es petits et tr\`es grands, avec une bonne
pr\'ecision.

\subsection{La repr\'esentation flottante}

L'id\'ee de base est similaire \`a la notation des nombres
sous la forme d'une mantisse multipli\'eee par une puissance de 10. 
Par exemple $12.34 \times 10^{-30}$.

Cette notation est tr\`es employ\'ee en physique. Elle revient \`a d\'ecrire un
nombre par un couple $(m,e$) form\'e de la mantisse et de l'exposant, ici $(12.34, -30)$.
Il y a plusieurs couples qui repr\'esentent le m\^eme nombre -par exemple
$(1.234, -29), (1234, -32), \ldots$ - on peut donc
convenir d'une repr\'esentation {\em normalis\'ee}, par exemple en d\'ecidant
que $1 \leq | m | < 10 $. Tous les nombres (sauf 0) auront alors une 
repr\'esentation unique.

En binaire, c'est la m\^eme id\'ee~: le nombre sera de la forme 
$m \times 2^e$,
avec  $0 < | m | \leq 2$.  

\begin{exemple}{} Normaliser le nombre -13 en vue d'une repr\'esentation binaire.

Comme 13 est compris entre $8=2^3$ et $16=2^4$, on prend $e=3$ et
 $m=-13/8= -1.625 (10) = -1.101 (2) $.
\end{exemple}

\subsection{Le format flottant standard IEEE}

\subsubsection{Repr\'esentations normalis\'ees}

Le format standard IEEE simple pr\'ecision code les nombres r\'eels sur 32 bits
(soit 4 octets).
\begin{itemize}
\item le premier bit S code le signe du nombre (0:positif, 1:n\'egatif).
\item 8 bits pour l'exposant (zone E). La repr\'esentation est
biais\'ee: cette zone repr\'esente un nombre $e$ compris dans
l'intervalle $-127..+128$, cod\'e en binaire non sign\'e
avec un d\'ecalage de 127:
$$ e = E -127 $$
Attention les ``exposants'' E=0 et E=127 sont réservés pour le codage
de nombres ``spéciaux'' (voir plus loin).
\item 23 bits pour la mantisse. Cette zone M ne contient que la
partie d\'ecimale de la mantisse $m$, en effet
la convention de normalisation nous assure que la partie enti\`ere est
toujours 1. (Pour le codatge du 0, voir plus loin).
\end{itemize}


\begin{exemple}{} Coder $-13$ dans le format IEEE simple pr\'ecision.

On sait d\'ej\`a que $-13 = -1.101 (2) \times 2^3$.
\begin{itemize}
\item Le nombre \'etant  n\'egatif, le signe S est 1.
\item L'exposant est 3, on lui ajoute 127, et on obtient 130 qui,
en binaire sur 8 bits s'\'ecrit $1000 0010$.
\item La mantisse $1.101 (2)$ priv\'ee de sa partie enti\`ere est 
cod\'ee sur 23 bits: $101\ 0000\ 0000\ 0000\ 0000\ 0000$
\end{itemize}

Le nombre r\'eel -13 est donc cod\'e en binaire:
$$ 1100\ 0001\ 0101\ 0000\ 0000\ 0000\ 0000\ 0000 $$
ou $C150 0000_H$, si on pr\'ef\`ere l'hexad\'ecimal (plus maniable).
\end{exemple}

\begin{exercice}{} D\'ecoder le nombre $1234 0000_H$.
\end{exercice}

\begin{exercice}{} Quel est le plus grand nombre normalis\'e ? Le plus petit 
(en valeur absolue)~?
\end{exercice}

\subsubsection{Les z\'eros, les infinis et les autres}

Le nombre z\'ero est, par convention, repr\'esent\'e avec les zones E
et M remplies de zéros, le signe S \'etant indiff\'erent (il y a donc
deux z\'eros \ldots)~:
$$ x000\ 0000\ 0000\ 0000\ 0000\ 0000\ 0000\ 0000 $$

Les deux infinis $+\infty$ et  $-\infty$ ont leur zone E remplies de 1, 
et la zone M \`a 0:
$$ S111\ 1111\ 1000\ 0000\ 0000\ 0000\ 0000\ 0000 $$

Certains codes sp\'eciaux repr\'esentent des erreurs, on les appelle des NaN 
(Not a Number). On les
reconnait \`a leur zone E remplie de 1, et leur zone M non nulle.

\subsubsection{Pr\'ecision du codage}

L'\'eventail des nombres r\'eels que l'on peut coder ainsi est tr\`es large, 
mais la pr\'ecision n'est pas illimit\'ee. On peut mesurer cette pr\'ecision
en calculant l'\'ecart qu'il y a entre le nombre 1.0 et le plus petit
nombre repr\'esentable qui lui soit imm\'ediatement sup\'erieur.

En repr\'esentation normalis\'ee le nombre 1 s'\'ecrit 
$1.00\ldots \times 2^0$. Comme la zone M contient les 23 chiffres
``d\'ecimaux'' de la mantisse, le nombre suivant s'obtient en rempla\c{c}ant
le chiffre le plus \`a droite par un 1, ce qui correspond \`a $1+ 2^{-23}$.
La pr\'ecision est donc de $2^{-23}$, 
soit \`a peu pr\`es $1.2 \times 10^{-7}$. On compte donc, au mieux, sur 6
chiffres significatifs.

Attention, la pr\'ecision se perd tr\`es rapidement lors des calculs r\'ep\'etitifs,
par exemple lors d'\'evaluation de s\'eries de Taylor.

Pour avoir une meilleure pr\'ecision on peut employer des nombres en double
pr\'ecision, la mantisse occupant alors 32 bits suppl\'ementaires.

\section{Codage des caract\`eres}

L'\'echange de donn\'ees avec les p\'eriph\'eriques d'entr\'ee
(clavier) et de sortie (\'ecran, imprimantes) se fait caract\`ere par
caract\`ere. Dans les premiers dispositifs utilis\'es (télétypes)
les caract\`eres \'etaient cod\'es sur 6 bits (26 lettres + 10
chiffres + 28 symboles). La distinction entre majuscules et minuscules
a exig\'e le passage \`a 7 bits (codes ASCII et EBCDIC), et le passage
\`a 8 bits est rendu n\'ecessaire pour la repr\'esentation des
caract\`eres accentu\'es des langues europ\'eennes bas\'ees sur
l'alphabet romain (ISO 8859 Latin1).  Il existe également un jeu de
caract\`eres ``universel'' sur 16 bits appelé UNICODE, qui regroupe les
principaux alphabets utilisés dans le monde.

\subsection{Code ASCII 7 bits}

Le code ASCII (American Standard Code for Information Interchange)
 utilise une table de 128 positions, que l'on rep\`ere par leur num\'ero en 
d\'ecimal. Si on met cette table sous forme de 8 colonnes, de 16 lignes,
les 2 premi\`eres colonnes contiennent des caract\`eres 
dits ``non imprimables'' suivis par des symboles divers.
Les chiffres commencent en quatri\`eme colonne (position 48). Les majuscules
sont en cinqui\`eme et sixi\`eme colonne (\`a partir de 65), les minuscules dans les
deux derni\`eres (\`a partir de 97).

En pratique les caract\`eres ASCII sont transmis dans des octets, le bit 
suppl\'ementaire \'etant ignor\'e ou utilis\'e pour effectuer un {\em contr\^ole de
parit\'e}:
\begin{itemize}
\item bit de parit\'e \`a 0: l'\'emetteur met 0 dans le bit de gauche. Si le 
r\'ecepteur y lit un 1, il y a eu une erreur de transmission.
\item bit de parit\'e \`a 1 (semblable au pr\'ec\'edent).
\item parit\'e paire: l'\'emetteur met 0 ou 1 dans le bit de gauche de fa\c{c}on
\`a ce que le nombre total de 1 dans l'octet soit pair. Si le r\'ecepteur re\c{c}oit
un octet avec un nombre impair de 1, il y a eu une erreur. 
\item parit\'e impaire: comme ci-dessus.
\end{itemize}

Ce code suffit pour la langue anglaise, qui ignore les accents. 


\subsection{Code EBCDIC}

Autrefois pr\'econis\'e par le principal constructeur d'ordinateurs de 
l'\'epoque (IBM). En d\'esu\'etude maintenant.

\subsection{Code ANSI/ISO Latin 8859-1}

L'existence du march\'e europ\'een a rendu n\'ecessaire l'adoption de codes
contenant les lettres accentu\'ees de l'alphabet latin. Divers codes 8 bits
ont  \'et\'e utilis\'es, qui sont (\`a peu de choses pr\`es) des extensions du 
code ASCII 7 bits: ASCII-IBM (pour les PC sous DOS), ANSI (PC sous Windows),
ASCII-McIntosh etc. Ces codes sont bien s\^ur incompatibles entre eux.

Actuellement un consensus se d\'egage (c'est une cons\'equence de la
profusion des r\'eseaux h\'et\'erog\`enes) pour l'utilisation du code
ANSI, normalis\'e par l'ISO (International Standard Organization) sous
la r\'ef\'erence ISO-8859-1 (ou ISO-Latin-1).  Ce code 8 bits contient
le n\'ecessaire pour les alphabets d\'eriv\'es de l'alphabet romain:
langues latines mais aussi anglo-saxonnes, nordiques, polonais, turc,
vietnamien etc.  (Avec l'oubli f\^acheux du caract\`ere \OE).

\begin{exercice}{}Chercher dans les documentations accessibles par le Web
si le caractère ``euro'' a bien été ajouté. 
\end{exercice}

\subsection{En attendant UNICODE...}

Le code 16 bits UNICODE permet de repr\'esenter 65536 caract\`eres, 
ce qui suffit pour les alphabets les plus r\'epandus (romain, cyrillique,
arabe, h\'ebreu, grec, hiragana, katagana, cor\'een \ldots), mais pas pour
les id\'eogrammes chinois.  L'inconv\'enient d'UNICODE est de n\'ecessiter
deux fois plus de place pour repr\'esenter un caract\`ere.
