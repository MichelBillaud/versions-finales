\chapter{Alg\`ebre de Boole et circuits logiques}

En combinant les portes logiques, et en prenant garde de ne pas faire
de boucles dans les circuits\footnote {les circuits combinatoires
(sans boucles) et s\'equentiels (avec boucles) seront \'etudi\'es dans
les chapitres suivants}, on r\'ealise des circuits dont les sorties ne
d\'ependent que des valeurs des entr\'ees. On pourra alors d\'ecrire
la fonction r\'ealis\'ee par ce circuit par sa table de v\'erit\'e.

L'\'etude des ``tables de v\'erit\'e'' 
a \'et\'e entreprise il y a plus d'un si\`ecle, et constitue ce
qu'on appelle l'alg\`ebre de Boole\footnote{L'oeuvre la plus
c\'el\`ebre de George Boole (1815-1864) est intitul\'ee {\em Une investigation
dans les lois de la pens\'ee, sur lesquelles sont fond\'ees les th\'eories 
math\'ematiques de la logique et des probabilit\'es} (1854), plus connue
sous le titre abr\'eg\'e {\em Les lois de la pens\'ee}}.

Les propri\'et\'es de cette alg\`ebre nous permettront de {\em raisonner}
sur les circuits, et donc de les construire rigoureusement, plut\^ot que
par la (co\^uteuse) m\'ethode des essais et erreurs. En particulier
nous verrons des m\'ethodes syst\'ematiques pour simplifier les circuits.


\section{D\'efinitions}

% \subsubsection{Fonctions bool\'eennes}

On appelle {\em fonction bool\'eenne} toute fonction de $\{0,1\}^n$ 
dans $\{0,1\}$.

Il y a $2^{2^n}$ fonctions bool\'eennes \`a $n$ variables: en effet pour 
d\'ecrire une fonction bool\'eenne \`a $n$ variables il suffit de remplir les
$2^n$ cases de sa table de v\'erit\'e, chaque case pouvant contenir 0 ou 1.




%\subsubsection{Expressions bool\'eennes}

Une {\em expression bool\'eenne} est un terme construit \`a partir de variables
(prises dans un ensemble $V$), de constantes 0 ou 1, et d'op\'erateurs ET, OU, 
et NON. Exemple~:  $a + \overline{a . \overline{b}} + b$.

Une expression bool\'eenne {\em repr\'esente} une fonction
bool\'eenne~; nous dirons que deux expressions sont \'equivalentes si
elles d\'enotent la m\^eme fonction.

\begin{exemple}{} Si nous construisons la table de v\'erit\'e de l'expression ci-dessus,
et de l'expression $\overline{a}+b$:
$$ \begin{array}{|cc|ccc||c||c||c|}
\hline
a &b & \overline{b}& a.\overline{b}& \overline{a.\overline{b}} & 
  \overline{a.\overline{b}} + b & \overline{a} & \overline{a}+b \\
\hline
0 & 0  & 1 & 0 & 1 & 1 & 1 & 1\\
0 & 1  & 0 & 0 & 1 & 1 & 1 & 1\\
1 & 0  & 1 & 1 & 0 & 0 & 0 & 0\\
1 & 1  & 0 & 0 & 1 & 1 & 0 & 1\\
\hline
\end{array}$$
nous constatons qu'elles co\"{\i}ncident~: les deux expressions repr\'esentent
la m\^eme fonction.
\end{exemple}

Par commodit\'e, on appelle {\em somme} toute expression de la forme 
$t_1 + t_2 + \ldots + t_n$.
Elle peut se r\'eduire \`a un seul terme, ou aucun (dans ce cas c'est 0).
Un {\em produit} est de la forme $t_1.t_2 \ldots t_n$. Le produit vide
est $1$.



Un {mon\^ome} est un produit de variables ou de
n\'egations de variables, chaque variable n'apparaissant qu'au plus une fois.

\begin{exemple}{de mon\^omes: } $x, \overline{y}, x.\overline{y}.z, 1$
\end{exemple}

\begin{remarque}
À partir d'un ensemble de $n$ variables on peut construire $C^k_n.2^k$
mon\^omes distincts \`a $k$ variables~: pour choisir un tel
monôme il suffit de s\'electionner
$k$ variables parmi $n$ (d'o\`u le coefficient binomial), et
d'attribuer ou non \`a chacune d'entre elles une barre ($2^k$
possibilit\'es).


Il y a en tout $3^n$ mon\^omes, puisqu'il suffit de d\'ecider, pour chacune des
variables, de la faire figurer telle quelle, avec une barre, ou pas du tout.
Ceci prouve l'\'egalit\'e:
$$ \sum_{k=0}^n C^k_n.2^k = 3^n$$
que l'on peut d'ailleurs retrouver en d\'eveloppant $(1 + 2)^n$ \`a l'aide des
formules connues.
\end{remarque}

Un mon\^ome est {\em complet} par rapport \`a un ensemble donn\'e de variables
si toutes les variables de cet ensemble appara\^{\i}ssent une fois. Il y a 
$2^n$ mon\^omes complets sur un ensemble de $n$ variables, chaque mon\^ome
complet correspondant \`a une des cases d'une table de v\'erit\'e \`a $n$ variables.

Une expression est {\em en forme canonique} si elle est  \'ecrite sous forme
d'une somme sans r\'ep\'etition de mon\^omes complets.


\paragraph{Proposition} Toute fonction bool\'eenne peut \^etre construite
par composition d'op\'erations OU, ET et NON.


\paragraph{Preuve} Toute fonction bool\'eenne $f(a,b,c\ldots)$ peut \^etre 
d\'ecrite par sa table de v\'erit\'e. Chaque case de cette table correspond
\`a un {\em mon\^ome complet}. Une expression de la fonction $f$ s'obtient
en faisant la somme des mon\^omes complets dont la valeur est 1.

\begin{exemple}{} Consid\'erons une fonction $maj$ (majorit\'e) \`a trois variables $a,b,c$, qui 
vaut 1 quand au moins deux des entr\'ees sont \`a un. On dresse la table de
v\'erit\'e de $maj$, en faisant figurer dans la marge les mon\^omes correspondants:
$$\begin{array}{|ccc|c|l}
\cline{1-4}
a & b & c & maj(a,b,c) & \mbox{mon\^ome} \\
\cline{1-4}
0 & 0 & 0 & 0 & \overline{a}\overline{b}\overline{c} \\
0 & 0 & 1 & 0 & \overline{a}\overline{b}c \\
0 & 1 & 0 & 0 & \overline{a}b\overline{c} \\
0 & 1 & 1 & 1 & \overline{a}bc \\
1 & 0 & 0 & 0 & a\overline{b}\overline{c} \\
1 & 0 & 1 & 1 & a\overline{b}c \\
1 & 1 & 0 & 1 & ab\overline{c} \\
1 & 1 & 1 & 1 & abc \\
\cline{1-4}
\end{array} $$
et on obtient l'expression en forme canonique $$
	maj(a,b,c)=  \overline{a}bc + a\overline{b}c +
	ab\overline{c} + abc$$
\end{exemple}

\section{Propri\'et\'es des alg\`ebres de Boole}

Les operations ET, OU, NON definies sur l'ensemble $\{0,1\}$ poss\`edent les 
propri\'et\'es suivantes:
{\small
$$\begin{array}{rclrcll}
 \overline{\overline{A}} &=& A& &&	& \mbox{double n\'egation} \\
\overline{0} &=& 1	& \overline{1}	&=& 0	& \mbox{constantes} \\
\overline{A+B} &=& \overline{A}.\overline{B}	&
\overline{A.B}	&=& \overline{A}+\overline{B}	& 
\mbox{dualit\'e} \\
(A+B)+C &=& A+(B+C)	&
(AB)C	&=& A(BC)	& \mbox{associativit\'e} \\
A+B &=&	B+A & AB	&=& BA	& \mbox{commutativit\'e} \\
(A+B)C &=& AC+BC &
AB+C	&=& (A+C)(B+C)	& \mbox{distributivit\'e} \\
A+A &=&	A & AA	&=& A	& \mbox{idempotence} \\
A+0 &=&	A & A.1 &=& A	& \mbox{\'el\'ements neutres} \\
A+1 &=&	1 & A.0 &=& 0	& \mbox{absorption} \\
A+\overline{A} &=&	1 & A.\overline{A} &=& 0	
& \mbox{compl\'ementaires} \\
\end{array} $$
}
\paragraph{Remarques}
\begin{itemize}
\item Chaque \'egalit\'e peut-\^etre v\'erif\'ee en construisant 
les tables de v\'erit\'e de ses parties gauche et droite.  
\item La colonne de droite contient des \'equations {\em duales} de celles
de la colonne de gauche, obtenues en intervertissant $+$ et $\cdot$, 0 et 1.
On peut passer d'une \'equation \`a l'\'equation duale en utilisant seulement
$\overline{\overline{A}}=A$, $\overline{0}=1$ et 
$\overline{A+B}= \overline{A}.\overline{B}$.
\item On voit
assez facilement
que ces \'equations permettent de d\'evelopper n'importe quel terme sous forme
d'une expression canonique. L'expression canonique d'une fonction \'etant 
unique (\`a la commutation de $+$ et $.$ pr\`es), ces \'equations 
suffisent donc 
pour montrer l'\'equivalence de deux expressions.
\item Les lois de dualit\'e sont appel\'ees aussi lois de 
De Morgan.\footnote{Les travaux d'Augustus de Morgan (1806-1871)
influenc\`erent fortement George Boole, et ses vives recommandations 
permirent \`a ce dernier,
bien qu'autodidacte, d'obtenir
la Chaire de Math\'ematiques du Queen's College de Cork.}
\end{itemize}

\subsubsection{Alg\`ebre de Boole}


On appelle {\em alg\`ebre de Boole} tout ensemble qui poss\`ede deux \'el\'ements
0 et 1 et des op\'erations $+,\cdot,\neg$ qui satisfont les \'equations ci-dessus.

\begin{exercice}{} Montrez qu'il n'y a pas d'alg\`ebre de Boole \`a 
3 \'el\'ements $\{0,1,2\}$. (Indication~: que vaut $\overline{2}$ ?).
\end{exercice}

On voit facilement que le produit $A \times B$ de deux alg\`ebres de
Boole est \'egalement une alg\`ebre de Boole. Les op\'erations sur le
produit sont d\'efinies par
$\overline(a,b)=(\overline{a},\overline{b})$, $(a,b) + (a',b') =
(a+a',b+b')$, $(a,b).(a',b') = (a.a',b.b')$.  Dans le sens inverse, un
th\'eor\^eme important et difficile (Stone) dit que toute alg\`ebre de
Boole se ram\`ene en fait \`a un produit d'alg\`ebres de Boole \`a 2
\'el\'ements\footnote{Ce produit peut \^etre infini, 
mais c'est une autre histoire}.


\section{Simplification des expressions}

A partir d'une expression  bool\'eenne on pourra facilement fabriquer un
circuit. Pour r\'ealiser une m\^eme fonction, on aura tout int\'er\^et \`a 
avoir des circuits qui utilisent le moins possible
de portes logiques, pour des raisons de simplicit\'e, de co\^ut, de taille du 
circuit et de consommation de courant.


On peut essayer de simplifier les circuits en  utilisant les \'equations vues
plus haut. Par exemple, pour la fonction $maj$~:
$$ \begin{array}{rcll}
	maj(a,b,c)&=& \overline{a}bc + a\overline{b}c + ab\overline{c} + abc \\
&=&  \overline{a}bc + a\overline{b}c + ab\overline{c} + abc + abc + abc &
\mbox{(idempotence)} \\
&=&  \overline{a}bc + abc + a\overline{b}c +abc + ab\overline{c} + abc &
\mbox{(commutativit\'e)} \\
&=&  (\overline{a}+a)bc  + a(\overline{b}+b)c + ab(\overline{c}+c) &
\mbox{(distributivit\'e)} \\
&=&  1.bc  + a.1.c + ab.1 & \mbox{(compl\'ementarit\'e)} \\
&=&  bc  + ac + ab & \mbox{(\'el\'ements neutres)} 
\end{array} $$

Mais la difficult\'e est alors de mener le calcul vers une expression
simple que l'on ne connait pas a priori.


\begin{probleme}{} Donnez une expression aussi simple que possible de chacune des
$2^{2^2}=16$ fonctions \`a deux variables.

D\'eterminez celles qui sont croissantes, c'est à dire telles que $x \leq x'$ et $y \leq y'$ entra\^ine $f(x,y) \leq f(x',y')$.  Pour chacune d'elles montrez qu'elle peut s'exprimer (sans utiliser de n\'egation) 
par une somme de produits de variables.

R\'eciproquement, montrez que toute fonction dont l'expression contient
des ET et des OU mais pas de NON est n\'ecessairement croissante. 

G\'en\'eralisez au cas des fonctions ayant un nombre quelconque de variables. 
\end{probleme}

\section{M\'ethode de Karnaugh}

La m\'ethode de Karnaugh est une m\'ethode visuelle pour 
trouver une expression simple d'une fonction bool\'eenne de
quelques variables (jusqu\`a  6).\footnote{Nous ne montrerons ici
que la méthode pour 3 et 4 variables, le passage à 5 ou 6 variables
nécessite une bonne vision dans l'espace}.

Elle repose sur une pr\'esentation particuli\`ere de la table de v\'erit\'e de
la fonction \'etudi\'ee.  Voici la disposition adopt\'ee pour les fonctions 
de 3 et 4 variables: 

$$ \begin{array}{|r|cc|}
\cline{1-3}
ab\ c&  0 & 1 \\
\cline{1-3}
00 & & \\
01 & & \\
11 & & \\
10 & & \\
\cline{1-3}
\end{array}
\hspace{.5in}
\begin{array}{|r|cccc|}
\cline{1-5}
ab\ cd&  00 & 01 & 11 & 10 \\
\cline{1-5}
00 & & &&\\
01 & & &&\\
11 & & &&\\
10 & & &&\\
\cline{1-5}
\end{array}
$$

On remarque que dans ces tables, deux cases adjacentes ne diffèrent
que le changement de valeur d'une seule variable. Comme chaque case
correspond \`a un mon\^ome complet, un groupement de 2 cases
adjacentes repr\'esentera un mon\^ome \`a $n-1$ variables.

\begin{exemple}{} Les deux cases du bas de la premi\`ere table correspondent 
respectivement \`a $a\overline{b}\overline{c}$ et $a\overline{b}c$, 
qui diff\`erent par la variable $c$. En les regroupant  on obtient
 $a\overline{b}\overline{c} + a\overline{b}c = a\overline{b}(\overline{c}+c)
= a\overline{b}$
\end{exemple}

Attention~: il faut \'egalement consid\'erer les bords oppos\'es comme \'etant adjacents,
par exemple les cases $\overline{a}\overline{b}\overline{c}$ 
(en haut \`a gauche) et $a\overline{b}\overline{c}$ (en bas \`a gauche) se
regroupent en $b\overline{c}$.

Les mon\^omes \`a $n-2$ variables sont visualis\'es sous forme de carr\'es,
ou de rectangles $1 \times 4$.  

\begin{exemple}{} On a $bd = (a + \overline{a})b(c +\overline{c})d$. En d\'eveloppant
cette expression on trouve une somme de 4 mon\^omes complets qui occupent le
milieu de la seconde table.  Les 4 coins de cette m\^eme table forment
(virtuellement) un carr\'e dont l'expression est $\overline{a}\overline{d}$.
La colonne de droite  correspond \`a $c\overline{d}$, etc.
\end{exemple}

De m\^eme les produits de $n-3$ variables occupent 8 cases dispos\'ees
en rectangles $ 2 \times 4$.

La m\'ethode de Karnaugh consiste \`a 
 \'ecrire la table de v\'erit\'e de la fonction dans la table, puis
proc\'eder \`a des regroupements que l'on entoure et dont on \'ecrit
l`expression au fur et \`a mesure.


\paragraph{Algorithme} L'algorithme est le suivant:
\begin{quote}
\begin{itemize}
\item  Si tous les 1s ont \'et\'e entour\'es~: arr\^eter.
\item trouver, visuellement, le plus gros regroupement possible
      contenant au moins un 1 non entour\'e.
\item l'entourer sur le tableau, et \'ecrire son expression
\item recommencer
\end{itemize}
\end{quote}

La figure \ref{boole1} illustre le d\'eroulement possible de la m\'ethode
sur l'exemple de la fonction majorit\'e.
Le r\'esultat obtenu est $BC+AB+AC$.

\dessin{boole1}{boole1.eps}{M\'ethode de Karnaugh~: exemple}



\begin{exercice}{Le vote} Une commission est compos\'e d'un  pr\'esident
P et trois membres A,B,C.  Cette commission doit se prononcer
sur un vote par oui ou non \`a la majorit\'e.  Personne ne peut
s'abstenir. En cas d'égalit\'e entre les oui et les non, la voix
du pr\'esident est pr\'epond\'erante.  Em utilisant la m\'ethode
de Karnaugh, donnez une expression simple de la fonction 
$ vote(A,B,C,P)$.
\end{exercice}

\begin{exercice}{D\'ecodage 7 segments hexadécimal} Un afficheur 7 segments est compos\'e
de 7 diodes électro-luminescentes not\'es a,b,\ldots g (voir figure
\ref{boole2}). Un nombre est fourni, cod\'e en binaire
sur 4 bits $x_3x_2x_1x_0$, qui devra \^etre affich\'e. Donnez l'expression
des 7 fonctions $a(x_3,x_2,x_1,x_0), b(x_3,x_2,x_1,x_0), \ldots$

\dessin{boole2}{boole2.eps}{Afficheur 7 segments, chiffres}
\end{exercice}

La m\'ethode de Karnaugh est également applicable dans le cas des
fonctions incompl\`etement sp\'ecifi\'ees, c`est-\`a-dire qui ont des
``cases vides'' dans lesquelles on peut mettre ce que l'on veut, parce
que cela correspond \`a des situations qui ne peuvent pas se produire.

La m\'ethode est alors la suivante: former les plus gros paquets possibles
contenant au moins un 1 non entour\'e et aucun 0.

 
\begin{exercice}{Le d\'e \'electronique} (fig. \ref{boole3}) poss\`ede trois entr\'ees, par lesquelles
arrivent des nombres de 1 \`a 6 cod\'es en binaire.  Donnez une fonction
pour chacune des lampes qui indiquera si elle doit \^etre allum\'ee ou 
\'eteinte.
\dessin{boole3}{boole3.eps}{Le d\'e \'electronique}
\end{exercice}

\begin{exercice}{Décodage 7 segments décimal} Reprendre 
l'exercice de l'afficheur 7 segments, en supposant que les valeurs
d'entrée vont toujours de 0000 à 1001.
\end{exercice}

\begin{exercice}{Multiples de 3}
Proposez un circuit qui prendra en entr\'ee un nombre $N$ entre 0 et
15 (cod\'e en binaire \'evidemment) et dont la sortie indiquera si $N$
est un multiple de $3$.
\end{exercice}