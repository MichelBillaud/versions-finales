\chapter{Les données en mémoire}


Un programme C contient généralement des variables.  Pendant
l'exécution, chaque variable est stockée en mémoire {quand elle n'est
  pas sont pas éliminée par le compilateur, ce qui arrive quand il
  détecte que la variable est inutile, ou suffisamment temporaire pour
  être rangée dans une registre du processeur}, dans un
\textbf{emplacement en mémoire} formé d'octets consécutifs.

\section{Taille des données, opérateur \texttt{sizeof()}}

Le nombre d'octets (la \textbf{taille}) dépend du type de la variable.
On la détermine en appliquant l'opérateur \texttt{sizeof()} à une variable
ou à un type.

\lstinputlisting[style=csource,
  float,
  caption={Affichage de tailles de quelques types},
  label={source:affichagetaille}
]{../Sources/tailles-types.c}

\lstinputlisting[style=run,
  float, caption={Exécution du programme du listing \ref{source:affichagetaille}},
  label={run:affichagetaille}
]{../Sources/tailles-types.run}

Le listing \ref{source:affichagetaille} montre un programme qui fait
afficher les tailles de quelques types. Le résultat (listing
\ref{run:affichagetaille}) dépend des choix d'implémentation du
compilateur que vous utilisez, sauf pour le type le type \texttt{char}
qui correspond \textbf{toujours} à un octet exactement.
\footnote{Il a été nommé \texttt{char} à une époque où le codage d'un
  caractère tenait toujours sur un octet (codages ANSI, EBCDIC,
  ...). Si c'était à refaire, ce type s'appellerait certainement
  \texttt{byte}.}

Par exemple, un \texttt{int} occupe \emph{en général} 4 octets sur une
machine 32 bits, et 8 octets sur une machine 64 bits.



\paragraph{Remarques :} \begin{enumerate}
\item \texttt{sizeof()} retourne un \texttt{size\_t}, type qui
  correspond à un entier non signé assez grand pour stocker une
  taille. L'implémentation de \texttt{size\_t} (\texttt{unsigned int},
  \texttt{unsigned long}, ...) est dépendante de l'architecture.
\item Portabilité : utilisez la spécification de format \texttt{\%zu}
  pour le type \texttt{size\_t}.
  
\end{enumerate}

\section{Structures en mémoire}

Rappel : une structure contient un ou plusieurs membres (champs)
qui peuvent être de types differents. Exemple de définition d'un type et d'une variable :

\begin{lstlisting}[style=cextract]
struct Employe {
  char nom[40];
  int  age;
};

struct Employe cuistot  = { "Maurice", 63 };
\end{lstlisting}

\paragraph{Exercice :}  \label{exo:taillestructure} . Écrivez un
programme montrant un exemple de structure dont la taille n'est pas 
\emph{égale} à la somme des tailles des champs.

\section{Adresse des données, opérateur ``\texttt{\&}''}

Le programme du Listing \ref{source:adressesvariables} fait afficher
les adresses de quelques variables pendant l'exécution, obtenues en
leur appliquant l'opérateur ``\texttt{\&}''
(\texttt{address-of})\footnote{Il s'agit ici des \emph{adresses
    virtuelles}, dans l'espace mémoire où le système a chargé le
  processus.} du langage C. Résultat sur Listing \ref{run:adressesvariables}

\lstinputlisting[style=csource,
  float,
  caption={Affichage des adresses de quelques variables},
  label={source:adressesvariables}
]{../Sources/adresses-variables.c}



\lstinputlisting[style=run,
  float, caption={Exécution du programme},
  label={run:adressesvariables}
]{../Sources/adresses-variables.run}



Les adresses sont des données typées : par exemple l'adresse d'une
variable de type \texttt{int} est de type \texttt{int*}.
L'affichage se fait avec la spécification ``\texttt{\%p}" (pour
\emph{pointer})
en les convertissant en ``adresses non-typées'' (\texttt{void *}).
  
\paragraph{Remarques :}
\begin{itemize}
  \item Les variables globales du programme et les variables locales
    de la fonction \texttt{main()} sont dans des ``segments'' dont les
    adresses diffèrent considérablement : le segment de données pour
    les variables globales, et le segment de pile pour les
    autres\footnote{Sur une machine qui supporte la notion de
      segmentation, évidemment. Ce n'est pas le cas des petits
      micro-contrôleurs dans le domaine de l'informatique embarquée}.
  \item Sur les systèmes d'exploitation modernes, les adresses
    virtuelles qui s'affichent changent à chaque
    exécution\footnote{C'est une mesure de sécurité pour éviter
      l'exploitation de ``débordements de tampon'' et autres erreurs
      de programmation. Lors du chargement d'un programme, le système d'exploitation choisit des adresses aléatoires pour placer les segments dans l'espace mémoire virtuel du processus.}.
\end{itemize}


\paragraph{Exercice :} reprenez l'exemple de structure (\ref{exo:taillestructure}) dont la taille est supérieure à la somme des tailles des champs, et définissez une variable de ce type. Faites afficher l'adresse et la taille de la structure et de chacun de ses champs. Conclusion ?


\section{Tableaux et adresses}

Le programme du listing \ref{source:adressestableaux} affiche
l'adresse d'un tableau et de ses éléments. Résultat sur Listing \ref{run:adressestableaux}.

\lstinputlisting[style=csource,
  float,
  caption={Tableaux et adresses},
  label={source:adressestableaux}  
]{../Sources/adresses-tableaux.c}

\lstinputlisting[style=run,
  float,
  caption={Exécution du programme},
  label={run:adressestableaux}
]{../Sources/adresses-tableaux.run}


\paragraph{Remarque :} en C, une variable de type tableau désigne en fait
l'\emph{adresse} de l'emplacement qui a été réservé en mémoire pour
placer les éléments. Il n'y a dons pas besoin de mettre un
``\texttt{\&}'' devant \texttt{tab} dans le \texttt{printf}.


On constate qu'à l'exécution, les
éléments se suivent en mémoire, et l'adresse du tableau
correspond à celle du premier élément.




