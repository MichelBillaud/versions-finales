
\chapter{Allocation dynamique}

L'\textbf{allocation dynamique de mémoire} est un ensemble de 
fonctionnalités mises à la disposition du programmeur d'application
par la bibliothèque standard C.

Elle lui permet de gérer de l'espace mémoire supplémentaire (en plus
de la pile d'exécution et du segment de données) pour y placer des
données, en spécifiant le nombre d'octets voulu. Elle permet aussi
de libérer un espace alloué dont on n'a plus besoin.

\paragraph{Attention :} l'usage de l'allocation dynamique impose un soin très attentif
au programmeur qui est guetté par deux dangers :


\begin{itemize}
\item \textbf{la fuite mémoire} si un programme alloue en boucle des
  zones mémoires, sans les libèrer quand il n'en n'a plus
  besoin. L'espace mémoire du programme s'agrandit indéfiniment, ce
  qui finit mal.
\item la \textbf{corruption des données} 
si un programme utilise par erreur une zone qui a été libérée.
\end{itemize}

C'est une difficulté typique de la programmation en C.

\section{Fonctions \texttt{malloc()} et \texttt{free()}}


Nous utilisons essentiellement deux fonctions, définies dans
\texttt{stdlib.h} : 

\begin{itemize}
\item \texttt{malloc()} pour obtenir de l'espace mémoire supplémentaire,
\item \texttt{free()} pour restituer (libérer) de l'espace obtenu par \texttt{malloc()},
\end{itemize}

et occasionnellement  \texttt{realloc()} qui agrandit ou rétrécit un espace qu'on
a obtenu, quitte à le déménager ailleurs.

Le listing \ref{prog:exempleallocation} montre l'utilisation d'un tableau
de structures alloué dynamiquement.

\lstinputlisting[style=csource,
  float,
  caption={Exemple : tableau de structures alloué dynamiquement},
  label={prog:exempleallocation}
]{../Sources/exemple-allocation.c}


Pour l'allocation par \texttt{malloc()}, on indique en paramètre la
taille (nombre d'octets) souhaitée. La fonction retourne l'adresse
(non typée) de la zone allouée, ou \texttt{NULL} en cas d'échec..

Pour libérer une zone, on fournit son adresse à la fonction \texttt{free}.

\paragraph{Exercice :} écrire les fonctions manquantes.


\section{Réallocation}

Si on veut ajouter un employé supplémentaire, il faut agrandir le tableau. pour cela on fait un appel à \texttt{realloc()} en indiquant

\begin{itemize}
\item l'adresse de la zone que l'on veut redimensionner (ici \texttt{tableau}),
\item la nouvelle taille
\end{itemize}
et \texttt{realloc} retournera l'adresse de la nouvelle zone :

\begin{lstlisting}[style=cextract]
nbElements += 1;
tableau = realloc(tableau,
                  nbElements * sizeof(struct Employe));
\end{lstlisting}

\paragraph{À savoir} : \begin{itemize}
  \item si le premier paramètre de \texttt{realloc} est \texttt{NULL},
    la fonction se comporte comme \texttt{malloc()},
  \item vous l'aviez deviné : en cas d'échec, \texttt{realloc()} retourne \texttt{NULL}.
\end{itemize}

\paragraph{Exercice :} écrivez un programme qui \begin{itemize}
\item part d'un tableau vide
\item fait une boucle, en demandant si on veut en ajouter d'autres
  \item les affiche tous à la fin.
\end{itemize}

