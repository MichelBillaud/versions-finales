\chapter{Ensemble de chaînes, hachage}


Dans cette partie, nous montrons comment représenter
\textbf{efficacement} un ensemble\footnote{fini, donné par extension}
de chaînes de caractères en utilisant une \textbf{fonction de
  hachage}.

\section{Opérations de base}
  
Les opérations de base sur cet ensemble :

\begin{itemize}
\item l'initialiser,
\item y ajouter un élément (si il n'y est pas déjà),
\item savoir combien il y a d'éléments dans l'ensemble,
\item tester si un élément est présent,
\item enlever un élément,
\item libérer les ressources utilisées.
\end{itemize}


Sur le listing \ref{prog:utilisationensembles} figure un exemple
d'utilisation où l'on ajoute une suite de mots (éventuellement
en plusieurs exemplaires) et on fait afficher la taille (qui doit être 10).

Le programme affiche également (listing
\ref{run:utilisationensembles})  le contenu interne de l'ensemble
de chaînes, ce qui nous facilitera les explications.
 
\lstinputlisting[style=csource,
  caption={Utilisation d'un ensemble de chaînes},
  label={prog:utilisationensembles}
]{../Sources/utilisation-ensemble-chaines.c}

\lstinputlisting[style=run,
  caption={Exécution},
  label={run:utilisationensembles}
]{../Sources/utilisation-ensemble-chaines.run}

Nous ne présenterons que quelques opérations, les autres sont laissées en
exercice.


\section{Idée générale}


\subsection{Répartition en alvéoles}

\begin{itemize}
  \item les chaînes de caractères qui
    font partie de l'ensemble sont réparties dans des ``alvéoles''.
  \item les alvéoles forment un tableau, ce qui permet un
    accès rapide par indice.
  \item  le numéro de l'alvéole dans laquelle se trouve (ou
      devrait se trouver) une chaîne de caractères est calculé à
      partir du contenu de cette chaîne, par ce qu'on appelle une
      \textbf{fonction de hachage}.
    \item plus précisement, le numéro d'alvéole s'obtient comme reste (opération modulo)
      de la division de la valeur du hachage par le nombre d'alvéoles.
\end{itemize}

\paragraph{Intérêt :}
La répartition en alvéoles permet de diviser le nombre de comparaisons
nécessaires pour tester la présence d'une chaîne : on ne regarde que
celles présentes dans son alvéole.

\paragraph{Dans l'idéal,} la fonction de hachage serait parfaite, et conduirait à une alvéole où ne se trouve qu'une chaîne.

\paragraph{En pratique,} il va y avoir quand même plusieurs chaînes dans
certaines alvéoles. On va donc

\begin{itemize}
\item prévoir qu'une alvéole contient une \texttt{liste} de chaînes,
\item avoir un grand nombre d'alvéoles de
  façon à avoir statistiquement peu de chaînes par alvéole.
\end{itemize}

\subsection{Agrandissement par doublement}

La stratégie choisie est de doubler
le nombre d'alvéoles quand le nombre de chaînes
présentes dans l'ensemble atteint certain seuil (3/4 du nombre
d'alvéoles).  Les chaines sont alors redistribuées entre les
alvéoles.

Le respect de ce seuil
garantit qu'il a au maximum $0.75\%$ chaines par alvéole. Il y aura
donc peu d'alvéoles avec plus d'une chaine.


Comme pour les tableaux extensibles, la stratégie de doublement fait
qu'en moyenne chaque alvéole est copiée au plus une fois.

\subsection{Doublement et redistribution}

Le doublement a une autre propriété intéressante.
Quand on redistribue
les chaines d'une alvéole,
\begin{itemize}
\item soit elles restent dans la même alvéole,
\item soit elles vont dans une alvéole ``jumelle'' qui vient d'être ajoutée.
\end{itemize}

\paragraph{Exemple :} pour la chaîne \texttt{"dix"}, la fonction de hachage
vaut 30805.
\begin{itemize}
\item Si il y a 4 alvéoles, elle se trouve dans l'alvéole
  $30805 \ \%\  4 = 1$.
  \item En passant à 8 alvéoles, elle va dans la nouvelle
    alvéole $30805 \ \%\  8 = 5 = 4 + 1$.
    \item En passant à 16, elle reste en
      $30805 \ \%\  16 = 5$.
      \item En passant à 32, elle va
en $30805 \ \%\  32 = 21 =
16 + 5$.
\item etc.
  
\end{itemize}
Ceci nous autorise à redistribuer les chaines en traitant les
anciennes alvéoles une par une : on est sûr de ne pas avoir à déplacer
chaque chaine plus d'une fois.


\section{Détails d'implémentation}

\subsection{Sémantique de valeur pour les chaines}

Lorsqu'on appelle la fonction qui sert à ajouter une chaine, ce qu'on
veut c'est ajouter le contenu de la chaine.  Pour cela on ne peut pas
se contenter de stocker l'adresse de la chaine reçue, il faut en faire
une copie.

Ci-dessous une erreur classique de programmation en C : si on fait
ensuite afficher le tableau, on s'aperçoit qu'il ne contient pas ce
qu'on pense y avoir mis :

\begin{lstlisting}[style=cextract]
  char *joueurs[10];
  char nom[100];
  for (int i=0; i < 10; i++) {
    printf("donnez un nom :");
    scanf("%s", nom);
    joueurs[i] = nom;       // une erreur de débutant
}
\end{lstlisting}

puisqu'on a stocké 10 fois l'adresse de la même variable locale \texttt{nom}...

Lors de l'ajout d'une chaine, on stocke donc en réalité \emph{une
  copie} obtenue par \texttt{strdup()}. Cette copie est une ressource
appartenant à l'ensemble, et sera libérée quand
\begin{itemize}
\item on retire une chaine de l'ensemble
\item on libère l'ensemble
\end{itemize}

\subsection{Structure des alvéoles}

Les alvéoles, qui en principe ne contiendront que peu d'éléments
(moins de $0.75$ en moyenne), sont représentées ici par des listes
chainées non ordonnées.


\section{Choix de la fonction de hachage}

Une fonction de hachage retourne un nombre non signé, parce qu'elle
sert à calculer un indice (entier positif ou nul) comme reste d'une
division (opérateur modulo).

Ce modulo (par une puissance de 2) fait qu'on utilise comme indice
les bits de poids faible de la valeur retournée. Il est important que
ces bits soient, autant que possible, dépendants de tous les caractères de la chaine.

Un contre-exemple pour illustrer cette notion. Si le hachage était calculé
ainsi

\begin{lstlisting}[style=cextract]
unsigned int hash = 0;
    for (const char *c = chaine; *c != '\0'; c++) {
        hash = 16 * hash + *c;               // mauvais
    }
\end{lstlisting}

à cause du décalage produit par la multiplication par 16, les 4 bits
de droite ne dépendraient que du dernier caractère de la chaine; les 8
bits de droite des deux derniers, etc. Les chaines \texttt{"sept"} et
\texttt{"huit"} se retrouveraient toujours dans la même alvéole, pour
les ensembles qui ont moins de 256 alvéoles. Idem pour \texttt{"six"}
et \texttt{"dix"}. Et à partir d'une certaine taille, les premiers
octets de la chaine seront sans influence sur le résultat (ils seront
perdus dans le débordement).

La multiplication par 17 (16 + 1) garantit que chaque octet de la
chaine a une influence sur les bits de poids faible du résultat de la
fonction de hachage.

\section{Code source}

\subsection{Entête}

\lstinputlisting[style=csource,
  caption={Ensemble de chaines : entêtes},
  label={prog:entetesensemble}
]{../Sources/ens_chaines.h}


\subsection{Code}

\lstinputlisting[style=csource,basicstyle=\footnotesize,
  caption={Ensemble de chaines : implémentation},
  label={prog:codeensemble}
]{../Sources/ens_chaines.c}
