\chapter{Pointeurs}

On appelle \textbf{pointeur} une donnée qui contient l'adresse d'une
donnée en mémoire.\footnote{Ce terme est aussi (hélas) souvent employé
  par extension pour désigner les adresses elles-mêmes. Nous
  essaierons d'éviter ce regrettable manque de rigueur, source de
  confusions, qui permettrait d'écrire qu'un pointeur (au sens de
  variable) \emph{contient} un pointeur (au sens d'adresse)...}.


On emploie les pointeurs pour diverses raisons, en particulier :

\begin{itemize}
\item le passage de paramètres,
\item le parcours de tableaux,
 \item la manipulation des données allouées dynamiquement.
\end{itemize}


\section{Déclaration des pointeurs}

Pour déclarer un pointeur destiné à contenir des adresses d'objets de type T,
on précède son nom par une étoile. Exemples :

\begin{lstlisting}[style=cextract]
int *pi;               // pointeur sur un int
struct Personne *pp;   // pointeur sur struct Personne
\end{lstlisting}

Pour déclarer un tableau de pointeurs, le nom du tableau est précédé par une
étoile

\begin{lstlisting}[style=cextract]
char *noms[10];  // tableau de 10 pointeurs de caractères
\end{lstlisting}

La règle générale est qu'en C, la déclaration d'une variable ressemble
à son usage (voir l'indirection ci-dessous).

\section{Pointeurs non-typé, conversions}

Les pointeurs non-typés sont déclarés avec \texttt{void *}.

\begin{lstlisting}[style=cextract]
int entier = 123;
void *adresse = & entier;        // pointeur non typé
printf("valeur=%d, adresse=%p\n",
       entier, adresse);
\end{lstlisting}

\paragraph{Remarque :} l'expression "\texttt{\& entier}'' de la
seconde ligne est de type \texttt{int *}, mais il y a une
\textbf{conversion
implicite} entre les adresses typées et non-typées.


\section{Indirection, opérateur ``\texttt{*}''}

L'opérateur ``\texttt{*}'' fournit un  accès à la donnée dont l'adresse est
contenue dans un pointeur \textbf{typé}. Exemple :

\begin{lstlisting}[style=cextract]
int nombre = 12;
int *p = & nombre;            // pointeur typé (entiers)

*p = 33;                      // modif. à travers p
printf("= %d\n", *p);         // accès indirect
\end{lstlisting}

\paragraph{Terminologie :} on dit que 
\begin{itemize}
\item  la variable \texttt{nombre} \textbf{est pointée par}
  \texttt{p}.
\item on fait une \textbf{indirection} pour, à partir d'un pointeur,
  accéder à la donnée qu'il pointe.
\item on \textbf{déréference} le pointeur.
\end{itemize}

\paragraph{Remarque :} dans l'exemple d'un tableau de pointeurs de caractères vu plus haut,
\begin{itemize}
\item \texttt{noms[2]} est le troisième\footnote{le premier a l'indice 0...} pointeur ;
  \item \texttt{*noms[2]} est le \texttt{char} désigné par ce pointeur.
\end{itemize}

Et puisque \texttt{*noms[i]} est un \texttt{char}, dans la logique de C, il n'est pas anormal que la déclaration d'un tableau de pointeurs

\begin{lstlisting}[style=cextract]
  char   *noms[10];
\end{lstlisting}

ressemble fortement à l'usage qu'on a des éléments.

\section{Le pointeur \texttt{NULL}}

La constante \texttt{NULL} est une valeur conventionnelle (de type
\texttt{void*} que l'on affecte à un pointeur pour indiquer qu'il
\textbf{ne contient pas}, à un moment donné, l'adresse d'un objet en
mémoire. Le pointeur ne pointe sur rien.\footnote{Ne pas confondre avec
  un pointeur non-initialisé, qui contient une valeur aléatoire}


Quand un pointeur contient \texttt{NULL},
tenter de le déréférencer est un \textbf{comportement indéfini},
qui provoque généralement un arrêt brutal de l'exécution :

\begin{lstlisting}[style=cextract]
  int *p = NULL;
  *p = 12;            // crash
\end{lstlisting}


\section{Pointeurs de structures, notation ``\texttt{->}''}

Selon les règles de priorités d'opérateurs de C,
``\texttt{*a.b}'' se lit  ``\texttt{*(a.b)}''.

La notation ``pointeur->champ'' facilite la désignation
d'un champ d'une structure dont on a l'adresse dans un
pointeur. Exemple :

\begin{lstlisting}[style=cextract]
struct Point {
  float x, y;
};
...
struct Point *p;  // p pointeur de point

p->x = 0.0;          // au lieu de  (*p).x = 0.0
p->y = 0.0;
\end{lstlisting}

\section{Passage de paramètres}

Le langage C ne connaissant que le passage de paramètres \textbf{par valeur},
on utilise des pointeurs pour simuler le ``passage de référence''
dans deux situations :

\begin{enumerate}
  \item l'action que l'on veut coder modifie un objet
    qu'on lui indique,
\item les objets que l'on souhaite transmettre sont assez gros, et
  pour des raisons de performance, on veut éviter la copie inhérente à
  un passage par valeur.
\end{enumerate}

\subsection{Pointeur pour le passage par référence}

Exemple du listing \ref{prog:passagereference} : une action consistant à
échanger les nombres contenus dans deux variables. On la traduit par
une fonction à qui on passe les adresses des variables à modifier.

\begin{lstlisting}[style=cextract,
caption={Émulation d'un passage par référence},
label={prog:passagereference}]
void echanger(int *pa, int *pb) {
  int tmp = *pa;
  *pa = *pb;
  *pb = tmp;
}

// usage
int a = 34, b = 23;
echanger( & a, & b);
\end{lstlisting}

\subsection{Pointeur pour éviter de copier}

Exemple (listing \ref{prog:passagestructure}: affichage d'une structure. 

\begin{lstlisting}[style=csource,
  caption={Passage d'un structure volumineuse},
  label={prog:passagestructure}
]
  struct Personne {
    char nom[100];
    char prenom[100];
    ...
  };
  
  void afficher_personne(const struct Personne *p) {
    printf("nom = %s\n'', p->nom);
    ...
  }
\end{lstlisting}

Le mot-clé \texttt{const} annonce nos intentions.
La déclaration de paramètre se lit de droite à gauche : \texttt{p}
est un pointeur vers une structure \texttt{Personne} qu'on ne modifie pas.

\section{Parcours de tableau, arithmétique des pointeurs}

Une chaine de caractères est un tableau d'octets terminé par un
caractère nul.

\begin{lstlisting}[style=cextract]
  char test[] = "abc";  // tableau de _4_ octets
\end{lstlisting}

Pour parcourir une chaine, on peut\footnote{à la place d'un indice}
utiliser un pointeur qui va désigner tour à tour chaque octet :

\begin{lstlisting}[style=cextract]
  void affiche_codes(const char chaine[]) {
    char *p = chaine;
    while (*p != '\0') {
      printf("-> %d\n"; *p);
      p++;
    }
  }
\end{lstlisting}
\paragraph{Remarques} \begin{enumerate}
  \item Un tableau déclaré en paramètre est en réalité un pointeur.
  \item l'incrémentation d'un pointeur (\texttt{p++]}) modifie ce
    pointeur pour qu'il désigne l'élément suivant\footnote{la valeur numérique du pointeur - celle qu'on voit avec \texttt{printf} - est augmentée
      de la taille du type pointé (ici 1, parce que c'est un \texttt{char}.
    }
\end{enumerate}

      

