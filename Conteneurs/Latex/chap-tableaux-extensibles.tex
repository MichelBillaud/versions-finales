\chapter{Tableaux extensibles}

Nous appellons \emph{tableau extensible} une structure de
données qui sert à stocker des élements, et

\begin{itemize}
\item comme un tableau ordinaire, permet de désigner un
  élément par sa position (première =  0, seconde = 1, etc.),
  pour le modifier ou le consulter.
\item à la différence des tableaux, permet d'ajouter des éléments
  à la fin sans limite de taille\footnote{autre que les limitations
    de l'allocation dynamique}.
\end{itemize}

Ici nous allons prendre l'exemple des tableaux extensibles d'entiers.

\section{Choix d'une API, exemple d'utilisation}

\begin{itemize}
 \item
Le type ``tableau extensible d'entiers'' se matérialise par une
structure appelée \texttt{tab\_int}.
\item
Une famille de fonctions, dont le nom est préfixé par ``\texttt{ti}''
représentera les actions qui agissent dessus.
\item
  Le premier paramètre de ces fonctions sera toujours l'adresse du
  tableau concerné.\footnote{pour les deux raisons évoquées plus haut
    :
    \begin{itemize}
    \item c'est obligatoire pour les  fonctions qui  modifient le tableau
    \item c'est souhaitable pour les autres, pour éviter de faire des copies. Dans ce cas on mettra un \texttt{const}.
    \end{itemize}
    }
\end{itemize}

Le programme du listing \ref{prog:utilisationtableau}
montre l'emploi d'un tel tableau, et on voit
fig. \ref{run:utilisationtableau} le résultat de l'exécution.
 
\lstinputlisting[style=csource,
  %  float,
  label=prog:utilisationtableau,
  caption={Utilisation d'un tableau extensible d'entiers},
]{../Sources/utilisation-tableau-extensible.c}

\lstinputlisting[style=run,
%  float,
  caption={Exécution},
  label={run:utilisationtableau}
]{../Sources/utilisation-tableau-extensible.run}



\clearpage
\section{L'implémentation}

\subsection{Données}

Un tableau extensible est représenté par

\begin{itemize}
\item un \textbf{tableau} alloué dynamiquement, pouvant accueillir
  un certain nombre d'éléments (sa \textbf{capacité}),
\item un entier indiquant le nombre d'éléments utilisés, au début
  du tableau (sa \textbf{taille})
\end{itemize}


\lstinputlisting[style=csource,
  caption={Fichier \texttt{tab\_int.h}}
]{../Sources/tab_int.h}


\subsection{Code}

Le code comporte quelques choix d'implémentation

\begin{itemize}
\item la capacité initiale, lorsqu'on initialise un tableau extensible (ici, 4 éléments)
  \item la stratégie d'agrandissement en cas de débordement. Ici on
    double : l'ajout du 5ieme élément réalloue le tableau avec une
    capacité de 8, et l'ajout du 8ieme passe la capacité à 16.  Cette
    stratégie est justifiée section \ref{strategiedoublement}.
\end{itemize}

\lstinputlisting[style=csource,
  caption={Code du module \texttt{tab\_int.h}}
]{../Sources/tab_int.c}

\subsection{Stratégie de doublement de la capacité}

\label{strategiedoublement}

Lorsque le tableau est plein, on le réalloue
avec une capacité supérieure.

La stratégie de doublement de cette capacité est,
contrairement à ce que suggère l'intuition, très efficace
en terme de nombre de copies : au cours du remplissage,
chaque élément a été copié \textbf{au plus une fois}
en moyenne.

Imaginons qu'à un moment le vector ait grandi jusqu'à 500
éléments. Comme le tableau grandit en doublant de taille, sa capacité
est la première puissance de 2 supérieure à 500, soit 512.

Le tableau sera agrandi (et réalloué) en ajoutant le 513ieme, sa capacité
passera à 1024 élements, et pour cela il faudra réallouer ce qui provoquera
la copie des 
512 éléments existants. Cout : 512, si on prend comme unité la copie d'un élément.

Mais pour arriver à 513, il avait fallu copier 256 éléments. Et pour
arriver à 257, en copier 128.

Si on fait le total, si on en est au 513-ième élément ajouté (et jusqu'au
1024-ième) on a fait en tout $256 + 128 + 64 + \ldots$ copies
d'éléments, ce qui est plus petit que 512.

Dans le pire des cas (ajout du 513 ième), le coût moyen d'ajout d'un
élément est inférieur à $512/513$ : il y a donc eu \textbf{moins d'une
copie par élément}.

\paragraph{Exercice :} évaluez la stratégie consistant à augmenter d'un la capacité à chaque ajout.

