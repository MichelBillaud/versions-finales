

\subsection*{Objectifs}

Ce document présente quelques appels système utiles à la réalisation
d'application communicantes sous UNIX.

Pour écrire de telles applications il faut savoir faire communiquer de
processus entre eux, que ce soit sur la même machine ou sur des
machines reliées en réseau (Internet par exemple).


Pour cela, on passe par des \emph{appels systèmes} pour demander au
système d'exploitation d'effecter des actions : ouvrir des voies de
communication, expédier des données, créer des processus etc.  On
parle de \emph{programmation système} lorsqu'on utilise explicitement
ces appels sans passer par des bibliothèques ou des modules de haut
niveau qui les encapsulent pour en cacher la complexité (supposée).


Les principaux appels systèmes sont présentés ici, avec des exemples
d'utilisation.\footnote{Attention, ce sont des illustrations des
appels, pas des recommandations sur la bonne manière de les
employer. En particulier, les contrôles de sécurité sur les données
sont très sommaires.}



\subsection*{Copyright, versions}

(c) 1998-2021 Michel Billaud

Ce document peut être reproduit en totalité ou en partie, sans frais,
sous réserve des restrictions suivantes :
\begin{itemize}
\item  cette note de copyright et de permission doit être préservée
dans toutes copies partielles ou totales 
\item  toutes traductions ou
travaux dérivés doivent être approuvés par l'auteur en le prévenant
avant leur distribution 
\item  si vous distribuez une partie de ce
travail, des instructions pour obtenir la version complète doivent
également être fournies
\item de courts extraits peuvent être reproduits sans ces notes de
  permissions.
\end{itemize}

L'auteur décline toute responsabilité vis-à-vis des dommages résultant
de l'utilisation des informations et des programmes qui figurent dans
ce document.


\begin{itemize}
\item Version initiale 1998
\item Révision 2002
\item Révision 2014
\item Révision 2016 (Conformité 11)
\item Révision 2018 (POSIX 2017)
\item Révision 2021 (conformité C17)
\end{itemize}

La dernière version de ce document peut être obtenue depuis la page Web
\url{http://www.mbillaud.fr/}
