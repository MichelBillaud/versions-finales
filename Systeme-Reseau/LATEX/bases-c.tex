

\section{Exemple, compilation}

Un exemple classique de programme écrit en C,
à taper dans un fichier \texttt{hello.c}

\extrait
\lstinputlisting{../PROGS/Divers/hello.c}

On peut le compiler par la commande

\extraitbash
\begin{lstlisting}
$ gcc -std=c18 -Wall -Wextra -pedantic -D_XOPEN_SOURCE=700 hello.c
\end{lstlisting}

\index{Options de compilation!POSIX.1-2017}
\index{Option de compilation!C19}

Avec ces options

\begin{itemize}
  \item le compilateur vérifie la conformité  au dernier standard
    du langage C (C18)
  \item on  bénéficie de la bibliothèque POSIX.1-2017,
\footnote{
  identique à IEEE Standard 1003.1-2017 et
  The Open Group Technical Standard Base Specifications, Issue 7.
  voir \url{http://pubs.opengroup.org/onlinepubs/9699919799}
}
\item un maximum d'avertissements sont affichés.
\end{itemize}






\section{Lecture et affichage}

\subsection{Lecture et écriture standards : \texttt{printf()} et \texttt{scanf()}}

\index{printf(format, valeurs...)}  \index{scanf(format, adresses...)}
\index{Entrees-sorties formattees@Entrées-sorties formattées} \extrait
\begin{lstlisting}
#include <stdio.h>
  
int printf (const char *format, ...);     
int scanf  (const char *format, ...); 
\end{lstlisting}



Ces instructions font des écritures et des lectures \emph{formattées}
sur les flots de sortie et d'entrée standard.  Les spécifications de
format sont décrites dans la page de manuel \texttt{printf(3)}.



\subsection{Lecture et écriture dans une chaîne :  \texttt{sprintf()} et \texttt{sscanf()}}

\index{sprintf(tampon, format, valeurs...)}  \index{sscanf(tampon,
  format, adresses...)}  \extrait
\begin{lstlisting}
#include <stdio.h>
  
int sprintf (      char *str, const char *format, ...);     
int sscanf  (const char *str, const char *format, ...); 
\end{lstlisting}


Similaires aux précédentes, mais les opérations lisent ou écrivent
dans le tampon \texttt{str}. 

\paragraph*{Remarque : } la fonction \texttt{sprintf()} ne connaît pas la
taille du tampon \texttt{str} ; il y a donc un risque de
débordement. Il faut prévoir des tampons assez larges, ou (mieux)
utiliser la fonction \texttt{snprintf()} :

\index{snprintf(tampon, taille, format, valeurs...}
\extrait
\begin{lstlisting}
#include <stdio.h>
int snprintf (char *str, size_t size, const char *format, ...);     
\end{lstlisting}

qui permet d'indiquer un nombre d'octets à ne pas dépasser.


\subsection{Lancement d'une commande : \texttt{system()}}

\index{system(chaine)}
\extrait
\begin{lstlisting}
#include <stdlib.h>

int system (const char * string);
\end{lstlisting}
permet de lancer une ligne de commande (\emph{shell}) depuis un
programme.  L'entier retourné par la fonction \texttt{system()} est le
\emph{code de retour} fourni en paramètre à \texttt{exit()} par la
commande.


\paragraph*{Exemple} :


\source
%\lstinputlisting{../PROGS/Divers/envoifichier.c}
\lstinputlisting{../PROGS/Divers/imprimer.c}


\section{Communication avec l'environnement}

\subsection{Paramètres de \texttt{main()}}

\index{argc}
\index{argv}
Le lancement d'un programme C provoque
l'appel de sa fonction principale \texttt{main()}.
Le standard C autorise deux formes pour la déclaration
de \texttt{main()} :

 
\index{main!main(void)}
\index{main!main(argc,argv)}
\extrait
\begin{lstlisting}
int main(void);
int main(int argc, char *argv[]);   
\end{lstlisting}

\begin{itemize}
\item \texttt{argc}  est le nombre de paramètres sur la ligne de commande
(y compris le nom de l'exécutable lui-même) ;
\item \texttt{argv} est une tableau de chaînes contenant les
  paramètres de la ligne de commande.
  \item les noms \texttt{argc, argv} sont purement conventionnels.
  \item déclaration équivalente pour \texttt{argv}  : \verb+char **argv+
\end{itemize}

\paragraph*{Exemple :} programme qui affiche le tableau \texttt{argv} :


\source
\lstinputlisting{../PROGS/Divers/env.c}



\subsection{\texttt{getopt()} : analyse des paramètres de la ligne de commande}

La fonction \texttt{getopt} facilite l'analyse des options d'une ligne de commande. On lui fournit :

\begin{itemize}
\item   le tableau des  paramètres \texttt{argv} et sa taille
\texttt{argc}
\item une
\emph{chaîne de spécification d'options}.  Par exemple la chaîne
\texttt{"hxa:"} déclare 3 options possibles, la dernière (a) devra être
suivies d'un paramètre.
\end{itemize}

\index{getopt}

\extrait
\begin{lstlisting}
#include <unistd.h>

int getopt(int argc, char *const argv[], const char *optstring);

extern char *optarg;
extern int  optind, opterr, optopt;
\end{lstlisting}


À chaque étape, \texttt{getopt()} retourne le nom d'une l'option (ou un
point d'interrogation pour une option non reconnue), et fournit
éventuellement dans \texttt{optarg} la valeur du paramètre associé.

À la fin de l'analyse, \texttt{getopt()} retourne \texttt{-1}, et le
tableau \texttt{argv} a été réarrangé pour que les paramètres
supplémentaires (non liés aux options) soient stockés à
partir de l'indice \texttt{optind}.


\source
\lstinputlisting{../PROGS/Divers/essai-getopt.c}


\paragraph*{Exemple.}

\extraitbash
\begin{lstlisting}
$ essai-getopt -a un deux trois -x quatre
= option `-x' activée
= paramètre `-a' présent = un
3 paramètres supplémentaires
    ->  deux
    ->  trois
    ->  quatre
\end{lstlisting}



\subsection{Variables d'environnement}

\index{getenv(), variables d'environnement} \index{variables
  d'environnement!getenv()}
La fonction \texttt{getenv()} permet de
consulter les variables d'environnement : \extrait
\begin{lstlisting}
#include <stdlib.h>

char *getenv(const char *name);
\end{lstlisting}


\paragraph*{Exemple} :

\source
\lstinputlisting{../PROGS/Divers/getlang.c}



\paragraph*{Exercice : } Ecrire un programme \texttt{exoenv.c} qui affiche les 
valeurs des variables d'environnement indiquées.
Exemple d'exécution:

\extrait
\begin{lstlisting}
$ exoenv TERM LOGNAME PWD
TERM=xterm
LOGNAME=billaud
PWD=/net/profs/billaud/essais
$
\end{lstlisting}

\paragraph*{Voir aussi} les fonctions 


\index{setenv(), variables d'environnement}
\index{variables d'environnement!setenv()}

\index{putenv(), variables d'environnement}
\index{variables d'environnement!putenv()}


\index{unsetenv(), variables d'environnement}
\index{variables d'environnement!unsetenv()}
\extrait
\begin{lstlisting}
#include <stdlib.h>
  
int  putenv  (const char *string);
int  setenv  (const char *name, const char *value, int overwrite);
void unsetenv(const char *name);
\end{lstlisting}

qui permettent de modifier les variables d'environnement
du processus courant et de ses fils.

\paragraph*{Exercice} vérifiez que ça ne modifie pas l'environnement
du processus père.

\subsection{\texttt{exit()} : Fin de programme}

Pour arrêter un programme, deux solutions simples :
\begin{itemize}
\item soit par un \texttt{return} dans la fonction \texttt{main()}
\item soit par un appel à la fonction \texttt{exit()}
\end{itemize}


\index{exit(status)}
\extrait
\begin{lstlisting}
#include <stdlib.h>
  
void exit(int status);
\end{lstlisting}


Le paramètre \texttt{status} est le \emph{code de retour} du processus.
On utilisera de préférence les deux constantes 
\texttt{EXIT\_SUCCESS} et \texttt{EXIT\_FAILURE} qui sont définies dans
\texttt{stdlib.h}.

\section{Erreurs}


\subsection{Variable \texttt{errno}, fonction \texttt{perror()} }

\index{errno}
\index{perror(message)}
\index{erreurs!errno}
\index{erreurs!perror(message)}

La plupart des fonctions du système peuvent échouer pour diverses
raisons. Habituellement, elles le signalent en retournant une valeur
spéciale.  On peut alors examiner la variable globale \texttt{errno}
pour déterminer plus précisément la cause de l'échec, et agir en
conséquence.

\extrait
\begin{lstlisting}
#include <stdio.h>

void perror(const char *s);

#include <errno.h>

extern int errno;
\end{lstlisting}


La fonction \texttt{perror()} imprime sur la sortie
d'erreur standard un message qui décrit la dernière erreur qui s'est
produite, précédé par la chaîne \texttt{s}.


\extrait
\begin{lstlisting}
#include <stdio.h>
void perror(const char *s);
\end{lstlisting}

\index{strerror(errnum)}
\index{erreurs!strerror(errnum)}

Enfin, la fonction \texttt{strerror()} retourne le texte (en anglais)
du message d'erreur correspondant à un numéro.


\extrait
\begin{lstlisting}
#include <string.h>

char *strerror(int errnum);
\end{lstlisting}


\index{chmod()!exemple}

\paragraph*{Exemple} : programme qui change les droits d'accès à des fichiers
grâce à l'appel système \emph{chmod(2)}.


\source
\lstinputlisting{../PROGS/Divers/droits.c}


\subsection{Traitement des erreurs, branchements non locaux }

\index{erreurs!setjmp()}
\index{erreurs!longjmp()}
\index{setjmp()}
\index{longjmp}

\extrait
\begin{lstlisting}
#include <setjmp.h>

int   setjmp(jmp_buf env);
void longjmp(jmp_buf env, int val);
\end{lstlisting}


Ces deux fonctions permettent de réaliser un \emph{branchement} d'une
fonction à une autre (la première doit avoir été appelée, au moins
indirectement, par la seconde). C'est un moyen primitif de réaliser un
semblant de traitement d'erreurs par \emph{exceptions}. À employer
avec précaution.

La fonction \texttt{setjmp()} sauve l'environnement (contexte d'exécution)
dans la variable tampon \texttt{env}, et retourne 0 si elle a été appelée
directement. 

La fonction \texttt{longjmp()} rétablit le dernier environnement qui a
été sauvé dans \texttt{env}. Le programme continue à l'endroit du
\texttt{setjmp()} comme si celui-ci avait retourné la valeur
\texttt{val}.  (Si le paramètre \texttt{val} à 0, la valeur retournée
est 1).

\paragraph*{Exemple} :


\source
\lstinputlisting{../PROGS/Divers/jump.c}



\section{Allocation dynamique}

\index{allocation dynamique!malloc(taille)}
\index{allocation dynamique!free(pointeur)}
\index{allocation dynamique!realloc(pointeur,taille}
\index{malloc(taille)}
\index{free(pointeur}
\index{realloc(pointeur,taille)}

\extrait
\begin{lstlisting}
#include <stdlib.h>

void *malloc(size_t size);
void  free(void *ptr);
void *realloc(void *ptr, size_t size);
\end{lstlisting}


\begin{itemize}
\item \texttt{malloc()} - memory allocation -  demande au système d'exploitation 
l'attribution d'un espace mémoire
de taille supérieure ou égale à \texttt{size} octets.
La valeur retournée est un pointeur sur cet espace
(\texttt{NULL} en cas d'échec). 
\item \texttt{free()} restitue cet espace au système. 
\item \texttt{realloc()} permet de redimensionner la
zone allouée en conservant son contenu.
\end{itemize}


\paragraph*{Exemple} : la fonction \texttt{lire\_nouvelle\_ligne()}
ci-dessous lit une ligne de l'entrée standard et retourne cette ligne
dans un tampon d'une taille suffisante. Elle renvoie le pointeur
\texttt{NULL} si il n'y a plus de place en mémoire.


\source
\lstinputlisting{../PROGS/Divers/lireligne.c}


Attention : dans l'exemple ci-dessus, la fonction
\texttt{lire\_nouvelle\_ligne()} alloue un nouveau tampon à chaque
invocation. Il est donc de la responsabilité du programmeur de libérer
ce tampon après usage pour éviter les \emph{fuites mémoire}. C'est ce
que fait l'appel de \texttt{free()} dans la fonction \texttt{main()}.

\paragraph{Copie de chaîne :}


\index{strdup(chaine)}
\index{chaine@chaîne!duplication, strdup(chaine)}

Il est très fréquent de devoir allouer une zone mémoire pour y loger
une copie d'une chaîne de caractères. On utilise pour cela la fonction
\texttt{strdup()}, qui ne fait pas pour l'instant partie des
bibliothèques standards du langage C lui-même, mais (ouf !) de la
bibliothèque POSIX. Elle apparaîtra dans C23.

\extrait
\begin{lstlisting}
#include <string.h>
  
char *strdup (const char *s);
\end{lstlisting}


\index{chaine@chaîne!copie, strcpy(dest, src)}

Dans un environnement non-POSIX, la fonction peut être redéfinie aisément :


\source
\begin{lstlisting}
char *strdup (const char *s)
{
  char *new_string = malloc(strlen(s) + 1);
  return strcpy(new_string, s);
}
\end{lstlisting}

