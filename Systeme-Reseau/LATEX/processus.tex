
Dans beaucoup de programmes il est nécessaire d'avoir plusieurs
processus qui tournent simultanément pour réaliser ensemble un certain
travail. Ils collaboreront à travers des tuyaux, du partage de mémoire
et autres techniques décrites ailleurs dans ce document.

Ici nous présentons deux types de processus. Le premier, présent dès
les premières versions d'UNIX, s'obtient essentiellement en dupliquant
le processus qui le lance (contenu de la mémoire, fichiers ouverts,
ressource diverses, etc).  Le processus ``fils'' possède alors un
espace mémoire totalement séparé de celui du ``père qui l'a lancé''.


Cette duplication fait qu'on l'appelle souvent \emph{processus lourd},
par opposition aux \emph{processus légers} introduits plus tard qui,
eux, partagent le même espace mémoire et les mêmes ressources.
Cette ``lourdeur'' est à relativiser, le système met en oeuvre
des techniques comme le copy-on-write qui limitent les dégâts.

Contrairement aux signaux POSIX par rapport aux signaux Unix, on ne
peut pas dire que les processus légers rendent obsolètes les processus
``lourds'' : les deux ont leur utilité, leurs avantages et inconvénients.


\section{Les processus lourds}
\sloppy
\subsection{\texttt{fork()}, \texttt{wait()}}
\label{fork}

\index{processus!fork()}
\index{fork()}
\index{processus!wait()}
\index{wait()}

\extrait
\begin{lstlisting}
#include <unistd.h>
pid_t fork(void);
pid_t wait(int *status)
\end{lstlisting}


La fonction \texttt{fork()} crée un nouveau processus (\emph{fils})
semblable au processus courant (\emph{père}). La valeur renvoyée
n'est pas la même pour le fils (\texttt{0}) et pour le père (numéro de
processus du fils). \texttt{-1} indique un échec.


La fonction \texttt{wait()} attend qu'un des processus fils soit
terminé.  Elle renvoie le numéro du fils, et son \texttt{status} (voir
\texttt{exit()}) en paramètre passé par adresse.


\textbf{Attention. } Le processus fils \emph{hérite} des descripteurs
ouverts de son père. Il convient que chacun des processus ferme
les descripteurs qui ne le concernent pas.

\textbf{Exemple :}



\source
\lstinputlisting{../PROGS/Divers/biproc.c}



\textbf{Exercice :} Observez ce qui se passe si, dans la fonction
\texttt{affiche()}, on remplace l'appel à \texttt{lire()} par un
\texttt{read()} ? Et si on ne fait pas le \texttt{wait()} ?

\subsection{ \texttt{waitpid()} : attente de changement d'état}

\index{processus!waitpid()}
\index{waitpid()}

La fonction \texttt{waitpid()} permet d'attendre un \emph{changement
  d'état} d'un des processus fils désigné par son \emph{pid}
(n'importe lequel si \emph{pid = -1}), et de récupérer éventuellement
son code de retour.


\extrait
\begin{lstlisting}
#include <sys/types.h>
#include <sys/wait.h>

pid_t  waitpid (pid_t pid, int *status, int options);
\end{lstlisting}

L'option \texttt{WNOHANG} rend \texttt{waitpid} non bloquant (qui
retourne alors \texttt{-1} si le processus attendu n'est pas terminé).

\textbf{Exemple :}


\extrait
\begin{lstlisting}
int pid_fils;
int status;

if( (pid_fils = fork()) != 0) {
   code_processus_fils();
   exit(EXIT_SUCCESS);
   };
...
if (waitpid(pid_fils,NULL,WMNOHANG) == -1)
   printf("Le processus fils n'est pas encore terminé\n");
...
\end{lstlisting}


La fonction retourne 
\begin{itemize}
\item
si un processus fils s'est terminé, le numéro du processus fils ;
\item
  \texttt{-1} si le processus a reçu un signal, 
et la variable \texttt{errno} contient alors  \texttt{EINTR} ;
\item
  \texttt{0} si l'option \texttt{WNOHANG} était indiqué, et qu'aucun
  processus n'a changé d'état
\end{itemize}

Les macros suivantes permettent de connaître la nature du changement d'état

\index{processus!status}

\begin{itemize}
\item 
  \texttt{WIFEXITED(status)} Vrai si le fils s'est terminé normalement
  (appel à \texttt{exit()} ou \texttt{return} depuis \texttt{main()}.
  Dans ce cas \texttt{WEXITSTATUS(status)} donne la
  valeur du code de retour status fourni par le processus fils.
\item
  \texttt{WIFSIGNALED(status)}
  Vrai si le fils s'est terminé à cause d'un signal non intercepté.
  Dans ce cas \texttt{WTERMSIG(status)}
              donne le numéro du signal qui a causé la fin du fils. 
\item \texttt{WIFSTOPPED(status)}
  Vrai si le fils est actuellement stoppé.
  Dans ce cas \texttt{WSTOPSIG(status)}
  donne le numéro du signal qui a causé l'arrêt du fils. 
\item \texttt{WIFCONTINUED(status)}
  indique si le processus a été continué.
\end{itemize}


\source
\lstinputlisting{../PROGS/Essais/test-waitpid.c}


\subsection{\texttt{exec()}}

\index{exec()!famille de fonctions}

\extrait
\begin{lstlisting}
#include <unistd.h>

int execv (const char *FILENAME, char *const ARGV[])
int execl (const char *FILENAME, const char *ARG0,...)
int execve(const char *FILENAME, char *const ARGV[], char *const ENV[])
int execle(const char *FILENAME, const char *ARG0,...char *const ENV[])
int execvp(const char *FILENAME, char *const ARGV[])
int execlp(const char *FILENAME, const char *ARG0, ...)
\end{lstlisting}

  
Ces fonctions font toutes la même chose : activer un exécutable
\emph{à la place} du processus courant. Elles diffèrent par
la manière d'indiquer les paramètres.

\begin{itemize}
\item  \texttt{execv()} : les paramètres de la commande sont transmis
sous forme d'un tableau de pointeurs sur des chaînes de caractères (le
dernier étant \texttt{NULL}). Exemple:


\source
\lstinputlisting{../PROGS/Divers/execv.c}



\item \texttt{execl()} reçoit un nombre variable de paramètres.
Le dernier est \texttt{NULL}). Exemple:


\source
\lstinputlisting{../PROGS/Divers/execl.c}



\item  \texttt{execve()} et \texttt{execle()} ont un paramètre
supplémentaire pour préciser l'\emph{environnement}.  

\item 
\texttt{execvp()} et \texttt{execlp()} utilisent la variable
d'environnement \texttt{PATH} pour localiser l'exécutable à lancer. On
pourrait donc écrire simplement:

\extrait
\begin{lstlisting}
  execlp("gcc", "gcc", fichier, "-o", prefixe, NULL);

  // avec execvp
  char * args[] = { "gcc", fichier, "-o", prefixe, NULL };
  execlp("gcc", args);
\end{lstlisting}
\end{itemize}

\subsection{Numéros de processus : \texttt{getpid()}, \texttt{getppid()}}


\index{getpid}
\index{getppid}
\index{processus!getpid}
\index{processus!getppid}

\extrait
\begin{lstlisting}
#include <unistd.h>

pid_t getpid(void);
pid_t getppid(void);
\end{lstlisting}


\texttt{getpid()} permet à un processus de connaître son propre numéro, et 
\texttt{getppid()} celui de son père.


\subsection{Programmation d'un démon}

Les \emph{démons}\footnote{Traduction de l'anglais \emph{daemon}, 
acronyme de  Disk And Extension MONitor'', qui désignait une
des parties résidentes d'un des premiers systèmes d'exploitation}
 sont des processus qui tournent normalement en
arrière-plan pour assurer un service. Pour programmer correctement un
démon, il ne suffit pas de faire un \texttt{fork()}, il faut aussi
s'assurer que le processus restant ne bloque pas de ressources. Par
exemple il doit libérer le terminal de contrôle du processus, revenir
à la racine, faute de quoi il empêchera le démontage éventuel du système de
fichiers à partir duquel il a été lancé.

\index{demon@démon}
\index{processus!démon}


\source
\lstinputlisting{../PROGS/Divers/demon.c}


Voir FAQ Unix : \emph{1.7 How do I get my program to act like a daemon}


\section{Les processus légers (Posix 1003.1c)}

\index{processus!légers}
\index{processus!thread}

Les processus classiques d'UNIX possèdent des ressources
séparées (espace mémoire, table des fichiers ouverts...).
Lorsqu'un nouveau \emph{fil d'exécution} (processus fils) 
est créé par \texttt{fork()}, il se voit attribuer une 
\emph{copie} des ressources du
processus père. 


Il s'ensuit deux problèmes : 
\begin{itemize}
\item  problème de  performances, puisque la 
duplication est un mécanisme coûteux
\item  problème de communication entre les processus, qui ont 
des variables séparées.
\end{itemize}

Il existe des moyens d'atténuer ces problèmes : technique du 
\emph{copy-on-write} dans le noyau pour ne dupliquer les
pages mémoires que lorsque
      c'est strictement nécessaire), utilisation
de segments de mémoire partagée (IPC) pour mettre des données en commun.
Il est cependant apparu utile de définir un mécanisme permettant
d'avoir plusieurs \emph{fils d'exécution} (threads) dans un même espace de
ressources non dupliqué : c'est ce qu'on appelle les 
\emph{processus légers}. Ces processus légers peuvent se voir affecter 
des priorités.


On remarquera que la commutation entre deux threads d'un même groupe est
une opération économique, puisqu'il n'est pas utile de recharger 
entièrement la table des pages de la MMU.


Ces processus légers ayant vocation à communiquer entre eux, la norme
POSIX 1003.1c définit également des mécanismes de synchronisation :
exclusion mutuelle (\emph{mutex}), \emph{sémaphores}, et \emph{conditions}.

\textbf{Remarque :} les sémaphores ne sont pas définis dans les
bibliothèques de AIX 4.2 et SVR4 d'ATT/Motorola. Ils existent dans
Solaris et les bibliothèques pour Linux.


\subsection{Threads}

\index{processus léger!pthread\_create()}
\index{processus léger!pthread\_exit()}
\index{pthread\_join()}
\index{pthread\_create()}
\index{pthread\_exit()}
\index{pthread\_join()}
\extrait
\begin{lstlisting}
#include <pthread.h>

int pthread_create(pthread_t      *thread, 
                   pthread_attr_t *attr, 
                   void           *(*start_routine)(void *), 
                   void           *arg);
void pthread_exit(void *retval);
int pthread_join(pthread_t th, void **thread_return);
\end{lstlisting}


La fonction \texttt{pthread\_create} demande le lancement d'un nouveau
processus léger, avec les attributs indiqués par la structure
pointée par \texttt{attr} (\texttt{NULL} = attributs par défaut).
Ce processus exécutera la fonction \texttt{start\_routine}, en lui donnant le
pointeur \texttt{arg} en paramètre.  L'identifiant du processus léger est
rangé à l'endoit pointé par \texttt{thread}.


Ce processus léger se termine (avec un code de retour) 
lorsque la fonction qui lui est
associée se termine par \texttt{return} \emph{retcode}, ou lorsque 
le processus léger exécute un
\texttt{pthread\_exit} \emph{(retcode)}.


La fonction \texttt{pthread\_join} permet au processus père d'attendre 
la fin d'un processus léger, et de récupérer éventuellement
son code de retour.



\textbf{Priorités :} Le fonctionnement des processus légers peut être
modifié (priorités, algorithme d'ordonnancement, etc.) en manipulant
les \emph{attributs} qui lui sont associés. Voir les fonctions
\texttt{pthread\_attr\_init}, \texttt{pthread\_attr\_destroy},
\texttt{pthread\_attr\_set-detachstate},
\texttt{pthread\_attr\_getdetachstate},
\texttt{pthread\_attr\_setschedparam},
\texttt{pthread\_attr\_getschedparam},
\texttt{pthread\_attr\_setschedpolicy},
\texttt{pthread\_attr\_getschedpolicy},
\texttt{pthread\_attr\_setinheritsched},
\texttt{pthread\_attr\_getinheritsched}, \texttt{pthread\_attr\_setscope},
\texttt{pthread\_attr\_getscope}.

\subsection{Verrous d'exclusion mutuelle (mutex)}

\index{mutex}
\index{processus léger!pthread\_mutex\_init()}
\index{processus léger!pthread\_mutex\_destroy()}
\index{processus léger!pthread\_mutex\_lock()}
\index{processus léger!pthread\_mutex\_unlock()}
\index{processus léger!pthread\_mutex\_trylock()}
\index{pthread\_mutex\_init()}
\index{pthread\_mutex\_destroy()}
\index{pthread\_mutex\_lock()}
\index{pthread\_mutex\_unlock()}
\index{pthread\_mutex\_trylock()}



\extrait
\begin{lstlisting}
#include <pthread.h>

int pthread_mutex_init(pthread_mutex_t   *mutex,   
                       const pthread_mutexattr_t *mutexattr);
int pthread_mutex_destroy(pthread_mutex_t *mutex);

int pthread_mutex_lock(pthread_mutex_t *mutex));
int pthread_mutex_unlock(pthread_mutex_t *mutex);

int pthread_mutex_trylock(pthread_mutex_t *mutex);
\end{lstlisting}


Les verrous d'exclusion mutuelle (\texttt{mutex}) sont  créés
par \texttt{pthread\_mutex\_init}. Il en est de différents types
(rapides, récursifs, etc.), selon les attributs pointés par
le paramètre \texttt{mutexattr}. La valeur par défaut (\texttt{mutexattr=NULL})
fait généralement l'affaire. L'identificateur du verrou est placé
dans la variable pointée par \texttt{mutex}.


\texttt{pthread\_mutex\_destroy} détruit le verrou. 
\texttt{pthread\_mutex\_lock} tente de le bloquer (et met le thread en attente
si le verrou est déjà bloqué), \texttt{pthread\_mutex\_unlock} le débloque.
\texttt{pthread\_mutex\_trylock} tente de bloquer le verrou, et échoue si
le verrou est déjà bloqué.


\subsection{Exemple}

\textbf{Source :}

\source
\lstinputlisting{../PROGS/Threads/leger_mutex.c}


\textbf{Compilation}:

Sous Linux, les programmes doivent être compilés avec 
l'option  \texttt{-pthread}:

\index{Options de compilation!bibliothèque des threads}



\extrait
\begin{lstlisting}
gcc -std=c18 -g -Wall -pedantic  leger_mutex.c -o leger_mutex -pthread
\end{lstlisting}


\textbf{Exécution}:

\extrait
\begin{lstlisting}
% leger_mutex 
[1] Hello
[2] World
[3] Hello
[4] Hello
[5] World
 5 lignes.
%
\end{lstlisting}



\subsection{Sémaphores}


Les sémaphores, qui font partie de la norme POSIX,
ne sont pas implémentés 
dans toutes les bibliothèques de threads.

\index{semaphore POSIX@sémaphore POSIX!sem\_init}
\index{semaphore POSIX@sémaphore POSIX!sem\_destroy}
\index{semaphore POSIX@sémaphore POSIX!sem\_wait}
\index{semaphore POSIX@sémaphore POSIX!sem\_post}
\index{semaphore POSIX@sémaphore POSIX!sem\_trywait}
\index{semaphore POSIX@sémaphore POSIX!sem\_getvalue}


\extrait
\begin{lstlisting}
#include <semaphore.h>

int sem_init(sem_t *sem, int pshared, unsigned int value);
int sem_destroy(sem_t * sem);

int sem_wait(sem_t * sem);
int sem_post(sem_t * sem);

int sem_trywait(sem_t * sem);
int sem_getvalue(sem_t * sem, int * sval);
\end{lstlisting}


Les sémaphores sont créés par \texttt{sem\_init}, qui place
l'identificateur du sémaphore à l'endroit pointé par \texttt{sem}. La
valeur initiale du sémaphore est dans \texttt{value}. Si
\texttt{pshared} est nul, le sémaphore est local au processus lourd
(le partage de sémaphores entre plusieurs processus lourds n'est pas
implémenté dans la version courante de \texttt{linuxthreads}.).



\texttt{sem\_wait} et \texttt{sem\_post} sont les équivalents respectifs
des primitives \texttt{P} et \texttt{V} de Dijkstra. La fonction
\texttt{sem\_trywait} échoue (au lieu de bloquer) si la valeur du
sémaphore est nulle. Enfin, \texttt{sem\_getvalue} consulte la valeur
courante du sémaphore.




\textbf{Exercice}: Utiliser un sémaphore au lieu d'un \texttt{mutex} 
pour sécuriser l'exemple.

\subsection{Conditions}

Les \emph{conditions}  servent à mettre en attente 
des processus légers derrière un \texttt{mutex}. 
Une primitive permet de débloquer d'un seul
coup tous les threads bloqués par une même condition.

\index{condition!pthread\_cond\_init}
\index{condition!pthread\_cond\_destroy}
\index{condition!pthread\_cond\_wait}
\index{condition!pthread\_cond\_signal}
\index{condition!pthread\_cond\_broadcast}

\extrait
\begin{lstlisting}
#include <pthread.h>

int pthread_cond_init(pthread_cond_t  *cond, 
                      pthread_condattr_t *cond_attr);
int pthread_cond_destroy(pthread_cond_t *cond);

int pthread_cond_wait(pthread_cond_t *cond,
                      pthread_mutex_t *mutex);

int pthread_cond_signal(pthread_cond_t *cond);
int pthread_cond_broadcast(pthread_cond_t *cond);
\end{lstlisting}




Les conditions sont créées par
\texttt{phtread\_cond\_init}, et  détruites par \texttt{phtread\_cond\_destroy}.

 
Un processus se met en attente en effectuant un 
\texttt{phtread\_cond\_wait} (ce qui bloque au passage un \texttt{mutex}).
La primitive 
\texttt{phtread\_cond\_broadcast} débloque tous les processus qui attendent
sur une condition, \texttt{phtread\_cond\_signal} en débloque un seul.



