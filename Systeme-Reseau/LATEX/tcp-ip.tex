
\section{Communication par TCP-IP, spécificités}

Les concepts fondamentaux de la transmission d'informations par le
réseau Internet (sockets, adresses, communication par datagrammes et flots,
client-serveur etc.) sont les mêmes que pour les sockets locaux (voir
\ref{sockets}).

Les spécificités concernent essentiellement l'adressage : comment
\textbf{fabriquer une adresse} de socket à partir d'un nom de machine
(résolution) et d'un numéro de port, comment retrouver le nom d'une
machine à partir d'une adresse (résolution inverse) etc.


\section{Sockets, addresses}

\index{socket!TCP-IP}

\extrait
\begin{lstlisting}
#include <sys/types.h>
#include <sys/socket.h>

int socket(int domain, int type, int protocol);
\end{lstlisting}


Pour communiquer, les applications doivent créer des \emph{sockets}
(prises bidirectionnelles) par la fonction \texttt{socket()} et les
relier entre elles.  On peut ensuite utiliser ces sockets comme des
fichiers ordinaires (par \texttt{read}, \texttt{write}, ...) ou par
des opérations spécifiques ( \texttt{send}, \texttt{sendto},
\texttt{recv}, \texttt{recvfrom}, ...).


\subsection{ \texttt{struct sockaddr} : adresses de sockets}

Pour désigner un socket sur une machine il faut une \emph{adresse de
socket}. Les fonctions réseau comme \texttt{bind}


\begin{lstlisting}
int bind(int socket,
         const struct sockaddr *address,
         socklen_t address_len);
\end{lstlisting}

prennent commme paramètre une \textbf{adresse de socket} désignée par
un pointeur.

\paragraph{Terminologie :} pour éviter les confusions, dans ce texte
\begin{itemize}
\item le terme \textbf{adresse} se réfère toujours une \textbf{adresse de socket} (concept réseau)
  \item on parle de \textbf{numéro IP} pour une adresse d'une machine (\textbf{hôte}), indépendamment du port (\textbf{service}),
\item on utilise systématiquement le mot \textbf{pointeur} pour parler des
  adresses de données en mémoire (concept du langage C)
\end{itemize}

Ce pointeur est obtenu par transtypage (cast) d'un
pointeur sur une structure contenant une adresse de socket. Le type de
la structure peut être différent pour ipv4 ou ipv6.

Les adresses de sockets ont un premier champ
\footnote{qui s'appelle  \texttt{sa\_family} pour \texttt{struct sockaddr}}
de type
\texttt{sa\_family\_t}, qui comme son nom l'indique est
une indication de la famille d'adresses : constantes
\texttt{AF\_INET} pour ipv4, \texttt{AF\_INET6} pour ipv6.

On utilisera 3 types de structures pour contenir des adresses

\begin{itemize}
\item \texttt{struct sockaddr\_in},  spécifiquement pour les adresses IPv4 ;
\item \texttt{struct sockaddr\_in6}, pour IPv6 ;
\item \texttt{struct sockaddr\_storage}, assez grande pour contenir
  n'importe quel type d'adresse ;
\end{itemize}

et les pointeurs de type \texttt{struct sockaddr *} serviront à
désigner ces structures de façon générique.\footnote{C'est une façon
courante de réaliser un genre de polymorphisme en C}


\subsection{\texttt{struct sockaddr\_in} : adresses de sockets IPv4}

\index{socket!TCP-IP!adresses IPv4}


Pour la communication par IP en général, l'adresse d'un socket est
formée à partir d'un numéro IP, et un numéro de port (service).

Les \texttt{struct sockaddr\_in} sont destinées spécifiquement
aux adresses IPv4, elles ont 3 champs importants :
\begin{itemize}
\item \texttt{sin\_family}, la famille d'adresses, valant \texttt{AF\_INET}
\item \texttt{sin\_addr}, pour le numéro IP,
\item \texttt{sin\_port}, pour le numéro de port
\end{itemize}

Attention, les octets du numéro IP et du numéro de port sont
stockés dans \emph{l'ordre réseau} (big-endian), qui n'est pas forcément
celui de la machine hôte sur laquelle s'exécute le programme. 

Pour renseigner correctement cette structure, il existe des fonctions
de conversion, en particulier \texttt{getaddrinfo()} que nous verrons
plus loin, et qui assure la \emph{résolution d'adresse}, c'est-à-dire
la traduction\footnote{sn consultant éventuellement des serveurs de
  noms (DNS)} d'un nom de machine (exemple \texttt{www.u-bordeaux.fr})
en numéro IP.

\subsection{\texttt{struct sockaddr\_in6} : adresses de socket IPv6}

\index{socket!TCP-IP!adresses IPv6}

Pour IPv6, on utilise
des \texttt{struct sockaddr\_in6}, avec
\begin{itemize}
\item \texttt{sin6\_family}, valant \texttt{AF\_INET6}
\item \texttt{sin6\_addr}, le numéro IP sur 6 octets
\item \texttt{sin6\_port}, pour le numéro de port.
\end{itemize}

Même remarque : on utilisera \texttt{getaddrinfo()} pour fabriquer ces
adresses de socket.


\subsection{\texttt{struct sockaddr\_storage} : conteneur d'adresse}

Cette structure est destinée à contenir une adresse de socket réseau de
n'importe quel type. Son premier champ \texttt{ss\_family} contient la
famille de l'adresse.

\section{Remplissage d'une adresse de socket : \texttt{getaddrinfo()}}

Comme pour les sockets locaux, on crée les sockets par \texttt{socket()}
et on  ``nomme la prise'' par \texttt{bind()}, à partir d'une
\emph{adresse de socket} que l'on a renseignée préalablement.

Pour constituer cette adresse, on utilise \texttt{getaddrinfo()} !

\index{getaddrinfo()}
\extrait
\begin{lstlisting}
#include <sys/types.h>
#include <sys/socket.h>
#include <netdb.h>

int getaddrinfo(const char *node, const char *service,
                const struct addrinfo *hints,
                struct addrinfo **res);
\end{lstlisting}

qui retourne une liste d'adresses de sockets IPv4 et/ou IPv6, selon
les arguments qui lui sont données :

\begin{itemize}
\item \texttt{node} et \texttt{service} : chaines contenant le
nom de la machine et du port,
\item un pointeur sur une structure \texttt{hints} contenant des
  indications diverses.
\end{itemize}

Le résultat est une liste chainée de structures \texttt{addrinfo}
contenant des adresses de sockets :

\begin{lstlisting}
struct addrinfo {
      int              ai_flags;
      int              ai_family;
      int              ai_socktype;
      int              ai_protocol;
      socklen_t        ai_addrlen;
      struct sockaddr *ai_addr;
      char            *ai_canonname;
      struct addrinfo *ai_next;
};
\end{lstlisting}

ce type de structure sert aussi pour les indications complémentaires
pour la requête (voir plus loin)

Les champs qui nous intéressent, dans les résultats, sont

\begin{itemize}
\item \texttt{ai\_family} qui indique la famille d'adresses : 
constante \texttt{AF\_INET}  pour ipv4, \texttt{AF\_INET6}  pour ipv6 ; 
\item \texttt{ai\_addr}  qui est un pointeur sur une adresse de socket
\item \texttt{ai\_addrlen}, la longueur de cette adresse ;
\item \texttt{ai\_next}, un pointeur sur le résultat suivant.
\end{itemize}
Attention, il faudra libérer la liste de résultats après usage, 
en appelant \texttt{freeaddrinfo()}.

\begin{lstlisting}
  void freeaddrinfo(struct addrinfo *res);
\end{lstlisting}


\subsection{Préparation d'une adresse distante}

Dans le cas le plus fréquent, on cherche à joindre une machine

\begin{itemize}
\item dont on connait le nom  \texttt{"www.elysee.fr"} ou le numéro
  \texttt{"8.253.7.126"};
\item  du \emph{service} que l'on veut joindre, que ce soit par un nom
  (exemple \texttt{"http"}) ou un numéro de port \texttt{"80"}.\footnote{
  le fichier \texttt{/etc/services} des machines Unix
  contient une table de correspondance entre les noms de services et les
  numéros de port}
%  \item avec un protocole de type ``flot de données'' ou ``datagrammes''.
\end{itemize}


\paragraph{Un exemple simple : } ouverture d'un socket ``flot de données''
vers le port 80 de la machine \texttt{www.elysee.fr}

\begin{lstlisting}
  struct addrinfo * adr_premier;

  // récupérer dans adr_premier la première adresse
  // qui correspond
  getaddrinfo("www.elysee.fr", "http", NULL, & adr_premier);

  // utilisation de l'adresse
  int fd = socket(adr_premier->ai_family, SOCK_STREAM, 0);
  bind(fd, adr_premier->ai_addr, adr_premier_>ai_addrlen);

  freeaddrinfo(adr_premier);
\end{lstlisting}

Il faudrait y ajouter des vérifications : échec des fonctions, absece
de résultats, etc.

%% \index{gethostbyname(nom)}

%% \extrait
%% \begin{lstlisting}
%%        #include <netdb.h>
%%        struct hostent *gethostbyname(const char *name);
%%        \end{lstlisting}

%% \texttt{gethostbyname()} retourne un pointeur vers une structure
%% \texttt{hostent} qui contient diverses informations sur la machine en
%% question, en particulier une adresse \texttt{h\_addr} \footnote{Une
%% machine peut avoir plusieurs adresses} que l'on mettra dans
%% \texttt{sin\_addr}.


%% Le champ \texttt{sin\_port} est un entier court \emph{en ordre réseau}, pour
%% y mettre un entier ordinaire il faut le convertir par \texttt{htons()}
%% 		    \footnote{Host TO Network Short}.


On trouvera une utilisation plus détaillée dans l'exemple
\texttt{client-echo.c}.

\subsection{Préparation d'une adresse locale}

%% Pour une adresse sur la machine locale\footnote{ne pas confondre avec
%%   la notion de \emph{socket du domaine local} vue en \ref{sockets}},
%% on utilise
%% \begin{itemize}
%% \item  dans le cas le plus fréquent, l'adresse \texttt{INADDR\_ANY}
%% (\texttt{0.0.0.0}). Le
%% socket est alors ouvert (avec le même numéro de port) sur toutes les
%% adresses IP de toutes les interfaces de la machine.  
%% \item  l'adresse
%% \texttt{INADDR\_LOOPBACK} correspondant à l'adresse locale
%% \texttt{127.0.0.1} (alias \texttt{localhost}). Le socket n'est alors
%% accessible que depuis la machine elle-même.  
%% \item  une des adresses IP de la la machine.
%% \item  l'adresse de diffusion générale (\emph{broadcast}) 
%% \texttt{INADDR\_BROADCAST} 
%% (\texttt{255.255.255.255})\footnote{L'utilisation de l'adresse de diffusion est
%% soumise à restrictions, voir manuel}
%% \end{itemize}


%% Voir exemple \texttt{serveur-echo.c}.

%% Pour IPV6
%% \begin{itemize}
%% \item Famille d'adresses \texttt{AF\_INET6}, famille de protocoles
%% \texttt{PF\_INET6}
%% \item adresses prédéfinies 
%% \texttt{IN6ADD\_LOOPBACK\_INIT} (\texttt{::1})
%% \texttt{IN6ADD\_ANY\_INIT}
%% \end{itemize}



Dans l'exemple ci-dessus, le troisième paramètre de \texttt{getaddrinfo()}
est un pointeur nul, mais on peut s'en servir pour transmettre
l'adresse d'une structure \texttt{addrinfo} qui sert à préciser ce que l'on
veut.

Les champs pertinents (les plus intéressants) :

\begin{itemize}
\item \texttt{ai\_family} qui indique la ou les familles d'adresses
  voulues : constante \texttt{AF\_INET} pour ipv4, \texttt{AF\_INET6}
  pour ipv6, \texttt{AF\_UNSPEC} (par défaut) pour l'un ou l'autre.

\item \texttt{ai\_socktype} indique si on ne doit rechercher que certains
types de sockets (\texttt{SOCK\_STREAM}, \texttt{SOCK\_DGRAM}) ou tous (O).

\item \texttt{ai\_flags} : combinaison d'indicateurs divers. Y mettre
  \texttt{AI\_PASSIVE} pour un serveur qui doit accepter des
  connexions sur ses différentes adresses réseau.
\end{itemize}

Les autres doivent être initialisés, 0 est une valeur par défaut
raisonnable.

L'initialisation se fait élégamment grâce aux ``Designated
Initializers'' introduits par la norme C99. Exemple\footnote{
  Avec ce type d'initialisation des structures, les champs
  non spécifiés sont initialisés à 0.}

\begin{lstlisting}
struct addrinfo indications_client = {    // ipv4 ou ipv6
   .ai_family   = AF_UNSPEC,
   .ai_socktype = SOCK_STREAM
};
   
struct addrinfo indications_serveur = {   // pour un serveur IPv6
   .ai_family = AF_INET6,
   .ai_flags  = AI_PASSIVE
};
\end{lstlisting}



\subsection{Examen d'une adresse : \texttt{getnameinfo()}}
  
\index{getaddrinfo()}
La fonction `getnameinfo()` réalise la conversion dans l'autre sens :
à partir d'une adresse de socket réseau, obtenir une description du
numéro IP de la machine et du numéro de service/port.

\extrait
\begin{lstlisting}
int getnameinfo(const struct sockaddr *addr, socklen_t addrlen,
                char *host, socklen_t hostlen,
                char *serv, socklen_t servlen, 
		int flags);
\end{lstlisting}


Les paramètres sont

\begin{itemize}
\item un pointeur sur la structure contenant l'adresse, et sa taille
\item l'adresse d'un tampon pour y ranger le nom de la machine, et sa longueur
\item l'adresse d'un tampon pour le port, et sa longueur.
\item une combinaison d'indicateurs : \texttt{NI\_NUMERICHOST} et
  \texttt{NI\_NUMERICSERV}, pour avoir les indications sous forme
  numérique.
\end{itemize}

\paragraph{Un exemple :} programme de résolution d'adresses.

Le programme qui suit s'exécute en ligne de commande.
Il prend en paramètre une série de noms de machines ou d'adresses

Pour chaque machine il affiche les adresses IP numériques
sous forme numérique IPv4 (décimal pointé) ou IPv6 (hexadécimal).


\source
\lstinputlisting{../PROGS/Resolution/resolution.c}



\subsection{Adresse associée à un socket}

\index{getsockname()}
\index{getpeername()}

A partir d'un ``field descriptor'' ouvert, ces deux fonctions
permettent de retrouver l'adresse

\begin{itemize}
\item du socket auquel il est lié (local)
\item du socket ``pair'' auquel il est connecté (distant)
\end{itemize}




\extrait
\begin{lstlisting}
#include <sys/socket.h>

int   getsockname(int sockfd,  
                  struct sockaddr  * name,
                  socklen_t        * namelen )
int   getpeername(int sockfd,
                  struct sockaddr  * addr,
                  socklen_t        * addrlen);
\end{lstlisting}

Paramètres :
\begin{itemize}

\item le descripteur,
  \item un pointeur sur un ``conteneur d'adresse
de socket'' (de préférence une structure \texttt{sockaddr\_storage}
pour avoir la compatibilité ipv4/ipv6),
\item
  un pointeur sur un entier
qui recevra la longueur de l'adresse.
\end{itemize}

 

%% On peut également tenter une \emph{résolution inverse}, c'est-à-dire
%% de retrouver le nom à partir de l'adresse de socket, en passant par 
%% \texttt{gethostbyaddr}, qui retourne un pointeur vers une structure \texttt{hostent},
%% dont le champ \texttt{h\_name} désigne le nom officiel de la machine.


%% Cette résolution n'aboutit pas toujours, parce que tous les numéros IP
%% ne correspondent pas à des machines  déclarées. 


%% \extrait
%% \begin{lstlisting}
%% #include <netdb.h>
%% extern int h_errno;

%% struct hostent *gethostbyaddr(const char *addr, int len, int type);

%% struct hostent {
%%   char    *h_name;        /* official name of host */
%%   char    **h_aliases;    /* alias list */
%%   int     h_addrtype;     /* host address type */
%%   int     h_length;       /* length of address */
%%   char    **h_addr_list;  /* list of addresses */
%% }
%% #define h_addr  h_addr_list[0]  /* for backward compatibility */
%% \end{lstlisting}


\section{Fermeture d'un socket}

\index{close(socket)}
\index{shutdown(socket,indic)}

Un socket peut être fermé par \texttt{close()} ou par \texttt{shutdown()}.


\extrait
\begin{lstlisting}
int shutdown(int fd, int how);
\end{lstlisting}


Un socket est bidirectionnel, le paramètre \texttt{how} indique quelle(s) 
moitié(s) on ferme :
\texttt{SHUT\_RD} pour l'entrée,
\texttt{SHUT\_WR} pour la sortie,
\texttt{SHUT\_RDWR} pour les deux (équivaut à \texttt{close()}).


\section{Communication par datagrammes (UDP)}

\subsection{Création d'un socket}       


\extrait
\begin{lstlisting}
#include <sys/types.h>
#include <sys/socket.h>

int socket(int domain, int type, int protocol);
\end{lstlisting}



Cette fonction construit un socket et retourne un numéro de
descripteur.

Pour une liaison par datagrammes via Internet, indiquez
famille d'adresses, le type \texttt{SOCK\_DGRAM} et le protocole
par défaut \texttt{0}.



Retourne \texttt{-1} en cas d'échec.


\subsection{Connexion de sockets}

\index{socket!connect(socket,adresse,longueur)!réseau}

La fonction \texttt{connect} met en relation un socket (de cette
machine) avec un autre socket désigné, qui sera le correspondant par
défaut'' pour la suite des opérations.


\extrait
\begin{lstlisting}
#include <sys/types.h>
#include <sys/socket.h>

int  connect(int sockfd, 
             const struct sockaddr *serv_addr,
             socklen_t addrlen);
             \end{lstlisting}


\subsection{Envoi de datagrammes}

Sur un socket connecté (voir ci-dessus), on peut expédier
des datagrammes (contenus dans un tampon
\texttt{t} de longueur \texttt{n}) par \texttt{write(sockfd,t,n)}.


La fonction \texttt{send()} 

\extrait
\begin{lstlisting}
int send(int s, const void *msg, size_t len, int flags);
\end{lstlisting}

permet d'indiquer des \emph{flags}, par exemple
\texttt{MSG\_DONTWAIT} pour  une écriture non bloquante.


Enfin, \texttt{sendto()} envoie un datagramme à une adresse
spécifiée, sur un socket connecté ou non.



\extrait
\begin{lstlisting}
int sendto(int s, const void *msg, size_t len, int flags,
           const struct sockaddr *to, socklen_t tolen);
           \end{lstlisting}


\subsection{Réception de datagrammes}

Inversement, la réception peut se faire par un simple 
\texttt{read()}, par un \texttt{recv()} (avec des flags), ou
par un \texttt{recvfrom}, qui permet de remplir une structure avec
l'adresse
\texttt{from}
du socket émetteur.

\extrait
\begin{lstlisting}
int recv(int s, void *buf, size_t len, int flags);
int recvfrom(int  s,  void  *buf,  size_t len, int flags,
             struct sockaddr *from, socklen_t *fromlen);
             \end{lstlisting}


\subsection{Exemple UDP : serveur d'écho}

Principe:
\begin{itemize}
\item 
Le client envoie une chaîne de caractères au serveur.
\item 
Le serveur l'affiche, la convertit en minuscules,
et la réexpédie.
\item 
Le client affiche la réponse.
\end{itemize}

Usage:
\begin{itemize}
\item  sur le serveur : \texttt{serveur-echo} \emph{numéro-de-port} 
\item 
pour chaque client : \texttt{client-echo} \emph{nom-serveur}
\emph{numéro-de-port} \emph{``message à expédier''}
\end{itemize}

\paragraph{Le client}.

 
\source
\lstinputlisting{../PROGS/Echo-Datagrammes/client-echo.c}


\textbf{Exercice} : faire en sorte que le client réexpédie sa requête
si il ne reçoit pas la réponse dans un délai fixé. Fixer une limite
au nombre
de tentatives. 

\paragraph{Le serveur} :

 
\source
\lstinputlisting{../PROGS/Echo-Datagrammes/serveur-echo.c}


\section{Communication par flots de données (TCP)}

La création d'un socket pour TCP se fait ainsi

\extrait
\begin{lstlisting}
  int fd;
  ..
  fd = socket(AF_INET,SOCK_STREAM,0);
  \end{lstlisting}


\subsection{Programmation des clients TCP}

Le socket d'un client TCP doit être relié (par \texttt{connect()}) à celui
du serveur, et il est utilisé ensuite par des \texttt{read()} et des
\texttt{write()}, ou des entrées-sorties de haut niveau \texttt{fprintf()},
\texttt{fscanf()}, etc. si on a défini des flots par \texttt{fdopen()}.



\subsection{Exemple : client web}

 
\source
\lstinputlisting{../PROGS/TCP-Flots/client-web.c}


Remarque: souvent il est plus commode de créer des flots de haut
niveau au dessus du socket (voir \texttt{fdopen()}) que de manipuler
des \texttt{read} et des \texttt{write}. Voir dans l'exemple suivant.

\subsection{Réaliser un serveur TCP}

Un serveur TCP doit traiter des connexions venant de plusieurs clients.


Après avoir créé et nommé le socket,  
le serveur spécifie qu'il
accepte les communications entrantes par \texttt{listen()}, et se met 
effectivement en attente
d'une connexion de client par \texttt{accept()}.


\extrait
\begin{lstlisting}
#include <sys/types.h>
#include <sys/socket.h> 

int listen(int s, int backlog);
int accept(int s,  struct  sockaddr  *addr,  
                   socklen_t  *addrlen);
\end{lstlisting}


Le paramètre \texttt{backlog} indique la taille maximale de la file des
 connexions en attente. Sous Linux la limite est donnée par
la constante \texttt{SOMAXCONN} (qui vaut 128), sur d'autres systèmes elle
est limitée à 5.


La fonction \texttt{accept()} retourne un autre socket, qui sert
à la communication avec le client.
L'adresse du client peut être obtenue par les paramètres
\texttt{addr} et \texttt{addrlen}.



En général, les serveurs TCP doivent traiter simultanément des
connexions venant de plusieurs clients. La solution habituelle est de
lancer, après l'appel à \texttt{accept()} un processus fils (par
\texttt{fork()})qui traite la communication avec un seul client.

Ceci
induit une gestion des processus, donc des signaux liés à la
terminaison des processus fils.


