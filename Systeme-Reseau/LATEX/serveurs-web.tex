
Dans ce qui suit nous présentons un serveur Web rudimentaire,
capable de fournir des pages Web construites à partir des fichiers
d'un répertoire. Nous donnons deux implémentations possibles,
à l'aide de processus lourds et légers.

Attention : ces serveurs ne traitent que les requêtes \texttt{GET} de
\texttt{HTTP/1.0}, et ignorent les entêtes HTTP, les ``keep-alive'',
qui gardent les connexions ouvertes etc.


\section{Serveur Web (avec processus)}

\subsection{Principe et pseudo-code}
Cette version suit l'approche traditionnelle. Un processus
est créé chaque fois qu'un client contacte le serveur.



Pseudo-code:


\extrait
\begin{lstlisting}
ouvrir socket serveur (socket/bind/listen)
répéter indéfiniment
|    attendre l'arrivée d'un client (accept)
|    créer un processus (fork) et lui déléguer
|      la communication avec le client
fin-répeter
\end{lstlisting}


\subsection{Code du serveur}

\source
\lstinputlisting{../PROGS/Serveurs-Web/serveur-web-processus.c}


\subsection{Discussion de la solution}

Un avantage est la simplicité de la solution, et sa robustesse:
une erreur d'exécution dans un processus fils est normalement
sans incidence sur le fonctionnement des autres parties du serveur.


En revanche, la création d'un processus est une opération
relativement coûteuse, qui introduit un temps de latence
au début de chaque communication.  D'où l'idée de lancer
les processus avant d'en avoir besoin (préchargement), 
et de réutiliser ceux qui ont terminé leur tâche.

\section{Serveur Web (avec threads)}

\subsection{Principe et pseudo-code}

Les processus légers permettent une autre approche : on 
crée préalablement un pool'' de processus que l'on bloque.
Lorsqu'un client se présente, on confie la communication
à un processus inoccupé.


\extrait
\begin{lstlisting}
ouvrir socket serveur (socket/bind/listen)
créer un pool de processus
répéter indéfiniment
|    attendre l'arrivée d'un client (accept)
|    trouver un processus libre, et lui
|        confier la communication avec le client
fin-répeter
\end{lstlisting}



\subsection{Code du serveur}
 
\source
\lstinputlisting{../PROGS/Serveurs-Web/serveur-web-threads.c}


\subsection{Discussion de la solution}

Les inconvénients de cette solution sont symétriques de ses avantages.
Les processus légers partageant une grande partie leur espace mémoire, le
crash d'un processus léger risque d'emporter le reste du serveur.


On peut utiliser le même mécanisme de pool de processus'' avec des
processus classiques. La difficulté technique réside dans la
transmission le descripteur résultant de l'\texttt{accept()} du
serveur vers un processus fils. Dans la première solution
(\texttt{fork()} pour chaque connexion) le fils est lancé \emph{après}
l'\texttt{accept()}, et peut donc hériter du descripteur. Dans le cas
d'un préchargement de processus fils, ce n'est plus possible.



Pour ce faire, on peut utiliser le mécanisme (assez baroque) 
de transmission des \emph{informations
de service} (\emph{Ancillary messages} à travers une liaison par 
\emph{socket Unix} entre deux processus : des options convenables
de \texttt{sendmsg()}
permettent de faire passer un ensemble de descripteurs de fichiers
d''un processus à un autre (qui tournent sur la même machine, puisqu'ils
communiquent par un socket Unix).
% Voir exemple à la fin de cette section.



\section{Parties communes aux deux serveurs}

\subsection{Déclarations et entêtes de fonctions}

\source
\lstinputlisting{../PROGS/Serveurs-Web/constantes.h}


\subsection{Les fonctions réseau}
\begin{itemize}
\item  \texttt{serveur\_tcp()} : création du socket du serveur TCP.
\item \texttt{attendre\_client()}
\end{itemize}

 
\source
\lstinputlisting{../PROGS/Serveurs-Web/reseau.c}


\subsection{Les fonctions de dialogue avec le client}
\begin{itemize}
\item \texttt{dialogue\_client()} : lecture et traitement de la requête d'un client
\item \texttt{envoyer\_document()},
\item \texttt{document\_non\_trouve()},
\item \texttt{requete\_invalide()}.
\end{itemize}

 
\source
\lstinputlisting{../PROGS/Serveurs-Web/traitement-client.c}


\subsection{Exercices, extensions...}

\textbf{Exercice : } modifier \texttt{traitement-client} pour
qu'il traite le cas des répertoires. Il devra alors afficher
le nom des objets de ce répertoire, leur type, leur taille
et un lien.

\textbf{Exercice : } Utiliser le mécanisme de transmission
de descripteur (voir exemple plus loin) pour réaliser un serveur
à processus préchargés.
