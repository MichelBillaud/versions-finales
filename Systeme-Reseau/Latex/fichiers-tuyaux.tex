\section{Manipulation des fichiers, opérations de haut niveau}


\subsection{Flots standards, entrées et sorties sur la console}

\index{flots d'entree-sortie@flots d'entrée-sortie}
\index{stdin, entrée standard}
\index{stdout, sortie standard}
\index{stderr, sortie d'erreur}
\index{FILE*}

Quand un programme est lancé, il y a trois \emph{flots} pré-déclarés et
ouverts, qui correspondent à l'entrée et la sortie standards, ainsi
qu'à la sortie d'erreur :

\extrait
\begin{lstlisting}
#include <stdio.h>

FILE *stdin;
FILE *stdout;
FILE *stderr;
\end{lstlisting}


Vous avez déjà rencontré quelques fonctions qui agissent sur ces
flots, implicitement, sans les nommer en paramètre\footnote{Elles
  correspondent à des de fonctions plus générales \texttt{fprintf()},
  \texttt{fscanf()}, etc.  que nous verrons plus loin.}

\index{getchar()}
\index{printf()}
\index{scanf()}
\extrait
\begin{lstlisting}
int  printf(const char *format, ...);     
int  scanf(const char *format, ...);      
int  getchar(void);                       
\end{lstlisting}

\begin{itemize}
\item \texttt{printf()} écrit sur \texttt{stdout} la valeur d'expressions
  selon un format donné.
  \item \texttt{scanf()} lit sur \texttt{stdin} la
    valeur de variables.
  \item pour lire une ligne complète, on fait appel à \texttt{fgets()} en
    utilisant le flot \texttt{stdin}, voir plus loin. 
\end{itemize}


\source
\lstinputlisting{../PROGS/Divers/facture.c}


\begin{itemize}
  \item
\texttt{scanf()} renvoie le nombre d'objets qui ont pu être effectivement
lus sans erreur. 

%% \texttt{gets()} lit des caractères jusqu'à une marque de fin de ligne ou de 
%% fin de fichier, et les place (marque non comprise) dans le tampon
%% donné en paramètre, suivis par un caractère NUL. \texttt{gets()} renvoie
%% finalement l'adresse du tampon. Attention, prévoir un tampon assez grand,
%% ou (beaucoup mieux) utilisez \texttt{fgets()}. 

\item
\texttt{getchar()} lit un caractère sur l'entrée standard et retourne sa 
valeur sous forme d'entier positif, ou la constante \texttt{EOF} (= -1) 
en fin de fichier.
\end{itemize}

\subsection{Opérations sur les flots}

\index{fopen(chemin,mode)}
\index{fclose(fichier)}
\index{fprintf(fichier,format,valeurs...)}
\index{fscanf(fichier,format,adresses...)}
\index{fgetc(fichier)}


\extrait
\begin{lstlisting}
#include <stdio.h>
  
FILE *fopen (char *path, char *mode);
int   fclose  (FILE *stream);
\end{lstlisting}

\begin{itemize}
  \item 
\texttt{fopen()} tente d'ouvrir le fichier désigné par la chaîne \texttt{path}
selon le mode indiqué, qui peut être 
\begin{itemize}
\item \texttt{"r"} (lecture seulement),
\item \texttt{"r+"} (lecture et écriture),
\item \texttt{"w"} (écriture seulement),
\item \texttt{"w+"} (lecture et écriture, effacement si le fichier existe déjà),
\item \texttt{"a"} (écriture à partir de la fin du fichier si il existe déjà),
\item \texttt{"a+"} (lecture et écriture, positionnement à la fin du fichier
si il existe déjà).
\end{itemize}

Si l'ouverture échoue, \texttt{fopen()} retourne le pointeur \texttt{NULL}.
\end{itemize}

\extrait
\begin{lstlisting}
int fprintf(FILE *stream, const char *format, ...);
int fscanf (FILE *stream, const char *format, ...);
int fgetc  (FILE *stream);
\end{lstlisting}

Ces fonctions ne diffèrent de \texttt{printf()},
\texttt{scanf()} et \texttt{getchar()} que par le premier paramètre,
qui précise sur quel flot porte l'opération.


\subsection{Lecture d'une ligne : \texttt{fgets} et \texttt{getline}}


\subsubsection{À l'ancienne : \texttt{fgets}}

La fermeture par \texttt{fclose(flot)} provoquera la fermeture de \texttt{fd}
à la fois en entrée et en sortie.

Quand on fait de la programmation réseau, on a parfois besoin de ne
fermer la communication que dans un sens. Dans ce cas, on duplique le
descripteur, et on crée un flot pour chacun :

\index{dup()!utilisation avec fopen()}
\extrait
\begin{lstlisting}]
    int fd = fopen(.....);
    ...
    FILE *entree = fdopen(    fd,  "r");
    FILE *sortie = fdopen(dup(fd), "w");
\end{lstlisting}

La fermeture d'un des deux flots ne clôt qu'un sens de communication.



\subsection{Positionnement}

\index{feof(fichier)}
\index{ftell(fichier)}
\index{fseek(fichier,offset,repere)}

\extrait
\begin{lstlisting}
int  feof (FILE *stream);
long ftell(FILE *stream);
int  fseek(FILE *stream, long offset, int whence);
\end{lstlisting}

\begin{itemize}
  \item 
\texttt{feof()} indique si la fin de fichier est atteinte.
\item \texttt{ftell()} indique la \emph{position courante} dans le fichier
  (0 = début).
  \item \texttt{fseek()} déplace la position courante : si
    \texttt{whence} contient
    \begin{itemize}
      \item \texttt{SEEK\_SET} la position est donnée par
        rapport au début du fichier,
        \item \texttt{SEEK\_CUR} : déplacement
          par rapport à la position courante,
          \item \texttt{SEEK\_END} : déplacement
            par rapport à la fin.
    \end{itemize}
\end{itemize}



\subsection{Divers}

\extrait
\begin{lstlisting}
int   ferror   (FILE *stream);
void  clearerr (FILE *stream);
\end{lstlisting}

\index{ferror(dichier)}
\index{clearerr(fichier)}

\begin{itemize}
\item La fonction \texttt{ferror()} indique si une erreur a eu lieu
  sur un flot,
  \item
    \texttt{clearerr()} efface l'indicateur d'erreur
\end{itemize}


\section{Manipulation des fichiers, opérations de bas niveau}

Pendant l'exécution d'un programme, un certain nombre de fichiers sont
\emph{ouverts} (en cours d'utilisation).  I

Il existe dans le système
une table des fichiers actuellement ouverts par le programme,
les opérations de bas niveau désignent les fichiers par leur indice
dans cette table. On appelle aussi  ce numéro de fichier (\texttt{fileno})
un ``descripteur de fichier'' (\texttt{fd} = file descriptor).

Les numéros 0, 1 et 2 correspondent respectivement à l'entrée standard,
la sortie standard, et la sortie d'erreur. Dans la programmation,
utilisez plutôt les constantes \texttt{STDIN\_FILENO}
\texttt{STDOUT\_FILENO},
\texttt{STDERR\_FILENO} définies dans \texttt{unistd.h}.


Les opérations de bas niveau communiquent en général avec les fichiers
par l'intermédiaire d'un tampon, un tableau d'octets, en indiquant le
nombre d'octets à transmettre. Exemple :

\source
\begin{lstlisting}
char * message = "Hello, world";

write(STDOUT_FILENO, message, 5);
\end{lstlisting}

envoie les 5 premiers caractères du message sur la sortie standard.



\subsection{Ouverture, fermeture, lecture, écriture}

\index{open(chemin,flags,mode)}
  
\extrait
\begin{lstlisting}
#include <sys/types.h>
#include <sys/stat.h>
#include <fcntl.h>
  
int open(const char *pathname, int flags, mode_t mode);
\end{lstlisting}


Ouverture d'un fichier nommé \texttt{pathname}. Les \texttt{flags}
peuvent prendre l'une des valeurs suivantes :
\begin{itemize}
  \item \texttt{O\_RDONLY}
    (lecture seulement),
    \item \texttt{O\_WRONLY} (écriture seulement),
    \item \texttt{O\_RDWR} (lecture et écriture).
\end{itemize}
  Cette valeur peut être
  combinée éventuellement (par un ``ou logique'') avec des options:
  \begin{itemize}
\item \texttt{O\_CREAT} (création du fichier si il n'existe pas déjà),
\item \texttt{O\_TRUNC} (si le fichier existe il sera tronqué),
\item \texttt{O\_APPEND} (chaque écriture se fera à la fin du fichier),
  \item etc. 
  \end{itemize}

En cas de création d'un nouveau fichier, le \texttt{mode} sert à
préciser les droits d'accès.  Lorsqu'un nouveau fichier est créé,
\texttt{mode} est combiné avec le \texttt{umask} du processus pour
former les droits d'accès du fichier. Les permissions effectives sont
alors \verb/(mode & ~umask)/ 

Le paramètre \texttt{mode}
doit être présent quand les \texttt{flags} contiennent
\texttt{O\_CREAT}.

La fonction \texttt{open()}
retourne  le numéro de \emph{descripteur de fichier} (-1 en cas d'erreur),
un nombre entier qui sert à référencer le fichier par la suite. 

%% Les descripteurs 0, 1 et 2 correspondent aux fichiers standards 
%% \texttt{stdin}, \texttt{stdout} et \texttt{stderr} qui sont
%% normalement déjà ouverts (0=entrée, 1=sortie, 2=erreurs).

\index{close(descripteur)}
\index{read!read(descripteur,tampon,taille)}
\index{write!write(descripteur,tampon,taille)}

\extrait
\begin{lstlisting}
#include <unistd.h>
#include <sys/types.h>

int close(int fd);
int read(int fd, char *buf, size_t count);
size_t write(int fd, const char *buf, size_t count);
\end{lstlisting}


\texttt{close()} ferme le fichier indiqué par le descripteur
\texttt{fd}.  Retourne 0 en cas de succès, -1 en cas d'échec.

\texttt{read()} demande à lire \emph{au plus} \texttt{count} octets
sur \texttt{fd}, à placer dans le tampon \texttt{buf}. Retourne le
nombre d'octets qui ont été effectivement lus, qui peut être inférieur
à la limite donnée pour cause de non-disponibilité (-1 en cas
d'erreur, 0 en fin de fichier).

\texttt{write()} tente d'écrire sur
le fichier les \texttt{count} premiers octets du tampon
\texttt{buf}. Retourne le nombre d'octets qui ont été effectivement
écrits, -1 en cas d'erreur.

\paragraph*{Exemple} :

\source
\lstinputlisting{../PROGS/Divers/copie.c}


\paragraph*{Problème.} Montrez que la taille du tampon influe sur les
performances des opérations d'entrée-sortie.  Pour cela, modifiez le
programme précédent pour qu'il accepte 3 paramètres : les noms des
fichiers source et destination, et la taille du tampon (ce tampon sera
alloué dynamiquement).


\subsection{Duplication de descripteurs}

\index{descripteur!duplication}
\index{dup(descripteur)}
\index{dup2(existant,nouveau)}
\extrait
\begin{lstlisting}
#include <unistd.h>

int dup  (int oldfd);
int dup2 (int oldfd, int newfd);
\end{lstlisting}


Ces deux fonctions créent une copie du descripteur \texttt{oldfd}.
\texttt{dup()} utilise le plus petit numéro de descripteur libre.
\texttt{dup2()} réutilise le descripteur \texttt{newfd}, en fermant
éventuellement le fichier qui lui était antérieurement associé.

La valeur retournée est celle du descripteur, ou \texttt{-1} en cas d'erreur. 

L'effet sera que le nouveau descripteur désignera la même
fichier que l'ancien.

\paragraph*{Exemple} :

\source
\lstinputlisting{../PROGS/Divers/redirection.c}


\paragraph*{Exercice} : que se produit-il si on essaie de rediriger la 
\emph{sortie} standard d'une commande à la manière de l'exemple précédent ?
(essayer avec ``\texttt{ls}'', ``\texttt{ls -l}'').


\subsection{Positionnement}

\index{lseek(descripteur,position,repère)}
\extrait
\begin{lstlisting}
#include <unistd.h>
#include <sys/types.h>

off_t lseek(int fildes, off_t offset, int whence);
\end{lstlisting}


\texttt{lseek()} repositionne le pointeur de lecture. Similaire à
\texttt{fseek()}. Pour connaître la position courante, faire un appel
à \texttt{stat()}. 

\paragraph*{Exercice. } Écrire un programme pour manipuler un fichier relatif 
d'enregistrements de taille fixe.

\subsection{Verrouillage}

\index{flock(descripteur,opération)}
\extrait
\begin{lstlisting}
#include <sys/file.h>

int flock(int fd, int operation)
\end{lstlisting}


Lorsque \texttt{operation} est \texttt{LOCK\_EX}, il y a verrouillage
du fichier désigné par le descripteur \texttt{fd}. Le fichier est
déverrouillé par l'option \texttt{LOCK\_UN}.

\paragraph*{Problème. } Écrire une fonction \texttt{mutex()}
qui permettra de 
délimiter une section critique dans un programme C. Exemple
d'utilisation :

\extrait
\begin{lstlisting}
#include "mutex.h"
...
mutex("/tmp/foobar",MUTEX_BEGIN);
...
mutex("/tmp/foobar",MUTEX_END);

\end{lstlisting}

Le premier paramètre indique le nom du fichier utilisé comme
verrou. Le second précise si il s'agit de verrouiller ou
déverrouiller. Faut-il prévoir des options \texttt{MUTEX\_CREATE},
\texttt{MUTEX\_DELETE} ? Qu'arrive-t'il si un programme se termine en
``oubliant'' de fermer des sections critiques ? 

Fournir le fichier
d'interface \texttt{mutex.h}, l'implémentation \texttt{mutex.c}, et
des programmes illustrant l'utilisation de cette fonction.

 
\subsection{\texttt{mmap()} : fichiers "mappés" en mémoire}

Un fichier "mappé en mémoire" apparaît comme un tableau d'octets, ce
qui permet de le parcourir en tous sens plus commodément qu'avec des
\texttt{seek()}, \texttt{read()} et \texttt{write()}. 

C'est beaucoup plus économique que de copier le fichier dans
une zone allouée en mémoire : c'est le système
de mémoire virtuelle qui s'occupe de lire et écrire physiquement les
pages du fichier au moment où on tente d'y accéder, et gère tous les
tampons.


\index{mmap()}
\index{munmap()}  
  

\extrait
\begin{lstlisting}
#include <unistd.h>
#include <sys/mman.h>

void * mmap (void  *start,  size_t length, 
             int prot , int flags, 
             int fd, off_t offset);
int munmap  (void *start, size_t length);
\end{lstlisting}


La fonction \texttt{mmap()} "mappe" en mémoire un morceau (de longueur
\texttt{length}, en partant du \texttt{offset}-ième octet) du
fichier désigné par le descripteur \texttt{fd} , et retourne
un pointeur sur la zone de mémoire correspondante.

On peut définir quelques options  (protection en lecture seule, 
partage, etc)
grâce à \texttt{prot} et \texttt{flags}. Voir pages de manuel.
\texttt{munmap()} "libère" la mémoire.



\source
\lstinputlisting{../PROGS/Divers/inverse.c}


\section{Fichiers, répertoires etc.}

Ici ``fichier'' est compris dans son sens large (élément d'un système
de fichiers), qui inclue aussi les répertoires, les périphériques, les
tuyaux et sockets etc. (voir plus loin).


\subsection{Suppression}

\index{remove(chemin)}
  
\extrait
\begin{lstlisting}
#include <stdio.h>

int remove(const char *pathname);
\end{lstlisting}


Cette fonction supprime le fichier \texttt{pathname}, et retourne 0 en
cas de succès (-1 sinon).

\paragraph*{Exercice: } écrire un substitut pour la commande \emph{rm}.

\index{stat(chemin,adressetampon)}
\index{fstat(descripteur,adressetampon)}
\extrait
\begin{lstlisting}
#include <sys/stat.h>
#include <unistd.h>

int stat(const char *file_name, struct stat *buf);
int fstat(int filedes, struct stat *buf);
\end{lstlisting}

\subsection{Informations sur les fichiers/répertoires/...}


Ces fonctions retournent diverses informations sur un
fichier désigné par un chemin d'accès (\texttt{stat()}) ou par un 
descripteur (\texttt{fstat()}). 

\paragraph*{Exemple} :

\source
\lstinputlisting{../PROGS/Divers/taille.c}


\subsection{Parcours de répertoires}

Le parcours d'un répertoire, pour obtenir la liste des fichiers et
répertoires qu'il contient, se fait grâce aux fonctions:

\index{repertoire@répertoire!parcours}
\index{opendir(chemin)}
\index{closedir(DIR *dir)}
\index{rewinddir(DIR *dir)}
\index{seekdir(DIR *dir, position)}
\index{telldir(DIR *dir)}

\extrait
\begin{lstlisting}
#include <sys/types.h>
#include <dirent.h>
  
DIR *opendir   (const char *name);
int  closedir  (DIR *dir);
void rewinddir (DIR *dir);
void seekdir   (DIR *dir, off_t offset);
off_t telldir  (DIR *dir);
\end{lstlisting}


Voir la documentation pour des exemples. 

\paragraph*{Exercice : } écrire une version  simplifiée de la
commande \texttt{ls}.

\paragraph*{Exercice : } écrire une commande qui fasse apparaître la 
structure d'une arborescence. Exemple d'affichage :

\extrait
\begin{lstlisting}
C++
| CompiSep
| Fichiers
Systeme
| Semaphores
| Pipes
| Poly
  | SVGD
  | Essais
| Fifos
\end{lstlisting}

Conseil: écrire une fonction à deux paramètres: le chemin d'accès du 
répertoire et le niveau de récursion.


\section{Tuyaux de communication }

\index{tuyau} Les \emph{tuyaux de communication} (\emph{pipes})
permettent de faire communiquer des processus qui s'exécutent d'une
même machine. Une fois ouverts, ils sont accessibles comme les fichiers
à travers un ``file descriptor''. Ce que les processus y écrivent peut
être lu immédiatement par d'autres processus.

Nous en présentons deux variétés :

\begin{itemize}
\item les tuyaux ``simples'' qui sont créés par un processus
  et servent à  la communication entre ses descendants et lui.
\item \emph{tuyaux nommés},
  qui sont visibles (comme les fichiers et répertoires),
  dans le système de fichiers. Ils  peuvent donc être partagés
par des programmes indépendants.
\end{itemize}

Nous verrons plus loin (\ref{sockets}) une forme de communication plus
générale, les sockets.


Nous commençons par les tuyaux nommés, dont l'usage est proche des fichiers
ordinaires.
  

\subsection{Tuyaux nommés (FIFO)}

\index{tuyau!nommé}
\index{FIFO!tuyau nommé}

Les ``tuyaux nommés'' visibles dans l'arborescence des fichiers et
répertoires. Ils sont créés par \texttt{mkfifo()}, et utilisés ensuite
par
\texttt{open()},
\texttt{read()},
\texttt{write()},
\texttt{close()},
\texttt{fdopen()}, etc.


\index{mkfifo(chemin,mode)}

\extrait
\begin{lstlisting}
#include <stdio.h>

int mkfifo (const char *path, mode_t mode);
\end{lstlisting}


La fonction \texttt{mkfifo()} crée un FIFO ayant le chemin indiqué par
\texttt{path} et les droits d'accès donnés par \texttt{mode}. Si la
création réussit, la fonction renvoie 0, sinon -1. Exemple:

\extrait
\begin{lstlisting}
if (mkfifo("/tmp/fifo.courrier", 0644) != 0) {
  perror("mkfifo");
}
\end{lstlisting}


\paragraph*{Exercice.} Regarder ce qui se passe quand
\begin{itemize}
\item plusieurs processus écrivent dans une même FIFO (faire une boucle
\texttt{sleep-write}).
\item plusieurs processus lisent la même FIFO.
\end{itemize}

\paragraph*{Exercice. } Écrire une commande \emph{mutex} qui permettra de 
délimiter une section critique dans des shell-scripts. Exemple
d'utilisation :

\extrait
\begin{lstlisting}
mutex -b /tmp/foobar
...
mutex -e /tmp/foobar
\end{lstlisting}

Le premier paramètre indique si il s'agit de verrouiller (\texttt{-b}
= begin) ou de déverrouiller (\texttt{-e} = end). Le second paramètre
est le nom du verrou.

\emph{Conseil} : la première option peut s'obtenir en tentant de lire
une information quelconque (\emph{jeton}) dans une FIFO. C'est la
seconde option qui dépose le jeton, ce qui débloquera le processus
lecteur.  Prévoir une option pour créer une FIFO ?


\subsection{Tuyaux (pipe)}

On utilise un \emph{pipe} (tuyau) pour faire communiquer un processus
et un de ses descendants\footnote{ou des descendants - au sens large -
du processus qui a créé le tuyau}.

\index{tuyau!pipe(fd[2])}
\index{pipe(fd[2])}

\extrait
\begin{lstlisting}
#include <unistd.h>

int pipe(int filedes[2]);
\end{lstlisting}


L'appel \texttt{pipe()} fabrique un tuyau de communication et renvoie dans
un tableau une paire de descripteurs.  On lit à un bout du tuyau (sur
le descripteur de sortie \texttt{fildes[0]})
 ce qu'on a écrit dans l'autre (\texttt{filedes[1]}). 
 Voir exemple dans \ref{fork}.


Les \emph{pipes} ne sont pas visibles dans l'arborescence des fichiers et
répertoires, par contre ils sont hérités lors de la création d'un 
processus.

\index{socketpair}
La fonction \texttt{socketpair()} (voir
\ref{socketpair}) généralise la fonction
\texttt{pipe}.

\subsection{Pipes depuis/vers une commande}

\index{tuyau!de/vers commande}
\index{popen(commande,type)}
\index{pclose(fichier)}

\extrait
\begin{lstlisting}
#include <stdio.h>
  
FILE *popen(const char *command, const char *type);
int pclose(FILE *stream);
\end{lstlisting}


\texttt{popen()} lance la commande décrite par la chaîne \texttt{command}
et retourne un flot.


Si \texttt{type} est \texttt{"r"} le flot retourné est celui 
de la sortie standard de la commande (on peut y lire). 
Si \texttt{type} est \texttt{"w"} 
c'est son entrée standard.

\texttt{pclose()} referme ce flot.



\paragraph*{Exemple} :    envoi d'un  ``\texttt{ls -l }'' par courrier 

\source
\lstinputlisting{../PROGS/Divers/avis.c}




\section{\texttt{select()} : attente de données }

Il est assez courant de devoir attendre des données en provenance
de plusieurs sources. On utilise pour cela la fonction \texttt{select()} qui
permet de surveiller plusieurs descripteurs simultanément.

\index{select(...)}

\extrait
\begin{lstlisting}
#include <sys/time.h>
#include <sys/types.h>
#include <unistd.h>

int  select(int n, fd_set *readfds, 
                   fd_set *writefds, 
                   fd_set *exceptfds, 
                   struct timeval *timeout);

FD_CLR   (int fd, fd_set *set);
FD_ISSET (int fd, fd_set *set);
FD_SET   (int fd, fd_set *set);
FD_ZERO  (fd_set *set);      
\end{lstlisting}


Cette fonction attend que des données soient prêtes à être lues sur un
des descripteurs de l'ensemble \texttt{readfs}, ou que l'un des
descripteurs de \texttt{writefds} soit prêt à recevoir des écritures,
que des exceptions se produisent (\texttt{exceptfds}), ou encore que
le temps d'attente \texttt{timeout} soit épuisé.  

 
Lorsque
\texttt{select()} se termine, \texttt{readfds}, \texttt{writefds} et
\texttt{exceptfds} contiennent les descripteurs qui ont changé d'état.
\texttt{select()} retourne le nombre de descripteurs qui ont changé
d'état, ou \texttt{-1} en cas de problème.
 
L'entier \texttt{n} doit être
supérieur (strictement) au plus grand des descripteurs contenus dans
les 3 ensembles (c'est en fait le nombre de bits significatifs du
masque binaire qui représente les ensembles). On peut utiliser la
constante \texttt{FD\_SETSIZE}.
 
Les pointeurs sur les ensembles (ou
le délai) peuvent être \texttt{NULL}, ils représentent alors des
ensembles vides (ou une absence de limite de temps).



Les macros \texttt{FD\_CLR, FD\_ISSET, FD\_SET, FD\_ZERO} permettent de 
manipuler les ensembles de descripteurs. 


\subsection{Attente de données provenant de plusieurs sources}

\paragraph*{Exemple} :

\source
\lstinputlisting{../PROGS/Divers/mix.c}


\subsection{Attente de données avec limite de temps}

L'exemple suivant montre comment utiliser la limite de temps dans le
cas (fréquent) d'attente sur un seul descripteur.

\source
\lstinputlisting{../PROGS/SelectFifo/lecteur.c}


