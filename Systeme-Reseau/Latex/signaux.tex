Ce chapitre présente la communication par signaux entre processus.

Il y a essentiellement deux opérations sur les signaux. En gros :

\begin{itemize}
\item un appel de fonction demande l'envoi d'un \emph{signal} à un
  autre processus (destinataire), désigné par son numéro de processus.
\item le destinataire a indiqué, par autre appel de fonction
  (\texttt{signal} ou \texttt{sigaction}) quel traitement
  doit être exécuté quand il reçoit un
  signal.
\end{itemize}

Un signal est un simple nombre, dont la valeur correspond à un
particulier. Il peut être émis par un programme (la fonction d'envoi
s'appelle \texttt{kill()}) (pour des raisons historiques), ou résulter
d'un évènement système (fin d'un processus fils, ...), ou d'une
division par zéro, etc.

En particulier, quand on fait tourner un processus depuis un shell et
qu'on tape controle-c, ce caractère est reçu par le shell qui envoie
alors le signal \texttt{SIGINT} (numéro 2) au processus qui tourne en
avant-plan et rend la main à la boucle interactive du shell. De même,
contrôle-Z envoie un \texttt{SIGTSTP} (20) qui demande au destinataire
de se mettre en pause (terminal stop). Ce qu'il fait en s'envoyant
lui-même le signal \texttt{SIGSTOP} (19).


Un signal a un comportement par défaut (arrêter le
programme, ne rien faire, ...), la seconde fonction
\texttt{signal/sigaction} sert à le changer.

On regarde ici deux bibliothèques liées aux signaux :

\begin{itemize}
\item la bibliothèque des signaux ``classiques'' d'UNIX, qui est
  simple à utiliser, mais n'est pas vraiment portable\footnote{le
    comportement peut être ``légèrement'' différent selon les versions
    d'UNIX.}
\item la bibliothèque définie par la norme POSIX, plus riche mais aussi
  plus complexe.
\end{itemize}

\section{Les signaux Unix}

\index{signal()}
\index{signal unix!signal()}

\subsection{\texttt{signal()}}

L'implémentation GNU se présente sous la forme

\extrait
\begin{lstlisting}
#include <stdio.h>

sighandler_t signal(int signum, sighandler_t handler);
\end{lstlisting}

où le type \verb+sighandler_t+ désigne les fonctions qui prennent
comme paramètre un \texttt{int} :

\begin{lstlisting}
  typedef void (*sighandler_t)(int);
\end{lstlisting}

La déclaration ``standard'' est un peu plus difficile à lire :
\begin{lstlisting}
  void ( *signal(int signum, void (*handler)(int)) ) (int);
\end{lstlisting}


\paragraph{Rôle : } \texttt{signal()} demande au système de lancer la fonction
\texttt{handler} lorsque le signal \texttt{signum} est reçu par le processus
courant.  La fonction \texttt{signal()} renvoie la fonction qui était
précédemment associée au même signal. 


Il y a une trentaine de signaux différents\footnote{La liste
complète des signaux, leur signification et leur comportement sont
décrits dans la page de manuel \texttt{signal} (chapitre 7 pour
Linux)},
parmi lesquels
\begin{itemize}
\item  \texttt{SIGINT} (program interrupt, émis par Ctrl-C), 
\item  \texttt{SIGTST} (terminal stop, émis par Ctrl-Z)
\item  \texttt{SIGTERM} (demande de fin de processus)
\item  \texttt{SIGKILL} (arrêt immédiat de processus)
\item  \texttt{SIGFPE} (erreur arithmétique),
\item  \texttt{SIGALRM} (fin de délai, voir fonction \texttt{alarm()}), etc.
\end{itemize}

La fonction \emph{handler()} prend en paramètre le numéro
du signal reçu, et ne renvoie rien. 


\textbf{Exemple : }

\source
\lstinputlisting{../PROGS/Signaux/sig-unix.c}


\subsection{\texttt{kill()} }

\index{kill()}
\index{signal unix!kill()}

\extrait
\begin{lstlisting}
#include <unistd.h>
int kill(pid_t pid, int sig);
\end{lstlisting}


La fonction \texttt{kill()} envoie un signal à un processus.

\subsection{\texttt{alarm()} }


\index{alarm()}
\index{signal unix!alarm()}


\extrait
\begin{lstlisting}
#include <unistd.h>

long alarm(long delai);
\end{lstlisting}


La fonction \texttt{alarm()} demande au système d'envoyer un signal
\texttt{SIGALRM} au processus dans un délai fixé (en secondes). Si 
une alarme était déjà positionnée, elle est remplacée. Un délai
nul supprime l'alarme existante.


\subsection{\texttt{pause()} }

\index{pause()}
\index{signal unix!pause()}

La fonction \texttt{pause()} bloque le processus courant jusqu'à ce qu'il
reçoive un signal.


\extrait
\begin{lstlisting}
#include<unistd.h>

int pause(void);
\end{lstlisting}


\textbf{Exercice} : Écrire une fonction équivalente à \texttt{sleep()}. 


\section{Les signaux Posix}

\index{signal POSIX}

Le comportement des signaux classiques d'UNIX est
 malheureusement différent d'une version à l'autre.
On emploie donc de préférence les mécanismes définis par
la norme POSIX, qui offrent de plus la possibilité de masquer des signaux.


\subsection{Manipulation des ensembles de signaux}

Le type \texttt{sigset\_t} représente les ensembles de signaux.


\extrait
\begin{lstlisting}
#include <signal.h>

int sigemptyset(sigset_t *set);
int sigfillset(sigset_t *set);
int sigaddset(sigset_t *set, int signum);
int sigdelset(sigset_t *set, int signum);
int sigismember(const sigset_t *set, int signum);
\end{lstlisting}



La fonction \texttt{sigemptyset()} crée un ensemble vide,
\texttt{sigaddset()} ajoute un élément, etc.

\subsection{\texttt{sigaction()} }


\index{signal POSIX sigaction()}
\index{sigaction()}

\extrait
\begin{lstlisting}
#include <signal.h>

int sigaction(int signum,  
              const  struct  sigaction  *act,
              struct sigaction *oldact);
              \end{lstlisting}


La fonction \texttt{sigaction()} change l'action qui sera
exécutée lors de la réception d'un signal. Cette action
est décrite par une structure \texttt{struct sigaction}


\extrait
\begin{lstlisting}
struct sigaction {
  void (*sa_handler)(int);
  void (*sa_sigaction)(int, siginfo_t *, void *);
  sigset_t sa_mask;
  int sa_flags;
  void (*sa_restorer)(void); /* non utilisé */
}            
\end{lstlisting}


\begin{itemize}
\item \texttt{sa\_handler} indique l'action associée au signal
\texttt{signum}. Il peut valoir
\texttt{SIG\_DFL} (action par défaut),
\texttt{SIG\_IGN} (ignorer),
ou un pointeur vers une fonction de traitement de lu signal.

\item 
le masque \texttt{sa\_mask} indique l'ensemble de signaux qui seront bloqués
pendant l'exécution de ce signal. Le signal lui-même sera bloqué, sauf
si \texttt{SA\_NODEFER} ou \texttt{SA\_NOMASK} figurent parmi les \emph{flags}.
\end{itemize}

Le champ \texttt{sa\_flags} contient une combinaison d'indicateurs, 
parmi lesquels
\begin{itemize}
\item \texttt{SA\_NOCLDSTOP} pour le signal \texttt{SIGCHLD}, ne pas 
recevoir la notification d'arrêt des processus fils 
(quand les processus fils reçoivent
\texttt{SIGSTOP},
\texttt{SIGTSTP}, \texttt{SIGTTIN} ou \texttt{SIGTTOU}).

\item \texttt{SA\_ONESHOT} ou \texttt{SA\_RESETHAND}
remet l'action par défaut quand le handler a été appelé
(c'est le comportement par défaut du  \texttt{signal()} classique).

\item \texttt{SA\_SIGINFO} indique qu'il faut utiliser la fonction
\texttt{sa\_sigaction()} à trois paramètres à la place de
\texttt{sa\_handler()}.
\end{itemize}

\textbf{Exemple : }

\source
\lstinputlisting{../PROGS/Signaux/sig-posix.c}
