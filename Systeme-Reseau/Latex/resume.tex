Ce document est un support de cours pour les enseignements de Système
et de Réseau. Il présente quelques appels système Unix nécessaires à la
réalisation d'applications communicantes. Une première partie rappelle
les notions de base indispensables à la programmation en C :
\texttt{printf}, \texttt{scanf}, \texttt{exit}, communication avec
l'environnement, allocation dynamique, gestion des erreurs.

Ensuite on présente de façon plus détaillées les 
entrées-sorties générales d'UNIX : fichiers, tuyaux, répertoires etc.,
ainsi que la communication inter-processus par le mécanisme des
sockets locaux par flots et datagrammes.

Viennent ensuite les processus  et les signaux. Les mécanismes
associés aux \emph{threads  Posix} sont détaillés : sémaphores,
verrous, conditions. Une autre partie décrit les IPC, que l'on trouve
plus couramment sur les divers UNIX : segments partagés
sémaphores et files de messages. La dernière partie aborde
la communication réseau par l'interface des \emph{sockets},
et montre des exemples d'applications client-serveur avec TCP et UDP.
