%rubber: module xelatex
\documentclass[10pt,twoside,a4paper,openright]{extreport}

\usepackage[ %  showframe,
  total={17cm,25cm},
  includehead,includefoot,
  centering]{geometry}

\usepackage{fontspec}
\usepackage[french]{babel}

\usepackage{listings}
\usepackage{hyperref}
\usepackage{makeidx}

\usepackage{titling}
\usepackage{textpos}
\usepackage{rotating}

\usepackage{xcolor}
\definecolor{msdarkblue}{RGB}{54,95,145}
\definecolor{msblue}{RGB}{79,129,189}
 
\usepackage{titlesec}

\titleformat{\chapter}[frame]{\normalfont\huge\bfseries}
{\chaptertitlename\ \thechapter}{16pt}{\Huge}

\titleformat*{\section}{\color{msdarkblue}\Large\sffamily\bfseries}
\titleformat*{\subsection}{\color{msblue}\large\sffamily\bfseries}
\titleformat*{\subsubsection}{\color{msblue}\normalsize\sffamily\bfseries}

\titlespacing{\section}{0pt}{*6}{*1.5}
\titlespacing{\subsection}{0pt}{*8}{*1.5}
\titlespacing{\subsubsection}{0pt}{*8}{*1.5}

\usepackage{bera}
\renewcommand{\familydefault}{\sfdefault}

\usepackage{graphicx}

% \usepackage{picins}

% \usepackage{fancyhdr}


\renewcommand{\maketitle}{

\begin{titlepage}
% les dimensions pour les coordonnées de textblock
\setlength{\TPHorizModule}{1mm}
\setlength{\TPVertModule}{1mm}
 
% barre bleue 
 \begin{textblock}{50}(-20,-20)
  \begin{color}{msdarkblue}
    \rule{3cm}{25.5cm}    
  \end{color}
\end{textblock}

% serie
\begin{textblock}{20}(-2,220)
\begin{rotate}{90}
{\huge\bfseries\textcolor{white}{Système - Réseaux}}
\end{rotate}
\end{textblock}
 
 \begin{textblock}{130}(20,30)
 {\noindent\rule{2cm}{2cm}}
 \\[5em]

{
\noindent\Huge\bf \noindent \thetitle } \\[5em]
 
{\noindent\Large\bfseries\sf \thedate} \\[10em]

{\noindent\Large\bfseries\sf Michel Billaud}

 \end{textblock}
\end{titlepage}
\makeatother

\clearpage
}



\title{
  Programmation Système \\
  et Réseau \\ en C sous Unix
}
\author{M. Billaud \url{michel.billaud@laposte.net}}
\date{Juillet 2021}


\newcommand{\extrait}[0]{\lstset{language=C,
    frame=leftline,
    numbers=none}}
\newcommand{\extraitbash}[0]{\lstset{language=bash,
    frame=leftline,
    numbers=none}}
\newcommand{\source}[0]{\lstset{language=C,
    frame=single,
    xleftmargin=1cm,
    numbers=left,stepnumber=5,numberfirstline=true,firstnumber=1}}

% \renewcommand{\familydefault}{\sfdefault}

\makeindex
\pagestyle{headings}
\begin{document} 
\sloppy
\maketitle
\begin{abstract}
Ce document est un support de cours pour les enseignements de Système
et de Réseau. Il présente quelques appels système Unix nécessaires à la
réalisation d'applications communicantes. Une première partie rappelle
les notions de base indispensables à la programmation en C :
\texttt{printf}, \texttt{scanf}, \texttt{exit}, communication avec
l'environnement, allocation dynamique, gestion des erreurs.

Ensuite on présente de façon plus détaillées les 
entrées-sorties générales d'UNIX : fichiers, tuyaux, répertoires etc.,
ainsi que la communication inter-processus par le mécanisme des
sockets locaux par flots et datagrammes.

Viennent ensuite les processus  et les signaux. Les mécanismes
associés aux \emph{threads  Posix} sont détaillés : sémaphores,
verrous, conditions. Une autre partie décrit les IPC, que l'on trouve
plus couramment sur les divers UNIX : segments partagés
sémaphores et files de messages. La dernière partie aborde
la communication réseau par l'interface des \emph{sockets},
et montre des exemples d'applications client-serveur avec TCP et UDP.

\end{abstract}

\newpage 
\tableofcontents

\newpage

\chapter*{Avant-propos}



\subsection*{Objectifs}

Ce document présente quelques appels système utiles à la réalisation
d'application communicantes sous UNIX.

Pour écrire de telles applications il faut savoir faire communiquer de
processus entre eux, que ce soit sur la même machine ou sur des
machines reliées en réseau (Internet par exemple).


Pour cela, on passe par des \emph{appels systèmes} pour demander au
système d'exploitation d'effecter des actions : ouvrir des voies de
communication, expédier des données, créer des processus etc.  On
parle de \emph{programmation système} lorsqu'on utilise explicitement
ces appels sans passer par des bibliothèques ou des modules de haut
niveau qui les encapsulent pour en cacher la complexité (supposée).


Les principaux appels systèmes sont présentés ici, avec des exemples
d'utilisation.\footnote{Attention, ce sont des illustrations des
appels, pas des recommandations sur la bonne manière de les
employer. En particulier, les contrôles de sécurité sur les données
sont très sommaires.}



\subsection*{Copyright, versions}

(c) 1998-2021 Michel Billaud

Ce document peut être reproduit en totalité ou en partie, sans frais,
sous réserve des restrictions suivantes :
\begin{itemize}
\item  cette note de copyright et de permission doit être préservée
dans toutes copies partielles ou totales 
\item  toutes traductions ou
travaux dérivés doivent être approuvés par l'auteur en le prévenant
avant leur distribution 
\item  si vous distribuez une partie de ce
travail, des instructions pour obtenir la version complète doivent
également être fournies
\item de courts extraits peuvent être reproduits sans ces notes de
  permissions.
\end{itemize}

L'auteur décline toute responsabilité vis-à-vis des dommages résultant
de l'utilisation des informations et des programmes qui figurent dans
ce document.


\begin{itemize}
\item Version initiale 1998
\item Révision 2002
\item Révision 2014
\item Révision 2016 (Conformité 11)
\item Révision 2018 (POSIX 2017)
\item Révision 2021 (conformité C17)
\end{itemize}

La dernière version de ce document peut être obtenue depuis la page Web
\url{http://www.mbillaud.fr/}


\newpage


\chapter{Bases de C}



\section{Exemple, compilation}

Un exemple classique de programme écrit en C,
à taper dans un fichier \texttt{hello.c}

\extrait
\lstinputlisting{../PROGS/Divers/hello.c}

On peut le compiler par la commande

\extraitbash
\begin{lstlisting}
$ gcc -std=c18 -Wall -Wextra -pedantic -D_XOPEN_SOURCE=700 hello.c
\end{lstlisting}

\index{Options de compilation!POSIX.1-2017}
\index{Option de compilation!C19}

Avec ces options

\begin{itemize}
  \item le compilateur vérifie la conformité  au dernier standard
    du langage C (C18)
  \item on  bénéficie de la bibliothèque POSIX.1-2017,
\footnote{
  identique à IEEE Standard 1003.1-2017 et
  The Open Group Technical Standard Base Specifications, Issue 7.
  voir \url{http://pubs.opengroup.org/onlinepubs/9699919799}
}
\item un maximum d'avertissements sont affichés.
\end{itemize}






\section{Lecture et affichage}

\subsection{Lecture et écriture standards : \texttt{printf()} et \texttt{scanf()}}

\index{printf(format, valeurs...)}  \index{scanf(format, adresses...)}
\index{Entrees-sorties formattees@Entrées-sorties formattées} \extrait
\begin{lstlisting}
#include <stdio.h>
  
int printf (const char *format, ...);     
int scanf  (const char *format, ...); 
\end{lstlisting}



Ces instructions font des écritures et des lectures \emph{formattées}
sur les flots de sortie et d'entrée standard.  Les spécifications de
format sont décrites dans la page de manuel \texttt{printf(3)}.



\subsection{Lecture et écriture dans une chaîne :  \texttt{sprintf()} et \texttt{sscanf()}}

\index{sprintf(tampon, format, valeurs...)}  \index{sscanf(tampon,
  format, adresses...)}  \extrait
\begin{lstlisting}
#include <stdio.h>
  
int sprintf (      char *str, const char *format, ...);     
int sscanf  (const char *str, const char *format, ...); 
\end{lstlisting}


Similaires aux précédentes, mais les opérations lisent ou écrivent
dans le tampon \texttt{str}. 

\paragraph*{Remarque : } la fonction \texttt{sprintf()} ne connaît pas la
taille du tampon \texttt{str} ; il y a donc un risque de
débordement. Il faut prévoir des tampons assez larges, ou (mieux)
utiliser la fonction \texttt{snprintf()} :

\index{snprintf(tampon, taille, format, valeurs...}
\extrait
\begin{lstlisting}
#include <stdio.h>
int snprintf (char *str, size_t size, const char *format, ...);     
\end{lstlisting}

qui permet d'indiquer un nombre d'octets à ne pas dépasser.


\subsection{Lancement d'une commande : \texttt{system()}}

\index{system(chaine)}
\extrait
\begin{lstlisting}
#include <stdlib.h>

int system (const char * string);
\end{lstlisting}
permet de lancer une ligne de commande (\emph{shell}) depuis un
programme.  L'entier retourné par la fonction \texttt{system()} est le
\emph{code de retour} fourni en paramètre à \texttt{exit()} par la
commande.


\paragraph*{Exemple} :


\source
%\lstinputlisting{../PROGS/Divers/envoifichier.c}
\lstinputlisting{../PROGS/Divers/imprimer.c}


\section{Communication avec l'environnement}

\subsection{Paramètres de \texttt{main()}}

\index{argc}
\index{argv}
Le lancement d'un programme C provoque
l'appel de sa fonction principale \texttt{main()}.
Le standard C autorise deux formes pour la déclaration
de \texttt{main()} :

 
\index{main!main(void)}
\index{main!main(argc,argv)}
\extrait
\begin{lstlisting}
int main(void);
int main(int argc, char *argv[]);   
\end{lstlisting}

\begin{itemize}
\item \texttt{argc}  est le nombre de paramètres sur la ligne de commande
(y compris le nom de l'exécutable lui-même) ;
\item \texttt{argv} est une tableau de chaînes contenant les
  paramètres de la ligne de commande.
  \item les noms \texttt{argc, argv} sont purement conventionnels.
  \item déclaration équivalente pour \texttt{argv}  : \verb+char **argv+
\end{itemize}

\paragraph*{Exemple :} programme qui affiche le tableau \texttt{argv} :


\source
\lstinputlisting{../PROGS/Divers/env.c}



\subsection{\texttt{getopt()} : analyse des paramètres de la ligne de commande}

La fonction \texttt{getopt} facilite l'analyse des options d'une ligne de commande. On lui fournit :

\begin{itemize}
\item   le tableau des  paramètres \texttt{argv} et sa taille
\texttt{argc}
\item une
\emph{chaîne de spécification d'options}.  Par exemple la chaîne
\texttt{"hxa:"} déclare 3 options possibles, la dernière (a) devra être
suivies d'un paramètre.
\end{itemize}

\index{getopt}

\extrait
\begin{lstlisting}
#include <unistd.h>

int getopt(int argc, char *const argv[], const char *optstring);

extern char *optarg;
extern int  optind, opterr, optopt;
\end{lstlisting}


À chaque étape, \texttt{getopt()} retourne le nom d'une l'option (ou un
point d'interrogation pour une option non reconnue), et fournit
éventuellement dans \texttt{optarg} la valeur du paramètre associé.

À la fin de l'analyse, \texttt{getopt()} retourne \texttt{-1}, et le
tableau \texttt{argv} a été réarrangé pour que les paramètres
supplémentaires (non liés aux options) soient stockés à
partir de l'indice \texttt{optind}.


\source
\lstinputlisting{../PROGS/Divers/essai-getopt.c}


\paragraph*{Exemple.}

\extraitbash
\begin{lstlisting}
$ essai-getopt -a un deux trois -x quatre
= option `-x' activée
= paramètre `-a' présent = un
3 paramètres supplémentaires
    ->  deux
    ->  trois
    ->  quatre
\end{lstlisting}



\subsection{Variables d'environnement}

\index{getenv(), variables d'environnement} \index{variables
  d'environnement!getenv()}
La fonction \texttt{getenv()} permet de
consulter les variables d'environnement : \extrait
\begin{lstlisting}
#include <stdlib.h>

char *getenv(const char *name);
\end{lstlisting}


\paragraph*{Exemple} :

\source
\lstinputlisting{../PROGS/Divers/getlang.c}



\paragraph*{Exercice : } Ecrire un programme \texttt{exoenv.c} qui affiche les 
valeurs des variables d'environnement indiquées.
Exemple d'exécution:

\extrait
\begin{lstlisting}
$ exoenv TERM LOGNAME PWD
TERM=xterm
LOGNAME=billaud
PWD=/net/profs/billaud/essais
$
\end{lstlisting}

\paragraph*{Voir aussi} les fonctions 


\index{setenv(), variables d'environnement}
\index{variables d'environnement!setenv()}

\index{putenv(), variables d'environnement}
\index{variables d'environnement!putenv()}


\index{unsetenv(), variables d'environnement}
\index{variables d'environnement!unsetenv()}
\extrait
\begin{lstlisting}
#include <stdlib.h>
  
int  putenv  (const char *string);
int  setenv  (const char *name, const char *value, int overwrite);
void unsetenv(const char *name);
\end{lstlisting}

qui permettent de modifier les variables d'environnement
du processus courant et de ses fils.

\paragraph*{Exercice} vérifiez que ça ne modifie pas l'environnement
du processus père.

\subsection{\texttt{exit()} : Fin de programme}

Pour arrêter un programme, deux solutions simples :
\begin{itemize}
\item soit par un \texttt{return} dans la fonction \texttt{main()}
\item soit par un appel à la fonction \texttt{exit()}
\end{itemize}


\index{exit(status)}
\extrait
\begin{lstlisting}
#include <stdlib.h>
  
void exit(int status);
\end{lstlisting}


Le paramètre \texttt{status} est le \emph{code de retour} du processus.
On utilisera de préférence les deux constantes 
\texttt{EXIT\_SUCCESS} et \texttt{EXIT\_FAILURE} qui sont définies dans
\texttt{stdlib.h}.

\section{Erreurs}


\subsection{Variable \texttt{errno}, fonction \texttt{perror()} }

\index{errno}
\index{perror(message)}
\index{erreurs!errno}
\index{erreurs!perror(message)}

La plupart des fonctions du système peuvent échouer pour diverses
raisons. Habituellement, elles le signalent en retournant une valeur
spéciale.  On peut alors examiner la variable globale \texttt{errno}
pour déterminer plus précisément la cause de l'échec, et agir en
conséquence.

\extrait
\begin{lstlisting}
#include <stdio.h>

void perror(const char *s);

#include <errno.h>

extern int errno;
\end{lstlisting}


La fonction \texttt{perror()} imprime sur la sortie
d'erreur standard un message qui décrit la dernière erreur qui s'est
produite, précédé par la chaîne \texttt{s}.


\extrait
\begin{lstlisting}
#include <stdio.h>
void perror(const char *s);
\end{lstlisting}

\index{strerror(errnum)}
\index{erreurs!strerror(errnum)}

Enfin, la fonction \texttt{strerror()} retourne le texte (en anglais)
du message d'erreur correspondant à un numéro.


\extrait
\begin{lstlisting}
#include <string.h>

char *strerror(int errnum);
\end{lstlisting}


\index{chmod()!exemple}

\paragraph*{Exemple} : programme qui change les droits d'accès à des fichiers
grâce à l'appel système \emph{chmod(2)}.


\source
\lstinputlisting{../PROGS/Divers/droits.c}


\subsection{Traitement des erreurs, branchements non locaux }

\index{erreurs!setjmp()}
\index{erreurs!longjmp()}
\index{setjmp()}
\index{longjmp}

\extrait
\begin{lstlisting}
#include <setjmp.h>

int   setjmp(jmp_buf env);
void longjmp(jmp_buf env, int val);
\end{lstlisting}


Ces deux fonctions permettent de réaliser un \emph{branchement} d'une
fonction à une autre (la première doit avoir été appelée, au moins
indirectement, par la seconde). C'est un moyen primitif de réaliser un
semblant de traitement d'erreurs par \emph{exceptions}. À employer
avec précaution.

La fonction \texttt{setjmp()} sauve l'environnement (contexte d'exécution)
dans la variable tampon \texttt{env}, et retourne 0 si elle a été appelée
directement. 

La fonction \texttt{longjmp()} rétablit le dernier environnement qui a
été sauvé dans \texttt{env}. Le programme continue à l'endroit du
\texttt{setjmp()} comme si celui-ci avait retourné la valeur
\texttt{val}.  (Si le paramètre \texttt{val} à 0, la valeur retournée
est 1).

\paragraph*{Exemple} :


\source
\lstinputlisting{../PROGS/Divers/jump.c}



\section{Allocation dynamique}

\index{allocation dynamique!malloc(taille)}
\index{allocation dynamique!free(pointeur)}
\index{allocation dynamique!realloc(pointeur,taille}
\index{malloc(taille)}
\index{free(pointeur}
\index{realloc(pointeur,taille)}

\extrait
\begin{lstlisting}
#include <stdlib.h>

void *malloc(size_t size);
void  free(void *ptr);
void *realloc(void *ptr, size_t size);
\end{lstlisting}


\begin{itemize}
\item \texttt{malloc()} - memory allocation -  demande au système d'exploitation 
l'attribution d'un espace mémoire
de taille supérieure ou égale à \texttt{size} octets.
La valeur retournée est un pointeur sur cet espace
(\texttt{NULL} en cas d'échec). 
\item \texttt{free()} restitue cet espace au système. 
\item \texttt{realloc()} permet de redimensionner la
zone allouée en conservant son contenu.
\end{itemize}


\paragraph*{Exemple} : la fonction \texttt{lire\_nouvelle\_ligne()}
ci-dessous lit une ligne de l'entrée standard et retourne cette ligne
dans un tampon d'une taille suffisante. Elle renvoie le pointeur
\texttt{NULL} si il n'y a plus de place en mémoire.


\source
\lstinputlisting{../PROGS/Divers/lireligne.c}


Attention : dans l'exemple ci-dessus, la fonction
\texttt{lire\_nouvelle\_ligne()} alloue un nouveau tampon à chaque
invocation. Il est donc de la responsabilité du programmeur de libérer
ce tampon après usage pour éviter les \emph{fuites mémoire}. C'est ce
que fait l'appel de \texttt{free()} dans la fonction \texttt{main()}.

\paragraph{Copie de chaîne :}


\index{strdup(chaine)}
\index{chaine@chaîne!duplication, strdup(chaine)}

Il est très fréquent de devoir allouer une zone mémoire pour y loger
une copie d'une chaîne de caractères. On utilise pour cela la fonction
\texttt{strdup()}, qui ne fait pas pour l'instant partie des
bibliothèques standards du langage C lui-même, mais (ouf !) de la
bibliothèque POSIX. Elle apparaîtra dans C23.

\extrait
\begin{lstlisting}
#include <string.h>
  
char *strdup (const char *s);
\end{lstlisting}


\index{chaine@chaîne!copie, strcpy(dest, src)}

Dans un environnement non-POSIX, la fonction peut être redéfinie aisément :


\source
\begin{lstlisting}
char *strdup (const char *s)
{
  char *new_string = malloc(strlen(s) + 1);
  return strcpy(new_string, s);
}
\end{lstlisting}



\chapter{Fichiers et tuyaux}

\section{Manipulation des fichiers, opérations de haut niveau}


\subsection{Flots standards, entrées et sorties sur la console}

\index{flots d'entree-sortie@flots d'entrée-sortie}
\index{stdin, entrée standard}
\index{stdout, sortie standard}
\index{stderr, sortie d'erreur}
\index{FILE*}

Quand un programme est lancé, il y a trois \emph{flots} pré-déclarés et
ouverts, qui correspondent à l'entrée et la sortie standards, ainsi
qu'à la sortie d'erreur :

\extrait
\begin{lstlisting}
#include <stdio.h>

FILE *stdin;
FILE *stdout;
FILE *stderr;
\end{lstlisting}


Vous avez déjà rencontré quelques fonctions qui agissent sur ces
flots, implicitement, sans les nommer en paramètre\footnote{Elles
  correspondent à des de fonctions plus générales \texttt{fprintf()},
  \texttt{fscanf()}, etc.  que nous verrons plus loin.}

\index{getchar()}
\index{printf()}
\index{scanf()}
\extrait
\begin{lstlisting}
int  printf(const char *format, ...);     
int  scanf(const char *format, ...);      
int  getchar(void);                       
\end{lstlisting}

\begin{itemize}
\item \texttt{printf()} écrit sur \texttt{stdout} la valeur d'expressions
  selon un format donné.
  \item \texttt{scanf()} lit sur \texttt{stdin} la
    valeur de variables.
  \item pour lire une ligne complète, on fait appel à \texttt{fgets()} en
    utilisant le flot \texttt{stdin}, voir plus loin. 
\end{itemize}


\source
\lstinputlisting{../PROGS/Divers/facture.c}


\begin{itemize}
  \item
\texttt{scanf()} renvoie le nombre d'objets qui ont pu être effectivement
lus sans erreur. 

%% \texttt{gets()} lit des caractères jusqu'à une marque de fin de ligne ou de 
%% fin de fichier, et les place (marque non comprise) dans le tampon
%% donné en paramètre, suivis par un caractère NUL. \texttt{gets()} renvoie
%% finalement l'adresse du tampon. Attention, prévoir un tampon assez grand,
%% ou (beaucoup mieux) utilisez \texttt{fgets()}. 

\item
\texttt{getchar()} lit un caractère sur l'entrée standard et retourne sa 
valeur sous forme d'entier positif, ou la constante \texttt{EOF} (= -1) 
en fin de fichier.
\end{itemize}

\subsection{Opérations sur les flots}

\index{fopen(chemin,mode)}
\index{fclose(fichier)}
\index{fprintf(fichier,format,valeurs...)}
\index{fscanf(fichier,format,adresses...)}
\index{fgetc(fichier)}


\extrait
\begin{lstlisting}
#include <stdio.h>
  
FILE *fopen (char *path, char *mode);
int   fclose  (FILE *stream);
\end{lstlisting}

\begin{itemize}
  \item 
\texttt{fopen()} tente d'ouvrir le fichier désigné par la chaîne \texttt{path}
selon le mode indiqué, qui peut être 
\begin{itemize}
\item \texttt{"r"} (lecture seulement),
\item \texttt{"r+"} (lecture et écriture),
\item \texttt{"w"} (écriture seulement),
\item \texttt{"w+"} (lecture et écriture, effacement si le fichier existe déjà),
\item \texttt{"a"} (écriture à partir de la fin du fichier si il existe déjà),
\item \texttt{"a+"} (lecture et écriture, positionnement à la fin du fichier
si il existe déjà).
\end{itemize}

Si l'ouverture échoue, \texttt{fopen()} retourne le pointeur \texttt{NULL}.
\end{itemize}

\extrait
\begin{lstlisting}
int fprintf(FILE *stream, const char *format, ...);
int fscanf (FILE *stream, const char *format, ...);
int fgetc  (FILE *stream);
\end{lstlisting}

Ces fonctions ne diffèrent de \texttt{printf()},
\texttt{scanf()} et \texttt{getchar()} que par le premier paramètre,
qui précise sur quel flot porte l'opération.


\subsection{Lecture d'une ligne : \texttt{fgets} et \texttt{getline}}


\subsubsection{À l'ancienne : \texttt{fgets}}

La fermeture par \texttt{fclose(flot)} provoquera la fermeture de \texttt{fd}
à la fois en entrée et en sortie.

Quand on fait de la programmation réseau, on a parfois besoin de ne
fermer la communication que dans un sens. Dans ce cas, on duplique le
descripteur, et on crée un flot pour chacun :

\index{dup()!utilisation avec fopen()}
\extrait
\begin{lstlisting}]
    int fd = fopen(.....);
    ...
    FILE *entree = fdopen(    fd,  "r");
    FILE *sortie = fdopen(dup(fd), "w");
\end{lstlisting}

La fermeture d'un des deux flots ne clôt qu'un sens de communication.



\subsection{Positionnement}

\index{feof(fichier)}
\index{ftell(fichier)}
\index{fseek(fichier,offset,repere)}

\extrait
\begin{lstlisting}
int  feof (FILE *stream);
long ftell(FILE *stream);
int  fseek(FILE *stream, long offset, int whence);
\end{lstlisting}

\begin{itemize}
  \item 
\texttt{feof()} indique si la fin de fichier est atteinte.
\item \texttt{ftell()} indique la \emph{position courante} dans le fichier
  (0 = début).
  \item \texttt{fseek()} déplace la position courante : si
    \texttt{whence} contient
    \begin{itemize}
      \item \texttt{SEEK\_SET} la position est donnée par
        rapport au début du fichier,
        \item \texttt{SEEK\_CUR} : déplacement
          par rapport à la position courante,
          \item \texttt{SEEK\_END} : déplacement
            par rapport à la fin.
    \end{itemize}
\end{itemize}



\subsection{Divers}

\extrait
\begin{lstlisting}
int   ferror   (FILE *stream);
void  clearerr (FILE *stream);
\end{lstlisting}

\index{ferror(dichier)}
\index{clearerr(fichier)}

\begin{itemize}
\item La fonction \texttt{ferror()} indique si une erreur a eu lieu
  sur un flot,
  \item
    \texttt{clearerr()} efface l'indicateur d'erreur
\end{itemize}


\section{Manipulation des fichiers, opérations de bas niveau}

Pendant l'exécution d'un programme, un certain nombre de fichiers sont
\emph{ouverts} (en cours d'utilisation).  I

Il existe dans le système
une table des fichiers actuellement ouverts par le programme,
les opérations de bas niveau désignent les fichiers par leur indice
dans cette table. On appelle aussi  ce numéro de fichier (\texttt{fileno})
un ``descripteur de fichier'' (\texttt{fd} = file descriptor).

Les numéros 0, 1 et 2 correspondent respectivement à l'entrée standard,
la sortie standard, et la sortie d'erreur. Dans la programmation,
utilisez plutôt les constantes \texttt{STDIN\_FILENO}
\texttt{STDOUT\_FILENO},
\texttt{STDERR\_FILENO} définies dans \texttt{unistd.h}.


Les opérations de bas niveau communiquent en général avec les fichiers
par l'intermédiaire d'un tampon, un tableau d'octets, en indiquant le
nombre d'octets à transmettre. Exemple :

\source
\begin{lstlisting}
char * message = "Hello, world";

write(STDOUT_FILENO, message, 5);
\end{lstlisting}

envoie les 5 premiers caractères du message sur la sortie standard.



\subsection{Ouverture, fermeture, lecture, écriture}

\index{open(chemin,flags,mode)}
  
\extrait
\begin{lstlisting}
#include <sys/types.h>
#include <sys/stat.h>
#include <fcntl.h>
  
int open(const char *pathname, int flags, mode_t mode);
\end{lstlisting}


Ouverture d'un fichier nommé \texttt{pathname}. Les \texttt{flags}
peuvent prendre l'une des valeurs suivantes :
\begin{itemize}
  \item \texttt{O\_RDONLY}
    (lecture seulement),
    \item \texttt{O\_WRONLY} (écriture seulement),
    \item \texttt{O\_RDWR} (lecture et écriture).
\end{itemize}
  Cette valeur peut être
  combinée éventuellement (par un ``ou logique'') avec des options:
  \begin{itemize}
\item \texttt{O\_CREAT} (création du fichier si il n'existe pas déjà),
\item \texttt{O\_TRUNC} (si le fichier existe il sera tronqué),
\item \texttt{O\_APPEND} (chaque écriture se fera à la fin du fichier),
  \item etc. 
  \end{itemize}

En cas de création d'un nouveau fichier, le \texttt{mode} sert à
préciser les droits d'accès.  Lorsqu'un nouveau fichier est créé,
\texttt{mode} est combiné avec le \texttt{umask} du processus pour
former les droits d'accès du fichier. Les permissions effectives sont
alors \verb/(mode & ~umask)/ 

Le paramètre \texttt{mode}
doit être présent quand les \texttt{flags} contiennent
\texttt{O\_CREAT}.

La fonction \texttt{open()}
retourne  le numéro de \emph{descripteur de fichier} (-1 en cas d'erreur),
un nombre entier qui sert à référencer le fichier par la suite. 

%% Les descripteurs 0, 1 et 2 correspondent aux fichiers standards 
%% \texttt{stdin}, \texttt{stdout} et \texttt{stderr} qui sont
%% normalement déjà ouverts (0=entrée, 1=sortie, 2=erreurs).

\index{close(descripteur)}
\index{read!read(descripteur,tampon,taille)}
\index{write!write(descripteur,tampon,taille)}

\extrait
\begin{lstlisting}
#include <unistd.h>
#include <sys/types.h>

int close(int fd);
int read(int fd, char *buf, size_t count);
size_t write(int fd, const char *buf, size_t count);
\end{lstlisting}


\texttt{close()} ferme le fichier indiqué par le descripteur
\texttt{fd}.  Retourne 0 en cas de succès, -1 en cas d'échec.

\texttt{read()} demande à lire \emph{au plus} \texttt{count} octets
sur \texttt{fd}, à placer dans le tampon \texttt{buf}. Retourne le
nombre d'octets qui ont été effectivement lus, qui peut être inférieur
à la limite donnée pour cause de non-disponibilité (-1 en cas
d'erreur, 0 en fin de fichier).

\texttt{write()} tente d'écrire sur
le fichier les \texttt{count} premiers octets du tampon
\texttt{buf}. Retourne le nombre d'octets qui ont été effectivement
écrits, -1 en cas d'erreur.

\paragraph*{Exemple} :

\source
\lstinputlisting{../PROGS/Divers/copie.c}


\paragraph*{Problème.} Montrez que la taille du tampon influe sur les
performances des opérations d'entrée-sortie.  Pour cela, modifiez le
programme précédent pour qu'il accepte 3 paramètres : les noms des
fichiers source et destination, et la taille du tampon (ce tampon sera
alloué dynamiquement).


\subsection{Duplication de descripteurs}

\index{descripteur!duplication}
\index{dup(descripteur)}
\index{dup2(existant,nouveau)}
\extrait
\begin{lstlisting}
#include <unistd.h>

int dup  (int oldfd);
int dup2 (int oldfd, int newfd);
\end{lstlisting}


Ces deux fonctions créent une copie du descripteur \texttt{oldfd}.
\texttt{dup()} utilise le plus petit numéro de descripteur libre.
\texttt{dup2()} réutilise le descripteur \texttt{newfd}, en fermant
éventuellement le fichier qui lui était antérieurement associé.

La valeur retournée est celle du descripteur, ou \texttt{-1} en cas d'erreur. 

L'effet sera que le nouveau descripteur désignera la même
fichier que l'ancien.

\paragraph*{Exemple} :

\source
\lstinputlisting{../PROGS/Divers/redirection.c}


\paragraph*{Exercice} : que se produit-il si on essaie de rediriger la 
\emph{sortie} standard d'une commande à la manière de l'exemple précédent ?
(essayer avec ``\texttt{ls}'', ``\texttt{ls -l}'').


\subsection{Positionnement}

\index{lseek(descripteur,position,repère)}
\extrait
\begin{lstlisting}
#include <unistd.h>
#include <sys/types.h>

off_t lseek(int fildes, off_t offset, int whence);
\end{lstlisting}


\texttt{lseek()} repositionne le pointeur de lecture. Similaire à
\texttt{fseek()}. Pour connaître la position courante, faire un appel
à \texttt{stat()}. 

\paragraph*{Exercice. } Écrire un programme pour manipuler un fichier relatif 
d'enregistrements de taille fixe.

\subsection{Verrouillage}

\index{flock(descripteur,opération)}
\extrait
\begin{lstlisting}
#include <sys/file.h>

int flock(int fd, int operation)
\end{lstlisting}


Lorsque \texttt{operation} est \texttt{LOCK\_EX}, il y a verrouillage
du fichier désigné par le descripteur \texttt{fd}. Le fichier est
déverrouillé par l'option \texttt{LOCK\_UN}.

\paragraph*{Problème. } Écrire une fonction \texttt{mutex()}
qui permettra de 
délimiter une section critique dans un programme C. Exemple
d'utilisation :

\extrait
\begin{lstlisting}
#include "mutex.h"
...
mutex("/tmp/foobar",MUTEX_BEGIN);
...
mutex("/tmp/foobar",MUTEX_END);

\end{lstlisting}

Le premier paramètre indique le nom du fichier utilisé comme
verrou. Le second précise si il s'agit de verrouiller ou
déverrouiller. Faut-il prévoir des options \texttt{MUTEX\_CREATE},
\texttt{MUTEX\_DELETE} ? Qu'arrive-t'il si un programme se termine en
``oubliant'' de fermer des sections critiques ? 

Fournir le fichier
d'interface \texttt{mutex.h}, l'implémentation \texttt{mutex.c}, et
des programmes illustrant l'utilisation de cette fonction.

 
\subsection{\texttt{mmap()} : fichiers "mappés" en mémoire}

Un fichier "mappé en mémoire" apparaît comme un tableau d'octets, ce
qui permet de le parcourir en tous sens plus commodément qu'avec des
\texttt{seek()}, \texttt{read()} et \texttt{write()}. 

C'est beaucoup plus économique que de copier le fichier dans
une zone allouée en mémoire : c'est le système
de mémoire virtuelle qui s'occupe de lire et écrire physiquement les
pages du fichier au moment où on tente d'y accéder, et gère tous les
tampons.


\index{mmap()}
\index{munmap()}  
  

\extrait
\begin{lstlisting}
#include <unistd.h>
#include <sys/mman.h>

void * mmap (void  *start,  size_t length, 
             int prot , int flags, 
             int fd, off_t offset);
int munmap  (void *start, size_t length);
\end{lstlisting}


La fonction \texttt{mmap()} "mappe" en mémoire un morceau (de longueur
\texttt{length}, en partant du \texttt{offset}-ième octet) du
fichier désigné par le descripteur \texttt{fd} , et retourne
un pointeur sur la zone de mémoire correspondante.

On peut définir quelques options  (protection en lecture seule, 
partage, etc)
grâce à \texttt{prot} et \texttt{flags}. Voir pages de manuel.
\texttt{munmap()} "libère" la mémoire.



\source
\lstinputlisting{../PROGS/Divers/inverse.c}


\section{Fichiers, répertoires etc.}

Ici ``fichier'' est compris dans son sens large (élément d'un système
de fichiers), qui inclue aussi les répertoires, les périphériques, les
tuyaux et sockets etc. (voir plus loin).


\subsection{Suppression}

\index{remove(chemin)}
  
\extrait
\begin{lstlisting}
#include <stdio.h>

int remove(const char *pathname);
\end{lstlisting}


Cette fonction supprime le fichier \texttt{pathname}, et retourne 0 en
cas de succès (-1 sinon).

\paragraph*{Exercice: } écrire un substitut pour la commande \emph{rm}.

\index{stat(chemin,adressetampon)}
\index{fstat(descripteur,adressetampon)}
\extrait
\begin{lstlisting}
#include <sys/stat.h>
#include <unistd.h>

int stat(const char *file_name, struct stat *buf);
int fstat(int filedes, struct stat *buf);
\end{lstlisting}

\subsection{Informations sur les fichiers/répertoires/...}


Ces fonctions retournent diverses informations sur un
fichier désigné par un chemin d'accès (\texttt{stat()}) ou par un 
descripteur (\texttt{fstat()}). 

\paragraph*{Exemple} :

\source
\lstinputlisting{../PROGS/Divers/taille.c}


\subsection{Parcours de répertoires}

Le parcours d'un répertoire, pour obtenir la liste des fichiers et
répertoires qu'il contient, se fait grâce aux fonctions:

\index{repertoire@répertoire!parcours}
\index{opendir(chemin)}
\index{closedir(DIR *dir)}
\index{rewinddir(DIR *dir)}
\index{seekdir(DIR *dir, position)}
\index{telldir(DIR *dir)}

\extrait
\begin{lstlisting}
#include <sys/types.h>
#include <dirent.h>
  
DIR *opendir   (const char *name);
int  closedir  (DIR *dir);
void rewinddir (DIR *dir);
void seekdir   (DIR *dir, off_t offset);
off_t telldir  (DIR *dir);
\end{lstlisting}


Voir la documentation pour des exemples. 

\paragraph*{Exercice : } écrire une version  simplifiée de la
commande \texttt{ls}.

\paragraph*{Exercice : } écrire une commande qui fasse apparaître la 
structure d'une arborescence. Exemple d'affichage :

\extrait
\begin{lstlisting}
C++
| CompiSep
| Fichiers
Systeme
| Semaphores
| Pipes
| Poly
  | SVGD
  | Essais
| Fifos
\end{lstlisting}

Conseil: écrire une fonction à deux paramètres: le chemin d'accès du 
répertoire et le niveau de récursion.


\section{Tuyaux de communication }

\index{tuyau} Les \emph{tuyaux de communication} (\emph{pipes})
permettent de faire communiquer des processus qui s'exécutent d'une
même machine. Une fois ouverts, ils sont accessibles comme les fichiers
à travers un ``file descriptor''. Ce que les processus y écrivent peut
être lu immédiatement par d'autres processus.

Nous en présentons deux variétés :

\begin{itemize}
\item les tuyaux ``simples'' qui sont créés par un processus
  et servent à  la communication entre ses descendants et lui.
\item \emph{tuyaux nommés},
  qui sont visibles (comme les fichiers et répertoires),
  dans le système de fichiers. Ils  peuvent donc être partagés
par des programmes indépendants.
\end{itemize}

Nous verrons plus loin (\ref{sockets}) une forme de communication plus
générale, les sockets.


Nous commençons par les tuyaux nommés, dont l'usage est proche des fichiers
ordinaires.
  

\subsection{Tuyaux nommés (FIFO)}

\index{tuyau!nommé}
\index{FIFO!tuyau nommé}

Les ``tuyaux nommés'' visibles dans l'arborescence des fichiers et
répertoires. Ils sont créés par \texttt{mkfifo()}, et utilisés ensuite
par
\texttt{open()},
\texttt{read()},
\texttt{write()},
\texttt{close()},
\texttt{fdopen()}, etc.


\index{mkfifo(chemin,mode)}

\extrait
\begin{lstlisting}
#include <stdio.h>

int mkfifo (const char *path, mode_t mode);
\end{lstlisting}


La fonction \texttt{mkfifo()} crée un FIFO ayant le chemin indiqué par
\texttt{path} et les droits d'accès donnés par \texttt{mode}. Si la
création réussit, la fonction renvoie 0, sinon -1. Exemple:

\extrait
\begin{lstlisting}
if (mkfifo("/tmp/fifo.courrier", 0644) != 0) {
  perror("mkfifo");
}
\end{lstlisting}


\paragraph*{Exercice.} Regarder ce qui se passe quand
\begin{itemize}
\item plusieurs processus écrivent dans une même FIFO (faire une boucle
\texttt{sleep-write}).
\item plusieurs processus lisent la même FIFO.
\end{itemize}

\paragraph*{Exercice. } Écrire une commande \emph{mutex} qui permettra de 
délimiter une section critique dans des shell-scripts. Exemple
d'utilisation :

\extrait
\begin{lstlisting}
mutex -b /tmp/foobar
...
mutex -e /tmp/foobar
\end{lstlisting}

Le premier paramètre indique si il s'agit de verrouiller (\texttt{-b}
= begin) ou de déverrouiller (\texttt{-e} = end). Le second paramètre
est le nom du verrou.

\emph{Conseil} : la première option peut s'obtenir en tentant de lire
une information quelconque (\emph{jeton}) dans une FIFO. C'est la
seconde option qui dépose le jeton, ce qui débloquera le processus
lecteur.  Prévoir une option pour créer une FIFO ?


\subsection{Tuyaux (pipe)}

On utilise un \emph{pipe} (tuyau) pour faire communiquer un processus
et un de ses descendants\footnote{ou des descendants - au sens large -
du processus qui a créé le tuyau}.

\index{tuyau!pipe(fd[2])}
\index{pipe(fd[2])}

\extrait
\begin{lstlisting}
#include <unistd.h>

int pipe(int filedes[2]);
\end{lstlisting}


L'appel \texttt{pipe()} fabrique un tuyau de communication et renvoie dans
un tableau une paire de descripteurs.  On lit à un bout du tuyau (sur
le descripteur de sortie \texttt{fildes[0]})
 ce qu'on a écrit dans l'autre (\texttt{filedes[1]}). 
 Voir exemple dans \ref{fork}.


Les \emph{pipes} ne sont pas visibles dans l'arborescence des fichiers et
répertoires, par contre ils sont hérités lors de la création d'un 
processus.

\index{socketpair}
La fonction \texttt{socketpair()} (voir
\ref{socketpair}) généralise la fonction
\texttt{pipe}.

\subsection{Pipes depuis/vers une commande}

\index{tuyau!de/vers commande}
\index{popen(commande,type)}
\index{pclose(fichier)}

\extrait
\begin{lstlisting}
#include <stdio.h>
  
FILE *popen(const char *command, const char *type);
int pclose(FILE *stream);
\end{lstlisting}


\texttt{popen()} lance la commande décrite par la chaîne \texttt{command}
et retourne un flot.


Si \texttt{type} est \texttt{"r"} le flot retourné est celui 
de la sortie standard de la commande (on peut y lire). 
Si \texttt{type} est \texttt{"w"} 
c'est son entrée standard.

\texttt{pclose()} referme ce flot.



\paragraph*{Exemple} :    envoi d'un  ``\texttt{ls -l }'' par courrier 

\source
\lstinputlisting{../PROGS/Divers/avis.c}




\section{\texttt{select()} : attente de données }

Il est assez courant de devoir attendre des données en provenance
de plusieurs sources. On utilise pour cela la fonction \texttt{select()} qui
permet de surveiller plusieurs descripteurs simultanément.

\index{select(...)}

\extrait
\begin{lstlisting}
#include <sys/time.h>
#include <sys/types.h>
#include <unistd.h>

int  select(int n, fd_set *readfds, 
                   fd_set *writefds, 
                   fd_set *exceptfds, 
                   struct timeval *timeout);

FD_CLR   (int fd, fd_set *set);
FD_ISSET (int fd, fd_set *set);
FD_SET   (int fd, fd_set *set);
FD_ZERO  (fd_set *set);      
\end{lstlisting}


Cette fonction attend que des données soient prêtes à être lues sur un
des descripteurs de l'ensemble \texttt{readfs}, ou que l'un des
descripteurs de \texttt{writefds} soit prêt à recevoir des écritures,
que des exceptions se produisent (\texttt{exceptfds}), ou encore que
le temps d'attente \texttt{timeout} soit épuisé.  

 
Lorsque
\texttt{select()} se termine, \texttt{readfds}, \texttt{writefds} et
\texttt{exceptfds} contiennent les descripteurs qui ont changé d'état.
\texttt{select()} retourne le nombre de descripteurs qui ont changé
d'état, ou \texttt{-1} en cas de problème.
 
L'entier \texttt{n} doit être
supérieur (strictement) au plus grand des descripteurs contenus dans
les 3 ensembles (c'est en fait le nombre de bits significatifs du
masque binaire qui représente les ensembles). On peut utiliser la
constante \texttt{FD\_SETSIZE}.
 
Les pointeurs sur les ensembles (ou
le délai) peuvent être \texttt{NULL}, ils représentent alors des
ensembles vides (ou une absence de limite de temps).



Les macros \texttt{FD\_CLR, FD\_ISSET, FD\_SET, FD\_ZERO} permettent de 
manipuler les ensembles de descripteurs. 


\subsection{Attente de données provenant de plusieurs sources}

\paragraph*{Exemple} :

\source
\lstinputlisting{../PROGS/Divers/mix.c}


\subsection{Attente de données avec limite de temps}

L'exemple suivant montre comment utiliser la limite de temps dans le
cas (fréquent) d'attente sur un seul descripteur.

\source
\lstinputlisting{../PROGS/SelectFifo/lecteur.c}






\chapter{Communication interprocessus par sockets locaux}


\section{Les sockets}

\label{sockets}

\index{socket!prise}
\index{prise!socket}

Les \emph{sockets} (\emph{prises}) sont une interface générique
pour la communication entre processus, par divers moyens.


On s'en sert pour la communication à travers les réseaux (Internet ou
autres), mais ils sont utilisables également pour la communication
locale entre processus qui s'exécutent sur une même machine (comme les
tuyaux déjà vus, et les files de messages IPC que nous verrons plus
loin).

Dans ce chapitre, nous montrons comment s'en servir pour la
communication locale, la communication sur le réseau sera vue plus
loin.


\subsection{Création d'un socket}

\index{socket!création}

\extrait
\begin{lstlisting}
#include <sys/types.h>
#include <sys/socket.h>

int socket(int domain, int type, int protocol);
\end{lstlisting}

\index{socket!domaine}
\index{socket!type}
\index{socket!type!local}
\index{socket!type!internet}

\index{socket!protocole}

La fonction \texttt{socket()} crée une nouvelle prise et retourne un
descripteur qui servira ensuite aux lectures et écritures. Le
paramètre \texttt{domain} indique le domaine de communication''
utilisé, qui est \texttt{PF\_LOCAL} ou (synonyme) \texttt{PF\_UNIX} pour
les communications locales.\footnote{ Le domaine définit une famille
de protocoles (protocol family) utilisables. Autres familles
disponibles : \texttt{PF\_INET} protocoles internet IPv4,
\texttt{PF\_INET6} protocoles internet IPv6, \texttt{PF\_IPX} protocoles
Novel IPX, \texttt{PF\_X25} protocoles X25 (ITU-T X.25 / ISO-8208),
\texttt{PF\_APPLETALK}, etc.}

\index{socket!protocole!communication par datagramme}
\index{socket!protocole!communication par flot}

Le \emph{type} indique le style de communication désiré
entre les deux participants. Les deux styles principaux sont
\begin{itemize}
\item \texttt{SOCK\_DGRAM} : communication par messages (blocs contenant des
octets) appelés \emph{datagrammes}
\item \texttt{ SOCK\_STREAM} : la communication se fait par un flot 
(bidirectionnel) d'octets une fois que la connexion est établie.
\end{itemize}

\textbf{Fiabilité}: la fiabilité des communication par datagrammes est
garantie pour le domaine local, mais ce n'est pas le cas pour les
``domaines réseau'' que nous verrons plus loin : les datagrammes
peuvent être perdus, dupliqués, arriver dans le désordre etc. et c'est
au programmeur d'application d'en tenir compte. Par contre la
fiabilité des ``streams'' est assurée par les couches basses du
système de communication, évidemment au prix d'un surcoût
(numérotation des paquets, accusés de réception, temporisations,
retransmissions, etc).


Enfin, le paramètre \texttt{protocol} indique le protocole sélectionné.
La valeur \texttt{0} correspond au protocole par défaut pour le domaine
et le type indiqué.

\subsection{Adresses}

\index{socket!adresse!socket local}
La fonction \texttt{socket()} crée un socket anonyme. Pour qu'un autre
processus puisse le désigner, il faut lui associer un \emph{nom} par
l'intermédiaire d'une \emph{adresse} contenue dans une structure
\texttt{sockaddr\_un} :



\extrait
\begin{lstlisting}
#include <sys/un.h>

struct sockaddr_un {
        sa_family_t  sun_family;              /* AF_UNIX */
        char         sun_path[UNIX_PATH_MAX]; /* pathname */
};
\end{lstlisting}

Ces adresses sont des chemins d'accès dans l'arborescence des fichiers et
répertoires.


Exemple :

\extrait
\begin{lstlisting}
#include <sys/types.h>
#include <sys/socket.h>
#include <sys/un.h>

socklen_t longueur_adresse;

struct sockaddr_un  adresse;

adresse.sun_family      = AF_LOCAL;
strcpy(adresse.sun_path, "/tmp/xyz");
longueur_adresse = sizeof adresse;
\end{lstlisting}

\index{socket!bind()}

L'association d'une adresse à un socket se fait par \texttt{bind()}
(voir exemples plus loin).

\section{Communication par datagrammes}

Dans l'exemple développé ici, un serveur affiche les datagrammes
émis par les clients.

\subsection{La réception de datagrammes}

\index{bind(socket,adresse,longueur)}
\index{socket!bind()}
\index{socket!recvfrom()}
\index{recvfrom()}

\extrait
\begin{lstlisting}
#include <sys/types.h>
#include <sys/socket.h>

int  bind(int  sockfd, struct sockaddr *my_addr, 
                       socklen_t addrlen);

int  recvfrom(int  s,  void  *buf,  size_t len, int flags,
              struct sockaddr *from, socklen_t *fromlen);

\end{lstlisting}


La fonction \texttt{bind()} permet de nommer le socket de réception.
La fonction \texttt{recvfrom()} attend l'arrivée d'un datagramme qui
est stocké dans les \texttt{len} premiers octets du tampon
\texttt{buff}.  Si \texttt{from} n'est pas \texttt{NULL}, l'adresse du
socket émetteur\footnote{qui peut servir à expédier une réponse} est
placée dans la structure pointée par \texttt{from}, dont la longueur
maximale est contenue dans l'entier pointé par \texttt{fromlen}.


Si la lecture a réussi, la fonction retourne le nombre d'octets du
message lu, et la longueur de l'adresse est mise à jour.


Le paramètre \texttt{flags} permet de préciser des options.


\source
\lstinputlisting{../PROGS/LocalDatagrammes/serveur-dgram-local.c}


\subsection{Émission de datagrammes}

\index{socket!sendto()}
\index{sendto()}

\extrait
\begin{lstlisting}
#include <sys/types.h>
#include <sys/socket.h>

int  sendto(int s, const void *msg, size_t len, int flags,
            const struct sockaddr *to, socklen_t tolen);
\end{lstlisting}


\texttt{sendto} utilise le descripteur de socket \texttt{s} pour
envoyer le message formé des \texttt{len} premiers octets de
\texttt{msg} à l'adresse de longueur \texttt{tolen} pointée par
\texttt{to}.


Le même descripteur peut être utilisé pour des envois à des
adresses différentes.


\source
\lstinputlisting{../PROGS/LocalDatagrammes/client-dgram-local.c}


\textbf{Exercice} : modifier les programmes précédents pour que le
serveur envoie une réponse qui sera affichée par le client\footnote{
Pensez à attribuer un nom au socket du client pour que le serveur puisse lui
répondre, par exemple avec l'aide de
la fonction \texttt{tempnam()}.}. 


\subsection{Émission et réception en mode connecté}

\index{socket!mode connecté}
\index{connect(socket,addresse,longueur)}
\index{send(socket,tampon,longueur,flags)}
\index{recv(socket,tampon,longueur,flags)}

\extrait
\begin{lstlisting}
#include <sys/types.h>
#include <sys/socket.h>

int  connect(int sockfd, 
             const struct sockaddr *serv_addr,
             socklen_t addrlen);

int send(int s, const void *msg, size_t len, int flags);
int recv(int s, void *buf, size_t len, int flags);
\end{lstlisting}


Un émetteur qui va envoyer une série de messages au même destinataire
par le même socket peut faire préalablement un \texttt{connect()} pour
indiquer une destination par défaut, et employer ensuite
\texttt{send()} à la place de \texttt{sendto()}.


Le récepteur qui ne s'intéresse pas à l'adresse de l'envoyeur peut
utiliser \texttt{recv()}.  


\textbf{Exercice} : modifier les programmes précédents pour utiliser 
\texttt{recv()} et \texttt{send()}.

\section{Communication par flots}

Dans ce type de communication, c'est une suite d'octets qui est
transmises (et non pas une suite de messages comme dans la
communication par datagrammes). 


Les sockets locaux de ce type sont créés par 

\extrait
\begin{lstlisting}
int fd = socket(PF_LOCAL,SOCK_STREAM,0);
\end{lstlisting}


\section{Architecture client-serveur}

La plupart des applications communicantes sont conçues selon
une architecture client-serveur, asymétrique, 
dans laquelle un processus serveur est contacté par plusieurs
clients.


\index{socket!read(socket,tampon,taille)}
\index{socket!write(socket,tampon,taille)}
\index{read!read(socket,tampon,taille)}
\index{write!write(socket,tampon,taille)}

Le client crée un socket (\texttt{socket()}), qu'il met en relation 
(par \texttt{connect()}) avec celui du serveur. Les données sont
échangées par \texttt{read()}, \texttt{write()} ... et le socket est
fermé par \texttt{close()}. 


\index{connect(socket,adresse,longueur)}
\index{socket!connect(socket,adresse,longueur)}
\extrait
\begin{lstlisting}
#include <sys/types.h>
#include <sys/socket.h>

int  connect(int sockfd, 
             const struct sockaddr *serv_addr,
             socklen_t addrlen);
\end{lstlisting}


\index{listen(socket,nombre)}
\index{socket!listen(socket,nombre)}
Du côté serveur : un socket est créé, et une adresse lui est associée
(\texttt{socket()} + \texttt{bind()}). Un \texttt{listen()} prévient
le système que ce socket recevra des demandes de connexion, et précise
le nombre de connexions que l'on peut mettre en file d'attente.

\index{accept(socket)}
\index{socket!accept(socket)}


Le serveur attend les demandes de connexion par la fonction
\texttt{accept()} qui retourne un descripteur, lequel permet la
communication avec le client.



\extrait
\begin{lstlisting}
#include <sys/types.h>
#include <sys/socket.h>

int  bind(int             sockfd, 
          struct sockaddr *my_addr, 
          socklen_t       addrlen);

int  listen(int s, int backlog);       

int  accept(int s,  
            struct  sockaddr  *addr,  
            socklen_t         *addrlen);
\end{lstlisting}


Remarque : il est possible ne fermer qu'une ``moitié'' de socket :
\texttt{shutdown(1)} met fin aux émissions (causant une ``fin de fichier''
chez le correspondant), \texttt{shutdown(0)} met fin aux réceptions.

\index{shutdown(socket,indic)}
\index{socket!shutdown(socket,indic)}

\extrait
\begin{lstlisting}
#include <sys/socket.h>

int shutdown(int s, int how);
\end{lstlisting}



\textbf{Le client :}

\source
\lstinputlisting{../PROGS/LocalStream/client-stream.c}


\textbf{Le serveur} est programmé ici de façon atypique, puisqu'il traite 
 qu'il traîte une seule communication à la fois. Si le client fait traîner
les choses, les autres clients en attente resteront bloqués longtemps.



\source
\lstinputlisting{../PROGS/LocalStream/serveur-stream.c}


Dans une programmation plus classique, le serveur lance un processus
(par \texttt{fork()}, voir plus loin) dès qu'une connexion est établie, et 
délègue le traitement de la connexion
à ce processus.


Une autre technique est envisageable pour traiter plusieurs connexions par un 
processus unique : le serveur maintient une liste de descripteurs 
ouverts, et fait une boucle autour d'un 
\texttt{select()}, en attente de données venant
\begin{itemize}
\item soit du descripteur ``principal'' ouvert par le serveur. Dans ce cas il
effectue ensuite un \texttt{accept(...)} qui permettra d'ajouter un
nouveau client à la liste.
\item soit d'un des descripteurs des clients, et il traite alors les données
venant de ce client (il l'enlève de la liste en fin de communication).
\end{itemize}
Cette technique conduit à des performances nettement supérieures aux
serveurs multiprocessus ou multithreads (pas de temps perdu à lancer
des processus), au prix d'une programmation qui oblige le programmeur
à gérer lui-même le ``contexte de déroulement'' de chaque processus.



\source
\lstinputlisting{../PROGS/LocalStream/serveur-stream-monotache.c}

\index{socket!socketpair())}
\index{socketpair()}

\section{\texttt{socketpair()}}

\label{socketpair}

La fonction \texttt{socketpair()} construit une paire de sockets
locaux, bi-directionnels,  reliés l'un à l'autre.


\extrait
\begin{lstlisting}
#include <sys/types.h>
#include <sys/socket.h>

int socketpair(int d, int type, int protocol, int sv[2]);
\end{lstlisting}


Dans l'état actuel des implémentations, le paramètre \texttt{d}
(domaine) doit être égal à \texttt{AF\_LOCAL}, et \texttt{type} à
\texttt{SOCK\_DGRAM} ou \texttt{SOCK\_STREAM}, avec le protocole par
défaut (valeur \texttt{0}).


Cette fonction remplit le tableau \texttt{sv[]} avec les descripteurs
de deux sockets du type indiqué. Ces deux sockets sont reliés entre
eux et bidirectionnels : ce qu'on écrit sur le descripteur
\texttt{sv[0]} peut être lu sur \texttt{sv[1]}, et réciproquement.


On utilise \texttt{socketpair()} comme \texttt{pipe()}, pour la
communication entre descendants d'un même processus \footnote{au sens
large, ce qui inclue la communication d'un processus avec un de ses
fils}. \texttt{socketpair()} possède deux avantages sur
\texttt{pipe()} : la possibilité de transmettre des datagrammes, et la
bidirectionnalité.


\textbf{Exemple :}


\source
\lstinputlisting{../PROGS/LocalDatagrammes/paire-send.c}





\chapter{Communication par signaux}

Ce chapitre présente la communication par signaux entre processus.

Il y a essentiellement deux opérations sur les signaux. En gros :

\begin{itemize}
\item un appel de fonction demande l'envoi d'un \emph{signal} à un
  autre processus (destinataire), désigné par son numéro de processus.
\item le destinataire a indiqué, par autre appel de fonction
  (\texttt{signal} ou \texttt{sigaction}) quel traitement
  doit être exécuté quand il reçoit un
  signal.
\end{itemize}

Un signal est un simple nombre, dont la valeur correspond à un
particulier. Il peut être émis par un programme (la fonction d'envoi
s'appelle \texttt{kill()}) (pour des raisons historiques), ou résulter
d'un évènement système (fin d'un processus fils, ...), ou d'une
division par zéro, etc.

En particulier, quand on fait tourner un processus depuis un shell et
qu'on tape controle-c, ce caractère est reçu par le shell qui envoie
alors le signal \texttt{SIGINT} (numéro 2) au processus qui tourne en
avant-plan et rend la main à la boucle interactive du shell. De même,
contrôle-Z envoie un \texttt{SIGTSTP} (20) qui demande au destinataire
de se mettre en pause (terminal stop). Ce qu'il fait en s'envoyant
lui-même le signal \texttt{SIGSTOP} (19).


Un signal a un comportement par défaut (arrêter le
programme, ne rien faire, ...), la seconde fonction
\texttt{signal/sigaction} sert à le changer.

On regarde ici deux bibliothèques liées aux signaux :

\begin{itemize}
\item la bibliothèque des signaux ``classiques'' d'UNIX, qui est
  simple à utiliser, mais n'est pas vraiment portable\footnote{le
    comportement peut être ``légèrement'' différent selon les versions
    d'UNIX.}
\item la bibliothèque définie par la norme POSIX, plus riche mais aussi
  plus complexe.
\end{itemize}

\section{Les signaux Unix}

\index{signal()}
\index{signal unix!signal()}

\subsection{\texttt{signal()}}

L'implémentation GNU se présente sous la forme

\extrait
\begin{lstlisting}
#include <stdio.h>

sighandler_t signal(int signum, sighandler_t handler);
\end{lstlisting}

où le type \verb+sighandler_t+ désigne les fonctions qui prennent
comme paramètre un \texttt{int} :

\begin{lstlisting}
  typedef void (*sighandler_t)(int);
\end{lstlisting}

La déclaration ``standard'' est un peu plus difficile à lire :
\begin{lstlisting}
  void ( *signal(int signum, void (*handler)(int)) ) (int);
\end{lstlisting}


\paragraph{Rôle : } \texttt{signal()} demande au système de lancer la fonction
\texttt{handler} lorsque le signal \texttt{signum} est reçu par le processus
courant.  La fonction \texttt{signal()} renvoie la fonction qui était
précédemment associée au même signal. 


Il y a une trentaine de signaux différents\footnote{La liste
complète des signaux, leur signification et leur comportement sont
décrits dans la page de manuel \texttt{signal} (chapitre 7 pour
Linux)},
parmi lesquels
\begin{itemize}
\item  \texttt{SIGINT} (program interrupt, émis par Ctrl-C), 
\item  \texttt{SIGTST} (terminal stop, émis par Ctrl-Z)
\item  \texttt{SIGTERM} (demande de fin de processus)
\item  \texttt{SIGKILL} (arrêt immédiat de processus)
\item  \texttt{SIGFPE} (erreur arithmétique),
\item  \texttt{SIGALRM} (fin de délai, voir fonction \texttt{alarm()}), etc.
\end{itemize}

La fonction \emph{handler()} prend en paramètre le numéro
du signal reçu, et ne renvoie rien. 


\textbf{Exemple : }

\source
\lstinputlisting{../PROGS/Signaux/sig-unix.c}


\subsection{\texttt{kill()} }

\index{kill()}
\index{signal unix!kill()}

\extrait
\begin{lstlisting}
#include <unistd.h>
int kill(pid_t pid, int sig);
\end{lstlisting}


La fonction \texttt{kill()} envoie un signal à un processus.

\subsection{\texttt{alarm()} }


\index{alarm()}
\index{signal unix!alarm()}


\extrait
\begin{lstlisting}
#include <unistd.h>

long alarm(long delai);
\end{lstlisting}


La fonction \texttt{alarm()} demande au système d'envoyer un signal
\texttt{SIGALRM} au processus dans un délai fixé (en secondes). Si 
une alarme était déjà positionnée, elle est remplacée. Un délai
nul supprime l'alarme existante.


\subsection{\texttt{pause()} }

\index{pause()}
\index{signal unix!pause()}

La fonction \texttt{pause()} bloque le processus courant jusqu'à ce qu'il
reçoive un signal.


\extrait
\begin{lstlisting}
#include<unistd.h>

int pause(void);
\end{lstlisting}


\textbf{Exercice} : Écrire une fonction équivalente à \texttt{sleep()}. 


\section{Les signaux Posix}

\index{signal POSIX}

Le comportement des signaux classiques d'UNIX est
 malheureusement différent d'une version à l'autre.
On emploie donc de préférence les mécanismes définis par
la norme POSIX, qui offrent de plus la possibilité de masquer des signaux.


\subsection{Manipulation des ensembles de signaux}

Le type \texttt{sigset\_t} représente les ensembles de signaux.


\extrait
\begin{lstlisting}
#include <signal.h>

int sigemptyset(sigset_t *set);
int sigfillset(sigset_t *set);
int sigaddset(sigset_t *set, int signum);
int sigdelset(sigset_t *set, int signum);
int sigismember(const sigset_t *set, int signum);
\end{lstlisting}



La fonction \texttt{sigemptyset()} crée un ensemble vide,
\texttt{sigaddset()} ajoute un élément, etc.

\subsection{\texttt{sigaction()} }


\index{signal POSIX sigaction()}
\index{sigaction()}

\extrait
\begin{lstlisting}
#include <signal.h>

int sigaction(int signum,  
              const  struct  sigaction  *act,
              struct sigaction *oldact);
              \end{lstlisting}


La fonction \texttt{sigaction()} change l'action qui sera
exécutée lors de la réception d'un signal. Cette action
est décrite par une structure \texttt{struct sigaction}


\extrait
\begin{lstlisting}
struct sigaction {
  void (*sa_handler)(int);
  void (*sa_sigaction)(int, siginfo_t *, void *);
  sigset_t sa_mask;
  int sa_flags;
  void (*sa_restorer)(void); /* non utilisé */
}            
\end{lstlisting}


\begin{itemize}
\item \texttt{sa\_handler} indique l'action associée au signal
\texttt{signum}. Il peut valoir
\texttt{SIG\_DFL} (action par défaut),
\texttt{SIG\_IGN} (ignorer),
ou un pointeur vers une fonction de traitement de lu signal.

\item 
le masque \texttt{sa\_mask} indique l'ensemble de signaux qui seront bloqués
pendant l'exécution de ce signal. Le signal lui-même sera bloqué, sauf
si \texttt{SA\_NODEFER} ou \texttt{SA\_NOMASK} figurent parmi les \emph{flags}.
\end{itemize}

Le champ \texttt{sa\_flags} contient une combinaison d'indicateurs, 
parmi lesquels
\begin{itemize}
\item \texttt{SA\_NOCLDSTOP} pour le signal \texttt{SIGCHLD}, ne pas 
recevoir la notification d'arrêt des processus fils 
(quand les processus fils reçoivent
\texttt{SIGSTOP},
\texttt{SIGTSTP}, \texttt{SIGTTIN} ou \texttt{SIGTTOU}).

\item \texttt{SA\_ONESHOT} ou \texttt{SA\_RESETHAND}
remet l'action par défaut quand le handler a été appelé
(c'est le comportement par défaut du  \texttt{signal()} classique).

\item \texttt{SA\_SIGINFO} indique qu'il faut utiliser la fonction
\texttt{sa\_sigaction()} à trois paramètres à la place de
\texttt{sa\_handler()}.
\end{itemize}

\textbf{Exemple : }

\source
\lstinputlisting{../PROGS/Signaux/sig-posix.c}


\chapter{Processus lourds et légers}


Dans beaucoup de programmes il est nécessaire d'avoir plusieurs
processus qui tournent simultanément pour réaliser ensemble un certain
travail. Ils collaboreront à travers des tuyaux, du partage de mémoire
et autres techniques décrites ailleurs dans ce document.

Ici nous présentons deux types de processus. Le premier, présent dès
les premières versions d'UNIX, s'obtient essentiellement en dupliquant
le processus qui le lance (contenu de la mémoire, fichiers ouverts,
ressource diverses, etc).  Le processus ``fils'' possède alors un
espace mémoire totalement séparé de celui du ``père qui l'a lancé''.


Cette duplication fait qu'on l'appelle souvent \emph{processus lourd},
par opposition aux \emph{processus légers} introduits plus tard qui,
eux, partagent le même espace mémoire et les mêmes ressources.
Cette ``lourdeur'' est à relativiser, le système met en oeuvre
des techniques comme le copy-on-write qui limitent les dégâts.

Contrairement aux signaux POSIX par rapport aux signaux Unix, on ne
peut pas dire que les processus légers rendent obsolètes les processus
``lourds'' : les deux ont leur utilité, leurs avantages et inconvénients.


\section{Les processus lourds}
\sloppy
\subsection{\texttt{fork()}, \texttt{wait()}}
\label{fork}

\index{processus!fork()}
\index{fork()}
\index{processus!wait()}
\index{wait()}

\extrait
\begin{lstlisting}
#include <unistd.h>
pid_t fork(void);
pid_t wait(int *status)
\end{lstlisting}


La fonction \texttt{fork()} crée un nouveau processus (\emph{fils})
semblable au processus courant (\emph{père}). La valeur renvoyée
n'est pas la même pour le fils (\texttt{0}) et pour le père (numéro de
processus du fils). \texttt{-1} indique un échec.


La fonction \texttt{wait()} attend qu'un des processus fils soit
terminé.  Elle renvoie le numéro du fils, et son \texttt{status} (voir
\texttt{exit()}) en paramètre passé par adresse.


\textbf{Attention. } Le processus fils \emph{hérite} des descripteurs
ouverts de son père. Il convient que chacun des processus ferme
les descripteurs qui ne le concernent pas.

\textbf{Exemple :}



\source
\lstinputlisting{../PROGS/Divers/biproc.c}



\textbf{Exercice :} Observez ce qui se passe si, dans la fonction
\texttt{affiche()}, on remplace l'appel à \texttt{lire()} par un
\texttt{read()} ? Et si on ne fait pas le \texttt{wait()} ?

\subsection{ \texttt{waitpid()} : attente de changement d'état}

\index{processus!waitpid()}
\index{waitpid()}

La fonction \texttt{waitpid()} permet d'attendre un \emph{changement
  d'état} d'un des processus fils désigné par son \emph{pid}
(n'importe lequel si \emph{pid = -1}), et de récupérer éventuellement
son code de retour.


\extrait
\begin{lstlisting}
#include <sys/types.h>
#include <sys/wait.h>

pid_t  waitpid (pid_t pid, int *status, int options);
\end{lstlisting}

L'option \texttt{WNOHANG} rend \texttt{waitpid} non bloquant (qui
retourne alors \texttt{-1} si le processus attendu n'est pas terminé).

\textbf{Exemple :}


\extrait
\begin{lstlisting}
int pid_fils;
int status;

if( (pid_fils = fork()) != 0) {
   code_processus_fils();
   exit(EXIT_SUCCESS);
   };
...
if (waitpid(pid_fils,NULL,WMNOHANG) == -1)
   printf("Le processus fils n'est pas encore terminé\n");
...
\end{lstlisting}


La fonction retourne 
\begin{itemize}
\item
si un processus fils s'est terminé, le numéro du processus fils ;
\item
  \texttt{-1} si le processus a reçu un signal, 
et la variable \texttt{errno} contient alors  \texttt{EINTR} ;
\item
  \texttt{0} si l'option \texttt{WNOHANG} était indiqué, et qu'aucun
  processus n'a changé d'état
\end{itemize}

Les macros suivantes permettent de connaître la nature du changement d'état

\index{processus!status}

\begin{itemize}
\item 
  \texttt{WIFEXITED(status)} Vrai si le fils s'est terminé normalement
  (appel à \texttt{exit()} ou \texttt{return} depuis \texttt{main()}.
  Dans ce cas \texttt{WEXITSTATUS(status)} donne la
  valeur du code de retour status fourni par le processus fils.
\item
  \texttt{WIFSIGNALED(status)}
  Vrai si le fils s'est terminé à cause d'un signal non intercepté.
  Dans ce cas \texttt{WTERMSIG(status)}
              donne le numéro du signal qui a causé la fin du fils. 
\item \texttt{WIFSTOPPED(status)}
  Vrai si le fils est actuellement stoppé.
  Dans ce cas \texttt{WSTOPSIG(status)}
  donne le numéro du signal qui a causé l'arrêt du fils. 
\item \texttt{WIFCONTINUED(status)}
  indique si le processus a été continué.
\end{itemize}


\source
\lstinputlisting{../PROGS/Essais/test-waitpid.c}


\subsection{\texttt{exec()}}

\index{exec()!famille de fonctions}

\extrait
\begin{lstlisting}
#include <unistd.h>

int execv (const char *FILENAME, char *const ARGV[])
int execl (const char *FILENAME, const char *ARG0,...)
int execve(const char *FILENAME, char *const ARGV[], char *const ENV[])
int execle(const char *FILENAME, const char *ARG0,...char *const ENV[])
int execvp(const char *FILENAME, char *const ARGV[])
int execlp(const char *FILENAME, const char *ARG0, ...)
\end{lstlisting}

  
Ces fonctions font toutes la même chose : activer un exécutable
\emph{à la place} du processus courant. Elles diffèrent par
la manière d'indiquer les paramètres.

\begin{itemize}
\item  \texttt{execv()} : les paramètres de la commande sont transmis
sous forme d'un tableau de pointeurs sur des chaînes de caractères (le
dernier étant \texttt{NULL}). Exemple:


\source
\lstinputlisting{../PROGS/Divers/execv.c}



\item \texttt{execl()} reçoit un nombre variable de paramètres.
Le dernier est \texttt{NULL}). Exemple:


\source
\lstinputlisting{../PROGS/Divers/execl.c}



\item  \texttt{execve()} et \texttt{execle()} ont un paramètre
supplémentaire pour préciser l'\emph{environnement}.  

\item 
\texttt{execvp()} et \texttt{execlp()} utilisent la variable
d'environnement \texttt{PATH} pour localiser l'exécutable à lancer. On
pourrait donc écrire simplement:

\extrait
\begin{lstlisting}
  execlp("gcc", "gcc", fichier, "-o", prefixe, NULL);

  // avec execvp
  char * args[] = { "gcc", fichier, "-o", prefixe, NULL };
  execlp("gcc", args);
\end{lstlisting}
\end{itemize}

\subsection{Numéros de processus : \texttt{getpid()}, \texttt{getppid()}}


\index{getpid}
\index{getppid}
\index{processus!getpid}
\index{processus!getppid}

\extrait
\begin{lstlisting}
#include <unistd.h>

pid_t getpid(void);
pid_t getppid(void);
\end{lstlisting}


\texttt{getpid()} permet à un processus de connaître son propre numéro, et 
\texttt{getppid()} celui de son père.


\subsection{Programmation d'un démon}

Les \emph{démons}\footnote{Traduction de l'anglais \emph{daemon}, 
acronyme de  Disk And Extension MONitor'', qui désignait une
des parties résidentes d'un des premiers systèmes d'exploitation}
 sont des processus qui tournent normalement en
arrière-plan pour assurer un service. Pour programmer correctement un
démon, il ne suffit pas de faire un \texttt{fork()}, il faut aussi
s'assurer que le processus restant ne bloque pas de ressources. Par
exemple il doit libérer le terminal de contrôle du processus, revenir
à la racine, faute de quoi il empêchera le démontage éventuel du système de
fichiers à partir duquel il a été lancé.

\index{demon@démon}
\index{processus!démon}


\source
\lstinputlisting{../PROGS/Divers/demon.c}


Voir FAQ Unix : \emph{1.7 How do I get my program to act like a daemon}


\section{Les processus légers (Posix 1003.1c)}

\index{processus!légers}
\index{processus!thread}

Les processus classiques d'UNIX possèdent des ressources
séparées (espace mémoire, table des fichiers ouverts...).
Lorsqu'un nouveau \emph{fil d'exécution} (processus fils) 
est créé par \texttt{fork()}, il se voit attribuer une 
\emph{copie} des ressources du
processus père. 


Il s'ensuit deux problèmes : 
\begin{itemize}
\item  problème de  performances, puisque la 
duplication est un mécanisme coûteux
\item  problème de communication entre les processus, qui ont 
des variables séparées.
\end{itemize}

Il existe des moyens d'atténuer ces problèmes : technique du 
\emph{copy-on-write} dans le noyau pour ne dupliquer les
pages mémoires que lorsque
      c'est strictement nécessaire), utilisation
de segments de mémoire partagée (IPC) pour mettre des données en commun.
Il est cependant apparu utile de définir un mécanisme permettant
d'avoir plusieurs \emph{fils d'exécution} (threads) dans un même espace de
ressources non dupliqué : c'est ce qu'on appelle les 
\emph{processus légers}. Ces processus légers peuvent se voir affecter 
des priorités.


On remarquera que la commutation entre deux threads d'un même groupe est
une opération économique, puisqu'il n'est pas utile de recharger 
entièrement la table des pages de la MMU.


Ces processus légers ayant vocation à communiquer entre eux, la norme
POSIX 1003.1c définit également des mécanismes de synchronisation :
exclusion mutuelle (\emph{mutex}), \emph{sémaphores}, et \emph{conditions}.

\textbf{Remarque :} les sémaphores ne sont pas définis dans les
bibliothèques de AIX 4.2 et SVR4 d'ATT/Motorola. Ils existent dans
Solaris et les bibliothèques pour Linux.


\subsection{Threads}

\index{processus léger!pthread\_create()}
\index{processus léger!pthread\_exit()}
\index{pthread\_join()}
\index{pthread\_create()}
\index{pthread\_exit()}
\index{pthread\_join()}
\extrait
\begin{lstlisting}
#include <pthread.h>

int pthread_create(pthread_t      *thread, 
                   pthread_attr_t *attr, 
                   void           *(*start_routine)(void *), 
                   void           *arg);
void pthread_exit(void *retval);
int pthread_join(pthread_t th, void **thread_return);
\end{lstlisting}


La fonction \texttt{pthread\_create} demande le lancement d'un nouveau
processus léger, avec les attributs indiqués par la structure
pointée par \texttt{attr} (\texttt{NULL} = attributs par défaut).
Ce processus exécutera la fonction \texttt{start\_routine}, en lui donnant le
pointeur \texttt{arg} en paramètre.  L'identifiant du processus léger est
rangé à l'endoit pointé par \texttt{thread}.


Ce processus léger se termine (avec un code de retour) 
lorsque la fonction qui lui est
associée se termine par \texttt{return} \emph{retcode}, ou lorsque 
le processus léger exécute un
\texttt{pthread\_exit} \emph{(retcode)}.


La fonction \texttt{pthread\_join} permet au processus père d'attendre 
la fin d'un processus léger, et de récupérer éventuellement
son code de retour.



\textbf{Priorités :} Le fonctionnement des processus légers peut être
modifié (priorités, algorithme d'ordonnancement, etc.) en manipulant
les \emph{attributs} qui lui sont associés. Voir les fonctions
\texttt{pthread\_attr\_init}, \texttt{pthread\_attr\_destroy},
\texttt{pthread\_attr\_set-detachstate},
\texttt{pthread\_attr\_getdetachstate},
\texttt{pthread\_attr\_setschedparam},
\texttt{pthread\_attr\_getschedparam},
\texttt{pthread\_attr\_setschedpolicy},
\texttt{pthread\_attr\_getschedpolicy},
\texttt{pthread\_attr\_setinheritsched},
\texttt{pthread\_attr\_getinheritsched}, \texttt{pthread\_attr\_setscope},
\texttt{pthread\_attr\_getscope}.

\subsection{Verrous d'exclusion mutuelle (mutex)}

\index{mutex}
\index{processus léger!pthread\_mutex\_init()}
\index{processus léger!pthread\_mutex\_destroy()}
\index{processus léger!pthread\_mutex\_lock()}
\index{processus léger!pthread\_mutex\_unlock()}
\index{processus léger!pthread\_mutex\_trylock()}
\index{pthread\_mutex\_init()}
\index{pthread\_mutex\_destroy()}
\index{pthread\_mutex\_lock()}
\index{pthread\_mutex\_unlock()}
\index{pthread\_mutex\_trylock()}



\extrait
\begin{lstlisting}
#include <pthread.h>

int pthread_mutex_init(pthread_mutex_t   *mutex,   
                       const pthread_mutexattr_t *mutexattr);
int pthread_mutex_destroy(pthread_mutex_t *mutex);

int pthread_mutex_lock(pthread_mutex_t *mutex));
int pthread_mutex_unlock(pthread_mutex_t *mutex);

int pthread_mutex_trylock(pthread_mutex_t *mutex);
\end{lstlisting}


Les verrous d'exclusion mutuelle (\texttt{mutex}) sont  créés
par \texttt{pthread\_mutex\_init}. Il en est de différents types
(rapides, récursifs, etc.), selon les attributs pointés par
le paramètre \texttt{mutexattr}. La valeur par défaut (\texttt{mutexattr=NULL})
fait généralement l'affaire. L'identificateur du verrou est placé
dans la variable pointée par \texttt{mutex}.


\texttt{pthread\_mutex\_destroy} détruit le verrou. 
\texttt{pthread\_mutex\_lock} tente de le bloquer (et met le thread en attente
si le verrou est déjà bloqué), \texttt{pthread\_mutex\_unlock} le débloque.
\texttt{pthread\_mutex\_trylock} tente de bloquer le verrou, et échoue si
le verrou est déjà bloqué.


\subsection{Exemple}

\textbf{Source :}

\source
\lstinputlisting{../PROGS/Threads/leger_mutex.c}


\textbf{Compilation}:

Sous Linux, les programmes doivent être compilés avec 
l'option  \texttt{-pthread}:

\index{Options de compilation!bibliothèque des threads}



\extrait
\begin{lstlisting}
gcc -std=c18 -g -Wall -pedantic  leger_mutex.c -o leger_mutex -pthread
\end{lstlisting}


\textbf{Exécution}:

\extrait
\begin{lstlisting}
% leger_mutex 
[1] Hello
[2] World
[3] Hello
[4] Hello
[5] World
 5 lignes.
%
\end{lstlisting}



\subsection{Sémaphores}


Les sémaphores, qui font partie de la norme POSIX,
ne sont pas implémentés 
dans toutes les bibliothèques de threads.

\index{semaphore POSIX@sémaphore POSIX!sem\_init}
\index{semaphore POSIX@sémaphore POSIX!sem\_destroy}
\index{semaphore POSIX@sémaphore POSIX!sem\_wait}
\index{semaphore POSIX@sémaphore POSIX!sem\_post}
\index{semaphore POSIX@sémaphore POSIX!sem\_trywait}
\index{semaphore POSIX@sémaphore POSIX!sem\_getvalue}


\extrait
\begin{lstlisting}
#include <semaphore.h>

int sem_init(sem_t *sem, int pshared, unsigned int value);
int sem_destroy(sem_t * sem);

int sem_wait(sem_t * sem);
int sem_post(sem_t * sem);

int sem_trywait(sem_t * sem);
int sem_getvalue(sem_t * sem, int * sval);
\end{lstlisting}


Les sémaphores sont créés par \texttt{sem\_init}, qui place
l'identificateur du sémaphore à l'endroit pointé par \texttt{sem}. La
valeur initiale du sémaphore est dans \texttt{value}. Si
\texttt{pshared} est nul, le sémaphore est local au processus lourd
(le partage de sémaphores entre plusieurs processus lourds n'est pas
implémenté dans la version courante de \texttt{linuxthreads}.).



\texttt{sem\_wait} et \texttt{sem\_post} sont les équivalents respectifs
des primitives \texttt{P} et \texttt{V} de Dijkstra. La fonction
\texttt{sem\_trywait} échoue (au lieu de bloquer) si la valeur du
sémaphore est nulle. Enfin, \texttt{sem\_getvalue} consulte la valeur
courante du sémaphore.




\textbf{Exercice}: Utiliser un sémaphore au lieu d'un \texttt{mutex} 
pour sécuriser l'exemple.

\subsection{Conditions}

Les \emph{conditions}  servent à mettre en attente 
des processus légers derrière un \texttt{mutex}. 
Une primitive permet de débloquer d'un seul
coup tous les threads bloqués par une même condition.

\index{condition!pthread\_cond\_init}
\index{condition!pthread\_cond\_destroy}
\index{condition!pthread\_cond\_wait}
\index{condition!pthread\_cond\_signal}
\index{condition!pthread\_cond\_broadcast}

\extrait
\begin{lstlisting}
#include <pthread.h>

int pthread_cond_init(pthread_cond_t  *cond, 
                      pthread_condattr_t *cond_attr);
int pthread_cond_destroy(pthread_cond_t *cond);

int pthread_cond_wait(pthread_cond_t *cond,
                      pthread_mutex_t *mutex);

int pthread_cond_signal(pthread_cond_t *cond);
int pthread_cond_broadcast(pthread_cond_t *cond);
\end{lstlisting}




Les conditions sont créées par
\texttt{phtread\_cond\_init}, et  détruites par \texttt{phtread\_cond\_destroy}.

 
Un processus se met en attente en effectuant un 
\texttt{phtread\_cond\_wait} (ce qui bloque au passage un \texttt{mutex}).
La primitive 
\texttt{phtread\_cond\_broadcast} débloque tous les processus qui attendent
sur une condition, \texttt{phtread\_cond\_signal} en débloque un seul.






\chapter{IPC : Communication locale entre processus}


\section{Les mécanismes IPC System V}
  
\index{IPC,Inter Process Communication}
  
Les mécanismes de communication entre processus (\emph{InterProcess
  Communication}, ou \emph{IPC}), issus d'Unix System V ont été repris
dans de nombreuses variantes d'Unix. Il y a 3 mécanismes, permettant
le partage d'informations entre processus tournant sur une même machine :

\begin{itemize}
\item  les segments, permettant de partager des zones de mémoire
\item  les sémaphores, qui fournissent un moyen d'en contrôler l'accès,
\item  les files de messages,
\end{itemize}


Ces  objets sont identifiés par des \emph{clés}.

\index{Options de compilation!\_X\_OPEN\_SOURCE}

Pour compiler ces programmes, ajoutez l'option
\verb+-D_XOPEN_SOURCE=700+ à la compilation

\section{\texttt{ftok()} constitution d'une clé }

\extrait
\begin{lstlisting}
# include <sys/types.h>
# include <sys/ipc.h>

key_t ftok ( char *pathname, char project )
\end{lstlisting}


La fonction \texttt{ftok()} constitue une clé à partir d'un chemin d'accès
polet d'un caractère indiquant un projet''. Plutôt que de risquer une
explication abstraite, étudions deux cas fréquents :
\begin{itemize}
\item On dispose d'un logiciel commun dans \texttt{/opt/jeux/OuiOui}. 
Ce logiciel utilise deux objets partagés. On pourra utiliser les clés
\texttt{ftok("/opt/jeux/OuiOui",'A')} et
\texttt{ftok("/opt/jeux/OuiOui",'B')}.
Ainsi tous les processus de ce logiciel se réfèreront aux mêmes objets
qui seront partagés entre tous les utilisateurs.
\item On distribue un exemple aux étudiants, qui le recopient chez eux
et le font tourner.
On souhaite que les processus d'un même étudiant communiquent entre eux,
mais qu'ils n'interfèrent pas avec d'autres. On basera donc la clé
sur une donnée personnelle, par exemple le répertoire d'accueil, avec les
clés
\texttt{ftok(getenv("HOME"),'A')} et
\texttt{ftok(getenv("HOME"),'B')}.

\end{itemize}


\section{Mémoires partagées}

Ce mécanisme permet à plusieurs programmes de partager des \emph{segments
mémoire}. Chaque segment mémoire est identifié, au niveau du système,
par une \texttt{clé} à laquelle correspond un \emph{identifiant}. Lorsqu'un
segment est \emph{attaché} à un programme, les données qu'il contient
sont accessibles en mémoire par l'intermédiaire d'un pointeur.

\index{memoire partagee@mémoire partagée!shmget()}
\index{memoire partagee@mémoire partagée!shmat()}
\index{memoire partagee@mémoire partagée!shmdt()}
\index{memoire partagee@mémoire partagée!shmctl()}


\extrait
\begin{lstlisting}
#include <sys/ipc.h>
#include <sys/shm.h>

int shmget(key_t key, int size, int shmflg);
char *shmat (int shmid, char *shmaddr, int shmflg )
int shmdt (char *shmaddr)
int shmctl(int shmid, int cmd, struct shmid_ds *buf);
\end{lstlisting}



La fonction \texttt{shmget()} donne l'identifiant du segment ayant la
clé \texttt{key}. Un nouveau segment (de taille \texttt{size}) est
créé si \texttt{key} est \texttt{IPC\_PRIVATE}, ou bien si les
indicateurs de \texttt{shmflg} contiennent \texttt{IPC\_CREAT}.
Combinées, les options \texttt{IPC\_EXCL | IPC\_CREAT} indiquent que
le segment ne doit pas exister préalablement.  Les bits de poids
faible de \texttt{shmflg} indiquent les droits d'accès.




\texttt{shmat()} attache le segment \texttt{shmid} en mémoire, avec les droits
spécifiés dans \texttt{shmflag} (\texttt{SHM\_R, SHM\_W, SHM\_RDONLY}). 
\texttt{shmaddr}
précise où ce segment doit être situé dans l'espace mémoire
(la valeur \texttt{NULL} demande un placement automatique). 
\texttt{shmat()} renvoie
l'adresse où le segment a été placé.


\texttt{shmdt()} ``libère'' le segment. \texttt{shmctl()} permet diverses 
opérations, dont la destruction d'une mémoire partagée (voir exemple).

\textbf{Exemple} (deux programmes):

Le producteur :

\source
\lstinputlisting{../PROGS/IPC/prod.c}


Le consommateur :


\source
\lstinputlisting{../PROGS/IPC/cons.c}



\textbf{Question} : le second programme n'affiche pas forcément des
informations cohérentes. Pourquoi ? Qu'y faire ? 



\textbf{Problème} : écrire deux programmes qui partagent deux variables
\texttt{i, j}. Voici le pseudo-code:

\extrait
\begin{lstlisting}
processus P1                 	processus P2
| i=0 j=0			| tant que i==j 
| repeter indefiniment		|   faire rien
|   i++	j++			| ecrire i
fin				fin
\end{lstlisting}

Au bout de combien de temps le processus P2 s'arrête-t-il ?
Faire plusieurs essais. 


\textbf{Exercice} : la commande \texttt{ipcs} affiche des informations sur les
segments qui existent. Ecrire une commande qui permet d'afficher le contenu 
d'un segment (on donne le \emph{shmid} et la longueur en paramètres).
 
\section{Sémaphores}

\index{IPC!sémaphores}

\index{semaphore IPC@sémaphore IPC!semget}
\index{semaphore IPC@sémaphore IPC!semop}
\index{semaphore IPC@sémaphore IPC!semctl}

\extrait
\begin{lstlisting}
#include <sys/types.h>
#include <sys/ipc.h>
#include <sys/sem.h>   

int semget(key_t key, int nsems, int semflg )
int semop(int semid, struct sembuf *sops, unsigned nsops)
int semctl(int semid, int semnum, int cmd, union semun arg )
\end{lstlisting}


Les opérations System V travaillent en fait sur des tableaux de sémaphores 
généralisés (pouvant évoluer par une valeur entière quelconque). 


La fonction \texttt{semget()} demande à travailler sur le 
sémaphore généralisé qui est identifié par la clé \texttt{key} (même notion que pour les clés
des segments partagés) et qui contient \texttt{nsems} sémaphores individuels.
Un nouveau sémaphore est créé, avec les droits donnés par les 9 bits de 
poids faible de \texttt{semflg}, si
\texttt{key} est \texttt{IPC\_PRIVATE}, ou si
\texttt{semflg} contient \texttt{IPC\_CREAT}. 



\texttt{semop()} agit sur le sémaphore \texttt{semid} en appliquant
simultanément à plusieurs sémaphores individuels les actions décrites
dans les \texttt{nsops} premiers éléments du tableau
\texttt{sops}. Chaque \texttt{sembuf} est une structure de la forme:


\extrait
\begin{lstlisting}
struct sembuf
{ 
  ...
  short sem_num;  /* semaphore number: 0 = first */
  short sem_op;   /* semaphore operation */
  short sem_flg;  /* operation flags */
  ...
}
\end{lstlisting}

\texttt{sem\_flg} est une combinaison d'indicateurs qui peut contenir
\texttt{IPC\_NOWAIT} et  \texttt{SEM\_UNDO} (voir manuel). Ici nous supposons
que \texttt{sem\_flg} est 0. 


\texttt{sem\_num} indique le numéro du sémaphore individuel sur lequel 
porte l'opération. \texttt{sem\_op} est un entier destiné (sauf si il est nul) 
à être ajouté à la valeur courante \emph{semval} du sémaphore. 
L'opération se bloque si \verb/sem_op + semval < 0/.

 
\textbf{Cas particulier : } si  \emph{sem\_op} est 0, l'opération est bloquée 
tant que  \texttt{semval}
est non nul.


Les valeurs des sémaphores ne sont  mises à jour que lorsque
aucun d'eux n'est bloqué. 


\texttt{semctl} permet de réaliser diverses opérations sur les sémaphores,
selon la commande demandée. En particulier, on peut fixer le
\texttt{n}-ième sémaphore à la valeur \texttt{val} en faisant :

\extrait
\begin{lstlisting}
semctl(sem,n,SETVAL,val);
\end{lstlisting}


\textbf{Exemple: } primitives sur les sémaphores traditionnels.



\source
\lstinputlisting{../PROGS/IPC/sem.c}



\textbf{Exercice : } que se passe-t-il si on essaie d'interrompre \texttt{semop()} ?

\textbf{Exercice : } utilisez les sémaphores pour ``sécuriser'' l'exemple
présenté sur les mémoires partagées.
 
\section{Files de messages}

\index{IPC!files de messages}

Ce mécanisme permet l'échange de messages par des processus. Chaque message
possède un \emph{corps} de longueur variable, et un \emph{type} (entier 
strictement positif) qui peut servir à préciser la nature des informations
contenues dans le corps. 

Au moment de la réception, on peut choisir de sélectionner les messages
d'un type donné. 

\index{file de messages IPC!msgget()}
\index{file de messages IPC!msgsnd()}
\index{file de messages IPC!msgrcv()}
\index{file de messages IPC!msgctl()}

\extrait
\begin{lstlisting}
#include <sys/types.h>
#include <sys/ipc.h>
#include <sys/msg.h>

int msgget (key_t key, int msgflg)
int msgsnd (int msqid, struct msgbuf *msgp, int msgsz, int msgflg)
int msgrcv (int msqid, struct msgbuf *msgp, int msgsz, 
                       long msgtyp, int msgflg)
int msgctl ( int msqid, int  cmd, struct msqid_ds *buf )
\end{lstlisting}


\texttt{msgget()} demande l'accès à (ou la création de) la file de
message avec la clé \texttt{key}. \texttt{msgget()} retourne la valeur
de l'\emph{identificateur de file}. 
 \texttt{msgsnd()} envoie un
message dans la file \texttt{msqid}. Le \emph{corps} de ce message
contient \texttt{msgsz} octets, il est placé, précédé par le
\emph{type} dans le tampon pointé par \texttt{msgp}. Ce tampon de la
forme:

\extrait
\begin{lstlisting}
struct msgbuf {
     long mtype;     /* message type, must be > 0 */
     char mtext[...] /* message data */
};
\end{lstlisting}


\texttt{msgrcv()} lit dans la file un message d'un type donné (si
\verb/type < 0/) ou indifférent (si \texttt{type==0}), et le
place dans le tampon pointé par \texttt{msgp}. La taille du corps ne
pourra excéder \texttt{msgsz} octets, sinon il sera
tronqué. \texttt{msgrcv()} renvoie la taille du corps du message. 


\textbf{Exemple.} Deux programmes, l'un pour envoyer des messages (lignes
de texte) sur une file avec un type donné, l'autre pour afficher les
messages reçus.


 
\source
\lstinputlisting{../PROGS/IPC/snd.c}



 
\source
\lstinputlisting{../PROGS/IPC/rcv.c}



\chapter{TCP-IP : communication par le réseau}


\section{Communication par TCP-IP, spécificités}

Les concepts fondamentaux de la transmission d'informations par le
réseau Internet (sockets, adresses, communication par datagrammes et flots,
client-serveur etc.) sont les mêmes que pour les sockets locaux (voir
\ref{sockets}).

Les spécificités concernent essentiellement l'adressage : comment
\textbf{fabriquer une adresse} de socket à partir d'un nom de machine
(résolution) et d'un numéro de port, comment retrouver le nom d'une
machine à partir d'une adresse (résolution inverse) etc.


\section{Sockets, addresses}

\index{socket!TCP-IP}

\extrait
\begin{lstlisting}
#include <sys/types.h>
#include <sys/socket.h>

int socket(int domain, int type, int protocol);
\end{lstlisting}


Pour communiquer, les applications doivent créer des \emph{sockets}
(prises bidirectionnelles) par la fonction \texttt{socket()} et les
relier entre elles.  On peut ensuite utiliser ces sockets comme des
fichiers ordinaires (par \texttt{read}, \texttt{write}, ...) ou par
des opérations spécifiques ( \texttt{send}, \texttt{sendto},
\texttt{recv}, \texttt{recvfrom}, ...).


\subsection{ \texttt{struct sockaddr} : adresses de sockets}

Pour désigner un socket sur une machine il faut une \emph{adresse de
socket}. Les fonctions réseau comme \texttt{bind}


\begin{lstlisting}
int bind(int socket,
         const struct sockaddr *address,
         socklen_t address_len);
\end{lstlisting}

prennent commme paramètre une \textbf{adresse de socket} désignée par
un pointeur.

\paragraph{Terminologie :} pour éviter les confusions, dans ce texte
\begin{itemize}
\item le terme \textbf{adresse} se réfère toujours une \textbf{adresse de socket} (concept réseau)
  \item on parle de \textbf{numéro IP} pour une adresse d'une machine (\textbf{hôte}), indépendamment du port (\textbf{service}),
\item on utilise systématiquement le mot \textbf{pointeur} pour parler des
  adresses de données en mémoire (concept du langage C)
\end{itemize}

Ce pointeur est obtenu par transtypage (cast) d'un
pointeur sur une structure contenant une adresse de socket. Le type de
la structure peut être différent pour ipv4 ou ipv6.

Les adresses de sockets ont un premier champ
\footnote{qui s'appelle  \texttt{sa\_family} pour \texttt{struct sockaddr}}
de type
\texttt{sa\_family\_t}, qui comme son nom l'indique est
une indication de la famille d'adresses : constantes
\texttt{AF\_INET} pour ipv4, \texttt{AF\_INET6} pour ipv6.

On utilisera 3 types de structures pour contenir des adresses

\begin{itemize}
\item \texttt{struct sockaddr\_in},  spécifiquement pour les adresses IPv4 ;
\item \texttt{struct sockaddr\_in6}, pour IPv6 ;
\item \texttt{struct sockaddr\_storage}, assez grande pour contenir
  n'importe quel type d'adresse ;
\end{itemize}

et les pointeurs de type \texttt{struct sockaddr *} serviront à
désigner ces structures de façon générique.\footnote{C'est une façon
courante de réaliser un genre de polymorphisme en C}


\subsection{\texttt{struct sockaddr\_in} : adresses de sockets IPv4}

\index{socket!TCP-IP!adresses IPv4}


Pour la communication par IP en général, l'adresse d'un socket est
formée à partir d'un numéro IP, et un numéro de port (service).

Les \texttt{struct sockaddr\_in} sont destinées spécifiquement
aux adresses IPv4, elles ont 3 champs importants :
\begin{itemize}
\item \texttt{sin\_family}, la famille d'adresses, valant \texttt{AF\_INET}
\item \texttt{sin\_addr}, pour le numéro IP,
\item \texttt{sin\_port}, pour le numéro de port
\end{itemize}

Attention, les octets du numéro IP et du numéro de port sont
stockés dans \emph{l'ordre réseau} (big-endian), qui n'est pas forcément
celui de la machine hôte sur laquelle s'exécute le programme. 

Pour renseigner correctement cette structure, il existe des fonctions
de conversion, en particulier \texttt{getaddrinfo()} que nous verrons
plus loin, et qui assure la \emph{résolution d'adresse}, c'est-à-dire
la traduction\footnote{sn consultant éventuellement des serveurs de
  noms (DNS)} d'un nom de machine (exemple \texttt{www.u-bordeaux.fr})
en numéro IP.

\subsection{\texttt{struct sockaddr\_in6} : adresses de socket IPv6}

\index{socket!TCP-IP!adresses IPv6}

Pour IPv6, on utilise
des \texttt{struct sockaddr\_in6}, avec
\begin{itemize}
\item \texttt{sin6\_family}, valant \texttt{AF\_INET6}
\item \texttt{sin6\_addr}, le numéro IP sur 6 octets
\item \texttt{sin6\_port}, pour le numéro de port.
\end{itemize}

Même remarque : on utilisera \texttt{getaddrinfo()} pour fabriquer ces
adresses de socket.


\subsection{\texttt{struct sockaddr\_storage} : conteneur d'adresse}

Cette structure est destinée à contenir une adresse de socket réseau de
n'importe quel type. Son premier champ \texttt{ss\_family} contient la
famille de l'adresse.

\section{Remplissage d'une adresse de socket : \texttt{getaddrinfo()}}

Comme pour les sockets locaux, on crée les sockets par \texttt{socket()}
et on  ``nomme la prise'' par \texttt{bind()}, à partir d'une
\emph{adresse de socket} que l'on a renseignée préalablement.

Pour constituer cette adresse, on utilise \texttt{getaddrinfo()} !

\index{getaddrinfo()}
\extrait
\begin{lstlisting}
#include <sys/types.h>
#include <sys/socket.h>
#include <netdb.h>

int getaddrinfo(const char *node, const char *service,
                const struct addrinfo *hints,
                struct addrinfo **res);
\end{lstlisting}

qui retourne une liste d'adresses de sockets IPv4 et/ou IPv6, selon
les arguments qui lui sont données :

\begin{itemize}
\item \texttt{node} et \texttt{service} : chaines contenant le
nom de la machine et du port,
\item un pointeur sur une structure \texttt{hints} contenant des
  indications diverses.
\end{itemize}

Le résultat est une liste chainée de structures \texttt{addrinfo}
contenant des adresses de sockets :

\begin{lstlisting}
struct addrinfo {
      int              ai_flags;
      int              ai_family;
      int              ai_socktype;
      int              ai_protocol;
      socklen_t        ai_addrlen;
      struct sockaddr *ai_addr;
      char            *ai_canonname;
      struct addrinfo *ai_next;
};
\end{lstlisting}

ce type de structure sert aussi pour les indications complémentaires
pour la requête (voir plus loin)

Les champs qui nous intéressent, dans les résultats, sont

\begin{itemize}
\item \texttt{ai\_family} qui indique la famille d'adresses : 
constante \texttt{AF\_INET}  pour ipv4, \texttt{AF\_INET6}  pour ipv6 ; 
\item \texttt{ai\_addr}  qui est un pointeur sur une adresse de socket
\item \texttt{ai\_addrlen}, la longueur de cette adresse ;
\item \texttt{ai\_next}, un pointeur sur le résultat suivant.
\end{itemize}
Attention, il faudra libérer la liste de résultats après usage, 
en appelant \texttt{freeaddrinfo()}.

\begin{lstlisting}
  void freeaddrinfo(struct addrinfo *res);
\end{lstlisting}


\subsection{Préparation d'une adresse distante}

Dans le cas le plus fréquent, on cherche à joindre une machine

\begin{itemize}
\item dont on connait le nom  \texttt{"www.elysee.fr"} ou le numéro
  \texttt{"8.253.7.126"};
\item  du \emph{service} que l'on veut joindre, que ce soit par un nom
  (exemple \texttt{"http"}) ou un numéro de port \texttt{"80"}.\footnote{
  le fichier \texttt{/etc/services} des machines Unix
  contient une table de correspondance entre les noms de services et les
  numéros de port}
%  \item avec un protocole de type ``flot de données'' ou ``datagrammes''.
\end{itemize}


\paragraph{Un exemple simple : } ouverture d'un socket ``flot de données''
vers le port 80 de la machine \texttt{www.elysee.fr}

\begin{lstlisting}
  struct addrinfo * adr_premier;

  // récupérer dans adr_premier la première adresse
  // qui correspond
  getaddrinfo("www.elysee.fr", "http", NULL, & adr_premier);

  // utilisation de l'adresse
  int fd = socket(adr_premier->ai_family, SOCK_STREAM, 0);
  bind(fd, adr_premier->ai_addr, adr_premier_>ai_addrlen);

  freeaddrinfo(adr_premier);
\end{lstlisting}

Il faudrait y ajouter des vérifications : échec des fonctions, absece
de résultats, etc.

%% \index{gethostbyname(nom)}

%% \extrait
%% \begin{lstlisting}
%%        #include <netdb.h>
%%        struct hostent *gethostbyname(const char *name);
%%        \end{lstlisting}

%% \texttt{gethostbyname()} retourne un pointeur vers une structure
%% \texttt{hostent} qui contient diverses informations sur la machine en
%% question, en particulier une adresse \texttt{h\_addr} \footnote{Une
%% machine peut avoir plusieurs adresses} que l'on mettra dans
%% \texttt{sin\_addr}.


%% Le champ \texttt{sin\_port} est un entier court \emph{en ordre réseau}, pour
%% y mettre un entier ordinaire il faut le convertir par \texttt{htons()}
%% 		    \footnote{Host TO Network Short}.


On trouvera une utilisation plus détaillée dans l'exemple
\texttt{client-echo.c}.

\subsection{Préparation d'une adresse locale}

%% Pour une adresse sur la machine locale\footnote{ne pas confondre avec
%%   la notion de \emph{socket du domaine local} vue en \ref{sockets}},
%% on utilise
%% \begin{itemize}
%% \item  dans le cas le plus fréquent, l'adresse \texttt{INADDR\_ANY}
%% (\texttt{0.0.0.0}). Le
%% socket est alors ouvert (avec le même numéro de port) sur toutes les
%% adresses IP de toutes les interfaces de la machine.  
%% \item  l'adresse
%% \texttt{INADDR\_LOOPBACK} correspondant à l'adresse locale
%% \texttt{127.0.0.1} (alias \texttt{localhost}). Le socket n'est alors
%% accessible que depuis la machine elle-même.  
%% \item  une des adresses IP de la la machine.
%% \item  l'adresse de diffusion générale (\emph{broadcast}) 
%% \texttt{INADDR\_BROADCAST} 
%% (\texttt{255.255.255.255})\footnote{L'utilisation de l'adresse de diffusion est
%% soumise à restrictions, voir manuel}
%% \end{itemize}


%% Voir exemple \texttt{serveur-echo.c}.

%% Pour IPV6
%% \begin{itemize}
%% \item Famille d'adresses \texttt{AF\_INET6}, famille de protocoles
%% \texttt{PF\_INET6}
%% \item adresses prédéfinies 
%% \texttt{IN6ADD\_LOOPBACK\_INIT} (\texttt{::1})
%% \texttt{IN6ADD\_ANY\_INIT}
%% \end{itemize}



Dans l'exemple ci-dessus, le troisième paramètre de \texttt{getaddrinfo()}
est un pointeur nul, mais on peut s'en servir pour transmettre
l'adresse d'une structure \texttt{addrinfo} qui sert à préciser ce que l'on
veut.

Les champs pertinents (les plus intéressants) :

\begin{itemize}
\item \texttt{ai\_family} qui indique la ou les familles d'adresses
  voulues : constante \texttt{AF\_INET} pour ipv4, \texttt{AF\_INET6}
  pour ipv6, \texttt{AF\_UNSPEC} (par défaut) pour l'un ou l'autre.

\item \texttt{ai\_socktype} indique si on ne doit rechercher que certains
types de sockets (\texttt{SOCK\_STREAM}, \texttt{SOCK\_DGRAM}) ou tous (O).

\item \texttt{ai\_flags} : combinaison d'indicateurs divers. Y mettre
  \texttt{AI\_PASSIVE} pour un serveur qui doit accepter des
  connexions sur ses différentes adresses réseau.
\end{itemize}

Les autres doivent être initialisés, 0 est une valeur par défaut
raisonnable.

L'initialisation se fait élégamment grâce aux ``Designated
Initializers'' introduits par la norme C99. Exemple\footnote{
  Avec ce type d'initialisation des structures, les champs
  non spécifiés sont initialisés à 0.}

\begin{lstlisting}
struct addrinfo indications_client = {    // ipv4 ou ipv6
   .ai_family   = AF_UNSPEC,
   .ai_socktype = SOCK_STREAM
};
   
struct addrinfo indications_serveur = {   // pour un serveur IPv6
   .ai_family = AF_INET6,
   .ai_flags  = AI_PASSIVE
};
\end{lstlisting}



\subsection{Examen d'une adresse : \texttt{getnameinfo()}}
  
\index{getaddrinfo()}
La fonction `getnameinfo()` réalise la conversion dans l'autre sens :
à partir d'une adresse de socket réseau, obtenir une description du
numéro IP de la machine et du numéro de service/port.

\extrait
\begin{lstlisting}
int getnameinfo(const struct sockaddr *addr, socklen_t addrlen,
                char *host, socklen_t hostlen,
                char *serv, socklen_t servlen, 
		int flags);
\end{lstlisting}


Les paramètres sont

\begin{itemize}
\item un pointeur sur la structure contenant l'adresse, et sa taille
\item l'adresse d'un tampon pour y ranger le nom de la machine, et sa longueur
\item l'adresse d'un tampon pour le port, et sa longueur.
\item une combinaison d'indicateurs : \texttt{NI\_NUMERICHOST} et
  \texttt{NI\_NUMERICSERV}, pour avoir les indications sous forme
  numérique.
\end{itemize}

\paragraph{Un exemple :} programme de résolution d'adresses.

Le programme qui suit s'exécute en ligne de commande.
Il prend en paramètre une série de noms de machines ou d'adresses

Pour chaque machine il affiche les adresses IP numériques
sous forme numérique IPv4 (décimal pointé) ou IPv6 (hexadécimal).


\source
\lstinputlisting{../PROGS/Resolution/resolution.c}



\subsection{Adresse associée à un socket}

\index{getsockname()}
\index{getpeername()}

A partir d'un ``field descriptor'' ouvert, ces deux fonctions
permettent de retrouver l'adresse

\begin{itemize}
\item du socket auquel il est lié (local)
\item du socket ``pair'' auquel il est connecté (distant)
\end{itemize}




\extrait
\begin{lstlisting}
#include <sys/socket.h>

int   getsockname(int sockfd,  
                  struct sockaddr  * name,
                  socklen_t        * namelen )
int   getpeername(int sockfd,
                  struct sockaddr  * addr,
                  socklen_t        * addrlen);
\end{lstlisting}

Paramètres :
\begin{itemize}

\item le descripteur,
  \item un pointeur sur un ``conteneur d'adresse
de socket'' (de préférence une structure \texttt{sockaddr\_storage}
pour avoir la compatibilité ipv4/ipv6),
\item
  un pointeur sur un entier
qui recevra la longueur de l'adresse.
\end{itemize}

 

%% On peut également tenter une \emph{résolution inverse}, c'est-à-dire
%% de retrouver le nom à partir de l'adresse de socket, en passant par 
%% \texttt{gethostbyaddr}, qui retourne un pointeur vers une structure \texttt{hostent},
%% dont le champ \texttt{h\_name} désigne le nom officiel de la machine.


%% Cette résolution n'aboutit pas toujours, parce que tous les numéros IP
%% ne correspondent pas à des machines  déclarées. 


%% \extrait
%% \begin{lstlisting}
%% #include <netdb.h>
%% extern int h_errno;

%% struct hostent *gethostbyaddr(const char *addr, int len, int type);

%% struct hostent {
%%   char    *h_name;        /* official name of host */
%%   char    **h_aliases;    /* alias list */
%%   int     h_addrtype;     /* host address type */
%%   int     h_length;       /* length of address */
%%   char    **h_addr_list;  /* list of addresses */
%% }
%% #define h_addr  h_addr_list[0]  /* for backward compatibility */
%% \end{lstlisting}


\section{Fermeture d'un socket}

\index{close(socket)}
\index{shutdown(socket,indic)}

Un socket peut être fermé par \texttt{close()} ou par \texttt{shutdown()}.


\extrait
\begin{lstlisting}
int shutdown(int fd, int how);
\end{lstlisting}


Un socket est bidirectionnel, le paramètre \texttt{how} indique quelle(s) 
moitié(s) on ferme :
\texttt{SHUT\_RD} pour l'entrée,
\texttt{SHUT\_WR} pour la sortie,
\texttt{SHUT\_RDWR} pour les deux (équivaut à \texttt{close()}).


\section{Communication par datagrammes (UDP)}

\subsection{Création d'un socket}       


\extrait
\begin{lstlisting}
#include <sys/types.h>
#include <sys/socket.h>

int socket(int domain, int type, int protocol);
\end{lstlisting}



Cette fonction construit un socket et retourne un numéro de
descripteur.

Pour une liaison par datagrammes via Internet, indiquez
famille d'adresses, le type \texttt{SOCK\_DGRAM} et le protocole
par défaut \texttt{0}.



Retourne \texttt{-1} en cas d'échec.


\subsection{Connexion de sockets}

\index{socket!connect(socket,adresse,longueur)!réseau}

La fonction \texttt{connect} met en relation un socket (de cette
machine) avec un autre socket désigné, qui sera le correspondant par
défaut'' pour la suite des opérations.


\extrait
\begin{lstlisting}
#include <sys/types.h>
#include <sys/socket.h>

int  connect(int sockfd, 
             const struct sockaddr *serv_addr,
             socklen_t addrlen);
             \end{lstlisting}


\subsection{Envoi de datagrammes}

Sur un socket connecté (voir ci-dessus), on peut expédier
des datagrammes (contenus dans un tampon
\texttt{t} de longueur \texttt{n}) par \texttt{write(sockfd,t,n)}.


La fonction \texttt{send()} 

\extrait
\begin{lstlisting}
int send(int s, const void *msg, size_t len, int flags);
\end{lstlisting}

permet d'indiquer des \emph{flags}, par exemple
\texttt{MSG\_DONTWAIT} pour  une écriture non bloquante.


Enfin, \texttt{sendto()} envoie un datagramme à une adresse
spécifiée, sur un socket connecté ou non.



\extrait
\begin{lstlisting}
int sendto(int s, const void *msg, size_t len, int flags,
           const struct sockaddr *to, socklen_t tolen);
           \end{lstlisting}


\subsection{Réception de datagrammes}

Inversement, la réception peut se faire par un simple 
\texttt{read()}, par un \texttt{recv()} (avec des flags), ou
par un \texttt{recvfrom}, qui permet de remplir une structure avec
l'adresse
\texttt{from}
du socket émetteur.

\extrait
\begin{lstlisting}
int recv(int s, void *buf, size_t len, int flags);
int recvfrom(int  s,  void  *buf,  size_t len, int flags,
             struct sockaddr *from, socklen_t *fromlen);
             \end{lstlisting}


\subsection{Exemple UDP : serveur d'écho}

Principe:
\begin{itemize}
\item 
Le client envoie une chaîne de caractères au serveur.
\item 
Le serveur l'affiche, la convertit en minuscules,
et la réexpédie.
\item 
Le client affiche la réponse.
\end{itemize}

Usage:
\begin{itemize}
\item  sur le serveur : \texttt{serveur-echo} \emph{numéro-de-port} 
\item 
pour chaque client : \texttt{client-echo} \emph{nom-serveur}
\emph{numéro-de-port} \emph{``message à expédier''}
\end{itemize}

\paragraph{Le client}.

 
\source
\lstinputlisting{../PROGS/Echo-Datagrammes/client-echo.c}


\textbf{Exercice} : faire en sorte que le client réexpédie sa requête
si il ne reçoit pas la réponse dans un délai fixé. Fixer une limite
au nombre
de tentatives. 

\paragraph{Le serveur} :

 
\source
\lstinputlisting{../PROGS/Echo-Datagrammes/serveur-echo.c}


\section{Communication par flots de données (TCP)}

La création d'un socket pour TCP se fait ainsi

\extrait
\begin{lstlisting}
  int fd;
  ..
  fd = socket(AF_INET,SOCK_STREAM,0);
  \end{lstlisting}


\subsection{Programmation des clients TCP}

Le socket d'un client TCP doit être relié (par \texttt{connect()}) à celui
du serveur, et il est utilisé ensuite par des \texttt{read()} et des
\texttt{write()}, ou des entrées-sorties de haut niveau \texttt{fprintf()},
\texttt{fscanf()}, etc. si on a défini des flots par \texttt{fdopen()}.



\subsection{Exemple : client web}

 
\source
\lstinputlisting{../PROGS/TCP-Flots/client-web.c}


Remarque: souvent il est plus commode de créer des flots de haut
niveau au dessus du socket (voir \texttt{fdopen()}) que de manipuler
des \texttt{read} et des \texttt{write}. Voir dans l'exemple suivant.

\subsection{Réaliser un serveur TCP}

Un serveur TCP doit traiter des connexions venant de plusieurs clients.


Après avoir créé et nommé le socket,  
le serveur spécifie qu'il
accepte les communications entrantes par \texttt{listen()}, et se met 
effectivement en attente
d'une connexion de client par \texttt{accept()}.


\extrait
\begin{lstlisting}
#include <sys/types.h>
#include <sys/socket.h> 

int listen(int s, int backlog);
int accept(int s,  struct  sockaddr  *addr,  
                   socklen_t  *addrlen);
\end{lstlisting}


Le paramètre \texttt{backlog} indique la taille maximale de la file des
 connexions en attente. Sous Linux la limite est donnée par
la constante \texttt{SOMAXCONN} (qui vaut 128), sur d'autres systèmes elle
est limitée à 5.


La fonction \texttt{accept()} retourne un autre socket, qui sert
à la communication avec le client.
L'adresse du client peut être obtenue par les paramètres
\texttt{addr} et \texttt{addrlen}.



En général, les serveurs TCP doivent traiter simultanément des
connexions venant de plusieurs clients. La solution habituelle est de
lancer, après l'appel à \texttt{accept()} un processus fils (par
\texttt{fork()})qui traite la communication avec un seul client.

Ceci
induit une gestion des processus, donc des signaux liés à la
terminaison des processus fils.




\chapter{Exemples TCP : serveurs Web}


Dans ce qui suit nous présentons un serveur Web rudimentaire,
capable de fournir des pages Web construites à partir des fichiers
d'un répertoire. Nous donnons deux implémentations possibles,
à l'aide de processus lourds et légers.

Attention : ces serveurs ne traitent que les requêtes \texttt{GET} de
\texttt{HTTP/1.0}, et ignorent les entêtes HTTP, les ``keep-alive'',
qui gardent les connexions ouvertes etc.


\section{Serveur Web (avec processus)}

\subsection{Principe et pseudo-code}
Cette version suit l'approche traditionnelle. Un processus
est créé chaque fois qu'un client contacte le serveur.



Pseudo-code:


\extrait
\begin{lstlisting}
ouvrir socket serveur (socket/bind/listen)
répéter indéfiniment
|    attendre l'arrivée d'un client (accept)
|    créer un processus (fork) et lui déléguer
|      la communication avec le client
fin-répeter
\end{lstlisting}


\subsection{Code du serveur}

\source
\lstinputlisting{../PROGS/Serveurs-Web/serveur-web-processus.c}


\subsection{Discussion de la solution}

Un avantage est la simplicité de la solution, et sa robustesse:
une erreur d'exécution dans un processus fils est normalement
sans incidence sur le fonctionnement des autres parties du serveur.


En revanche, la création d'un processus est une opération
relativement coûteuse, qui introduit un temps de latence
au début de chaque communication.  D'où l'idée de lancer
les processus avant d'en avoir besoin (préchargement), 
et de réutiliser ceux qui ont terminé leur tâche.

\section{Serveur Web (avec threads)}

\subsection{Principe et pseudo-code}

Les processus légers permettent une autre approche : on 
crée préalablement un pool'' de processus que l'on bloque.
Lorsqu'un client se présente, on confie la communication
à un processus inoccupé.


\extrait
\begin{lstlisting}
ouvrir socket serveur (socket/bind/listen)
créer un pool de processus
répéter indéfiniment
|    attendre l'arrivée d'un client (accept)
|    trouver un processus libre, et lui
|        confier la communication avec le client
fin-répeter
\end{lstlisting}



\subsection{Code du serveur}
 
\source
\lstinputlisting{../PROGS/Serveurs-Web/serveur-web-threads.c}


\subsection{Discussion de la solution}

Les inconvénients de cette solution sont symétriques de ses avantages.
Les processus légers partageant une grande partie leur espace mémoire, le
crash d'un processus léger risque d'emporter le reste du serveur.


On peut utiliser le même mécanisme de pool de processus'' avec des
processus classiques. La difficulté technique réside dans la
transmission le descripteur résultant de l'\texttt{accept()} du
serveur vers un processus fils. Dans la première solution
(\texttt{fork()} pour chaque connexion) le fils est lancé \emph{après}
l'\texttt{accept()}, et peut donc hériter du descripteur. Dans le cas
d'un préchargement de processus fils, ce n'est plus possible.



Pour ce faire, on peut utiliser le mécanisme (assez baroque) 
de transmission des \emph{informations
de service} (\emph{Ancillary messages} à travers une liaison par 
\emph{socket Unix} entre deux processus : des options convenables
de \texttt{sendmsg()}
permettent de faire passer un ensemble de descripteurs de fichiers
d''un processus à un autre (qui tournent sur la même machine, puisqu'ils
communiquent par un socket Unix).
% Voir exemple à la fin de cette section.



\section{Parties communes aux deux serveurs}

\subsection{Déclarations et entêtes de fonctions}

\source
\lstinputlisting{../PROGS/Serveurs-Web/constantes.h}


\subsection{Les fonctions réseau}
\begin{itemize}
\item  \texttt{serveur\_tcp()} : création du socket du serveur TCP.
\item \texttt{attendre\_client()}
\end{itemize}

 
\source
\lstinputlisting{../PROGS/Serveurs-Web/reseau.c}


\subsection{Les fonctions de dialogue avec le client}
\begin{itemize}
\item \texttt{dialogue\_client()} : lecture et traitement de la requête d'un client
\item \texttt{envoyer\_document()},
\item \texttt{document\_non\_trouve()},
\item \texttt{requete\_invalide()}.
\end{itemize}

 
\source
\lstinputlisting{../PROGS/Serveurs-Web/traitement-client.c}


\subsection{Exercices, extensions...}

\textbf{Exercice : } modifier \texttt{traitement-client} pour
qu'il traite le cas des répertoires. Il devra alors afficher
le nom des objets de ce répertoire, leur type, leur taille
et un lien.

\textbf{Exercice : } Utiliser le mécanisme de transmission
de descripteur (voir exemple plus loin) pour réaliser un serveur
à processus préchargés.


\appendix

\chapter{Transmission d'un descripteur}

L'exemple ci-dessous montre comment transmettre un descripteur
d'un processus à un autre.

\source
\lstinputlisting{../PROGS/Divers/passfd.c}



\chapter{Documentation}


Les documents suivants ont été très utiles (il y a longtemps) pour la rédaction
de ce texte et la programmation des exemples :

\begin{itemize}
  
\item \emph{Unix Frequently Asked Questions}
  \url{http://www.faqs.org/faqs/unix-faq/faq/}


\emph{HP-UX Reference, volume 2 (section 2 and 3, C programming routines).}
HP-UX release 8.0, Hewlett-Packard. (Janvier 1991).

\item 
\emph{Advanced Unix Programming}, Marc J. Rochkind, Prentice-Hall Software Series
(1985).
ISBN 0-13-011800-1.

\item \emph{Linux online man pages} 

\item \emph{The GNU C Library Reference Manual}, Sandra Loosemore,
Richard M. Stallman, Roland McGrath, and Andrew Oram. Edition 0.06,
23-Dec-1994, pour version 1.09beta. Free Software Foundation,
ISBN 1-882114-51-1.

\item 
\emph{What is multithreading ?}, Martin McCarthy, Linux Journal 34, 
Février 1997, pages 31 à 40.

\item \emph{Systèmes d'exploitation distribués}, Andrew Tanenbaum,
InterEditions 1994, ISBN 2-7296-0706-4. 

\item  
Page Web de Xavier Leroy sur les threads :
\url{http://pauillac.inria.fr/~xleroy/linuxthreads}
\end{itemize}

Le standard C 18 :

\begin{itemize}
  \item \emph{C18 Draft Standard} \url{http://www.open-std.org/jtc1/sc22/wg14/www/docs/n2479.pdf}
\end{itemize}


\printindex

\end{document}




