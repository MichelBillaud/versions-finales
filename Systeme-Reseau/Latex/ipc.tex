
\section{Les mécanismes IPC System V}
  
\index{IPC,Inter Process Communication}
  
Les mécanismes de communication entre processus (\emph{InterProcess
  Communication}, ou \emph{IPC}), issus d'Unix System V ont été repris
dans de nombreuses variantes d'Unix. Il y a 3 mécanismes, permettant
le partage d'informations entre processus tournant sur une même machine :

\begin{itemize}
\item  les segments, permettant de partager des zones de mémoire
\item  les sémaphores, qui fournissent un moyen d'en contrôler l'accès,
\item  les files de messages,
\end{itemize}


Ces  objets sont identifiés par des \emph{clés}.

\index{Options de compilation!\_X\_OPEN\_SOURCE}

Pour compiler ces programmes, ajoutez l'option
\verb+-D_XOPEN_SOURCE=700+ à la compilation

\section{\texttt{ftok()} constitution d'une clé }

\extrait
\begin{lstlisting}
# include <sys/types.h>
# include <sys/ipc.h>

key_t ftok ( char *pathname, char project )
\end{lstlisting}


La fonction \texttt{ftok()} constitue une clé à partir d'un chemin d'accès
polet d'un caractère indiquant un projet''. Plutôt que de risquer une
explication abstraite, étudions deux cas fréquents :
\begin{itemize}
\item On dispose d'un logiciel commun dans \texttt{/opt/jeux/OuiOui}. 
Ce logiciel utilise deux objets partagés. On pourra utiliser les clés
\texttt{ftok("/opt/jeux/OuiOui",'A')} et
\texttt{ftok("/opt/jeux/OuiOui",'B')}.
Ainsi tous les processus de ce logiciel se réfèreront aux mêmes objets
qui seront partagés entre tous les utilisateurs.
\item On distribue un exemple aux étudiants, qui le recopient chez eux
et le font tourner.
On souhaite que les processus d'un même étudiant communiquent entre eux,
mais qu'ils n'interfèrent pas avec d'autres. On basera donc la clé
sur une donnée personnelle, par exemple le répertoire d'accueil, avec les
clés
\texttt{ftok(getenv("HOME"),'A')} et
\texttt{ftok(getenv("HOME"),'B')}.

\end{itemize}


\section{Mémoires partagées}

Ce mécanisme permet à plusieurs programmes de partager des \emph{segments
mémoire}. Chaque segment mémoire est identifié, au niveau du système,
par une \texttt{clé} à laquelle correspond un \emph{identifiant}. Lorsqu'un
segment est \emph{attaché} à un programme, les données qu'il contient
sont accessibles en mémoire par l'intermédiaire d'un pointeur.

\index{memoire partagee@mémoire partagée!shmget()}
\index{memoire partagee@mémoire partagée!shmat()}
\index{memoire partagee@mémoire partagée!shmdt()}
\index{memoire partagee@mémoire partagée!shmctl()}


\extrait
\begin{lstlisting}
#include <sys/ipc.h>
#include <sys/shm.h>

int shmget(key_t key, int size, int shmflg);
char *shmat (int shmid, char *shmaddr, int shmflg )
int shmdt (char *shmaddr)
int shmctl(int shmid, int cmd, struct shmid_ds *buf);
\end{lstlisting}



La fonction \texttt{shmget()} donne l'identifiant du segment ayant la
clé \texttt{key}. Un nouveau segment (de taille \texttt{size}) est
créé si \texttt{key} est \texttt{IPC\_PRIVATE}, ou bien si les
indicateurs de \texttt{shmflg} contiennent \texttt{IPC\_CREAT}.
Combinées, les options \texttt{IPC\_EXCL | IPC\_CREAT} indiquent que
le segment ne doit pas exister préalablement.  Les bits de poids
faible de \texttt{shmflg} indiquent les droits d'accès.




\texttt{shmat()} attache le segment \texttt{shmid} en mémoire, avec les droits
spécifiés dans \texttt{shmflag} (\texttt{SHM\_R, SHM\_W, SHM\_RDONLY}). 
\texttt{shmaddr}
précise où ce segment doit être situé dans l'espace mémoire
(la valeur \texttt{NULL} demande un placement automatique). 
\texttt{shmat()} renvoie
l'adresse où le segment a été placé.


\texttt{shmdt()} ``libère'' le segment. \texttt{shmctl()} permet diverses 
opérations, dont la destruction d'une mémoire partagée (voir exemple).

\textbf{Exemple} (deux programmes):

Le producteur :

\source
\lstinputlisting{../PROGS/IPC/prod.c}


Le consommateur :


\source
\lstinputlisting{../PROGS/IPC/cons.c}



\textbf{Question} : le second programme n'affiche pas forcément des
informations cohérentes. Pourquoi ? Qu'y faire ? 



\textbf{Problème} : écrire deux programmes qui partagent deux variables
\texttt{i, j}. Voici le pseudo-code:

\extrait
\begin{lstlisting}
processus P1                 	processus P2
| i=0 j=0			| tant que i==j 
| repeter indefiniment		|   faire rien
|   i++	j++			| ecrire i
fin				fin
\end{lstlisting}

Au bout de combien de temps le processus P2 s'arrête-t-il ?
Faire plusieurs essais. 


\textbf{Exercice} : la commande \texttt{ipcs} affiche des informations sur les
segments qui existent. Ecrire une commande qui permet d'afficher le contenu 
d'un segment (on donne le \emph{shmid} et la longueur en paramètres).
 
\section{Sémaphores}

\index{IPC!sémaphores}

\index{semaphore IPC@sémaphore IPC!semget}
\index{semaphore IPC@sémaphore IPC!semop}
\index{semaphore IPC@sémaphore IPC!semctl}

\extrait
\begin{lstlisting}
#include <sys/types.h>
#include <sys/ipc.h>
#include <sys/sem.h>   

int semget(key_t key, int nsems, int semflg )
int semop(int semid, struct sembuf *sops, unsigned nsops)
int semctl(int semid, int semnum, int cmd, union semun arg )
\end{lstlisting}


Les opérations System V travaillent en fait sur des tableaux de sémaphores 
généralisés (pouvant évoluer par une valeur entière quelconque). 


La fonction \texttt{semget()} demande à travailler sur le 
sémaphore généralisé qui est identifié par la clé \texttt{key} (même notion que pour les clés
des segments partagés) et qui contient \texttt{nsems} sémaphores individuels.
Un nouveau sémaphore est créé, avec les droits donnés par les 9 bits de 
poids faible de \texttt{semflg}, si
\texttt{key} est \texttt{IPC\_PRIVATE}, ou si
\texttt{semflg} contient \texttt{IPC\_CREAT}. 



\texttt{semop()} agit sur le sémaphore \texttt{semid} en appliquant
simultanément à plusieurs sémaphores individuels les actions décrites
dans les \texttt{nsops} premiers éléments du tableau
\texttt{sops}. Chaque \texttt{sembuf} est une structure de la forme:


\extrait
\begin{lstlisting}
struct sembuf
{ 
  ...
  short sem_num;  /* semaphore number: 0 = first */
  short sem_op;   /* semaphore operation */
  short sem_flg;  /* operation flags */
  ...
}
\end{lstlisting}

\texttt{sem\_flg} est une combinaison d'indicateurs qui peut contenir
\texttt{IPC\_NOWAIT} et  \texttt{SEM\_UNDO} (voir manuel). Ici nous supposons
que \texttt{sem\_flg} est 0. 


\texttt{sem\_num} indique le numéro du sémaphore individuel sur lequel 
porte l'opération. \texttt{sem\_op} est un entier destiné (sauf si il est nul) 
à être ajouté à la valeur courante \emph{semval} du sémaphore. 
L'opération se bloque si \verb/sem_op + semval < 0/.

 
\textbf{Cas particulier : } si  \emph{sem\_op} est 0, l'opération est bloquée 
tant que  \texttt{semval}
est non nul.


Les valeurs des sémaphores ne sont  mises à jour que lorsque
aucun d'eux n'est bloqué. 


\texttt{semctl} permet de réaliser diverses opérations sur les sémaphores,
selon la commande demandée. En particulier, on peut fixer le
\texttt{n}-ième sémaphore à la valeur \texttt{val} en faisant :

\extrait
\begin{lstlisting}
semctl(sem,n,SETVAL,val);
\end{lstlisting}


\textbf{Exemple: } primitives sur les sémaphores traditionnels.



\source
\lstinputlisting{../PROGS/IPC/sem.c}



\textbf{Exercice : } que se passe-t-il si on essaie d'interrompre \texttt{semop()} ?

\textbf{Exercice : } utilisez les sémaphores pour ``sécuriser'' l'exemple
présenté sur les mémoires partagées.
 
\section{Files de messages}

\index{IPC!files de messages}

Ce mécanisme permet l'échange de messages par des processus. Chaque message
possède un \emph{corps} de longueur variable, et un \emph{type} (entier 
strictement positif) qui peut servir à préciser la nature des informations
contenues dans le corps. 

Au moment de la réception, on peut choisir de sélectionner les messages
d'un type donné. 

\index{file de messages IPC!msgget()}
\index{file de messages IPC!msgsnd()}
\index{file de messages IPC!msgrcv()}
\index{file de messages IPC!msgctl()}

\extrait
\begin{lstlisting}
#include <sys/types.h>
#include <sys/ipc.h>
#include <sys/msg.h>

int msgget (key_t key, int msgflg)
int msgsnd (int msqid, struct msgbuf *msgp, int msgsz, int msgflg)
int msgrcv (int msqid, struct msgbuf *msgp, int msgsz, 
                       long msgtyp, int msgflg)
int msgctl ( int msqid, int  cmd, struct msqid_ds *buf )
\end{lstlisting}


\texttt{msgget()} demande l'accès à (ou la création de) la file de
message avec la clé \texttt{key}. \texttt{msgget()} retourne la valeur
de l'\emph{identificateur de file}. 
 \texttt{msgsnd()} envoie un
message dans la file \texttt{msqid}. Le \emph{corps} de ce message
contient \texttt{msgsz} octets, il est placé, précédé par le
\emph{type} dans le tampon pointé par \texttt{msgp}. Ce tampon de la
forme:

\extrait
\begin{lstlisting}
struct msgbuf {
     long mtype;     /* message type, must be > 0 */
     char mtext[...] /* message data */
};
\end{lstlisting}


\texttt{msgrcv()} lit dans la file un message d'un type donné (si
\verb/type < 0/) ou indifférent (si \texttt{type==0}), et le
place dans le tampon pointé par \texttt{msgp}. La taille du corps ne
pourra excéder \texttt{msgsz} octets, sinon il sera
tronqué. \texttt{msgrcv()} renvoie la taille du corps du message. 


\textbf{Exemple.} Deux programmes, l'un pour envoyer des messages (lignes
de texte) sur une file avec un type donné, l'autre pour afficher les
messages reçus.


 
\source
\lstinputlisting{../PROGS/IPC/snd.c}



 
\source
\lstinputlisting{../PROGS/IPC/rcv.c}

