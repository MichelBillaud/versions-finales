
\section{Les sockets}

\label{sockets}

\index{socket!prise}
\index{prise!socket}

Les \emph{sockets} (\emph{prises}) sont une interface générique
pour la communication entre processus, par divers moyens.


On s'en sert pour la communication à travers les réseaux (Internet ou
autres), mais ils sont utilisables également pour la communication
locale entre processus qui s'exécutent sur une même machine (comme les
tuyaux déjà vus, et les files de messages IPC que nous verrons plus
loin).

Dans ce chapitre, nous montrons comment s'en servir pour la
communication locale, la communication sur le réseau sera vue plus
loin.


\subsection{Création d'un socket}

\index{socket!création}

\extrait
\begin{lstlisting}
#include <sys/types.h>
#include <sys/socket.h>

int socket(int domain, int type, int protocol);
\end{lstlisting}

\index{socket!domaine}
\index{socket!type}
\index{socket!type!local}
\index{socket!type!internet}

\index{socket!protocole}

La fonction \texttt{socket()} crée une nouvelle prise et retourne un
descripteur qui servira ensuite aux lectures et écritures. Le
paramètre \texttt{domain} indique le domaine de communication''
utilisé, qui est \texttt{PF\_LOCAL} ou (synonyme) \texttt{PF\_UNIX} pour
les communications locales.\footnote{ Le domaine définit une famille
de protocoles (protocol family) utilisables. Autres familles
disponibles : \texttt{PF\_INET} protocoles internet IPv4,
\texttt{PF\_INET6} protocoles internet IPv6, \texttt{PF\_IPX} protocoles
Novel IPX, \texttt{PF\_X25} protocoles X25 (ITU-T X.25 / ISO-8208),
\texttt{PF\_APPLETALK}, etc.}

\index{socket!protocole!communication par datagramme}
\index{socket!protocole!communication par flot}

Le \emph{type} indique le style de communication désiré
entre les deux participants. Les deux styles principaux sont
\begin{itemize}
\item \texttt{SOCK\_DGRAM} : communication par messages (blocs contenant des
octets) appelés \emph{datagrammes}
\item \texttt{ SOCK\_STREAM} : la communication se fait par un flot 
(bidirectionnel) d'octets une fois que la connexion est établie.
\end{itemize}

\textbf{Fiabilité}: la fiabilité des communication par datagrammes est
garantie pour le domaine local, mais ce n'est pas le cas pour les
``domaines réseau'' que nous verrons plus loin : les datagrammes
peuvent être perdus, dupliqués, arriver dans le désordre etc. et c'est
au programmeur d'application d'en tenir compte. Par contre la
fiabilité des ``streams'' est assurée par les couches basses du
système de communication, évidemment au prix d'un surcoût
(numérotation des paquets, accusés de réception, temporisations,
retransmissions, etc).


Enfin, le paramètre \texttt{protocol} indique le protocole sélectionné.
La valeur \texttt{0} correspond au protocole par défaut pour le domaine
et le type indiqué.

\subsection{Adresses}

\index{socket!adresse!socket local}
La fonction \texttt{socket()} crée un socket anonyme. Pour qu'un autre
processus puisse le désigner, il faut lui associer un \emph{nom} par
l'intermédiaire d'une \emph{adresse} contenue dans une structure
\texttt{sockaddr\_un} :



\extrait
\begin{lstlisting}
#include <sys/un.h>

struct sockaddr_un {
        sa_family_t  sun_family;              /* AF_UNIX */
        char         sun_path[UNIX_PATH_MAX]; /* pathname */
};
\end{lstlisting}

Ces adresses sont des chemins d'accès dans l'arborescence des fichiers et
répertoires.


Exemple :

\extrait
\begin{lstlisting}
#include <sys/types.h>
#include <sys/socket.h>
#include <sys/un.h>

socklen_t longueur_adresse;

struct sockaddr_un  adresse;

adresse.sun_family      = AF_LOCAL;
strcpy(adresse.sun_path, "/tmp/xyz");
longueur_adresse = sizeof adresse;
\end{lstlisting}

\index{socket!bind()}

L'association d'une adresse à un socket se fait par \texttt{bind()}
(voir exemples plus loin).

\section{Communication par datagrammes}

Dans l'exemple développé ici, un serveur affiche les datagrammes
émis par les clients.

\subsection{La réception de datagrammes}

\index{bind(socket,adresse,longueur)}
\index{socket!bind()}
\index{socket!recvfrom()}
\index{recvfrom()}

\extrait
\begin{lstlisting}
#include <sys/types.h>
#include <sys/socket.h>

int  bind(int  sockfd, struct sockaddr *my_addr, 
                       socklen_t addrlen);

int  recvfrom(int  s,  void  *buf,  size_t len, int flags,
              struct sockaddr *from, socklen_t *fromlen);

\end{lstlisting}


La fonction \texttt{bind()} permet de nommer le socket de réception.
La fonction \texttt{recvfrom()} attend l'arrivée d'un datagramme qui
est stocké dans les \texttt{len} premiers octets du tampon
\texttt{buff}.  Si \texttt{from} n'est pas \texttt{NULL}, l'adresse du
socket émetteur\footnote{qui peut servir à expédier une réponse} est
placée dans la structure pointée par \texttt{from}, dont la longueur
maximale est contenue dans l'entier pointé par \texttt{fromlen}.


Si la lecture a réussi, la fonction retourne le nombre d'octets du
message lu, et la longueur de l'adresse est mise à jour.


Le paramètre \texttt{flags} permet de préciser des options.


\source
\lstinputlisting{../PROGS/LocalDatagrammes/serveur-dgram-local.c}


\subsection{Émission de datagrammes}

\index{socket!sendto()}
\index{sendto()}

\extrait
\begin{lstlisting}
#include <sys/types.h>
#include <sys/socket.h>

int  sendto(int s, const void *msg, size_t len, int flags,
            const struct sockaddr *to, socklen_t tolen);
\end{lstlisting}


\texttt{sendto} utilise le descripteur de socket \texttt{s} pour
envoyer le message formé des \texttt{len} premiers octets de
\texttt{msg} à l'adresse de longueur \texttt{tolen} pointée par
\texttt{to}.


Le même descripteur peut être utilisé pour des envois à des
adresses différentes.


\source
\lstinputlisting{../PROGS/LocalDatagrammes/client-dgram-local.c}


\textbf{Exercice} : modifier les programmes précédents pour que le
serveur envoie une réponse qui sera affichée par le client\footnote{
Pensez à attribuer un nom au socket du client pour que le serveur puisse lui
répondre, par exemple avec l'aide de
la fonction \texttt{tempnam()}.}. 


\subsection{Émission et réception en mode connecté}

\index{socket!mode connecté}
\index{connect(socket,addresse,longueur)}
\index{send(socket,tampon,longueur,flags)}
\index{recv(socket,tampon,longueur,flags)}

\extrait
\begin{lstlisting}
#include <sys/types.h>
#include <sys/socket.h>

int  connect(int sockfd, 
             const struct sockaddr *serv_addr,
             socklen_t addrlen);

int send(int s, const void *msg, size_t len, int flags);
int recv(int s, void *buf, size_t len, int flags);
\end{lstlisting}


Un émetteur qui va envoyer une série de messages au même destinataire
par le même socket peut faire préalablement un \texttt{connect()} pour
indiquer une destination par défaut, et employer ensuite
\texttt{send()} à la place de \texttt{sendto()}.


Le récepteur qui ne s'intéresse pas à l'adresse de l'envoyeur peut
utiliser \texttt{recv()}.  


\textbf{Exercice} : modifier les programmes précédents pour utiliser 
\texttt{recv()} et \texttt{send()}.

\section{Communication par flots}

Dans ce type de communication, c'est une suite d'octets qui est
transmises (et non pas une suite de messages comme dans la
communication par datagrammes). 


Les sockets locaux de ce type sont créés par 

\extrait
\begin{lstlisting}
int fd = socket(PF_LOCAL,SOCK_STREAM,0);
\end{lstlisting}


\section{Architecture client-serveur}

La plupart des applications communicantes sont conçues selon
une architecture client-serveur, asymétrique, 
dans laquelle un processus serveur est contacté par plusieurs
clients.


\index{socket!read(socket,tampon,taille)}
\index{socket!write(socket,tampon,taille)}
\index{read!read(socket,tampon,taille)}
\index{write!write(socket,tampon,taille)}

Le client crée un socket (\texttt{socket()}), qu'il met en relation 
(par \texttt{connect()}) avec celui du serveur. Les données sont
échangées par \texttt{read()}, \texttt{write()} ... et le socket est
fermé par \texttt{close()}. 


\index{connect(socket,adresse,longueur)}
\index{socket!connect(socket,adresse,longueur)}
\extrait
\begin{lstlisting}
#include <sys/types.h>
#include <sys/socket.h>

int  connect(int sockfd, 
             const struct sockaddr *serv_addr,
             socklen_t addrlen);
\end{lstlisting}


\index{listen(socket,nombre)}
\index{socket!listen(socket,nombre)}
Du côté serveur : un socket est créé, et une adresse lui est associée
(\texttt{socket()} + \texttt{bind()}). Un \texttt{listen()} prévient
le système que ce socket recevra des demandes de connexion, et précise
le nombre de connexions que l'on peut mettre en file d'attente.

\index{accept(socket)}
\index{socket!accept(socket)}


Le serveur attend les demandes de connexion par la fonction
\texttt{accept()} qui retourne un descripteur, lequel permet la
communication avec le client.



\extrait
\begin{lstlisting}
#include <sys/types.h>
#include <sys/socket.h>

int  bind(int             sockfd, 
          struct sockaddr *my_addr, 
          socklen_t       addrlen);

int  listen(int s, int backlog);       

int  accept(int s,  
            struct  sockaddr  *addr,  
            socklen_t         *addrlen);
\end{lstlisting}


Remarque : il est possible ne fermer qu'une ``moitié'' de socket :
\texttt{shutdown(1)} met fin aux émissions (causant une ``fin de fichier''
chez le correspondant), \texttt{shutdown(0)} met fin aux réceptions.

\index{shutdown(socket,indic)}
\index{socket!shutdown(socket,indic)}

\extrait
\begin{lstlisting}
#include <sys/socket.h>

int shutdown(int s, int how);
\end{lstlisting}



\textbf{Le client :}

\source
\lstinputlisting{../PROGS/LocalStream/client-stream.c}


\textbf{Le serveur} est programmé ici de façon atypique, puisqu'il traite 
 qu'il traîte une seule communication à la fois. Si le client fait traîner
les choses, les autres clients en attente resteront bloqués longtemps.



\source
\lstinputlisting{../PROGS/LocalStream/serveur-stream.c}


Dans une programmation plus classique, le serveur lance un processus
(par \texttt{fork()}, voir plus loin) dès qu'une connexion est établie, et 
délègue le traitement de la connexion
à ce processus.


Une autre technique est envisageable pour traiter plusieurs connexions par un 
processus unique : le serveur maintient une liste de descripteurs 
ouverts, et fait une boucle autour d'un 
\texttt{select()}, en attente de données venant
\begin{itemize}
\item soit du descripteur ``principal'' ouvert par le serveur. Dans ce cas il
effectue ensuite un \texttt{accept(...)} qui permettra d'ajouter un
nouveau client à la liste.
\item soit d'un des descripteurs des clients, et il traite alors les données
venant de ce client (il l'enlève de la liste en fin de communication).
\end{itemize}
Cette technique conduit à des performances nettement supérieures aux
serveurs multiprocessus ou multithreads (pas de temps perdu à lancer
des processus), au prix d'une programmation qui oblige le programmeur
à gérer lui-même le ``contexte de déroulement'' de chaque processus.



\source
\lstinputlisting{../PROGS/LocalStream/serveur-stream-monotache.c}

\index{socket!socketpair())}
\index{socketpair()}

\section{\texttt{socketpair()}}

\label{socketpair}

La fonction \texttt{socketpair()} construit une paire de sockets
locaux, bi-directionnels,  reliés l'un à l'autre.


\extrait
\begin{lstlisting}
#include <sys/types.h>
#include <sys/socket.h>

int socketpair(int d, int type, int protocol, int sv[2]);
\end{lstlisting}


Dans l'état actuel des implémentations, le paramètre \texttt{d}
(domaine) doit être égal à \texttt{AF\_LOCAL}, et \texttt{type} à
\texttt{SOCK\_DGRAM} ou \texttt{SOCK\_STREAM}, avec le protocole par
défaut (valeur \texttt{0}).


Cette fonction remplit le tableau \texttt{sv[]} avec les descripteurs
de deux sockets du type indiqué. Ces deux sockets sont reliés entre
eux et bidirectionnels : ce qu'on écrit sur le descripteur
\texttt{sv[0]} peut être lu sur \texttt{sv[1]}, et réciproquement.


On utilise \texttt{socketpair()} comme \texttt{pipe()}, pour la
communication entre descendants d'un même processus \footnote{au sens
large, ce qui inclue la communication d'un processus avec un de ses
fils}. \texttt{socketpair()} possède deux avantages sur
\texttt{pipe()} : la possibilité de transmettre des datagrammes, et la
bidirectionnalité.


\textbf{Exemple :}


\source
\lstinputlisting{../PROGS/LocalDatagrammes/paire-send.c}


