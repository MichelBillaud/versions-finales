
\chapter{Supplément~: Notions de Sémantique}


%	6.1 Quelques généralités
%	6.2 Notion d'Environnement
%	6.3 Sémantique de l'Affectation
%	6.4 Sémantique de la Composition Séquentielle
%	6.5 Sémantique de la Répétition
%	6.6 Conclusion



\section{Quelques généralités}


La \emph{sémantique} est 
un terme de linguistique qui désigne 
\begin{citation}``l'étude du 
sens (ou contenu) des mots et des énoncés, par opposition à l'étude
des formes (morphologie) et à celle des rapports entre les termes dans
la phrase'' (dictionnaire Lexis).
\end{citation}

A la différence des langues naturelles, les langages de programmation
sont des objets linguistiques relativement simples~:

\begin{itemize}
\item Ayant une syntaxe assez rigide (il faut qu'un programme (le
compilateur) puisse analyser une "phrase" sans trop de peine) ;
\item Possédant, par rapport aux langues naturelles, une sémantique
volontairement très pauvre (une instruction d'un programme doit avoir
un sens clairement défini).
\end{itemize}

\subsection{Aspects Syntaxiques d'un langage de programmation}

Voici par exemple la description de la grammaire d'un petit langage de
programmation~:

\begin{verbatimtab}
<programme>    ::= <entête> <bloc>
<bloc>         ::= début <liste d'instructions> fin
<liste d'instructions> 
               ::=   rien
                 |   <instruction> ; <liste d'instructions>
<instruction>  ::= <affectation>
                 |   <bloc>
                 |   <boucle tantque>
                 |   <alternative>
                      ....
<affectation>  ::= <variable> := <expression>
<expression>   ::= <constante>
                 |   <variable>
                 |   <expression> + <expression>
                 |   <expression> * <expression>
                      ....
<tantque>      ::= tantque <condition> faire <instruction>
<alternative>  ::= si <condition> alors <instruction> sinon <instruction>
<condition>    ::= <expression>  <  <expression>
                 |  ....
                 |  <condition> et <condition>
                 | ...
\end{verbatimtab}
En Hope, nous pouvons définir des objets assez ressemblants~:

\begin{verbatimtab}
type id	 	== list(char);

data expr 	== nombre(num)
		++ variable(id)
		++ expr plus expr
		++ expr moins expr
		++ .... ;

data condition	== expr egal expr
		++ expr infegal expr
		++ ...
		++ condition et condition
		++ non(condition)
		++ ...
			
data inst	== id~:= expr
		++ bloc(list(inst))
		++ sialorssinon( condition X inst X inst )
		++ tantque( condition X inst )
		++ ... 

\end{verbatimtab}

\subsection{Sémantique d'un programme}

En informatique, la sémantique est l'étude de la ``signification'' des
programmes. Comment définir formellement la signification d'un
programme ? Prenons un exemple~:

\begin{verbatimtab}
programme m
donnée en entrée : x,y (entiers positifs)
résultat en sortie : r (entier)
début
     r := 0;
     tantque x>0 faire 
          début 
               x:=x-1; 
               r:=r+y; 
          fin
fin
\end{verbatimtab}

La signification de ce programme est  une certaine
fonction $S(m)$ (S pour Sémantique) qui, à deux entiers $x$ et $y$, fait
correspondre un résultat $r$ en fin de traitement~:

$$\begin{array}{l}
               S(m): N \times N \rightarrow N \\
               S(m)(n,p) = n \times p
\end{array}$$

Plus généralement, tout programme $p$ a une signification qui est
représentée par une certaine fonction $S(p) : D \rightarrow R$. Cette fonction
est souvent une fonction partielle, dans le sens o\`u le calcul de $p$
sur certaines données peut ne pas conduire à un résultat (le programme
boucle, il y a eu une division par zéro, ...) Le domaine de
définition de $S(p)$ est donc naturellement l'ensemble des données pour
lesquelles le calcul se termine normalement.


\paragraph{Remarques~:}

\begin{itemize}
\item On note fréquemment $(E \rightarrow F)$ l'ensemble des fonctions
partielles qui vont d'un ensemble E dans un ensemble F. On pourra donc
écrire indifféremment $$f : E \rightarrow F$$  ou $$f \in  (E
\rightarrow F)$$

\item Réfléchissons un peu sur le statut de $S$ dans la notation 
$$S(p) : D \rightarrow R$$
$ S$ est une fonction qui associe à un programme $p$ sa
signification, qui est elle-m\^eme une fonction qui fabrique un
résultat à partir d'une donnée.

Notons $P$ l'ensemble des programmes, on a alors~:
          $$    S : P \rightarrow (D \rightarrow R)    $$

Le problème qui se pose est de
savoir comment construire cette fonction $S$, c'est-à-dire calculer la
\emph{signification d'un programme} sans avoir besoin de le faire tourner sur
toutes les données possibles. 
\end{itemize}

\paragraph{Nota:} Nous limitons ici à une classe très simple de programmes
impératifs: programmes structurés, pas d'entrées-sorties, pas de
procédures.


\section{Notion d'Environnement}

Le petit programme donné en exemple va nous permettre de préciser
quelques notions importantes. Tout d'abord la notion d'\emph{environnement}~:
un environnement c'est l'association, à un moment donné de l'exécution
de ce programme, de certaines valeurs aux variables du programme.

Par exemple si on lance le programme pour $x=2$ et $y=3$, on a au départ 
l'environnement initial      	$$e_1 = \{x \equiv 2 , y \equiv 3\}$$
puis après l'initialisation  $$e_2 = \{ x \equiv 2, y \equiv 3 , r \equiv 0\}$$
et ensuite (dans la boucle)  	$$e_3 = \{ x \equiv 1 , y \equiv 3 , r \equiv 0\}$$
puis $$e_4 = \{x \equiv 1 , y \equiv , r \equiv 3 \}$$
etc.
à la fin on aura   $$	e_{48} : \{ x \rightarrow 0 , y \rightarrow 3 , r \rightarrow 6 \}$$

Deux choses à remarquer~:

\begin{itemize}
\item Un environnement peut \^etre vu comme une fonction qui part de
l'ensemble $Id$ des identificateurs du programme (ici $Id = {x,y,r}$) et
qui va dans l'ensemble des $V$ des valeurs (ici des entiers).  

\item
Lorsqu'une instruction, (ou une séquence d'instructions) s'exécute,
elle modifie l'environnement. On peut donc décrire la signification
d'une instruction $i$ (ou d'une séquence par une fonction $S(i)$ qui
associe à un environnement (celui o\`u on est juste avant d'exécuter
$i$) un autre environnement (celui d'après l'exécution de $i$).  
$$S(i) : E
\rightarrow E $$
Par exemple $S( x:=x-1 ) (e_2) = e_3$
\end{itemize}

En pratique, on représente souvent un environnement par une liste de
		doublets (identificateur, valeur).  

\begin{verbatim}
type env == list (chaine X num);
\end{verbatim}

Il nous faut alors quelques fonctions auxiliaires

\begin{alltt}
dec chercher : chaine X env  -> num ;
--- chercher ( id , [] ) <= 0 ;		! normalement, cas d'erreur
--- chercher ( id,(i,v)::r ) <= if id=i 
				then v 
				else chercher(id,r);

dec associer : chaine X num X env  ->  env ;
		....

\end{alltt}

La sémantique sera représentée par une fonction~:

\begin{verbatim}
dec S : inst -> (env -> env) ;
\end{verbatim}


\section{Sémantique de l'Affectation}

Considérons l'instruction ``\verb+r:=r+y+''. Nous voulons l'exécuter
dans un certain environnement $e$. Comment fait on ? Tout simplement on
additionne les valeurs de $r$ et de $y$ (prises dans l'environnement $e$) et
on modifie la valeur de la variable $r$ dans $e$; plus exactement on crée
un environnement $e'$ semblable à $e$, sauf en ce qui concerne la valeur
de $r$.

Nous avons besoin de définir une nouvelle fonction sémantique $V$
(comme valeur) qui renvoie la valeur d'une expression (si cette valeur
existe) dans un certain environnement.

par exemple   $V(r+y)(e) = 3$

On définit donc~:
\begin{verbatim}
dec V : expr -> (env -> num) ;
\end{verbatim}

Et, par induction sur la structure des expressions~:

\begin{verbatim}
--- V(nombre(n))(e) 	<= n;
--- V(variable(id))(e) 	<= chercher(id,e);
--- V( a plus b ) (e)	<= V(a)(e) + V(b)(e)
\end{verbatim}
etc ...


En général, si l'on a une affectation \verb/var := expr/, la fonction
sémantique $S(\mbox{\texttt{var := expr}})$ qui lui est associée est
définie de la manière suivante~:

\begin{verbatim}
--- S(id := expression)(e)  <= let v == V(expression)(e)
				in modifier(id,v,e);
\end{verbatim}


\section{Sémantique de la Composition Séquentielle}

Lorsqu'on sait faire une instruction, on peut raisonnablement essayer d'en faire deux à la suite l'une de l'autre ! Interrogeons-nous sur la séquence d'instructions ``\verb/x:=x-1; r:=r+y/''. 

Supposons que nous l'exécutions en partant de l'environnement
               $e$. Aprés la première instruction on se retrouve dans
               l'environnement $e' = S(\mbox{\texttt{x:=x-1}})(e)$; puis la seconde nous fait
               passer dans $e" = S(\mbox{\texttt{r:=r+y}})(e')$. Donc

$$\begin{array}{rl}
S(\mbox{\tt x:=x-1;r:=r+y})(e) &= S(\mbox{\tt r:=r+y})(S(\mbox{\tt x:=x-1})(e)) \\
&=               (S(\mbox{\tt r:=r+y}) \circ S(\mbox{\tt x:=x-1}))(e)
\end{array}$$

Pour résumer, si on a deux instructions $i$  et $i'$ ~:
$$                         S(i ; i' ) =  S(i' ) \circ S(i)S                         $$


Ici la composition séquentielle des instructions n'appara\^{\i}t que
dans la structure de bloc, et donc pour des listes
d'instructions. Définissons une fonction auxiliaire $Slist$ (sémantique
d'une liste d'instructions)~:

\begin{verbatim}
dec Slist : list(inst) -> (env -> env);
--- Slist ( [] )(e) 		<= e;
--- Slist ( p::r ) ( e )	<= Slist(r) (S (p)(e));
\end{verbatim}

On pourra alors poser~:

\begin{verbatim}
--- S ( bloc (l) ) (e) 	<= Slist(l);
\end{verbatim}


\section{Sémantique de la Répétition}

Examinons maintenant la boucle tant-que. Une boucle tant-que comporte
deux éléments~: une condition et un corps. Pour évaluer une condition,
il nous faut une fonction $B$ (booléenne):

\begin{verbatim}
dec B : condition -> (env -> truval);
--- B( a egal b)(e) <= V(a)(e) = V(b)(e);
 ...
--- B ( c1 et c2 )(e) <= B(c1)(e) and B(c2)(e);
....
\end{verbatim}

Nous savons également définir la signification du corps de la boucle. Comment recoller les morceaux ?

Soit $boucle$ l'instruction tantque <cond> faire <corps>/

pour exécuter $boucle$ dans l'environnement $e$, que fait on ?
\begin{itemize}
\item  on évalue la condition (dans $e$)
\item si la condition est fausse, on s'arr\^ete (l'environnement n'a pas changé)
\item- si elle est vraie, on exécute le corps (l'environnement est
modifié) et on recommence au début (avec ce nouvel environnement
$e'$).
\end{itemize}

Ce qui s'écrit~: 

\begin{verbatim}
--- S(tantque(c,i))(e) <= if B(c)(e)
			then   S(tantque(c,i))( S(i)(e) )
			else   e;
\end{verbatim}


\begin{exercice}
\begin{enumerate}
\item Définir la sémantique de l'alternative~:
``\verb/si <condition> alors <i1> sinon <i2>/''
\item Définir la sémantique de la répétition~:
     ``\verb/répéter <i> jusqu'à <condition>/''
\item Utiliser cet arsenal mathématique pour démontrer que 

    pour tout $e \in E$  avec $e(x)=a$ et $e(y)=b$  ($a,b \in N$)), on a~: 
$$ (S(m)(e))(r) = a \times b$$    
\end{enumerate}
\end{exercice}


