% rubber: module xelatex

\documentclass[twoside]{book}

\usepackage[a4paper,top=2cm, bottom=2cm, left=2.5cm,right=2.5cm]{geometry}

% compiler avec xelatex 
% ou : rubber --module xelatex fichier.tex 

\usepackage{fontspec}
\usepackage[french]{babel}

\usepackage{lmodern}
\usepackage{amssymb}
\usepackage{listings}
\usepackage{moreverb}
\usepackage{alltt}

\columnsep 1.0cm

\newcommand{\barre}{\hspace{0pt} \hrulefill \hspace{0pt}}
\newcommand{\tape}[1]{\underline{#1}}

\newtheorem{exemple}{Exemple}[chapter]
\newtheorem{exercice}{Exercice}[chapter]

\renewenvironment{citation}{\begin{quotation}\em}{\end{quotation}}

\def\siecle#1{\textsc #1\textsuperscript{e}}

\title {Programmation Fonctionnelle en HOPE}
\author{Michel Billaud \\
Département Informatique \\
IUT - Université de Bordeaux}
\date{Révision \today}

\sloppy

\begin{document}
\pagenumbering{roman}
\maketitle

\tableofcontents
% \makeindex




\chapter*{Préface}

\addtocontents{toc}{\contentsline{chapter}{\numberline{ }Préface}{\thepage}}

Ce document a été écrit vers 1991 pour un enseignement de Programmation
Fonctionnelle destiné aux étudiants de $2^e$ année du département informatique
de l'IUT ``A'' de Bordeaux, ainsi qu'à ceux du DEST IOE (Informatique des
Organisations Européennes) organisé conjointement avec le
\emph{Polytechnic} de Sheffield (Grande-Bretagne).

On y retrouve beaucoup des idées pédagogiques développées par Matthew
Love pour le cours de programmation fonctionnelle pour le \emph{MSC of
Computer Science} du Polytechnic, idées que j'ai empruntées lors d'un
séjour à Sheffield en 1990.

Le texte source (en Write pour Windows 2) a été converti en \LaTeX (et
légèrement édité) au mois d'août 1999, avec quelques modifications
très mineures.

Dernières corrections : juillet 2021.

Adresse mail perso de l'auteur : \texttt{michel . billaud @ laposte . net}






\pagenumbering{arabic}


\bibliographystyle{plain}


\chapter{Introduction}


%	1.1 Un peu d'histoire
%	1.2 La programmation imp\'erative 
%	1.3 La programmation d\'eclarative
%	1.4 Quelques langages fonctionnels 
%	1.5 La crise du logiciel
%	1.6 La programmation fonctionnelle, une solution d'avenir ?





\section{Un peu d'histoire}

Les ordinateurs sont, dans leur principe, des machines assez simples
construites autour de quelques types de circuits élémentaires~:
registres, mémoires, additionneurs, décodeurs, etc.

Ces machines exécutent séquentiellement des programmes, qui sont des
suites d'instructions élémentaires enregistrées dans la mémoire
centrale. Les données (également en mémoire) que peuvent manipuler ces
instructions appartiennent à quelques types bien connus~: nombres
entiers ou réels, caractères, adresses, etc.

Cette structure simple, appelée architecture Von Neumann \index{Von
Neumann, machine de} (du nom d'un mathématicien américain d'origine
hongroise qui contribua au développement du concept de programme
enregistré dans les projets ENIAC et EDVAC, vers 1946), convenait bien
aux premières applications des calculateurs (années 40)~: il
s'agissait d'effectuer des suites fastidieuses de calculs répétitifs,
pour établir des tables numériques (par exemple calculs balistiques
sur le ``directeur de tir antiaérien M9'' fabriqué par Bell vers 1942),
décrypter des messages secrets (machines britanniques Robinson et
Colossus), calculs numériques pour la recherche nucléaire (ENIAC),
etc.


\subsection*{Références}

L'étude détaillée du fonctionnement est faite dans le cours de
Première Année de DUT Informatique intitulé ``Architecture des
Systèmes Informatiques'', et on trouvera sans peine des ouvrages sur
ce sujet à la bibliothèque.

Pour une perspective historique, consulter {\em Préhistoire et
Histoire des Ordinateurs} de Robert LIGONNIÈRE (1987), aux éditions
Robert Laffont.


\section{La programmation impérative}

Les programmes écrits à l'époque ne pouvaient guère être
compliqués, ne serait-ce qu'en raison de la très faible capacité des
mémoires centrales (quelques milliers d'octets).  Cette capacité
augmentant naturellement au cours du temps (et par conséquent la
longueur et la complexité des programmes), on s'est avisé qu'il
pouvait être intéressant~:

\begin{itemize}

\item d'écrire des programmes en utilisant des noms mnémotechniques 
pour chacune des instructions de la machine (naissance du langage
d'assemblage), et en donnant des noms symboliques aux
emplacements-mémoire destinés à contenir des valeurs intermédiaires
du calcul (autrement dit les variables). Ainsi les programmes sont
plus faciles à écrire, et surtout à relire~;


\item d'utiliser des langages de programmation {\em évolués}~: chaque
instruction en langage évolué est traduite par un \emph{compilateur}
en une séquence d'instructions dans le langage de la machine (Fortran,
etc). Cela facilite la tâche du programmeur~;
\item de rendre les programmes indépendants de la machine utilisée~: un programme écrit dans un langage normalisé pourra tourner sans trop de modifications sur des machines de marques différentes  (Algol, Cobol, etc.);
\item d'autoriser le programmeur à se définir ses propres types de
données, à partir des types de base et de constructeurs~: tableaux,
enregistrements, pointeurs (PL/I, Algol/W, etc.); \item d'imposer une
certaine discipline au programmeur (programmation structurée), en
limitant l'emploi de l'instruction de branchement {\em goto} qui rend
les programmes spécialement illisibles (Pascal)~;
\item d'inciter à la \emph{réutilisation de modules} déjà écrits et à
la constitution de bibliothèques de modules, en intégrant aux langages
de programmation des mécanismes de modularisation (MODULA, ADA,
EIFFEL)~;
\end{itemize}

Cette évolution considérable nous à fait passer, en moins d'un
demi-siècle, du code binaire à l'assembleur, Cobol, Fortran, PL/I
jusqu'à ADA, Eiffel, etc. Elle préserve cependant deux traits
fondamentaux de l'informatique des origines, à savoir la {\em
séquentialité} des programmes et la notion d'{\em affectation}
(modification du contenu d'une variable), qui caractérisent la
``\emph{programmation impérative}''.

Dans ce style de programmation, il incombe au programmeur de décrire
la suite exhaustive des actions que la machine devra effectuer
(affectations, comparaisons, additions, etc.) dans un ordre précis
pour parvenir au résultat voulu. C'est tout à fait fastidieux. Le
programmeur qui aborde une tâche d'une certaine importance se
trouverait rapidement emporté par un flot de détails de programmation,
s'il n'avait une méthode de travail efficace~:

\begin{itemize}
	\item décomposer les problèmes compliqués en problèmes de plus
	en plus simples~; 
\item ne pas essayer de réinventer la roue~:
	connaître et utiliser les algorithmes classiques~; 
\item	réfléchir d'abord, programmer ensuite~;
\item laisser une
	trace écrite de son raisonnement (la fameuse documentation).
\end{itemize}

Ce genre de métier demande une certaine minutie et beaucoup de
ténacité\footnote{parfois à la limite du comportement obsessionnel}, mais il
n'est pas requis d'avoir un cerveau particulièrement brillant~: une
bonne formation suffit.


\section{La programmation déclarative}

\index{déclaratifs (langages)}
La plupart des langages de programmation (il en existe plusieurs
milliers) relèvent de la catégorie précédente~: Fortran, Cobol, PL/I,
Basic, Pascal, Ada, C, etc., pour les raisons historiques évoquées
ci-dessus (évolution progressive depuis le langage d'assemblage).

Il existe cependant une autre catégorie digne d'intérêt (pour des
raisons que nous expliciterons plus loin)~: les langages {\em
déclaratifs}
\index{langages déclaratifs}.

Dans un langage déclaratif, programmer c'est essentiellement indiquer
 à une machine la nature des données dont on dispose d'une part, la
 nature des résultats que l'on veut d'autre part, plutôt que la
 séquence de traitements qui mène des unes aux autres.

En quelque sorte, les langages déclaratifs décrivent des
\emph{spécifications} de traitements plutôt que des algorithmes~:
\begin{itemize}
\item  une \emph{spécification}  résume ce que fait une procédure,  
\item un \emph{algorithme} décrit comment la procédure le fait. 
\end{itemize}

Les spécifications et algorithmes font partie de la documentation
interne (commentaires) et externe (dossier de programmation) de tout
programme sérieux.  Voir l'exemple de la figure \ref{progpascal}.

\begin{figure*}[htb]
\barre
\begin{verbatim}
function PGD (n:integer) : integer;

(* Spécification : pour tout entier n>1,  
   PGD(n) est le plus grand diviseur de n 
   qui lui soit strictement inférieur *)

var d : integer; 

begin

(* Algorithme :
    boucle descendante de n-1 à 1
    sortie quand d vaut 1 ou divise n
    le résultat est dans d
*) 
    d := n - 1;
    while (n mod d) <> 0 
       do d := d - 1; 
    PGD := d;
end;
\end{verbatim}
\caption{Un programme Pascal bien documenté}
\label{progpascal}
\barre
\end{figure*}

\index{documentation} Traditionnellement ressentie comme une corvée
fastidieuse par les programmeurs\footnote{et les étudiants} qui la
remettent toujours à plus tard (``quand le programme tournera''), la
documentation des programmes contribue à réduire les coûts de
programmation~:
\begin{itemize}
\item Rédiger une spécification, c'est expliquer la vision que 
l'on a d'un problème. Vision que l'on doit confronter à celle de
l'utilisateur, avant de se lancer dans l'écriture d'un
programme. Combien de programmes ont été jetés à la poubelle parce
qu'ils ne correspondaient pas du tout à ce que le client avait
demandé~?
\item Lors de l'écriture des programmes, si on n'est pas capable
d'écrire, en bon français, ce que l'on veut faire, il est évident
que l'on aura de graves difficultés à dire à une machine stupide (par
définition), comment elle doit le faire, que ce soit en Pascal, en
Cobol, ou quoi que ce soit.\footnote{\em Le vent souffle toujours dans
le mauvais sens pour celui qui ne sait pas où il veut aller}. Il
faut donc rédiger la documentation \emph{avant} le programme, et non
l'inverse.
\item Lors de la mise au point~: une procédure est valide si 
les sous-procédures qu'elle utilise sont correctes et employées
conformément à leurs spécifications respectives, et si l'algorithme
correspond à ce que la procédure est censée faire. Il est plus facile
de vérifier une procédure lorsqu'on a sa spécification sous les yeux~:
la documentation doit faire partie du programme
\item Pour la maintenance~: il est inutile de relire tout le code 
pour savoir si l'on peut modifier une procédure sans conséquences
graves sur le reste du programme~: l'effet d'une procédure est
entièrement décrit dans sa spécification. Et on ne passe plus des
heures à se demander ce que fait la procédure \texttt{toto}.
\end{itemize}

La programmation déclarative existe sous deux formes~:

\begin{itemize}
\item La Programmation Logique~:  une formule logique décrit 
la relation qui existe entre les données et les résultats. Ce qui est
à rapprocher des langages d'interrogation de bases de données
relationnelles du type SQL. Le langage le plus connu est Prolog.
\item
La Programmation Fonctionnelle~: tout traitement informatique consiste
à calculer des résultats à partir de données, c'est donc une fonction
$f : \{Données\} \rightarrow \{Résultats\}$ au sens mathématique du terme.
\end{itemize}

\section{Quelques langages fonctionnels}

Il existe un très grand nombre de langages fonctionnels (La figure
\ref{ExemplesDivers} montre le même exemple écrit en 
Scheme, ML, Miranda, Hope et FP).

Le premier (et le plus connu) a été con\c{c}u par John MacCarthy à la
fin des années 50. LISP \index{LISP} était, au départ, un langage de
traitement de listes (\emph{LISt Processing language}) comportant un
``noyau'' purement fonctionnel et - pour des raisons d'efficacité -
diverses ``améliorations'' comme les notions de variable et
d'affectation.\footnote{Le concepteur d'un langage de programmation
doit toujours trouver un compromis entre un objectif de ``propreté''
(allant parfois jusqu'au minimalisme) du langage qu'il définit, et des
concessions souhaitables à l'{\em efficacité}.  Mais comme le souci
d'efficacité s'estompe peu à peu avec l'augmentation des performances
des machines, il est donc difficile de juger {\em a posteriori} du
caractère raisonnable des compromis qui ont été décidés il y a
plusieurs dizaines d'années.}

LISP a de très nombreux descendants. Un des plus prometteurs est
Scheme (1974), qui est très utilisé dans l'enseignement
\footnote{Notamment en 
DEUG et Licence-Maîtrise d'Informatique à Bordeaux}.



\begin{figure*}[htb]
\barre
\begin{itemize}
\item Scheme~:
\begin{verbatim}
(define (fac n)
  (if   (eqv? n 0) 
        1 
        (* n (fac (- n 1 )))
))
\end{verbatim}
\item ML~:
\begin{verbatim}
fun fac(n) =  if n = 0 
                 then 1 
                 else n * fac(n - 1)
\end{verbatim}
\item HOPE~: 
\begin{verbatim}
dec fac : num -> num;
--- fac 0 <= 1;
--- fac n <=  n * fac(n - 1);
\end{verbatim}
\item MIRANDA~: 
\begin{verbatim}
fac 0 = 1
fac n = n * fac (n - 1)
\end{verbatim}
\item FP~:
\begin{verbatim}
def fac = eq0 -> 1 ;  * o [ id, fac o ( - [id, 1] ) ]
\end{verbatim}
\end{itemize}
\caption{Une fonction dans plusieurs langages}
\barre
\label{ExemplesDivers}
\end{figure*}

En 1974 apparaissait ML  à l'Université d'Edimbourg (Écosse).  ML
était au départ le ``métalangage'' d'un système de preuve formelle de
fonctions récursives\emph{La démonstration automatique est une des
branches de l'intelligence artificielle}~: le système LCF \index{LCF
(système de démonstration automatique)}(\emph{Logic of Computable
Functions}).

En raison de ses nombreuses qualités, le langage ML a été ensuite
redéfini par les mêmes auteurs en reprenant des idées de ML et de HOPE
(voir plus loin) pour conduire à SML \index{SML (STandart ML}
(\emph{Standart ML}) (1986). Parallèlement, une équipe de l'INRIA (France) a
développé CAML, \index{CAML (langage fonctionnel)} basé sur le concept
de Machine Abstraite Catégorique (1986).

Le langage Hope a également été développé à Edimbourg vers 1980. Son
nom provient de l'ancienne adresse de l'Institut pour l'Intelligence
Artificielle~: \emph{Hope Park Square}. C'est un langage fonctionnel pur,
contrairement à Standard ML qui contient des concepts ``impératifs''
(variables, affectations, pointeurs, traitement des exceptions). Le
langage Miranda est également de la même famille.

À peu près à la même époque (1978), J.W. Backus - un des inventeurs de
FORTRAN (1955) et d'ALGOL (1958)
\footnote{Rappelez-vous, BNF = Backus-Naur Form\ldots}
\index{FP (Functional Programming)}- proposait FP.

\subsection*{Quelques références bibliographiques}
\begin{itemize}
\item Backus, J.W, (1978). 
{\em Can programming be liberated from the von Neumann style ? A
functional style and its algebra of programs}, Communications of the
ACM, 21, 613-41.

\item Cousineau, G., Curien, P.L, Mauny, M., (1985).
{\em The Categorical Abstract Machine, in Proc. Conference on
Functional Programming Languages and Computer Architecture}, Nancy,
50-64, LNCS 201, Springer Verlag.

\item Burstall, R.M., MacQueen, D.B., Sanella, D.T., (1980). 
{\em Hope, an Experimental Applicative Language, CSR-62-80, Department
of Computer Science}, University of Edimburgh.

\item  Gordon M.J., Milner, A.J., Wadsworth, C.P, (1979)
{\em  Edimburgh LCF}. LNCS 78. Springer Verlag.

\item MacCarthy, J., (1960). 
{\em Recursive functions of symbolic expressions and their computation
by machine}. Communications of the ACM, 3(4), 184-95.

\item Rees, J., Clinger, W., eds. (1986). {\em Revised3 Report on the Algorithmic Language Scheme}.  SIGPLAN Notices, 37-79, Vol 21 n 12, Dec. 1986.

\item Turner, D.A., (1985). {\em Miranda, a non-strict functional language with polymorphic types}, in Proc. Conference on Functional Programming Languages and Computer Architecture, Nancy, 1-16, LNCS 201, Springer Verlag.

\item Wirsig, M., Sannella, D., (1987). 
{\em Une Introduction à la Programmation Fonctionnelle~: Hope et ML},
in Technique et Science Informatiques, vol.6 n 6, 517-525,
AFCET-Bordas.
\end{itemize}

\section{La crise du logiciel}

Que l'informatique soit un secteur en pleine expansion, voila bien un
lieu commun journalistique d'une trompeuse évidence. Car
l'extraordinaire miniaturisation, l'amélioration fantastique des
performances et la chute vertigineuse des prix des composants
matériels se sont accompagnés, depuis une bonne dizaine d'années,
d'une gigantesque crise du logiciel~: la production du logiciel est de
plus en plus coûteuse. Quelques éléments~:

\begin{itemize}
\item la part du logiciel dans les coûts informatiques est de plus de 90 \%~;
\item un service informatique ``normal'' consacre 
plus de 80 \% de son activité à la maintenance d'applications
existantes~;
\item un programmeur moyen produit en moyenne 20 à 30 lignes
 de code par jour (indépendamment du langage de programmation
 utilisé!)~;
\item Les langages Cobol et Fortran ont été conçus dans 
les 10 premières années de l'informatique d'entreprise. 30 ans plus
tard, ils représentent encore 80 \% des programmes utilisés et
maintenus~;
\item Les langages de programmation 
classiques ne sont pas adaptés (et c'est un euphémisme) à
l'utilisation de machines massivement parallèles (par exemple réseaux
de 64000 processeurs).
\end{itemize}

{\bf Références}~: Feuilletez la presse informatique
(professionnelle), ainsi que les ouvrages consacrés au ``génie
logiciel''.


\section{La programmation fonctionnelle, une solution d'avenir ?}

Trois aspects de la programmation fonctionnelle permettent de la 
considérer comme une solution possible à cette crise du logiciel~:

\begin{itemize}
\item Les programmes fonctionnels sont généralement beaucoup plus 
courts que leurs homologues impératifs~: ils sont écrits plus
rapidement, à moindre coût. Ils sont également plus abstraits
\index{Abstraction}
\footnote{Il convient de ne pas confondre les différents sens 
de l'adjectif 'abstrait'. Cf. Dictionnaire de la Langue fran\c{c}aise
Lexis (éditions Larousse).  {\bf Abstrait,e}~: adj. (lat. abstractus,
isolé par la pensée; 1390)
\begin{enumerate}
\item Se dit d'une qualité considérée en elle-même,
 indépendamment de l'objet (concret) dont elle est un des caractères,
 de sa représentation, ou de tout ce qui dépasse le particulier pour
 atteindre le général~: la grandeur et la couleur sont des qualités
 abstraites (= concepts). Les noms abstraits, comme ``blancheur'' et
 ``politesse'', désignent en grammaire une qualité ou une manière
 d'être (contr. CONCRET).
\item Se dit d'une personne (de son esprit ou de son oeuvre) difficile 
à comprendre à cause de la généralité de son expression ou,
péjor. dont la pensée est vague et exprimée de manière confuse~: 
\emph{Je
suivais mal son raisonnement abstrait} (syn. non péjor. SUBTIL,
contr. CLAIR). \emph{C'est un écrivain abstrait, qui se refuse à illustrer
sa pensée par des exemples concrets} (syn. lit. ABSCONS, ABSTRUS). Un
exposé abstrait qui ennuyait l'auditoire (syn. pejor. et fam. FUMEUX;
contr. PRÉCIS).  \item Art abstrait, art qui tend à représenter la
réalité abstraite et non pas les apparences de la réalité~: \emph{L'art
abstrait utilise les lignes et les masses pour traduire l'idée ou le
sentiment} (contr. FIGURATIF).
\end{enumerate}
} (on pourra donc réutiliser tels quels des modules d'un programme
 déjà écrit) et plus faciles à comprendre (moins de ``petits détails''
 de programmation).
\item Les programmes fonctionnels se prêtent bien aux techniques
 de preuve de programmes et de manipulation formelle (transformation
 de programmes). C'est un style qui se rapproche des techniques
 modernes de ``spécifications formelles'' de programmes.
\item Ils peuvent facilement, 
et avec profit, être implantés sur des machines massivement
parallèles (par exemple réseaux de 64000 processeurs).
\end{itemize}

Ces aspects favorables proviennent de la nature mathématique des
programmes fonctionnels~: les éléments d'un langage fonctionnel sont
des fonctions qui décrivent l'obtention de résultats (sorties) à
partir de données (entrées) indépendamment de l'environnement où
elles sont utilisées (c'est la \emph{transparence référentielle}). On peut
très facilement les combiner entre elles, ce qui n'est pas le cas des
programmes impératifs.

\include{chap2-programmer-avec-fonctions}
\chapter{Induction et récursion}


%	3.1 Un peu d'histoire 
%	3.2 Axiomatique de PEANO
%	3.3 Définitions récursives
%	3.4 Raisonnement par récurrence
%	3.5 Fonctions d'ordre supérieur
%	3.6 À propos des fonctions auxiliaires
%	3.7 Exercices et problème



\section{Un peu d'histoire}

Le \siecle{XIX} est le siècle de la Révolution Industrielle et de la
Science Triomphante. C'est l'époque de la foi en un progrès
scientifique inéluctable et bénéfique, fondé sur l'étude rigoureuse
des faits (positivisme d'Auguste COMTE), et sur l'existence d'un
déterminisme qui régirait aussi bien les phénomènes physiques
(MAXWELL, BERTHELOT) biologiques (BERNARD, DARWIN), que l'organisation
sociale (TAINE, naturalisme de ZOLA) et politique (MARX).

À l'extr\^eme, c'est le scientisme~:
\begin{citation}
Une chose évidente d'abord, c'est que chaque découverte pratique de
l'esprit humain correspond à un progrès moral, à un progrès de dignité
pour l'universalité des hommes. [...] Je suis convaincu que les
progrès de la mécanique, de la chimie, seront la rédemption de
l'ouvrier~; que le travail matériel de l'humanité ira toujours en
diminuant et en devenant moins pénible~; que de la sorte l'humanité
deviendra plus libre de vaquer à une vie heureuse, morale,
intellectuelle. Aimez la science. Respectez-la, croyez-le, c'est la
meilleure amie du peuple, la plus s\^ure garantie de ses progrès.
{\em Ernest RENAN }
\end{citation}


\subsection*{La mécanisation du calcul}

La Révolution Industrielle na\^{\i}t de la mécanisation du travail
physique humain. Les travaux de Charles BABBAGE (1792-1871) ont montré
à ses contemporains la possibilité de mécaniser également le calcul,
c'est-à-dire de faire exécuter des calculs numériques par une machine
sous contr\^ole d'un programme (possibilité théorique surtout, car
Charles BABBAGE mourut sans voir sa ``Machine Analytique'' réalisée,
faute de moyens).

C'est le fils aîné de BABBAGE, Henry, qui fabriqua une version très
réduite de la Machine Analytique à partir de 1880. Le 21 janvier 1888,
la machine imprima une table des 44 premiers multiples de $\pi$, avec 29
décimales. Mais un incident technique d\^u à la technologie employée
(roues dentées, cylindres à picots) provoqua une erreur au 32$^e$
multiple. Découragé, Henry BABBAGE ne reprit les essais qu'en 1906, il
trouva la cause de l'erreur et y remédia. Il put alors faire une
démonstration réussie devant l'Académie d'Astronomie. La machine ayant
prouvé la justesse des idées de son inventeur, elle fut remise à un
musée en 1910.

Lady Ada LOVELACE (1815-1852), l'assistante de Charles BABBAGE,
écrivit un jour que la machine ``pourrait peut-\^etre ne pas traiter
que des nombres''.


\subsection*{La  formalisation du raisonnement}

En 1854, George BOOLE (1815-64) publia son ouvrage {\em Les Lois de la
 Pensée} dans lequel il expliquait que le raisonnement logique pouvait
 \^etre assimilé à du calcul algébrique, relan\c{c}ant alors la
 ``logique formelle'' - étude du raisonnement déductif, des
 syllogismes, etc. - qui sommeillait depuis quelques siècles. BOOLE
 rencontra BABBAGE en 1862, mais il mourut trop t\^ot
\footnote{Il contracta une congestion pulmonaire en allant à pied, sous la pluie, donner une conférence au Queen's College de Cork.}
pour qu'une collaboration ait pu na\^{\i}tre.

William Stanley JEVONS (1835-82) fut le seul mathématicien à
comprendre immédiatement la portée de l'oeuvre de BOOLE. Il réalisa un
``piano logique'' qui pouvait résoudre des équations logiques. Très
enthousiaste, il aurait sans doute exploré la mécanisation du
raisonnement beaucoup plus loin s'il ne s'était noyé accidentellement.


\subsection*{L'axiomatisation des mathématiques}

\subsubsection*{L'état de l'art au \siecle{XIX}  siècle}

Les connaissances mathématiques sont bien avancées au \siecle{XIX} siècle~: la
plupart des branches ont été créées depuis déjà longtemps~:

\begin{itemize}
\item Éléments d'EUCLIDE (\siecle{III} avant J.C)~: Algèbre, Géométrie,
  Arithmétique ...
\item Traité d'algèbre d'AL-KHWARIZMI (\siecle{IX} siècle)
\item Résolution de l'équation du 3$^e$ degré par Hieronimo CARDAN
(1545), 4$^e$ degré par son élève Ludovico FERRARI
\item Nombres Négatifs et Complexes~: Rapha\"{e}l BOMBELLI (1526-72)
\item Symbolisme Algébrique~:  Fran\c{c}ois VIETE (1540-1603)
\item Théorie des Nombres~: Pierre de FERMAT (1601-65)
\item Calcul Infinitésimal~: Isaac NEWTON (1642-1727) 
\item Calcul Différentiel~: Gottfried Wilheim LEIBNITZ (1646-1716)
\end{itemize}

Mais les mathématiciens se heurtent à des paradoxes inextricables,
notamment sur l'infiniment petit et l'infiniment grand. Appara\^{\i}t
alors un courant de pensée qui vise, dans le m\^eme esprit qu'EUCLIDE,
à donner des définitions précises des objets mathématiques que l'on
utilisait jusque-là de manière assez intuitive, dans l'intention
d'éliminer ces paradoxes.

\subsubsection*{La démarche d'Euclide}

L'oeuvre du grec EUCLIDE (\siecle{III} av. JC) est un modèle de
rigueur~: ses Eléments récapitulent en une quinzaine de volumes les
connaissances mathématiques de l'époque. PROCLUS (\siecle{V} siècle après J.C)
affirme qu'Euclide, en rassemblant ses Eléments, 
\begin{citation}
en a coordonné
beaucoup d'Eudoxe, perfectionné beaucoup de Théétète et qu'il a évoqué
dans d'irréfutables démonstrations ceux que ses prédécesseurs avaient
démontré d'une manière rel\^achée.
\end{citation}

Le Livre I des Éléments est précédé par une présentation assez
intuitive des concepts utilisés dans la suite~:
\begin{citation}
Un point est ce qui n'a aucune partie. Une ligne est une longueur sans largeur.
\end{citation}

et par une série de postulats (affirmations admises sans qu'il y ait
lieu de les démontrer) qui indiquent les relations entre ces concepts~:
\begin{citation}
On demande~:
\begin{enumerate}
\item qu'on puisse conduire une droite d'un point quelconque à un point quelconque,
	\item qu'on puisse prolonger continuellement, selon sa
	direction, une droite finie en une droite,
\item  que d'un point quelconque, et avec un intervalle quelconque, on puisse décrire une circonférence quelconque,
\item et que tous les angles droits soient égaux entre eux,
\item et que si une droite tombant sur deux droites fait les angles intérieurs du m\^eme c\^oté plus petits que deux droits, ces droites, prolongées à l'infini, se rencontreront du c\^oté o\`u les angles sont plus petits que deux droits.
\end{enumerate}
\end{citation}

Après avoir exposé systématiquement les définitions, postulats et
axiomes, EUCLIDE en déduit les propriétés élémentaires des triangles,
et de leurs bissectrices, milieux des c\^otés, etc.

\subsubsection*{Le courant formaliste}

Au \siecle{XIX} les mathématiciens-logiciens du courant ``formaliste'' (et
 plus tard Russell, Hilbert, etc.) essaient de formuler les quelques
 hypothèses (axiomes, postulats) qui sont vraiment indispensables pour
 définir les objets mathématiques, et dont on pourra déduire toutes
 les propriétés qui sont ``intuitivement vraies''.

En 1876 c'est un Allemand, Julius Wilheim Richard DEDEKIND
(1831-1916), qui donne une définition formelle de l'ensemble $\mathbb{R}$ des
nombres réels à partir des coupures de $\mathbb{Q}$ (l'ensemble des
rationnels).
\begin{citation}
Une coupure est un partage de $\mathbb{Q}$ en deux partitions non vides $A$ et $B$
telles que tout élément de $A$ est plus petit que tout élément de $B$. À
chaque coupure correspond alors un nombre réel unique~: par exemple,
le nombre irrationnel $\sqrt{2}$ est défini par la coupure 
$$\begin{array}{rl}
A &= \{ q | q>0 \mbox{\ et\ } q^2<2 \} \\
 B &= \{ q | q<0 \mbox{\ ou\ } q^2 \geq 2 \} 
\end{array}$$

\end{citation}

Vers 1880, Georg Ferdinand Ludwig Philipp CANTOR (1845-1918) met au
point la théorie des ensembles, qui permet de résoudre de nombreux
paradoxes de la théorie des limites liés, en fait, à l'existence de
plusieurs types d'infinis.

Quelques problèmes typiques, que vous pouvez essayer de résoudre~:

\begin{exercice}
  Montrez qu'il y a
  \begin{itemize}
\item - autant de couples d'entiers que d'entiers (on peut
  construire une bijection entre $\mathbb{N}$ et $\mathbb{N}^2$) ;
  \item plus de nombres réels dans l'intervalle $[0,1[$ que de nombres
    entiers~ : il n'existe pas d'injection de $[0,1[$ dans N).
  \end{itemize}
\end{exercice}

\begin{exercice}
 L'ensemble des parties d'un ensemble $E$ est ``plus gros'' que cet
 ensemble~: il n'existe pas d'injection de ${\cal P}(E)$ dans $ E$ (résultat
 d\^u à CANTOR).
\end{exercice}



Malheureusement pour CANTOR, sa théorie fait appara\^{\i}tre à nouveau
des paradoxes (comme le paradoxe de RUSSELL~: l'ensemble des ensembles
qui ne se contiennent pas eux-m\^emes se contient-il lui-m\^eme ? ).
La théorie des ensembles sera axiomatisée plus tard par Ernst ZERMELO
(1871-1953) en 1908. Génial mais incompris à son époque, CANTOR est
mort dans un asile psychiatrique.

En 1889, le mathématicien et logicien Giuseppe PEANO propose (enfin !) 
une définition formelle de l'ensemble des entiers naturels (positifs).  


\subsection*{Mathématiques et Réalité}

L'approche ``hypothético-déductive'' pose le problème du choix du système d'axiomes de base~: 
\begin{itemize}
\item pour LEIBNIZ (et d'autres), ``tout ce qui est vrai est démontrable''~: pour arriver à tout démontrer toute propriété vraie dans le cadre d'une théorie (c'est la complétude de la théorie),  il faut prendre suffisamment d'axiomes~;
\item il ne faut pas que les axiomes choisis conduisent à des contradictions (consistance de la théorie)~;
\item il faut éviter aussi de choisir des axiomes qui seraient des conséquences des autres axiomes (ils seraient inutiles).
\end{itemize}

Par exemple, la question s'est longtemps posée de savoir si le fameux cinquième postulat d'EUCLIDE
\begin{citation} par un point extérieur à une droite on peut mener une seule et unique droite parallèle à celle-ci
\end{citation}
 était ou non un axiome.
 De nombreux mathématiciens essayèrent en vain de montrer que ce postulat était une conséquence des autres axiomes de la géométrie d'Euclide, jusqu'en 1826, lorsque LOBATCHEVKI (1793-1856) présente une géométrie non-euclidienne dans laquelle 
\begin{citation} par un point extérieur à une droite passent une infinité de parallèles à cette droite.
\end{citation}
 Un peu plus t\^ot, vers 1813, GAUSS (1777-1844) avait inventé la
 géométrie hyperbolique, 
\begin{citation}
``une étrange géométrie,
 tout à fait différente de la n\^otre''
\end{citation} mais il n'avait pas
 osé publier ses travaux~: 
\begin{citation}
``J'appréhende les clameurs des
 Béotiens si je voulais exprimer complètement mes vues''
\end{citation}

 En
 effet, à l'époque, la géométrie euclidienne était censée {\em rendre
 compte de la réalité physique du monde} (KANT); par conséquent ses
 axiomes étaient donc considérés comme {\em vrais dans l'absolu}~: il
 était inimaginable de fonder une théorie sur leur négation.



De la m\^eme fa\c{c}on, l'hypothèse du continu 
\begin{citation} le cardinal de
l'ensemble des nombres réels est le premier qui soit supérieur à celui
de l'ensemble des nombres entiers
\end{citation}
 formulée par CANTOR qui essaya de
la démontrer jusqu'à sa mort, fut montrée
\emph{indécidable}\footnote{on ne peut pas la démontrer, ni la réfuter
: on peut l'ajouter comme nouvel axiome, aussi bien que la proposition
contraire} par P.J. COHEN en 1963.


L'espoir de trouver un jour quand m\^eme un ``bon'' système d'axiomes
qui suffirait à tout démontrer s'effondre en 1931 après la publication
du Théorème d'Incomplétude de Kurt G\"{O}DEL (né en 1906)~:
\begin{citation}
Toute formulation axiomatique de la théorie des nombres est soit incomplète, soit contradictoire.
\end{citation}

Autrement dit, dans toute théorie T (contenant la théorie des
nombres), il existe des propositions indécidables, dont on ne peut
démontrer ni la vérité ni la fausseté. Par exemple, la proposition
``la théorie T est non-contradictoire'' est indécidable dans la
théorie T elle-m\^eme.


\subsection*{Bonnes lectures~:}

\begin{itemize}
\item Douglas HOFSTADTER (1985), G\"{o}del, Escher et Bach, les Brins d'une Guirlande Éternelle,  InterEditions.
\item R. APERY, M.CAVEING, et al. (1982) Penser les mathématiques, Points Inédits S29, ed. du Seuil.
\item A. DAHAN-DALMEDICO, J. PFEIFFER, (1986). Une histoire des mathématiques, Points Sciences S49, ed. du Seuil.
\item Gustave FLAUBERT, Bouvard et Pécuchet, Livre de poche 440-441.
\end{itemize}



\section{Axiomatique de PEANO}

Cette définition des entiers naturels tient en 5 axiomes, elle repose
sur un objet de base (zéro) et un ``constructeur''~: la fonction qui a
tout entier $n$ associe son successeur $n+1$
\begin{enumerate}
\item zéro est un entier
\item tout entier a un successeur, qui est également un entier
\item zéro n'est le successeur d'aucun entier
\item	 deux entiers différents ont des successeurs différents
\item Principe d'induction~: si une propriété P est vraie pour
 zéro (cas de base)
		 et que P(n) entra\^{\i}ne
 P(successeur(n)) pour tout entier n,  (étape d'induction)
		alors P est vraie pour tout entier .
\end{enumerate}

C'est le premier exemple d'ensemble défini inductivement~: on part
d'un objet de base zero (0) auquel on adjoint son successeur (1) puis
le successeur du successeur (2), etc. Nous verrons plus loin quantité
d'autres ensembles (listes, arbres) définis de la m\^eme fa\c{c}on, à
partir d'objets de base et de constructeurs.

Le principe d'induction est à la base de la technique de preuve par
récurrence~: pour prouver qu'une propriété est vraie pour tous les
entiers on montre~:
\begin{itemize}
	\item quelle est vraie pour $0$ (en général c'est plut\^ot
	facile), \item que si elle est vraie pour un entier $n$
	(hypothèse de récurrence), alors elle est vraie pour son
	successeur $n+1$.
\end{itemize}
Nous avons donc le droit de définir des fonctions par induction naturelle~: une fonction $f$ est définie pour tout entier positif à partir du moment o\`u~:
\begin{itemize}
\item on connaît $f(0)$,
\item on peut exprimer $f(n+1)$ à partir de $f(n)$, pour tout $n$.
\end{itemize}
\begin{exercice} Construire un ensemble qui satisfasse tous les axiomes de PEANO sauf le troisième.
\end{exercice}
\begin{exercice} Idem, mais en excluant cette fois-ci le quatrième.
\end{exercice}
\begin{exercice} Idem, sans le cinquième.
\end{exercice}
\begin{exercice} Montrez que l'on peut remplacer le principe d'induction par la variante~:
	« si une propriété $Q$ est vraie pour $zero$ (cas de base) et que
	 $Q(zero)$ et $Q(successeur(zero)$ et .... $Q(n)$ implique
	 $Q(successeur(n))$ pour tout entier $n$, alors $Q$ est vraie pour
	 tout entier. » et comparez les mérites respectifs des deux
	 formulations.
\end{exercice}
 

Le lecteur attentif ne manquera pas de nous soup\c{c}onner~: peut-on
honn\^etement prétendre \emph{définir les nombres entiers} sans parler des
opérations arithmétiques comme l'addition, la comparaison, etc.~?  Et
bien oui~! L'addition, par exemple, n'est pas une notion primitive de
la théorie des nombres, mais une fonction que l'on définit par
récurrence~:

$$\begin{array}{rcrcl}
	plus ( &0& , 	&p ) =& p \\				
	plus ( &succ(n)& , &	p ) =& succ(  plus (n ,p ) )
\end{array}$$

		pour tous entiers $n$ et $p$.

Rassurez-vous, deux plus deux font toujours quatre, en effet 
$$deux = succ(succ(zero))$$
et
$$\begin{array}{rcl}
	plus(deux, deux) 	&=& plus ( succ(succ(zero)),  succ(succ(zero)) ) \\
				&=& succ ( plus ( succ(zero)),  succ(succ(zero)) )) \\
				&=& succ (succ ( plus (zero),  succ(succ(zero)) )) \\
				&=& succ ( succ (succ ( succ ( zero ) ) ) )
\end{array}$$



\begin{exercice} Définir la multiplication.
\end{exercice}

\begin{exercice} Définir la relation ``inférieur ou égal''.
\end{exercice}

L'axiomatique de PEANO nous suffit donc pour reconstruire les notions
connues de l'arithmétique.


\section{Définitions récursives}

\begin{quotation}
Il fut saisi par la frénésie des factorielles~: $1!~= 1~$; $2!~= 2~$; 
$3!~=~6~$; $4!~= 24$~; $5!~= 120~$; $6!~= 720~$; $7!~=~5~040~$; $8!~=~40~320~$; 
$9!~=362~880~$; $10!~=~3~628~800~$; $11!~=~39~916~800~$; $12!~=~479~001~600~$; [...]
$22!~=~1~124~000~727~777~607~680~000$, soit plus d'un milliard de fois
soixante-dix-sept milliards~!  Smautf en est aujourd'hui à $76!$ mais il
ne trouve plus de papier au format suffisant et en trouverait-il, il
n'y aurait pas de table assez grande pour l'étaler.

La vie mode d'emploi {\em Georges PEREC}, Livre de Poche 5341 (1978)
\end{quotation}

Nous allons essayer de spécifier formellement la fonction factorielle,
c'est-à-dire de préciser ce qu'elle fait, de manière aussi claire que
possible.


\subsection*{Comment spécifier la fonction factorielle ?}

Premier essai, par une phrase~:
\begin{citation}
	La fonction factorielle associe, à tout entier positif ou nul n, un autre entier qui est le produit des entiers de 1 à n.
\end{citation}

Cette spécification est correcte~: elle identifie clairement et sans
ambig\"{u}ité la fonction dont nous parlons. Mais sa formulation
littéraire la rend difficilement exploitable ensuite par le calcul
algébrique qui est l'outil de base du raisonnement mathématique~: nous
préférerions une bonne formule.

Seconde tentative, par une expression mathématique~:

\[factorielle (n) = 1 \times 2 \times 3 \ldots \times n\]

C'est une définition \emph{elliptique}~: il faut un certain effort de
la part du lecteur pour comprendre ce qu'il convient de mettre en lieu
et place des points de suspension. Elle n'est donc pas idéale.

Et m\^eme avant ces points, car avec un peu de mauvaise foi, on
pourrait conclure que 
$$factorielle (2) = 1\times 2 \times 3 \times 2 =
12$$
Bref, le ``$1 \times 2 \times 3 \ldots$'' est à prendre avec des
pincettes~: lorsqu'il y a des points de suspension ensuite, $1 \times
2 \times 3$ n'est plus égal à $6$, mais $1$, $2$ ou $6$ selon
les circonstances !

La troisième tentative, et la bonne, sera de définir la fonction
factorielle par une spécification récursive .


\subsection*{Construire une spécification récursive}

Une spécification est dite \emph{récursive} lorsqu'elle définit un
objet mathématique (un ensemble, une relation, une fonction) à l'aide
de lui-m\^eme. 

Ceci doit vous sembler pour le moins obscur, aussi
revenons à notre exemple et regardons quelques valeurs de la fonction
$factorielle$~:

$$\begin{array}{rcl}
	factorielle (0) 	&=& 1 \\
	factorielle (1) 	&=& 1 \\
	factorielle (2) 	&=& 1 \times 2 \\
	factorielle (3) 	&=& 1 \times 2 \times 3 \\
	factorielle (4) 	&=& 1 \times 2 \times 3 \times 4 \\
	... && \\
	factorielle (n - 2) &=& 1 \times 2 \times 3 \times ... \times (n-2)  \\
	factorielle (n - 1) &=& 1 \times 2 \times 3 \times ... \times (n-2) \times (n-1) \\
	factorielle (n ) 	&=& 1 \times 2 \times 3 \times ... \times (n-2) \times (n-1) \times n \\
\end{array}$$

Nous remarquons que chaque ligne ne diffère de la précédente que par
un seul terme. Par exemple, $factorielle(4)$ est égal à$
factorielle(3) \times 4$. Nous sommes donc tentés de définir
$factorielle$ par l'équation~
	$$factorielle (n) = factorielle (n-1) \times n $$ valable pour
tout entier positif $n$, dans laquelle la fonction $factorielle$ est
exprimée à partir d'elle-m\^eme. Mais il y a un hic~: en appliquant
cette équation au cas $n=0$, nous obtenons $ factorielle(0) =
factorielle(-1)\times0$ et donc $1=0$, ce qui n'est pas raisonnable
(sans parler du fait que $-1$ n'est pas dans le domaine de définition
de la fonction).

Il faut donc limiter l'usage de l'équation $factorielle (n) =
factorielle(n-1) \times n$ aux cas o\`u n est strictement positif, et
préciser ce qui se passe lorsque $n=0$ (le cas de base). Voici donc
une bonne définition de la factorielle~:
$$\begin{array}{rll}
factorielle (n) &= factorielle (n-1) \times n & 
\mbox{\ pour tout entier positif\ } n \\
factorielle (0) &= 	1
\end{array}$$


\subsection*{Traduction en HOPE}

Dans le langage HOPE, nous déclarerons cette fonction sous la forme~:
\begin{verbatim}
dec factorielle : num -> num ;
--- factorielle (0) <= 1 ;
--- factorielle (n) <=  factorielle (n-1) * n ;
\end{verbatim}

Attention, l'ordre des équations est important en Hope, car
l'interprèteur va essayer de les utiliser dans l'ordre o\`u elles ont
été déclarées. Le calcul de $factorielle(2)$ se déroule comme suit~:

\begin{itemize}
\item rejet de la première équation~: $2$ est différent de la constante $ 0 $, donc l'équation ne convient pas.
\item essai de la seconde équation~: on peut poser $n=2$. Donc on remplace
$ factorielle(2)$ par $factorielle(2-1)\times 2$ :
	$factorielle(2) = factorielle(2-1)\times 2 = factorielle(1)\times 2$
\end{itemize}

reste à résoudre $factorielle(1)$~:

\begin{itemize}
\item rejet de la première équation car $1$ est différent de la constante $0$.
\item essai de la seconde équation~: on peut poser $n=1$. Donc on remplace
$ factorielle(1)$ par $ factorielle(1-1)\times 2 $:

$$\begin{array}{rl}
factorielle(2) &= factorielle(1)\times 2 \\
		&= factorielle(1-1)\times 1\times 2 \\\
	& = factorielle(0)\times 1\times 2
\end{array}$$
\end{itemize}
reste à résoudre factorielle(0)~:
\begin{itemize}
\item la première équation ``colle''~: on remplace$ factorielle(0)$ par$ 1$, et on obtient~:

$$\begin{array}{rl}
factorielle(2)  &= factorielle(0)\times 1\times 2 \\
&= 1\times 1\times 2 \\
&= 2
\end{array}$$
\end{itemize}

Il est clair que si nous avions posé les équations dans l'ordre
inverse, nous n'aurions jamais pu détecter le cas de base, et le
calcul ne se serait jamais terminé.  En règle générale donc, il faut
écrire les équations de base en t\^ete, et ensuite les équations de
récurrence.


\paragraph{Remarque~: }on pouvait aussi écrire~:

\begin{verbatim}
dec factorielle : num -> num ;
--- factorielle (n) <= if n=0 
                       then 1 
                       else factorielle (n-1) * n ;
\end{verbatim}

Mais, sous cette forme, on voit moins bien les équations
sous-jacentes. Quand c'est possible, il faut éviter l'emploi de la
structure conditionnelle.


Exercices~:
\begin{exercice}
 Écrire une fonction \texttt{som} qui calcule la somme des $n$ premiers entiers strictement positifs, c'est-à-dire~:	$som(n) =   1+2+3+...+n $.

Indications~:
\begin{verbatim}
som(0) =
som(1) =
som(2) =
som(3) =
som(4) =
...
\end{verbatim}
\end{exercice}






\begin{exercice}
 Écrire une fonction \texttt{somcarre} qui calcule la somme des carrés des $n$ premiers entiers strictement positifs, c'est-à-dire~:	
$$ somcarre(n) =   1^2+2^2+3^2+\ldots.+n^2$$

\end{exercice}
\begin{exercice}
 Écrire une fonction \texttt{somcube} qui calcule la somme des cubes des n premiers entiers strictement positifs, c'est-à-dire~:	
$somcube(n) =   1^3+2^3+3^3+....+n^3$.
\end{exercice}	

\begin{exercice}	
 En utilisant le type prédéfini \texttt{truval} (\emph{truth value} =
 booléen) qui possède deux valeurs \texttt{true} et \texttt{false}, écrire une fonction
 \texttt{pair} qui détermine si un entier (positif) est pair ou non.
	
\begin{verbatim}
pair(0) =
pair(1) =
pair(2) =
pair(3) = 
...
\end{verbatim}
\end{exercice}



\begin{exercice}
Voici la suite de Fibonnacci  (Léonard de Pise 1170 ?- 1250)~:
$$ 1, 1, 2, 3, 5, 8, 13, 34, 55, 89 \ldots $$ Comme vous pouvez le remarquer,
chaque élément de cette suite est la somme des deux précédents (sauf
les deux premiers, qui sont les cas de base)~: il en sera évidemment
également ainsi pour les termes suivants.
\begin{verbatim}
dec fib : num -> num ;

--- fib (1) 	<=

--- fib (2) 	<=	

--- fib (n)	<=
\end{verbatim}
\end{exercice}

\begin{exercice}
Et vous ne pouvez ignorer le triangle de Pascal, les fameux
coefficients binomiaux~:
\begin{tabular}{lllll}
c(0,0)=1 \\
c(1,0)=1 &	c(1,1)=1\\
c(2,0)=1&	c(2,1)=2	&c(2,2)=1 \\
c(3,0)=1	&c(3,1)=3	&c(3,2)=3	&c(3,3)=1 \\
c(4,0)=1 	&c(4,1)=4 	&c(4,2)=6 	&c(4,3)=4	&c(4,4)=1\\
\end{tabular}


\end{exercice}


\section{Raisonnement par récurrence}

L'écriture d'une spécification récursive est une activité
intellectuelle très proche du raisonnement par récurrence. Soit à
démontrer par exemple la proposition~:

\paragraph*{Proposition}
	Pour tout entier $n$ positif ou nul,  on a  
$$som(n)= \frac{n\times(n+1)}{2}$$


\subsection{Preuve formelle}

Nous allons d'abord présenter une preuve formelle de cette
proposition~: la démonstration sera une suite d'étapes dont nous
montrerons systématiquement les justifications. Ainsi nous serons
certains que notre preuve est inattaquable.


\textbf{Début de la Preuve.}

Nous allons montrer que pour tout entier $n$ possède la
propriété  $P(n)$  définie par~: 
$$P(n) \equiv  \left( som(n)=\frac{n\times(n+1)}{2} \right)$$

\begin{itemize}
\item A. P(0) est vrai en effet~:

\begin{tabular}{rll}
1.&	$P(0) \equiv (som(0)=0)$	&	par définition de $P$	 \\
2.&	$som(0)=0$ &				par définition de $som(0)$ \\
3.&	$P(0)$ est vrai		&		conséquence de 1 et 2
\end{tabular}

\item B. pour tout entier $n$, $P(n) \Rightarrow  P(n+1)$, car

\begin{tabular}{rll}
4. &	$P(n)$ vrai		 & Hypothèse \\
5. &	$som(n)=\frac{n \times(n+1)}{2}$ &	conséquence de 4 et définition de P \\
6. &	$P(n+1) \equiv (som(n+1) = \frac{(n+1)\times(n+2)}{2})$ &par définition de P \\
7. &	$som(n+1) = som(n) + (n+1)$		& par définition de som \\
8. &	$som(n+1) = \frac{n\times(n+1)}{2}  + (n+1)$	&conséquence de 7 et 5 \\
9. &	$som(n+1) = \frac{n\times(n+1)}{2}  + \frac{2(n+1)}{2}$	&petit calcul \\
10.& 	$som(n+1) = \frac{(n+2)\times(n+1)}{2}$		&idem. \\
11.& 	$P(n+1)$ vrai				&conséquence de 6 et 10 \\
\hline 
12.&	$P(n) \Rightarrow P(n+1)$			&car l'hypothèse 4 entraîne 11 
\end{tabular}
\item C. D'après A et B, et en vertu du principe de récurrence, nous concluons~:
la propriété $P(n)$ est  vraie pour tout entier $n$ positif ou nul.
\end{itemize}
\textbf{Fin de la Preuve.}

On peut difficilement mettre en doute la rigueur d'un raisonnement
présenté sous cette forme !


\subsection{Preuve par transformation de programme}

Prouver qu'une fonction possède une certaine propriété revient à
démontrer qu'une autre fonction est constante. C'est évident~: sur
notre exemple, la propriété~:

\paragraph*{Propriété}	Pour tout entier positif ou nul, on a  $som(n)= \frac{n\times(n+1)}{2}$.

est vraie si et seulement si la fonction~:

\begin{verbatim}
dec P : num -> truval;
--- P(n) <= ( som(n) = n*(n+1)/2 );
\end{verbatim}
vaut \texttt{true} pour tout entier positif ou nul.

Nous allons modifier cette définition de \texttt{P}, pas à pas, jusqu'à ce que
nous puissions conclure que \texttt{P} vaut toujours \texttt{true}.

\begin{itemize}
\item Séparation en deux cas, selon que $n$ est nul ou non. On obtient une
définition équivalente~:

\begin{verbatim}
dec P : num -> truval
--- P(0) <= ( som(0) = 0*(0+1)/2 );
--- P(n) <= ( som(n) = n*(n+1)/2 );
\end{verbatim}

\item Comme $som(0)=0$ et, quand $n$ est différent de $0$, on a 
$som(n)=som(n-1)+n$, on simplifie~:
\begin{verbatim}
dec P : num -> truval;
--- P(0) <= true;
--- P(n) <= ( som(n-1)+n  = n*(n+1)/2 );
\end{verbatim}
\item Mais nous avons les équivalences~:
	
$$\begin{array}{llcl}
	& som(n-1) + n	&=&   \frac{n \times (n+1)}{2}    \\
\Leftrightarrow &   	som(n-1) 	&=&   \frac{n\times(n+1)}{2} - n    \\
\Leftrightarrow &	som(n-1) 	&=&   \frac{n\times(n -1)}{2}
\end{array}$$
\item  Mais, d'après la définition initiale de P, on a~:
$$\begin{array}{rl}
	P(n-1)  	&= (  som(n-1) = \frac{(n-1)\times(n+1-1)}{2})  \\
		&= (  som(n-1) = \frac{(n-1)\times n}{2} )
\end{array}$$

Ce qui amène à redéfinir une dernière fois \texttt{P}~:

\begin{verbatim}
dec P : num -> truval;
--- P(0) <= true ;
--- P(n) <= P(n-1) ;
\end{verbatim}
\end{itemize}

L'axiome d'induction nous permet de conclure que pour tout entier
positif ou nul n, \texttt{P(n) = true}.




\begin{exercice} Démontrez que 
$$\begin{array}{rl} 
somcarre(n) &= 1^2+2^2+3^2+\ldots+n^2) \\
	&= \frac{n(n+1)(2n+1)}{6}
\end{array}$$
\end{exercice}
\begin{exercice} $somcube(n) = (1^3+2^3+3^3+....+n^3) = som(n)^2$
  	(AL-KARAGI, fin du \siecle{X} siècle).\end{exercice}
\begin{exercice} $ 1\times factorielle(1)+2\times factorielle(2)+...+n\times factorielle(n) = factorielle(n+1)-1 $
\end{exercice}

\section{Fonctions d'ordre supérieur}

Vous avez sans doute remarqué une certaine ressemblance d'écriture
entre les fonctions \texttt{som}, \texttt{somcarre}, et
\texttt{somcube}. Si l'on vous demandait d'écrire une fonction \texttt{tresor}
qui calcule, pour tout entier $n$, la somme~:

\[	tresor(n) = azor(1) + azor(2) + .... azor(n) \]

dans laquelle \texttt{azor} est une fonction de type 
``\verb+num -> num+'', 
vous trouveriez que l'enseignant abuse de votre bonne volonté,
- ou manque sérieusement d'imagination - en vous posant toujours le
m\^eme genre d'exercices, que vous savez très bien faire.


% \citation{Programmer, c'est toujours la m\^eme chose (enfin presque).}

Dans la vie quotidienne du programmeur, cette situation se produit
fréquemment~: il vous faudra écrire un programme P2 qui fait presque
la m\^eme chose qu'un programme P1 que vous avez déjà écrit
auparavant~: t\^ache d'intér\^et uniquement alimentaire~! Dans un tel
cas bien s\^ur on ne repart pas de zéro~: on récupère le texte de P1,
et on le bricole un peu.

Par exemple, on ``bidouillerait'' la définition de \texttt{som} en la rebaptisant
\texttt{tresor}, et en rempla\c{c}ant \texttt{n+som(n-1)} par 
\texttt{azor(n)+tresor(n-1)}. Et le
tour est joué !


La programmation fonctionnelle offre une alternative plus séduisante
que ce bricolage indigne~: c'est l'utilisation de \emph{fonctions
d'ordre supérieur}. Expliquons-nous~: dans tous les cas il s'agit
d'additionner les valeurs $f(1)+f(2)+ ... + f(n)$, pour une certaine
fonction $f$ qui est soit l'identité, soit la fonction $carre$, la
fonction $cube$, $azor$ ou ce qu'on voudra. Alors nous allons écrire
une \emph{fonctionnelle} qui fera le calcul voulu pour tout entier $n$
et toute fonction $f$~:


\begin{verbatim}
dec somfonc : num X ( num -> num ) -> num ;
--- somfonc ( 0 , f )  <= 0 ;
--- somfonc ( n , f )  <= f(n) + somfonc (n-1 , f );
\end{verbatim}

\textbf{Vocabulaire}~: en mathématiques, on appelle
\emph{fonctionnelle} ou \emph{fonction d'ordre supérieur} une fonction
dont un (au moins) des paramètres est lui-m\^eme une fonction.

Maintenant \texttt{tresor} s'écrit plus simplement~:
\begin{verbatim}
dec tresor : num -> num ;
--- tresor ( n) <= somfonc ( n , azor ) ;
\end{verbatim}

Pour réécrire \texttt{somcarre} à l'aide de \texttt{trésor}, on peut~:

\begin{itemize}
\item soit se donner la peine d'écrire une fonction \texttt{carre}~:
\begin{verbatim}
dec carre : num -> num
--- carre(n) <= n*n ;

dec somcarre2 : num -> num ;
--- sommcarre2(n) <= somfonc( n , carre ) ;
\end{verbatim}

\item soit utiliser une \emph{fonction anonyme}~:
\begin{verbatim}
dec somcarre3 : num -> num ;
--- somcarre3(n) <= somfonc( n , lambda (n) => n*n  ) ;
\end{verbatim}
\end{itemize}

En HOPE la forme~:
\begin{alltt}
lambda ( \textrm{x, y, z} \ldots ) => \textrm{expression}
\end{alltt}
sert à désigner la fonction anonyme qui à $x, y, z \ldots$ fait correspondre
une certaine valeur décrite par l'expression. On peut utiliser les
lambda-expressions à volonté, tapez par exemple~:
\begin{alltt}
	(lambda (a,b) => 2*a - 3*b) (10,4) ;
\end{alltt}
et vous verrez appara\^{\i}tre ? 


\section{À propos des fonctions auxiliaires}

Pour résoudre un problème compliqué, on le décompose en problèmes plus
simples. En programmation fonctionnelle, on aura souvent besoin
d'écrire des fonctions auxiliaires, pour résoudre une partie d'un
problème. Pensez-y car~:

\begin{itemize}
\item Souvent on ne peut pas faire autrement ! Par exemple il semble
difficile décrire la fonction $f(n,k) = 1^k + 2^k + \ldots + n^k$ sans
écrire une fonction auxiliaire {\em a puissance b}
\item Vous aurez peut-\^etre besoin de la m\^eme fonction auxiliaire
ailleurs.
\item L'emploi d'une ``bonne'' fonction auxiliaire peut conduire à un
gain d'efficacité énorme.
\end{itemize}


Par exemple, avec la définition suivante :

\begin{verbatim}
dec fib : num -> num ;
--- fib(1) <= 1;
--- fib(2) <= 1 ;
--- fib(n) <= fib(n-1) + fib(n-2);
\end{verbatim}

Que vaut $mystere(1,n)$ ?

\begin{exercice}
Prouvez-le !
\end{exercice}


\begin{exercice}
Exercice 2 ( ``variable tampon'' )
\begin{verbatim}
dec bizarre : num X num X num -> num ;
--- bizarre ( a, 0, c) <= c;
--- bizarre ( a, b, c) <= bizarre(a-1, b, a*c);
\end{verbatim}
Que vaut $bizarre( n, p, 1)$ ? 
 
\end{exercice}


\begin{exercice}
\textbf{Test de primalité} : 
Un nombre (entier positif) est premier s'il admet exactement deux
diviseurs~: l'unité et lui m\^eme. Ècrire une fonction qui indique si
un nombre est premier.

Indications~:
\begin{itemize}
\item contrairement à une opinion très répandue, 1 n'est pas premier !
\item un nombre $ n \geq 2 $ n'est pas premier s'il a un diviseur dans
l'intervalle entier $\{ 2 \ldots n-1 \}$
\item $ \{ a \ldots b \} = \{ a \ldots b-1 \} \cup \{ b \}$ si  $a<b$.
\item $p$ est divisible par $q$ si et seulement si $p\ modulo\ q = 0$.
\end{itemize}

\end{exercice}


\chapter{Les listes}

%	4.1 Axiomatique des listes
%	4.2 Les listes en Hope
%	4.3 Induction naturelle sur les listes
%	4.4 Quelques méthodes de tri
%	4.5 Fonctionnelles usuelles
%	4.6 Exercices

\section{Axiomatique des listes}

Les \emph{listes} ou \emph{séquences} sont des groupements d'objets
qu'on ne peut accéder que dans un certain ordre préétabli.

Par exemple le troisième enregistrement d'un fichier séquentiel ne
peut \^etre consulté qu'après avoir lu les deux premiers
successivement.

Pour construire l'ensemble $ListeNombres$ (les listes de
nombres) par exemple, il nous suffit de deux choses~:
\begin{itemize}
\item la liste vide,
\item un moyen de rajouter un nombre quelconque à une liste.
De proche en proche, nous aurons donc la liste vide, puis les listes à un élément, puis à deux éléments, etc. 
\end{itemize}

Ce qui conduit à l'axiomatique suivante~:
\begin{enumerate}
\item $vide$ est une liste de nombres
\item si $n$ est un nombre et $l$ une liste de nombres, 
	alors $cons(n , l)$ est aussi une liste de nombres.

Ainsi, un élément de base (la liste vide) et une fonction 
\begin{verbatim}
cons : num X ListeNombres - > ListeNombres
\end{verbatim}
 nous permettent-ils de construire
des listes de plus en plus grandes~:

\begin{verbatimtab}
vide
cons(732,vide)
cons(1789, cons(732 , vide))
cons(1515, cons(1789 , cons(732 , vide))) 
\end{verbatimtab}
\item Il n'existe pas de couple $(n,l)$ tel que $cons(n , l) = vide$.
\item  cons est injective~: si $cons(n,l) = cons(n',l')$, c'est que 
forcément $n=n'$ et $l=l'$.
\item Si une propriété $P$ est vraie pour $vide$
	 et que $P(l) \Rightarrow  P( cons(n,l) )$ pour tout entier $n$ et toute liste $l$  
	alors $P$ est vraie pour toute liste de nombres.	(Principe d'induction)
\end{enumerate}
On retrouve là une axiomatique semblable à celle de PEANO pour les entiers. C'est bien normal, il suffit de faire correspondre à toute liste sa longueur~: 
\begin{itemize}
\item la liste vide a pour longueur zéro, 
\item si $l$ est de longueur $n$,  $cons( a,l )$ est de longueur $n+1 = succ(n)$.
\end{itemize}

\section{Les listes en Hope}

En Hope, il suffirait d'écrire (si les listes n'étaient pas déjà prédéfinies)~:

\begin{verbatimtab}
data ListeNombres == vide ++ cons( num X ListeNombres ) ;
\end{verbatimtab}

Ce qui définirait les listes de nombres (\texttt{num}), mais pas les
listes de caractères (\texttt{char}) ou de booléens (\texttt{truval}),
ni les autres.


\subsection{Le type générique \texttt{list}}

Le langage HOPE nous donne les moyens de construire des listes
d'objets d'un type arbitraire, par exemple on peut définir des listes
de nombres, des listes de caractères, des listes de listes de
caractères:

\begin{verbatimtab}
type ListeNombres == list(num);
type Chaine == list(char);
type Texte == list(Chaine);
\end{verbatimtab}

En quelque sorte les listes sont d'un type paramétrable (c'est ce
qu'on appelle la \emph{généricité}). Les \emph{listes génériques} HOPE sont
prédéfinies sous la forme~:

\begin{verbatimtab}
typevar truc ;
data list (truc) == nil 
                 ++  truc ::  list (truc) ;
\end{verbatimtab}

Remarque sur les notations~: par commodité ``\verb+::+'' a été
prédéclarée comme étant une opération infixe, ce qui explique pourquoi
on écrit~:
\begin{verbatimtab}
data list (truc) == nil 
                 ++  truc :: list (truc)  ;
\end{verbatimtab}
au lieu de~:
\begin{verbatimtab}
data list (truc) == nil 
		 ++  ::( truc X list (truc)  );
\end{verbatimtab}
mais la signification est  la m\^eme.

Attention, les listes doivent \^etre homogènes !

\begin{itemize}
\item \verb+{(3 :: (true :: nil))+ n'est pas une liste convenable.
\item \verb+(  ('a',true) :: (('b',false) :: nil ))+ est homogène~: 
ses éléments sont de type \verb+(char X truval)+.
\end{itemize}

\subsection{Notations simplifiées}

La notation des listes par des assemblages de \verb+::+ et \verb+nil+
est peu commode, à cause de la quantité de parenthèses. Hope permet
d'écrire les listes entre crochets, les éléments étant séparés par des
virgules. Exemples~:

\begin{verbatim}
[1,2,3,4]  =   (1 :: (2 :: (3 :: (4 :: nil) ) ) ) 
[ ]        =   nil
\end{verbatim}

Dans le cas des listes de caractères, on les écrit entre guillemets~:

\begin{verbatim}
"abcd"  =  [ 'a', 'b', 'c', 'd' ]  =  ('a' :: ('b' :: ('c' :: ('d' :: nil ))))
\end{verbatim}

Attention, il ne faudra pas confondre~:

\begin{itemize}
\item \verb+'a'+, 		qui est de type \texttt{ char},
\item \verb+"a" = ['a']+ 	qui est de type \texttt{list(char)},
\item \verb+["a"] = [ ['a']]+ 	qui est de type \texttt{list(list(char))}.
\end{itemize}

\begin{exercice}
Quels est le type des listes suivantes ? Vérifiez vos réponses sur machine.
\begin{verbatimtab}
1-      (1 :: ( 2::nil))				
2-      ('a' :: ('b' :: nil))
3-      (1 :: nil)
4-      ('w' :: nil)
5-      nil
6-      (false :: (true :: nil))
7-      "berlingot"
8-      [ "pommes","frites"]
9-      [ [ "pommes","vapeur"],["carottes","sautees"]]
10-     ( 12  :: (true :: nil) )
11-     (1 ::2)
\end{verbatimtab}
\end{exercice}

\section{Induction naturelle sur les listes}
\label{fonclistes}
Les listes sont définies à partir d'un objet de base 
\texttt{nil} et d'un constructeur  \verb+::+, selon le
\emph{schéma d'induction}~:

\begin{itemize}
\item $nil$ est une liste
\item si $a$ est un objet et $l$ une liste, alors $(a::l)$ est une liste
\end{itemize}

La majeure partie des fonctions que vous aurez à écrire seront
définies selon le m\^eme schéma d'induction. Par exemple, la fonction
qui retourne la longueur d'une liste~:
\begin{verbatimtab}
dec long : list(alpha) -> num;   ! le type des éléments est indifférent
--- long ( nil )  <= 0
--- long ( a::l ) <= 1+long(l);
\end{verbatimtab}
Ici nous avons défini la fonction $long$ en indiquant~:

\begin{itemize}
\item sa valeur pour l'élément de base $nil$,
\item comment on calcule, à partir de la valeur pour une liste $l$, la valeur de la fonction pour ($a::l$).
\end{itemize}

Nous pouvons suivre les étapes du calcul de $long( [ 12, 7, 4 ] )$~:
$$\begin{array}{rll}
long( [ 12, 7, 4 ] ) &= 1 + long( [7, 4 ] )&	\mbox{en appliquant la seconde équation} \\
			&= 1 + 1 + long( [ 4 ] )&\mbox{idem.}\\
			&= 1 + 1 + 1 + long ( [ ] )&\mbox{idem.}\\
			&= 1 + 1 + 1 + 0 &\mbox{en appliquant la première.} \\
			&= 3 &
\end{array}$$



\paragraph*{Remarque~:} C'est une fonction \emph{polymorphe}, 
dans la mesure o\`u elle peut agir sur plusieurs types de données~:
listes de nombres, de caractères, etc. Le polymorphisme, ajouté à la
génericité, donnent aux langages fonctionnels une grande puissance
d'abstraction. Dans les langages impératifs, si l'on veut calculer la
longueur de listes d'objets de 12 types différents, il faut écrire 12
sous-programmes.


\begin{exercice}
 Ecrire la fonction ``somme des éléments d'une liste de nombres''~:
\begin{verbatimtab}
dec somme : list ( num) -> num ;

--- somme ( [ ] )       <= 				

--- somme ( n :: l )    <=   somme(l)                    ;
\end{verbatimtab}
et montrez les étapes du calcul de somme ( [ 2,5,13,3 ] )~:
\begin{verbatimtab}
somme ( [ 2,5,13,3 ] ) 	=





\end{verbatimtab}
\end{exercice}






\begin{exercice}
 Ecrire une fonction qui indique si un élément est présent ou non dans une liste. Exemples~:
\begin{verbatimtab}
element (  'h' , "la cucarracha" ) = true
element ( 12 , [ 2,3,5,7,11,13] ) = false

dec element :  alpha X list(alpha) ->

--- element ( e , [ ] )         <=

--- element ( e , (a::l) )      <=  


\end{verbatimtab}
\end{exercice}
\begin{exercice}
 Ecrire la fonction ``nombre d'éléments qui sont supérieurs à une
 certaine valeur''.
Exemple~: \verb+super([1,3,12,4,6,2] , 5) = 2+	puisque 2 éléments sont 
plus grands que 5.
\begin{verbatimtab}
dec super : list ( num ) X num  -> num ; 

--- super ( [ ]     , val )     <=

--- super ( n :: l , val )      <= 		


\end{verbatimtab}
\end{exercice}

\begin{exercice}
Ecrire la fonction ``liste des éléments qui sont supérieurs à une certaine valeur''.

Exemple~: \verb+suplis([1,3,12,4,6,2] , 5) = [12,6]+
\begin{verbatimtab}
dec suplis : list ( num ) X num  -> 

--- suplis ( [ ]     , val ) 	<=

--- suplis ( n :: l , val )	<= 		

\end{verbatimtab}
\end{exercice}

\begin{exercice}

Ecrire une fonction qui, à tout entier positif $n$, fait correspondre la liste \texttt{[$n$,$n-1$,$n-2$,\ldots,$2$,$1$]}.
Exemple~: \verb+{yop(5) = [5,4,3,2,1]+
\begin{verbatimtab}
dec yop : 

---

---

\end{verbatimtab}
\end{exercice}
\begin{exercice}
Démontrez que pour tout entier $n$ on a 	
$$long(yop(n)) = n$$  
\end{exercice}


\begin{exercice}





Démontrez que $$somme(yop(n))=som(n)$$

\end{exercice}








\begin{exercice}

 Ecrire une fonction qui concatène deux listes (c-à-d qui les met bout-à-bout).
Exemple~: \verb+conc("abra","cadabra") = "abracadabra"+

\begin{verbatimtab}
dec conc : list(alpha) X list(alpha) -> list(alpha) ;

--- conc ( [  ]   ,  l2 ) 	<=

--- conc ( (a::l) , l2 )	<= 

\end{verbatimtab}
\end{exercice}

\paragraph*{Remarque~: } Cette fonction de concaténation 
est très utilisée en pratique. Par souci d'efficacité, elle a été
intégrée à l'interprèteur sous forme d'une fonction prédéfinie ``\verb+<>+''
(en notation infixe)~:

\begin{verbatimtab}
"etoile " <> "des neiges" = "etoile des neiges"
\end{verbatimtab}

\begin{exercice}
Démontrez les propriétés suivantes de la concaténation~:
\begin{itemize}
\item $nil$ est élément neutre pour l'opération $conc$~: 
$$\begin{array}{rcl}
conc(nil,l2) &=& l2\\
conc(l1,nil) &=& l1
\end{array}$$
\item la concaténation est associative~: 
$$ conc(l1,conc(l2,l3)) = conc( conc(l1,l2) , l3) $$
\item $long(conc(l1,l2)) = long(l1) + long(l2)$
\item $ suplis(conc(l1,l2),val) = conc( suplis(l1,val) , suplis(l2,val) )$
\end{itemize}
\end{exercice}

\section{Quelques méthodes de tri}

A titre d'exemple de programmes fonctionnels sur les listes, nous allons maintenant voir quelques méthodes pour trier une liste de nombres.

\subsection{Tri par insertion}

\paragraph*{Le principe~:}  Pour trier la liste à 4 
éléments \verb+[8,5,12,3]+~:
\begin{itemize}
\item on enlève un élément (le premier = 8)
\item on trie
 (récursivement) le reste de la liste~: on obtient la liste ordonnée 
\verb+[3,5,12]+
\item on insère le premier élément (8) à sa place
\item ce qui donne [3,5,8,12]~: le résultat voulu.
\end{itemize}
	
Mais comment a-t-on trié la liste à 3 éléments \verb+[5,12,3]+~? Et bien, de la m\^eme fa\c{c}on~:
\begin{itemize}
\item on a enlevé le premier (5)
\item on a trié le reste~: ce qui donnait \verb+[3,12]+
\item on a inséré le premier élément (5) à sa place
\item et on a obtenu \verb+[3,5,12]+.
\end{itemize}

Mais comment a-t-on trié \verb+[12,3]+~?
\begin{itemize}
\item ...
\end{itemize}

\paragraph*{Mise en oeuvre~:} Tout d'abord il 
nous faut une fonction auxiliaire pour insérer un élément à sa place
dans une liste ordonnée.

\begin{verbatimtab}
dec insertion : num X list(num) -> num ;

--- insertion (element , [ ] ) 	<=			;

--- insertion (element, premier :: reste)  
                           <= if element < premier
                                 then

                                 else
 
\end{verbatimtab}

Ceci fait, nous pouvons écrire la fonction TriInsertion~:

\begin{verbatimtab}
dec TriInsertion : list(num) - > list(num);

--- TriInsertion ( [ ] )              <=

--- TriInsertion ( premier :: reste ) <= 

\end{verbatimtab}

Cette méthode est facile à programmer, mais elle n'est pas très
efficace~: en effet dans le pire des cas, par exemple la liste
\verb+[15,13,8,5,1]+ les insertions se font toujours à la fin. Pour insérer
15 dans la liste triée \verb+[1,5,8,13]+ il faut 5 étapes de calcul. Pour
insérer 13 dans \verb+[1,5,8]+ il a fallu 4 étapes, etc.

Donc, toujours dans le pire des cas (une liste ordonnée en sens
inverse de longueur n), il faudra effectuer $n+(n-1)+(n-2)+\ldots+2+1 =
\frac{n(n+1)}{2}$ étapes. Le temps du calcul est donc de l'ordre de $n^2$.

\subsection{Tri par partition (version na\"{\i}ve)}

\paragraph*{Le principe~:}  Pour trier la liste à 5 éléments 
\verb+[8,12,5,3,9]+
\begin{itemize}
\item on met de c\^oté le premier élement (8)n
\item on extrait du reste deux listes : 
\begin{itemize}
	\item les éléments plus petits que 8, 
	\item ceux qui sont plus grands~;
\end{itemize}
\item ce qui donne deux listes \verb+[5,3]+  et \verb+[12,9]+
\item on trie ces deux listes (récursivement) ; on trouve alors
\verb+[3,5]+ et \verb+[9,12]+
\item on regroupe~:  \verb+[3,5] <> (  [8] <>  [9,12] )+
\item ce qui donne la liste triée \verb+[3,5,8,9,12]+.
\end{itemize}

\paragraph*{Mise en oeuvre~:} Il nous faut d'abord 
deux fonctions, qui extraient respectivement d'une liste les éléments
qui sont plus petits (ou plus grands) qu'un certain nombre.

\begin{verbatimtab}
dec PlusPetits : num X list(num) -> list(num);

--- PlusPetits ( n , [ ] ) <=

--- PlusPetits ( n , p :: r ) <=


dec PlusGrands : num X list(num) -> list(num);

--- PlusGrands ( n , [ ] ) <=

--- PlusGrands ( n , p :: r ) <=

\end{verbatimtab}

Et la définition du tri par partition s'en suit facilement~:

\begin{verbatimtab}
dec TriPartition : list(num) -> list(num) ;
--- TriPartition ( [ ] )    <= [ ]  ;
--- TriPartition ( p :: r ) <=  
              let pp == PlusPetits(p, r)
           in let pg == PlusGrands(p, r)
           in TriPartition(pp) <>( [p] <> TriPartition(pg) ) ;
\end{verbatimtab}

\subsection{Tri par partition (version améliorée)}

Le co\^ut de calcul peut \^etre diminué par une technique relativement
simple. D'abord o\`u est le problème~? Il vient de ce que le co\^ut
d'une concaténation \verb+g<>d+ est proportionnel à la longueur de la liste
\verb+g+. Et donc le co\^ut de l'évaluation de l'expression~:
\begin{verbatim}
TriPartition(pp) <> ( [p] <> TriPartition(pg) ) 
\end{verbatim}
est la somme du co\^ut
du tri de \verb+pp+ et \verb+pg+, et d'un facteur proportionnel à la taille de
\verb+pp+. (C'e\^ut été encore pire en groupant les parenthèses différemment)

Voici la technique~: on définit une nouvelle fonction à partir de 
\verb+TPConc+, apparemment plus compliquée, en ajoutant un paramètre  supplémentaire  (appelé parfois \emph{paramètre tampon}):
\begin{verbatimtab}
dec TPConc : list(num) X list(num) -> list(num);
--- TPConc( premiere , seconde ) <= TriPartition( premiere ) <> seconde ;
\end{verbatimtab}
Remarquez que~: 	
\begin{verbatim}
TriPartition( liste ) = TPConc( liste , [ ] ) 
\end{verbatim}


Maintenant nous allons voir que nous pouvons redéfinir \texttt{TPConc} de
manière à n'utiliser, dans sa définition, ni la concaténation , ni
\texttt{TriPartition}.

Dédoublons \texttt{TPConc}, selon que son premier argument est la liste vide ou pas~:
\begin{verbatimtab}
dec TPConc : list(num) X list(num) -> list(num);
--- TPConc( [ ]  , seconde ) <= TriPartition( [ ] ) <> seconde ;
--- TPConc(  p::r , seconde ) <= TriPartition( p::r ) <> seconde ;
\end{verbatimtab}
En utilisant la définition de \texttt{TriPartition} ceci équivaut à~:
\begin{verbatimtab}
dec TPConc : list(num) X list(num) -> list(num);
--- TPConc( [ ]  , seconde ) <= [ ] <> seconde ;
--- TPConc ( p :: r , seconde) )  
           <=  let pp == PlusPetits (p , r)
            in let pg == PlusGrands(p, r)
               in TriPartition(pp) <> ([p] <> TriPartition(pg)) <> seconde) ;
\end{verbatimtab}
La concaténation étant associative, on va pouvoir introduire TPConc~:
\begin{verbatimtab}
TriPartition(pp) <> ([p] <> TriPartition(pg) )  <> seconde
        = TriPartition(pp) <> ([p] <> (TriPartition(pg)   <> seconde)) ;
        = TriPartition(pp) <> ([p] <> TPConc(pg, seconde) );
        = TriPartition(pp) <> ( p :: TPConc(pg, seconde) ;
        = TPConc( pp , p::TPConc(pg,seconde) )
\end{verbatimtab}
Ce qui nous mène à une version nettement améliorée du tri par partition~:
\begin{verbatimtab}
dec TPConc : list(num) X list(num) -> list(num);
--- TPConc( [ ]  , seconde ) <=  seconde ;
--- TPConc ( p :: r , seconde) )  
           <=  let  pp == PlusPetits (p , r)
             in let pg == PlusGrands(p, r)
             in TPConc( pp , p::TPConc(pg,seconde) );

dec TriPartition : list(num) -> list(num) ;
--- TriPartition ( liste ) <= TPConc( liste , [ ] );
\end{verbatimtab}
On peut montrer que le c\^out moyen du tri d'une liste de n éléments
est proportionnel à $n \times log(n)$, ce qui est bien meilleur que pour le
tri par insertion. Cependant le co\^ut maximal (dans le pire des cas,
qui est statistiquement très rare) reste de l'ordre de $n^2$.

\subsection{Le tri-fusion}

Pour terminer, une méthode à la fois élégante et efficace, puisque ses co\^uts moyens et maximaux sont proportionnels à $n \times log(n)$.

\paragraph*{Principe~:} Pour trier une liste de 7 éléments 
\verb+[3,5,2,9,7,1,0]+~:
\begin{itemize}
\item on partage cette liste en deux sous-listes, en prenant un
	 élément sur deux.  On obtient alors deux listes \verb+[3,2,7,0]+ et
	\verb+[5,9,1]+ 
\item on les trie, récursivement. On obtient deux listes
	ordonnées \verb+[0,2,3,7]+ et \verb+[1,5,9]+
\item on fusionne ces deux listes,
	 ce qui donne le résultat \verb+[0,1,2,3,5,7,9]+.
\end{itemize}

L'opération de \emph{fusion} ou \emph{interclassement} (algorithme
classique en informatique de gestion) consiste à construire une liste
ordonnée à partir de deux listes également ordonnées~:
\begin{verbatimtab}
dec fusion : list(num) X list(num) -> list(num) ;

--- fusion ( [ ] , l2) <=

--- fusion ( l1, [ ] ) <=

--- fusion ( p1::r1 , p2::r2 ) <=  if p1<p2     then

                                                else
\end{verbatimtab}

Il faut savoir extraire un élément sur deux~:
\begin{verbatimtab}
dec RangPair, RangImpair : list(num) -> list(num);

--- RangPair ( [ ] ) 	<= [ ]
--- RangPair ( p::r ) 	<= RangImpair(r);

--- RangImpair ( [ ] ) 	<= [ ] ;
--- RangImpair ( p::r ) <= p :: RangPair(r) ;	!  récursivité croisée
\end{verbatimtab}
et le tri-fusion s'écrit facilement~:
\begin{verbatimtab}
dec TriFusion : list(num) X list(num) -> list(num) ;
--- TriFusion ( [ ] ) <= [ ] ;
--- TriFusion ( [ seul ] ) <= [ seul ] ;
--- TriFusion ( liste ) <= let (l1,l2) == (RangImpair(liste),RangPair(liste))
				in Fusion( TriFusion(l1) , TriFusion(l2) );
\end{verbatimtab}
Question~: Pourquoi devons nous traiter séparément le cas des listes à
un seul élément ?



\section{Fonctionnelles usuelles sur les listes}

Les fonctions simples vues en \ref{fonclistes} se généralisent
facilement en fonctionnelles ``d'intér\^et général''. Par exemple la
fonction ``nombre des éléments supérieurs à une certaine valeur''
\begin{verbatimtab}
dec super : list(num) X num -> num ;
--- super ( [ ] , val ) <= 0 ;
--- super ( n::l , val) <= if n>val 	then 1+super(l) 
					else super(l);
\end{verbatimtab}
est un cas particulier de la fonction ``nombre des éléments qui
possèdent une certaine propriété P''. Il suffit de passer en paramètre
le prédicat (fonction à résultat booléen) qui indique si un certain $x$
possède ou non la propriété recherchée. Par exemple on passera le
prédicat ``\verb+lambda (x) => x>5+'' 
pour compter les éléments supérieurs à 5. Voici la fonctionnelle~:
\begin{verbatimtab}
dec combien : list(num) X (num -> truval) -> num ;
--- combien ( [ ] , P ) <= 0 ;
--- combien ( n::l , P ) <= if P(n) 	
                            then 1+combien(l) 
		  	     else combien(l);
\end{verbatimtab}
De plus nous n'avons aucune raison de nous limiter aux listes de
nombres~: cette fonctionnelle marche pour des listes de tous types, à
condition bien s\^ur que le domaine du prédicat soit du type
convenable. Nous obtenons la fonctionnelle~:
\begin{verbatimtab}
dec combien : list(alpha) X (alpha -> truval) -> num ;
--- combien ( [ ] , P ) <= 0 ;
--- combien ( n::l , P ) <= if P(n) 	
                            then 1+combien(l,P) 
			    else combien(l,P);
\end{verbatimtab}

\begin{exercice}

\begin{enumerate}
\item Généraliser \texttt{suplis} (liste des
éléments supérieurs à une valeur) pour obtenir une fonctionnelle
\texttt{selection} qui extrait d'une liste les éléments qui
  possèdent une certaine propriété (par exemple ceux qui sont pairs,
  ou ceux qui sont premiers, ou qui sont entre \texttt{"BERTHE"} et
  \texttt{"CHARLOTTE"}, etc.)
\item
 Montrez comment écrire \texttt{suplis} à partir de \texttt{sélection}.
\end{enumerate}
\end{exercice}


\section{Exercices}

\subsection{Sur les fonctionnelles}
\begin{exercice}
 Ecrire une fonction de tri à tout faire (il faudra passer en paramètre un prédicat qui représente la relation d'ordre choisie). Par exemple~:
\begin{verbatim}
TriGeneral ( [ 1,5,4,3,2 ]  , lambda(a,b) => a < b ) = [1,2,3,4,5]
TriGeneral ( [ 1,5,4,3,2 ]  , lambda(a,b) => a > b ) = [5,4,3,2,1]
\end{verbatim}
\end{exercice}

\begin{exercice}
\begin{enumerate}
\item Ecrire une fonction \texttt{liscar} qui prend comme paramètre une
liste de nombres, et renvoie la liste des carrés de ces
nombres. Exemple~:
\begin{verbatim}
liscar [2, 8, 3] = [4, 64, 9]
\end{verbatim}
\item
Genéraliser cette fonction pour obtenir en une fonctionnelle \texttt{map}~:
liste des images par une certaine fonction.
\item Exprimer \texttt{liscar} à partir de \texttt{map}.
\end{enumerate}
\end{exercice}

\begin{exercice}
\begin{itemize}
\item 
   Trouvez une fonctionnelle \texttt{reduction} qui généralise les
   deux fonctions "somme des éléments d'une liste", et "produit des
   éléments d'une liste".
\item
 Montrez que cette fonctionnelle \texttt{reduction} permet
 d'exprimer \texttt{map} aussi bien que \texttt{combien}.
\end{itemize}
\end{exercice}


\subsection{Sur les transformations de programmes}

(revoir méthode du paramètre supplémentaire)~:

\begin{exercice}
Ecrire une fonction \texttt{iota} qui à tout entier $n$ fait
correspondre la liste des $n$ premiers entiers positifs dans le sens
croissant.  Exemple \verb/iota(5) = [1,2,3,4,5]/.

Soit \texttt{iotaconc} la fonction définie par 
\verb+iotaconc( n , liste ) = iota(n) <> liste +
\begin{itemize}
\item donnez une définition récursive directe de \texttt{iotaconc}.
\item en déduire une définition plus efficace de \texttt{iota}.
\end{itemize}
\end{exercice}
	
\begin{exercice}
Par la m\^eme méthode, donnez une définition efficace (conduisant à un
	co\^ut linéaire) de l'inverse d'une liste. Exemple~:
	\verb+inverse("pomme" ) = "emmop"+
\end{exercice}



\chapter{Structures arborescentes}



%	5.1 Les arbres binaires
%	5.2 Fonctions sur les arbres
%	5.3 Arbres binaires de recherche
%	5.4 Expressions arithmétiques
%	5.5 Calcul symbolique





\section{Les arbres binaires}

Nul besoin n'est de présenter ici la notion d'arbre~: répertoires et
fichiers, structure des programmes, tout ou presque est arborescence
en Informatique. Dans cette partie nous nous restreindrons à la
catégorie la plus simple~: les arbres binaires. Nous montrerons un peu
plus loin une utilisation (très classique) des arbres binaires pour le
tri.

Un arbre binaire est, en quelques mots, un arbre dont les noeuds
"portent" deux sous-arbres, à gauche et à droite.  Voici quelques
exemples (rappelons que, par convention, la racine de l'arbre est en
haut)~:

\begin{center}

\setlength{\unitlength}{1mm}
\begin{picture}(105,55)

\put(10,50){\line(-1,-3){5}}
\put(10,50){\line(1,-3){5}}

\put(15,35){\line(-1,-2){5}}
\put(15,35){\line(1,-2){5}}

\put(10,50){\circle*{1}}
\put(5,35){\circle*{1}}
\put(15,35){\circle*{1}}

\put(10,25){\circle*{1}}
\put(20,25){\circle*{1}}

\put(5,10){\makebox(10,10){1}}



\put(40,50){\line(-1,-3){5}}
\put(35,35){\line(-1,-2){5}}
\put(35,35){\line(1,-2){5}}
\put(15,35){\line(-1,-2){5}}
\put(40,24){\line(-1,-1){5}}

\put(40,50){\circle*{1}}
\put(35,35){\circle*{1}}
\put(30,25){\circle*{1}}
\put(40,25){\circle*{1}}
\put(35,20){\circle*{1}}

\put(35,10){\makebox(10,10){2}}

% \put(70,50){\line(1,-3){5}}
\put(70,50){\circle*{1}}
% \put(75,35){\circle*{1}}

\put(65,10){\makebox(10,10){3}}


\put(95,10){\makebox(10,10){4}}
\end{picture}


\end{center}

Les arbres binaires peuvent \^etre vides~: c'est le cas par exemple du
sous-arbre droit du second exemple, et aussi du quatrième exemple
(invisible). Le troisièmes est réduit à un simple racine.
On appelle \emph{feuilles de l'arbre}  les sommets dont les deux
sous-arbres sont vides.

\begin{exercice}
\begin{itemize}
\item
 Dessinez tous les arbres à  0, 1, 2, 3 et 4 sommets.
\item  En général, combien y a-t'il d'arbres à $n$ sommets ?
\end{itemize}
\end{exercice}


\subsection{Définition inductive des arbres binaires}

Le plus souvent, on associe une information (\emph{étiquette}) à
chaque noeud de l'arbre~.

\begin{center}

\setlength{\unitlength}{1mm}
\begin{picture}(60,50)

\put(25,40){\line(-1,-1){15}}
\put(25,40){\line(1,-1){15}}

\put(40,25){\line(-2,-3){10}}
\put(40,25){\line(2,-3){10}}

\put(25,40){\makebox(10,10)[b]{a}}
\put(10,25){\makebox(10,10)[b]{b}}
\put(40,25){\makebox(10,10)[b]{c}}
\put(30,10){\makebox(10,10)[b]{k}}
\put(50,10){\makebox(10,10)[b]{w}}

\end{picture}


\end{center}


Voici une définition inductive de l'ensemble $ArbreBin(E)$ arbres
binaires prenant leurs étiquettes dans un certain ensemble
(tout-à-fait arbitraire) $E$. 

Pour commencer, il nous faut un objet représentant l'arbre
vide, et une fonction $noeud~: ArbreBin(E) \times E \times ArbreBin(E) \rightarrow
ArbreBin(E)$ pour fabriquer un arbre à partir de deux arbres plus
petits et d'une étiquette.

On pose les axiomes suivants~:
\begin{enumerate}
\item L'arbre vide appartient à $ArbreBin(E)$
\item si $A1$ et $A2$ appartiennent à $ArbreBin(E)$ et $e$ appartient à $E$,
		alors $noeud(A1,e,A2)$ appartient à $ArbreBin(E)$
\item $vide$ n'est l'image par $noeud$ d'aucun triplet $(A1,e,A2)$
\item  $noeud$ est injective
\item soit $P$ une propriété sur $ArbreBin(E)$. Si
\begin{itemize}
\item $P(vide)$ est vraie, 
\item pour tous $A1,A2$ dans $ArbreBin(E)$ possédant la propriété P, 
et quelque soit $e$ dans $E$, on a  $P( noeud(A1,e,A2) )$;
\end{itemize}
		alors $P$ est vraie pour tout $a$ dans $ArbreBin(E)$
\end{enumerate}

\subsection{Définition en Hope}

En Hope, c'est une définition d'un type générique~:

\begin{verbatimtab}
data ArbreBin(alpha) == vide 
                     ++ noeud(ArbreBin(alpha) X alpha X ArbreBin(alpha));
\end{verbatimtab}

L'exemple précédent, qui est de type \texttt{ArbreBin(char)}, est représenté par l'expression~:
\begin{verbatimtab}
dec ex1 : ArbreBin(char);
--- ex1 <= noeud( noeud( vide,'b',vide ) , 
		'a' , 
		noeud( noeud(vide,'k',vide) , 
			'c', 
			noeud(vide,'v',vide) ) ):
\end{verbatimtab}

\section{Fonctions sur les arbres}

La plupart des fonctions sur les arbres sont construites à partir du
schéma d'induction naturelle. Par exemple la fonction ``Somme des
étiquettes d'un arbre de nombres''~:
\begin{verbatimtab}
dec somarbre : ArbreBin(num) -> num ;
	
--- somarbre ( vide ) 	<= 

--- somarbre ( noeud(a1,e,a2) )  <= 	somarbre(a1)		somarbre(a2)
\end{verbatimtab}

\begin{exercice}
Ecrire une fonction ``liste des étiquettes des feuilles d'un arbre''.
\begin{verbatimtab}
dec ListeFeuilles : ArbreBin(alpha) -> list(alpha) ;

--- ListeFeuilles ( vide )  <=

--- ListeFeuilles ( noeud (a1,e,a2) ) <= 

\end{verbatimtab}
\end{exercice}

\begin{exercice}

\begin{itemize}
\item Ecrire une fonction $ListePrefixe$ qui renvoie la liste des
étiquettes d'un arbre, dans l'ordre préfixe, c'est-à-dire d'abord
l'étiquette de la racine, puis celles des sous-arbres gauche et
droit. Par exemple~: \verb+ListePrefixe(ex1) = "abckw"+
\item
M\^eme question pour l'ordre infixe (d'abord le sous-arbre gauche,
puis la racine et enfin le sous-arbre droit). Par exemple~:
\verb+ListeInfixe(ex1) <= "bakcw"+
\item Montrez que, pour chacune de ces fonctions, le temps de calcul (évalué
en nombre d'étapes) est (dans le pire des cas) proportionnel au carré
du nombre de sommets de l'argument.
\item
Transformez ces définitions (méthode du ``param\^etre tampon'') pour obtenir des fonctions dont le temps de calcul soit linéaire.
\end{itemize}
\end{exercice}

\section{Arbres binaires de recherche}

Nous avons vu au chapitre précédent une version du tri par insertion
qui n'était guère efficace, parce que nous cherchions à insérer un
élément dans une liste. Au lieu de construire une liste, nous allons
fabriquer un arbre de recherche qui contiendra tous les objets à
trier.

\paragraph*{Définition~:} un \emph{arbre de recherche} 
est un arbre binaire dans lequel toutes les étiquettes du sous-arbre
gauche (resp. droit) sont inférieures (resp. supérieures ou égales) à
l'étiquette de la racine, et de m\^eme pour tous les sous-arbres de
cet arbre.

Exemple : 
\begin{center}
%
% Un arbre de recherche équilibré
%
\setlength{\unitlength}{1mm}
\begin{picture}(40,35)

\put(20,30){\line(-2,-3){10}}
\put(20,30){\line(2,-3){10}}
\put(10,15){\line(-1,-2){5}}
\put(10,15){\line(1,-2){5}}

\put(22,30){10}
\put(12,15){5}
\put(32,15){12}
\put(7,5){7}
\put(17,5){8}
\end{picture}


\end{center}

\begin{exercice}
Ecrire une fonction qui recherche la plus petite étiquette dans un
arbre binaire de recherche (non vide bien s\^ur).
\begin{verbatimtab}
dec PlusPetite : ArbreBin(num) -> num;

--- PlusPetite
\end{verbatimtab}

M\^eme question pour la plus grande étiquette.
\end{exercice}


\begin{exercice}
 Ecrire une fonction qui indique si un arbre binaire quelconque est oui ou non un arbre de recherche.
\begin{verbatimtab}
dec estOrdonne : ArbreBin(num) -> truval ;

--- estOrdonne( vide ) <= true ;

--- estOrdonne( noeud(a1,e,a2) ) <= 
\end{verbatimtab}

\end{exercice}

\begin{exercice}
Montrez que si l'arbre binaire A est un arbre de recherche, alors la
liste $ListeInfixe(A)$ est ordonnée.
\end{exercice}

\subsection{Insertion dans un arbre binaire de recherche}

Pour insérer 9 dans l'arbre de l'exemple, il faudra d'abord aller à
gauche car $9<10$, puis à droite puisque $5<9$, puis encore à droite car
$8<9$. La place étant libre, on peut alors y mettre 9.

\begin{verbatimtab}
dec Insertion : num  X ArbreBin(num) -> ArbreBin(num);

--- Insertion ( n , vide ) <= 

--- Insertion ( n , noeud(a1,e,a2) ) <= if n<e 

			then noeud( Insertion(n,a1) , e , a2 ) 

			else 

\end{verbatimtab}

\begin{exercice}

Montrez que si l'arbre A est ordonné, alors Insertion(n,A) est
également ordonné.
\end{exercice}

Rappelons le principe du tri par insertion. On dispose d'une liste de
nombres à trier. On prend les éléments de cette liste, et on les
insère un-à-un dans une structure de données qui était vide au
départ. Dans la première version la structure était une liste, ici la
structure de données c'est un arbre binaire de recherche. Voici
comment on fabrique un arbre à partir d'une liste~:
\begin{verbatimtab}
dec Arbre : list(num) -> ArbreBin(num) ;
--- Arbre ( [ ] ) <= vide ;
--- Arbre ( n::r ) <= Insertion ( n , Arbre( r ) );
\end{verbatimtab}
Et nous obtenons un nouveau tri par insertion~:
\begin{verbatimtab}
dec TriIns : list( num ) -> list( num ) ;
--- TriIns (liste) <= ListeInfixe (FabriquerArbre ( liste ) ) ;

\end{verbatimtab}

\subsection{ Expressions arithmétiques}

Les expressions arithmétiques, comme par exemple $3.a.x + b + 1$ sont
également des structures arborescentes familières. Les objets de base
en sont les nombres~: ici $1$ et $3$, les variables~: $a, x, b$, et ils sont
combinés par les opérateurs arithmétiques~: addition, multiplication,
etc.

En Hope on peut très facilement créer un ensemble \texttt{expr} semblable aux
expressions arithmétiques (nous nous limiterons aux 4 opérations de
base) en écrivant, dans un premier temps~:
\begin{verbatimtab}
type chaine == list(char);	! par commodité

data expr == 			! une expression peut être ...
   	nombre(num)		! 	- un nombre ayant une certaine valeur
++	var(chaine)		! 	- une variable avec un nom
++	plus( expr X expr )	! 	- la somme ...
++ 	moins ( expr X expr )	!	- la différence ...
++ 	mult ( expr X expr )	! 	- le produit ...
++ 	divis ( expr X expr ) ;	!	- ou le quotient de deux expressions
\end{verbatimtab}

L'expression ``$a.x + b$" sera représentée par~:  

\begin{verbatim}
plus( mult ( var("a") , var("x") ) , var("b") );
\end{verbatim}

\paragraph*{Remarque}

En Hope on ne peut pas réutiliser les symboles \verb/+/, \verb/-/,
etc. comme constructeurs pour de nouveaux types. En effet cette
\emph{surcharge} (\emph{overloading}) conduirait à des ambiguités sur les types
des objets~: la fonction ``\verb/+/'' serait à la fois de type 
``\verb/num X num -> num/'' et ``\verb+expr X expr -> expr+''.


Il est préférable de signaler préalablement que \verb+plus+,
\verb+moins+, etc. sont des opérateurs infixes avec des priorités
semblables à celles de ``\verb/+/'', ``\verb/-/'', etc. Ceci conduit à~:
\begin{verbatimtab}
infix plus, moins : 5 ;
infix mult, divis : 6 ;
type chaine == list(char);

data expr == nombre(num)
	  ++ var(chaine)
 	  ++ expr plus expr
	  ++ expr moins expr
 	  ++ expr mult expr
  	  ++ expr divis expr ;
\end{verbatimtab}
Déclarons maintenant quelques exemples, sous forme d'une fonction qui
renvoie une expression correspondant au numéro d'exemple que l'on
veut~:
\begin{verbatimtab}
dec exemple : num -> expr;
--- exemple(1) <= var "a" mult var "x"  plus var "b" ;
--- exemple(2) <= nombre 1 divis var "x";
--- exemple(3) <= nombre 1 divis exemple(1);
\end{verbatimtab}

\begin{exercice}

\begin{itemize}
\item  A quelles expressions correspondent ces trois exemples ?
\begin{verbatimtab}
exemple 1 -> 

exemple 2 ->

exemple 3 ->
\end{verbatimtab}
\item Ajouter un exemple 4 représentant $a.x^2 - b.x + c$.
\begin{verbatimtab}
exemple(4) <= 
\end{verbatimtab}
\end{itemize}
\end{exercice}
\section{Calcul symbolique}

Dans cette partie nous allons développer un petit exemple typique de ce qu'on appelle programmation symbolique  ou encore manipulation d'expressions symboliques.

Il s'agit d'écrire un programme capable d'effectuer, comme tout
honn\^ete bachelier, la différentiation (ou dérivation) d'une
expression par rapport à une variable. Pour un début, nous
nous contenterons des expressions simples, limitées aux 4 opérations,
que nous avons vues dans la partie précédente.

Vue de près, la différentiation est une opération \texttt{diff} qui, à partir
d'une expression $E$ et d'une variable $V$, permet de trouver une autre
expression $E'$ qui représente la dérivée de $E$ par rapport à $V$. Par
commodité, nous déclarons \texttt{diff} comme opération infixe, ce qui donne~:
\begin{verbatimtab}
infix diff : 4 ;
dec diff : expr X expr -> expr ;
\end{verbatimtab}
Il ne nous reste plus qu'à étudier les différents cas, qui
correspondent aux différentes manières de manières de fabriquer les
expressions~:
\begin{verbatimtab}
--- nombre n     	diff  v       <= nombre 0 ;

--- var x        	diff  v       <= if var x = v
					then 

					else

--- (f plus g)	diff v	<= let (ff,gg) == (f diff v, g diff v)

				in ff plus gg ;

--- (f moins g)	diff v	<= let (ff,gg) == (f diff v, g diff v)
                                  
				in

--- (f mult g)	diff v	<= let (ff,gg) == (f diff v, g diff v)

				in

--- (f divis g)	diff v	<= let (ff,gg) == (f diff v, g diff v)
                                  
				in 
\end{verbatimtab}

Et voilà tout. Ce programme de quelques lignes sait calculer la
dérivée d'une expression par rapport à une variable ! Quelques
exemples pour s'en convaincre~:
\begin{verbatimtab}
>: var "x" diff var "x";
>:  nombre ( 1) : expr

>: exemple 1 diff var "x";
>: ((( nombre ( 0) mult  var ("x")) plus ( nombre ( 1) mult  var ("a"))) 
plus  nombre ( 0)) : expr
\end{verbatimtab}
C'est-à-dire : $0.x + 1.a + 0$ 
\begin{verbatimtab}
>: exemple 2 diff var "x";
>: ((( nombre ( 0) mult  var ("x")) moins ( nombre ( 1) mult  nombre ( 1))) 
divis (var ("x") mult  var("x"))) : expr
\end{verbatimtab}
Autrement dit ~: $\frac{0.x - 1.1}{x.x}$
\begin{verbatimtab}
>: exemple 3 diff var "x" ;
>: ((( nombre ( 0) mult (( var ("a") mult  var ("x")) plus  var ("b"))) 
moins ((((nombre ( 0) mult  var("x")) plus ( nombre ( 1) mult  var ("a"))) 
plus  nombre ( 0)) mult nombre ( 1))) divis ((( var ("a") mult  var ("x")) 
plus  var ("b")) mult (( var("a") mult var ("x")) plus  var ("b")))) : expr
\end{verbatimtab}
En clair ?


Bien que surprenants, ces résultats sont tout-à-fait corrects~: il
suffit de faire quelques simplifications pour retrouver les
expressions attendues. Mais pourquoi la fonction n'a-t-elle pas fait
ces simplifications ? Tout simplement parce que nous ne l'avons pas
demandé !

La manière la plus simple de procéder est de faire les simplifications
au moment o\`u l'on construit les expressions. Par exemple nous
remplacerons l'équation ``dérivée d'une somme d'expressions'' par~:
\begin{verbatimtab}
--- f plus g	diff  v	<= let (ff,gg) == (f diff v, g diff v)
				in ff Plus gg ;
\end{verbatimtab}
o\`u \verb+Plus+ est une nouvelle fonction déclarée (préalablement) ainsi~:
\begin{verbatimtab}
infix Plus : 5 ;
dec Plus : expr X expr -> expr ;
\end{verbatimtab}
La fonction \verb+Plus+ construit une somme d'expressions, comme plus,
mais elle ``sait'' effectuer un certain nombre de simplifications~:
\begin{verbatimtab}
--- nombre n	Plus 	nombre p 	<= nombre (n+p);
--- nombre 0	Plus 	g      		<= g;
--- f 		Plus 	nombre 0 	<= f;
--- f          	Plus 	g      		<= f plus g;

\end{verbatimtab}

\begin{exercice}

\begin{itemize}
\item Modifier \texttt{diff} en introduisant de nouvelles fonctions
\texttt{Moins}, \texttt{Mult} et \texttt{Divis}. (Attention pour la
division, Hope ne connait que les nombres entiers.)
\item Testez sur machine~:
\begin{verbatimtab}
exemple 1 diff var "x" = 

exemple 2 diff var "x" = 

exemple 3 diff var "x" = 
\end{verbatimtab}
\end{itemize}
\end{exercice}


\chapter{Supplément~: Notions de Sémantique}


%	6.1 Quelques généralités
%	6.2 Notion d'Environnement
%	6.3 Sémantique de l'Affectation
%	6.4 Sémantique de la Composition Séquentielle
%	6.5 Sémantique de la Répétition
%	6.6 Conclusion



\section{Quelques généralités}


La \emph{sémantique} est 
un terme de linguistique qui désigne 
\begin{citation}``l'étude du 
sens (ou contenu) des mots et des énoncés, par opposition à l'étude
des formes (morphologie) et à celle des rapports entre les termes dans
la phrase'' (dictionnaire Lexis).
\end{citation}

A la différence des langues naturelles, les langages de programmation
sont des objets linguistiques relativement simples~:

\begin{itemize}
\item Ayant une syntaxe assez rigide (il faut qu'un programme (le
compilateur) puisse analyser une "phrase" sans trop de peine) ;
\item Possédant, par rapport aux langues naturelles, une sémantique
volontairement très pauvre (une instruction d'un programme doit avoir
un sens clairement défini).
\end{itemize}

\subsection{Aspects Syntaxiques d'un langage de programmation}

Voici par exemple la description de la grammaire d'un petit langage de
programmation~:

\begin{verbatimtab}
<programme>    ::= <entête> <bloc>
<bloc>         ::= début <liste d'instructions> fin
<liste d'instructions> 
               ::=   rien
                 |   <instruction> ; <liste d'instructions>
<instruction>  ::= <affectation>
                 |   <bloc>
                 |   <boucle tantque>
                 |   <alternative>
                      ....
<affectation>  ::= <variable> := <expression>
<expression>   ::= <constante>
                 |   <variable>
                 |   <expression> + <expression>
                 |   <expression> * <expression>
                      ....
<tantque>      ::= tantque <condition> faire <instruction>
<alternative>  ::= si <condition> alors <instruction> sinon <instruction>
<condition>    ::= <expression>  <  <expression>
                 |  ....
                 |  <condition> et <condition>
                 | ...
\end{verbatimtab}
En Hope, nous pouvons définir des objets assez ressemblants~:

\begin{verbatimtab}
type id	 	== list(char);

data expr 	== nombre(num)
		++ variable(id)
		++ expr plus expr
		++ expr moins expr
		++ .... ;

data condition	== expr egal expr
		++ expr infegal expr
		++ ...
		++ condition et condition
		++ non(condition)
		++ ...
			
data inst	== id~:= expr
		++ bloc(list(inst))
		++ sialorssinon( condition X inst X inst )
		++ tantque( condition X inst )
		++ ... 

\end{verbatimtab}

\subsection{Sémantique d'un programme}

En informatique, la sémantique est l'étude de la ``signification'' des
programmes. Comment définir formellement la signification d'un
programme ? Prenons un exemple~:

\begin{verbatimtab}
programme m
donnée en entrée : x,y (entiers positifs)
résultat en sortie : r (entier)
début
     r := 0;
     tantque x>0 faire 
          début 
               x:=x-1; 
               r:=r+y; 
          fin
fin
\end{verbatimtab}

La signification de ce programme est  une certaine
fonction $S(m)$ (S pour Sémantique) qui, à deux entiers $x$ et $y$, fait
correspondre un résultat $r$ en fin de traitement~:

$$\begin{array}{l}
               S(m): N \times N \rightarrow N \\
               S(m)(n,p) = n \times p
\end{array}$$

Plus généralement, tout programme $p$ a une signification qui est
représentée par une certaine fonction $S(p) : D \rightarrow R$. Cette fonction
est souvent une fonction partielle, dans le sens o\`u le calcul de $p$
sur certaines données peut ne pas conduire à un résultat (le programme
boucle, il y a eu une division par zéro, ...) Le domaine de
définition de $S(p)$ est donc naturellement l'ensemble des données pour
lesquelles le calcul se termine normalement.


\paragraph{Remarques~:}

\begin{itemize}
\item On note fréquemment $(E \rightarrow F)$ l'ensemble des fonctions
partielles qui vont d'un ensemble E dans un ensemble F. On pourra donc
écrire indifféremment $$f : E \rightarrow F$$  ou $$f \in  (E
\rightarrow F)$$

\item Réfléchissons un peu sur le statut de $S$ dans la notation 
$$S(p) : D \rightarrow R$$
$ S$ est une fonction qui associe à un programme $p$ sa
signification, qui est elle-m\^eme une fonction qui fabrique un
résultat à partir d'une donnée.

Notons $P$ l'ensemble des programmes, on a alors~:
          $$    S : P \rightarrow (D \rightarrow R)    $$

Le problème qui se pose est de
savoir comment construire cette fonction $S$, c'est-à-dire calculer la
\emph{signification d'un programme} sans avoir besoin de le faire tourner sur
toutes les données possibles. 
\end{itemize}

\paragraph{Nota:} Nous limitons ici à une classe très simple de programmes
impératifs: programmes structurés, pas d'entrées-sorties, pas de
procédures.


\section{Notion d'Environnement}

Le petit programme donné en exemple va nous permettre de préciser
quelques notions importantes. Tout d'abord la notion d'\emph{environnement}~:
un environnement c'est l'association, à un moment donné de l'exécution
de ce programme, de certaines valeurs aux variables du programme.

Par exemple si on lance le programme pour $x=2$ et $y=3$, on a au départ 
l'environnement initial      	$$e_1 = \{x \equiv 2 , y \equiv 3\}$$
puis après l'initialisation  $$e_2 = \{ x \equiv 2, y \equiv 3 , r \equiv 0\}$$
et ensuite (dans la boucle)  	$$e_3 = \{ x \equiv 1 , y \equiv 3 , r \equiv 0\}$$
puis $$e_4 = \{x \equiv 1 , y \equiv , r \equiv 3 \}$$
etc.
à la fin on aura   $$	e_{48} : \{ x \rightarrow 0 , y \rightarrow 3 , r \rightarrow 6 \}$$

Deux choses à remarquer~:

\begin{itemize}
\item Un environnement peut \^etre vu comme une fonction qui part de
l'ensemble $Id$ des identificateurs du programme (ici $Id = {x,y,r}$) et
qui va dans l'ensemble des $V$ des valeurs (ici des entiers).  

\item
Lorsqu'une instruction, (ou une séquence d'instructions) s'exécute,
elle modifie l'environnement. On peut donc décrire la signification
d'une instruction $i$ (ou d'une séquence par une fonction $S(i)$ qui
associe à un environnement (celui o\`u on est juste avant d'exécuter
$i$) un autre environnement (celui d'après l'exécution de $i$).  
$$S(i) : E
\rightarrow E $$
Par exemple $S( x:=x-1 ) (e_2) = e_3$
\end{itemize}

En pratique, on représente souvent un environnement par une liste de
		doublets (identificateur, valeur).  

\begin{verbatim}
type env == list (chaine X num);
\end{verbatim}

Il nous faut alors quelques fonctions auxiliaires

\begin{alltt}
dec chercher : chaine X env  -> num ;
--- chercher ( id , [] ) <= 0 ;		! normalement, cas d'erreur
--- chercher ( id,(i,v)::r ) <= if id=i 
				then v 
				else chercher(id,r);

dec associer : chaine X num X env  ->  env ;
		....

\end{alltt}

La sémantique sera représentée par une fonction~:

\begin{verbatim}
dec S : inst -> (env -> env) ;
\end{verbatim}


\section{Sémantique de l'Affectation}

Considérons l'instruction ``\verb+r:=r+y+''. Nous voulons l'exécuter
dans un certain environnement $e$. Comment fait on ? Tout simplement on
additionne les valeurs de $r$ et de $y$ (prises dans l'environnement $e$) et
on modifie la valeur de la variable $r$ dans $e$; plus exactement on crée
un environnement $e'$ semblable à $e$, sauf en ce qui concerne la valeur
de $r$.

Nous avons besoin de définir une nouvelle fonction sémantique $V$
(comme valeur) qui renvoie la valeur d'une expression (si cette valeur
existe) dans un certain environnement.

par exemple   $V(r+y)(e) = 3$

On définit donc~:
\begin{verbatim}
dec V : expr -> (env -> num) ;
\end{verbatim}

Et, par induction sur la structure des expressions~:

\begin{verbatim}
--- V(nombre(n))(e) 	<= n;
--- V(variable(id))(e) 	<= chercher(id,e);
--- V( a plus b ) (e)	<= V(a)(e) + V(b)(e)
\end{verbatim}
etc ...


En général, si l'on a une affectation \verb/var := expr/, la fonction
sémantique $S(\mbox{\texttt{var := expr}})$ qui lui est associée est
définie de la manière suivante~:

\begin{verbatim}
--- S(id := expression)(e)  <= let v == V(expression)(e)
				in modifier(id,v,e);
\end{verbatim}


\section{Sémantique de la Composition Séquentielle}

Lorsqu'on sait faire une instruction, on peut raisonnablement essayer d'en faire deux à la suite l'une de l'autre ! Interrogeons-nous sur la séquence d'instructions ``\verb/x:=x-1; r:=r+y/''. 

Supposons que nous l'exécutions en partant de l'environnement
               $e$. Aprés la première instruction on se retrouve dans
               l'environnement $e' = S(\mbox{\texttt{x:=x-1}})(e)$; puis la seconde nous fait
               passer dans $e" = S(\mbox{\texttt{r:=r+y}})(e')$. Donc

$$\begin{array}{rl}
S(\mbox{\tt x:=x-1;r:=r+y})(e) &= S(\mbox{\tt r:=r+y})(S(\mbox{\tt x:=x-1})(e)) \\
&=               (S(\mbox{\tt r:=r+y}) \circ S(\mbox{\tt x:=x-1}))(e)
\end{array}$$

Pour résumer, si on a deux instructions $i$  et $i'$ ~:
$$                         S(i ; i' ) =  S(i' ) \circ S(i)S                         $$


Ici la composition séquentielle des instructions n'appara\^{\i}t que
dans la structure de bloc, et donc pour des listes
d'instructions. Définissons une fonction auxiliaire $Slist$ (sémantique
d'une liste d'instructions)~:

\begin{verbatim}
dec Slist : list(inst) -> (env -> env);
--- Slist ( [] )(e) 		<= e;
--- Slist ( p::r ) ( e )	<= Slist(r) (S (p)(e));
\end{verbatim}

On pourra alors poser~:

\begin{verbatim}
--- S ( bloc (l) ) (e) 	<= Slist(l);
\end{verbatim}


\section{Sémantique de la Répétition}

Examinons maintenant la boucle tant-que. Une boucle tant-que comporte
deux éléments~: une condition et un corps. Pour évaluer une condition,
il nous faut une fonction $B$ (booléenne):

\begin{verbatim}
dec B : condition -> (env -> truval);
--- B( a egal b)(e) <= V(a)(e) = V(b)(e);
 ...
--- B ( c1 et c2 )(e) <= B(c1)(e) and B(c2)(e);
....
\end{verbatim}

Nous savons également définir la signification du corps de la boucle. Comment recoller les morceaux ?

Soit $boucle$ l'instruction tantque <cond> faire <corps>/

pour exécuter $boucle$ dans l'environnement $e$, que fait on ?
\begin{itemize}
\item  on évalue la condition (dans $e$)
\item si la condition est fausse, on s'arr\^ete (l'environnement n'a pas changé)
\item- si elle est vraie, on exécute le corps (l'environnement est
modifié) et on recommence au début (avec ce nouvel environnement
$e'$).
\end{itemize}

Ce qui s'écrit~: 

\begin{verbatim}
--- S(tantque(c,i))(e) <= if B(c)(e)
			then   S(tantque(c,i))( S(i)(e) )
			else   e;
\end{verbatim}


\begin{exercice}
\begin{enumerate}
\item Définir la sémantique de l'alternative~:
``\verb/si <condition> alors <i1> sinon <i2>/''
\item Définir la sémantique de la répétition~:
     ``\verb/répéter <i> jusqu'à <condition>/''
\item Utiliser cet arsenal mathématique pour démontrer que 

    pour tout $e \in E$  avec $e(x)=a$ et $e(y)=b$  ($a,b \in N$)), on a~: 
$$ (S(m)(e))(r) = a \times b$$    
\end{enumerate}
\end{exercice}





\chapter*{Conclusion}

La programmation s'apprend en général par t\^atonnements, ce qui fait
parfois conclure un peu h\^ativement qu'il s'agit d'un art (ou d'un
artisanat, voire un bricolage) plut\^ot que d'une technique. Il
convenait donc de montrer que l'édifice repose sur de robustes
fondations mathématiques~: composition de fonctions et calcul par
récurrence. Qu'en conclure ?

\begin{itemize}
\item La notion d'environnement est fondamentale pour la compréhension
  de la programmation impérative. Cette notion ne fait pas partie du
  bagage mathématique usuel. D'o\`u les réticences initiales à
  l'acceptation d'instructions du type~:\verb/i := i + 1/, qui contredisent
  l'aspect déclaratid.
 
\item pour comprendre  -et expliquer- une procédure
non triviale (comportant par exemple une boucle), on  utilise
nécessairement le raisonnement par récurrence, sous une forme plus ou
moins consciente. C'est donc une technique qu'il convient de
ma\^{\i}triser.

\item Ceci justifie l'apprentissage de la programmation fonctionnelle,
comme approche qui familiarise le programmeur avec
le raisonnement explicite par récurrence.  

\end{itemize}


% \clearpage

% \bibliography{hope}
\clearpage

\include{index}

\end{document}
