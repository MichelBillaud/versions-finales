\chapter{Induction et récursion}


%	3.1 Un peu d'histoire 
%	3.2 Axiomatique de PEANO
%	3.3 Définitions récursives
%	3.4 Raisonnement par récurrence
%	3.5 Fonctions d'ordre supérieur
%	3.6 À propos des fonctions auxiliaires
%	3.7 Exercices et problème



\section{Un peu d'histoire}

Le \siecle{XIX} est le siècle de la Révolution Industrielle et de la
Science Triomphante. C'est l'époque de la foi en un progrès
scientifique inéluctable et bénéfique, fondé sur l'étude rigoureuse
des faits (positivisme d'Auguste COMTE), et sur l'existence d'un
déterminisme qui régirait aussi bien les phénomènes physiques
(MAXWELL, BERTHELOT) biologiques (BERNARD, DARWIN), que l'organisation
sociale (TAINE, naturalisme de ZOLA) et politique (MARX).

À l'extr\^eme, c'est le scientisme~:
\begin{citation}
Une chose évidente d'abord, c'est que chaque découverte pratique de
l'esprit humain correspond à un progrès moral, à un progrès de dignité
pour l'universalité des hommes. [...] Je suis convaincu que les
progrès de la mécanique, de la chimie, seront la rédemption de
l'ouvrier~; que le travail matériel de l'humanité ira toujours en
diminuant et en devenant moins pénible~; que de la sorte l'humanité
deviendra plus libre de vaquer à une vie heureuse, morale,
intellectuelle. Aimez la science. Respectez-la, croyez-le, c'est la
meilleure amie du peuple, la plus s\^ure garantie de ses progrès.
{\em Ernest RENAN }
\end{citation}


\subsection*{La mécanisation du calcul}

La Révolution Industrielle na\^{\i}t de la mécanisation du travail
physique humain. Les travaux de Charles BABBAGE (1792-1871) ont montré
à ses contemporains la possibilité de mécaniser également le calcul,
c'est-à-dire de faire exécuter des calculs numériques par une machine
sous contr\^ole d'un programme (possibilité théorique surtout, car
Charles BABBAGE mourut sans voir sa ``Machine Analytique'' réalisée,
faute de moyens).

C'est le fils aîné de BABBAGE, Henry, qui fabriqua une version très
réduite de la Machine Analytique à partir de 1880. Le 21 janvier 1888,
la machine imprima une table des 44 premiers multiples de $\pi$, avec 29
décimales. Mais un incident technique d\^u à la technologie employée
(roues dentées, cylindres à picots) provoqua une erreur au 32$^e$
multiple. Découragé, Henry BABBAGE ne reprit les essais qu'en 1906, il
trouva la cause de l'erreur et y remédia. Il put alors faire une
démonstration réussie devant l'Académie d'Astronomie. La machine ayant
prouvé la justesse des idées de son inventeur, elle fut remise à un
musée en 1910.

Lady Ada LOVELACE (1815-1852), l'assistante de Charles BABBAGE,
écrivit un jour que la machine ``pourrait peut-\^etre ne pas traiter
que des nombres''.


\subsection*{La  formalisation du raisonnement}

En 1854, George BOOLE (1815-64) publia son ouvrage {\em Les Lois de la
 Pensée} dans lequel il expliquait que le raisonnement logique pouvait
 \^etre assimilé à du calcul algébrique, relan\c{c}ant alors la
 ``logique formelle'' - étude du raisonnement déductif, des
 syllogismes, etc. - qui sommeillait depuis quelques siècles. BOOLE
 rencontra BABBAGE en 1862, mais il mourut trop t\^ot
\footnote{Il contracta une congestion pulmonaire en allant à pied, sous la pluie, donner une conférence au Queen's College de Cork.}
pour qu'une collaboration ait pu na\^{\i}tre.

William Stanley JEVONS (1835-82) fut le seul mathématicien à
comprendre immédiatement la portée de l'oeuvre de BOOLE. Il réalisa un
``piano logique'' qui pouvait résoudre des équations logiques. Très
enthousiaste, il aurait sans doute exploré la mécanisation du
raisonnement beaucoup plus loin s'il ne s'était noyé accidentellement.


\subsection*{L'axiomatisation des mathématiques}

\subsubsection*{L'état de l'art au \siecle{XIX}  siècle}

Les connaissances mathématiques sont bien avancées au \siecle{XIX} siècle~: la
plupart des branches ont été créées depuis déjà longtemps~:

\begin{itemize}
\item Éléments d'EUCLIDE (\siecle{III} avant J.C)~: Algèbre, Géométrie,
  Arithmétique ...
\item Traité d'algèbre d'AL-KHWARIZMI (\siecle{IX} siècle)
\item Résolution de l'équation du 3$^e$ degré par Hieronimo CARDAN
(1545), 4$^e$ degré par son élève Ludovico FERRARI
\item Nombres Négatifs et Complexes~: Rapha\"{e}l BOMBELLI (1526-72)
\item Symbolisme Algébrique~:  Fran\c{c}ois VIETE (1540-1603)
\item Théorie des Nombres~: Pierre de FERMAT (1601-65)
\item Calcul Infinitésimal~: Isaac NEWTON (1642-1727) 
\item Calcul Différentiel~: Gottfried Wilheim LEIBNITZ (1646-1716)
\end{itemize}

Mais les mathématiciens se heurtent à des paradoxes inextricables,
notamment sur l'infiniment petit et l'infiniment grand. Appara\^{\i}t
alors un courant de pensée qui vise, dans le m\^eme esprit qu'EUCLIDE,
à donner des définitions précises des objets mathématiques que l'on
utilisait jusque-là de manière assez intuitive, dans l'intention
d'éliminer ces paradoxes.

\subsubsection*{La démarche d'Euclide}

L'oeuvre du grec EUCLIDE (\siecle{III} av. JC) est un modèle de
rigueur~: ses Eléments récapitulent en une quinzaine de volumes les
connaissances mathématiques de l'époque. PROCLUS (\siecle{V} siècle après J.C)
affirme qu'Euclide, en rassemblant ses Eléments, 
\begin{citation}
en a coordonné
beaucoup d'Eudoxe, perfectionné beaucoup de Théétète et qu'il a évoqué
dans d'irréfutables démonstrations ceux que ses prédécesseurs avaient
démontré d'une manière rel\^achée.
\end{citation}

Le Livre I des Éléments est précédé par une présentation assez
intuitive des concepts utilisés dans la suite~:
\begin{citation}
Un point est ce qui n'a aucune partie. Une ligne est une longueur sans largeur.
\end{citation}

et par une série de postulats (affirmations admises sans qu'il y ait
lieu de les démontrer) qui indiquent les relations entre ces concepts~:
\begin{citation}
On demande~:
\begin{enumerate}
\item qu'on puisse conduire une droite d'un point quelconque à un point quelconque,
	\item qu'on puisse prolonger continuellement, selon sa
	direction, une droite finie en une droite,
\item  que d'un point quelconque, et avec un intervalle quelconque, on puisse décrire une circonférence quelconque,
\item et que tous les angles droits soient égaux entre eux,
\item et que si une droite tombant sur deux droites fait les angles intérieurs du m\^eme c\^oté plus petits que deux droits, ces droites, prolongées à l'infini, se rencontreront du c\^oté o\`u les angles sont plus petits que deux droits.
\end{enumerate}
\end{citation}

Après avoir exposé systématiquement les définitions, postulats et
axiomes, EUCLIDE en déduit les propriétés élémentaires des triangles,
et de leurs bissectrices, milieux des c\^otés, etc.

\subsubsection*{Le courant formaliste}

Au \siecle{XIX} les mathématiciens-logiciens du courant ``formaliste'' (et
 plus tard Russell, Hilbert, etc.) essaient de formuler les quelques
 hypothèses (axiomes, postulats) qui sont vraiment indispensables pour
 définir les objets mathématiques, et dont on pourra déduire toutes
 les propriétés qui sont ``intuitivement vraies''.

En 1876 c'est un Allemand, Julius Wilheim Richard DEDEKIND
(1831-1916), qui donne une définition formelle de l'ensemble $\mathbb{R}$ des
nombres réels à partir des coupures de $\mathbb{Q}$ (l'ensemble des
rationnels).
\begin{citation}
Une coupure est un partage de $\mathbb{Q}$ en deux partitions non vides $A$ et $B$
telles que tout élément de $A$ est plus petit que tout élément de $B$. À
chaque coupure correspond alors un nombre réel unique~: par exemple,
le nombre irrationnel $\sqrt{2}$ est défini par la coupure 
$$\begin{array}{rl}
A &= \{ q | q>0 \mbox{\ et\ } q^2<2 \} \\
 B &= \{ q | q<0 \mbox{\ ou\ } q^2 \geq 2 \} 
\end{array}$$

\end{citation}

Vers 1880, Georg Ferdinand Ludwig Philipp CANTOR (1845-1918) met au
point la théorie des ensembles, qui permet de résoudre de nombreux
paradoxes de la théorie des limites liés, en fait, à l'existence de
plusieurs types d'infinis.

Quelques problèmes typiques, que vous pouvez essayer de résoudre~:

\begin{exercice}
  Montrez qu'il y a
  \begin{itemize}
\item - autant de couples d'entiers que d'entiers (on peut
  construire une bijection entre $\mathbb{N}$ et $\mathbb{N}^2$) ;
  \item plus de nombres réels dans l'intervalle $[0,1[$ que de nombres
    entiers~ : il n'existe pas d'injection de $[0,1[$ dans N).
  \end{itemize}
\end{exercice}

\begin{exercice}
 L'ensemble des parties d'un ensemble $E$ est ``plus gros'' que cet
 ensemble~: il n'existe pas d'injection de ${\cal P}(E)$ dans $ E$ (résultat
 d\^u à CANTOR).
\end{exercice}



Malheureusement pour CANTOR, sa théorie fait appara\^{\i}tre à nouveau
des paradoxes (comme le paradoxe de RUSSELL~: l'ensemble des ensembles
qui ne se contiennent pas eux-m\^emes se contient-il lui-m\^eme ? ).
La théorie des ensembles sera axiomatisée plus tard par Ernst ZERMELO
(1871-1953) en 1908. Génial mais incompris à son époque, CANTOR est
mort dans un asile psychiatrique.

En 1889, le mathématicien et logicien Giuseppe PEANO propose (enfin !) 
une définition formelle de l'ensemble des entiers naturels (positifs).  


\subsection*{Mathématiques et Réalité}

L'approche ``hypothético-déductive'' pose le problème du choix du système d'axiomes de base~: 
\begin{itemize}
\item pour LEIBNIZ (et d'autres), ``tout ce qui est vrai est démontrable''~: pour arriver à tout démontrer toute propriété vraie dans le cadre d'une théorie (c'est la complétude de la théorie),  il faut prendre suffisamment d'axiomes~;
\item il ne faut pas que les axiomes choisis conduisent à des contradictions (consistance de la théorie)~;
\item il faut éviter aussi de choisir des axiomes qui seraient des conséquences des autres axiomes (ils seraient inutiles).
\end{itemize}

Par exemple, la question s'est longtemps posée de savoir si le fameux cinquième postulat d'EUCLIDE
\begin{citation} par un point extérieur à une droite on peut mener une seule et unique droite parallèle à celle-ci
\end{citation}
 était ou non un axiome.
 De nombreux mathématiciens essayèrent en vain de montrer que ce postulat était une conséquence des autres axiomes de la géométrie d'Euclide, jusqu'en 1826, lorsque LOBATCHEVKI (1793-1856) présente une géométrie non-euclidienne dans laquelle 
\begin{citation} par un point extérieur à une droite passent une infinité de parallèles à cette droite.
\end{citation}
 Un peu plus t\^ot, vers 1813, GAUSS (1777-1844) avait inventé la
 géométrie hyperbolique, 
\begin{citation}
``une étrange géométrie,
 tout à fait différente de la n\^otre''
\end{citation} mais il n'avait pas
 osé publier ses travaux~: 
\begin{citation}
``J'appréhende les clameurs des
 Béotiens si je voulais exprimer complètement mes vues''
\end{citation}

 En
 effet, à l'époque, la géométrie euclidienne était censée {\em rendre
 compte de la réalité physique du monde} (KANT); par conséquent ses
 axiomes étaient donc considérés comme {\em vrais dans l'absolu}~: il
 était inimaginable de fonder une théorie sur leur négation.



De la m\^eme fa\c{c}on, l'hypothèse du continu 
\begin{citation} le cardinal de
l'ensemble des nombres réels est le premier qui soit supérieur à celui
de l'ensemble des nombres entiers
\end{citation}
 formulée par CANTOR qui essaya de
la démontrer jusqu'à sa mort, fut montrée
\emph{indécidable}\footnote{on ne peut pas la démontrer, ni la réfuter
: on peut l'ajouter comme nouvel axiome, aussi bien que la proposition
contraire} par P.J. COHEN en 1963.


L'espoir de trouver un jour quand m\^eme un ``bon'' système d'axiomes
qui suffirait à tout démontrer s'effondre en 1931 après la publication
du Théorème d'Incomplétude de Kurt G\"{O}DEL (né en 1906)~:
\begin{citation}
Toute formulation axiomatique de la théorie des nombres est soit incomplète, soit contradictoire.
\end{citation}

Autrement dit, dans toute théorie T (contenant la théorie des
nombres), il existe des propositions indécidables, dont on ne peut
démontrer ni la vérité ni la fausseté. Par exemple, la proposition
``la théorie T est non-contradictoire'' est indécidable dans la
théorie T elle-m\^eme.


\subsection*{Bonnes lectures~:}

\begin{itemize}
\item Douglas HOFSTADTER (1985), G\"{o}del, Escher et Bach, les Brins d'une Guirlande Éternelle,  InterEditions.
\item R. APERY, M.CAVEING, et al. (1982) Penser les mathématiques, Points Inédits S29, ed. du Seuil.
\item A. DAHAN-DALMEDICO, J. PFEIFFER, (1986). Une histoire des mathématiques, Points Sciences S49, ed. du Seuil.
\item Gustave FLAUBERT, Bouvard et Pécuchet, Livre de poche 440-441.
\end{itemize}



\section{Axiomatique de PEANO}

Cette définition des entiers naturels tient en 5 axiomes, elle repose
sur un objet de base (zéro) et un ``constructeur''~: la fonction qui a
tout entier $n$ associe son successeur $n+1$
\begin{enumerate}
\item zéro est un entier
\item tout entier a un successeur, qui est également un entier
\item zéro n'est le successeur d'aucun entier
\item	 deux entiers différents ont des successeurs différents
\item Principe d'induction~: si une propriété P est vraie pour
 zéro (cas de base)
		 et que P(n) entra\^{\i}ne
 P(successeur(n)) pour tout entier n,  (étape d'induction)
		alors P est vraie pour tout entier .
\end{enumerate}

C'est le premier exemple d'ensemble défini inductivement~: on part
d'un objet de base zero (0) auquel on adjoint son successeur (1) puis
le successeur du successeur (2), etc. Nous verrons plus loin quantité
d'autres ensembles (listes, arbres) définis de la m\^eme fa\c{c}on, à
partir d'objets de base et de constructeurs.

Le principe d'induction est à la base de la technique de preuve par
récurrence~: pour prouver qu'une propriété est vraie pour tous les
entiers on montre~:
\begin{itemize}
	\item quelle est vraie pour $0$ (en général c'est plut\^ot
	facile), \item que si elle est vraie pour un entier $n$
	(hypothèse de récurrence), alors elle est vraie pour son
	successeur $n+1$.
\end{itemize}
Nous avons donc le droit de définir des fonctions par induction naturelle~: une fonction $f$ est définie pour tout entier positif à partir du moment o\`u~:
\begin{itemize}
\item on connaît $f(0)$,
\item on peut exprimer $f(n+1)$ à partir de $f(n)$, pour tout $n$.
\end{itemize}
\begin{exercice} Construire un ensemble qui satisfasse tous les axiomes de PEANO sauf le troisième.
\end{exercice}
\begin{exercice} Idem, mais en excluant cette fois-ci le quatrième.
\end{exercice}
\begin{exercice} Idem, sans le cinquième.
\end{exercice}
\begin{exercice} Montrez que l'on peut remplacer le principe d'induction par la variante~:
	« si une propriété $Q$ est vraie pour $zero$ (cas de base) et que
	 $Q(zero)$ et $Q(successeur(zero)$ et .... $Q(n)$ implique
	 $Q(successeur(n))$ pour tout entier $n$, alors $Q$ est vraie pour
	 tout entier. » et comparez les mérites respectifs des deux
	 formulations.
\end{exercice}
 

Le lecteur attentif ne manquera pas de nous soup\c{c}onner~: peut-on
honn\^etement prétendre \emph{définir les nombres entiers} sans parler des
opérations arithmétiques comme l'addition, la comparaison, etc.~?  Et
bien oui~! L'addition, par exemple, n'est pas une notion primitive de
la théorie des nombres, mais une fonction que l'on définit par
récurrence~:

$$\begin{array}{rcrcl}
	plus ( &0& , 	&p ) =& p \\				
	plus ( &succ(n)& , &	p ) =& succ(  plus (n ,p ) )
\end{array}$$

		pour tous entiers $n$ et $p$.

Rassurez-vous, deux plus deux font toujours quatre, en effet 
$$deux = succ(succ(zero))$$
et
$$\begin{array}{rcl}
	plus(deux, deux) 	&=& plus ( succ(succ(zero)),  succ(succ(zero)) ) \\
				&=& succ ( plus ( succ(zero)),  succ(succ(zero)) )) \\
				&=& succ (succ ( plus (zero),  succ(succ(zero)) )) \\
				&=& succ ( succ (succ ( succ ( zero ) ) ) )
\end{array}$$



\begin{exercice} Définir la multiplication.
\end{exercice}

\begin{exercice} Définir la relation ``inférieur ou égal''.
\end{exercice}

L'axiomatique de PEANO nous suffit donc pour reconstruire les notions
connues de l'arithmétique.


\section{Définitions récursives}

\begin{quotation}
Il fut saisi par la frénésie des factorielles~: $1!~= 1~$; $2!~= 2~$; 
$3!~=~6~$; $4!~= 24$~; $5!~= 120~$; $6!~= 720~$; $7!~=~5~040~$; $8!~=~40~320~$; 
$9!~=362~880~$; $10!~=~3~628~800~$; $11!~=~39~916~800~$; $12!~=~479~001~600~$; [...]
$22!~=~1~124~000~727~777~607~680~000$, soit plus d'un milliard de fois
soixante-dix-sept milliards~!  Smautf en est aujourd'hui à $76!$ mais il
ne trouve plus de papier au format suffisant et en trouverait-il, il
n'y aurait pas de table assez grande pour l'étaler.

La vie mode d'emploi {\em Georges PEREC}, Livre de Poche 5341 (1978)
\end{quotation}

Nous allons essayer de spécifier formellement la fonction factorielle,
c'est-à-dire de préciser ce qu'elle fait, de manière aussi claire que
possible.


\subsection*{Comment spécifier la fonction factorielle ?}

Premier essai, par une phrase~:
\begin{citation}
	La fonction factorielle associe, à tout entier positif ou nul n, un autre entier qui est le produit des entiers de 1 à n.
\end{citation}

Cette spécification est correcte~: elle identifie clairement et sans
ambig\"{u}ité la fonction dont nous parlons. Mais sa formulation
littéraire la rend difficilement exploitable ensuite par le calcul
algébrique qui est l'outil de base du raisonnement mathématique~: nous
préférerions une bonne formule.

Seconde tentative, par une expression mathématique~:

\[factorielle (n) = 1 \times 2 \times 3 \ldots \times n\]

C'est une définition \emph{elliptique}~: il faut un certain effort de
la part du lecteur pour comprendre ce qu'il convient de mettre en lieu
et place des points de suspension. Elle n'est donc pas idéale.

Et m\^eme avant ces points, car avec un peu de mauvaise foi, on
pourrait conclure que 
$$factorielle (2) = 1\times 2 \times 3 \times 2 =
12$$
Bref, le ``$1 \times 2 \times 3 \ldots$'' est à prendre avec des
pincettes~: lorsqu'il y a des points de suspension ensuite, $1 \times
2 \times 3$ n'est plus égal à $6$, mais $1$, $2$ ou $6$ selon
les circonstances !

La troisième tentative, et la bonne, sera de définir la fonction
factorielle par une spécification récursive .


\subsection*{Construire une spécification récursive}

Une spécification est dite \emph{récursive} lorsqu'elle définit un
objet mathématique (un ensemble, une relation, une fonction) à l'aide
de lui-m\^eme. 

Ceci doit vous sembler pour le moins obscur, aussi
revenons à notre exemple et regardons quelques valeurs de la fonction
$factorielle$~:

$$\begin{array}{rcl}
	factorielle (0) 	&=& 1 \\
	factorielle (1) 	&=& 1 \\
	factorielle (2) 	&=& 1 \times 2 \\
	factorielle (3) 	&=& 1 \times 2 \times 3 \\
	factorielle (4) 	&=& 1 \times 2 \times 3 \times 4 \\
	... && \\
	factorielle (n - 2) &=& 1 \times 2 \times 3 \times ... \times (n-2)  \\
	factorielle (n - 1) &=& 1 \times 2 \times 3 \times ... \times (n-2) \times (n-1) \\
	factorielle (n ) 	&=& 1 \times 2 \times 3 \times ... \times (n-2) \times (n-1) \times n \\
\end{array}$$

Nous remarquons que chaque ligne ne diffère de la précédente que par
un seul terme. Par exemple, $factorielle(4)$ est égal à$
factorielle(3) \times 4$. Nous sommes donc tentés de définir
$factorielle$ par l'équation~
	$$factorielle (n) = factorielle (n-1) \times n $$ valable pour
tout entier positif $n$, dans laquelle la fonction $factorielle$ est
exprimée à partir d'elle-m\^eme. Mais il y a un hic~: en appliquant
cette équation au cas $n=0$, nous obtenons $ factorielle(0) =
factorielle(-1)\times0$ et donc $1=0$, ce qui n'est pas raisonnable
(sans parler du fait que $-1$ n'est pas dans le domaine de définition
de la fonction).

Il faut donc limiter l'usage de l'équation $factorielle (n) =
factorielle(n-1) \times n$ aux cas o\`u n est strictement positif, et
préciser ce qui se passe lorsque $n=0$ (le cas de base). Voici donc
une bonne définition de la factorielle~:
$$\begin{array}{rll}
factorielle (n) &= factorielle (n-1) \times n & 
\mbox{\ pour tout entier positif\ } n \\
factorielle (0) &= 	1
\end{array}$$


\subsection*{Traduction en HOPE}

Dans le langage HOPE, nous déclarerons cette fonction sous la forme~:
\begin{verbatim}
dec factorielle : num -> num ;
--- factorielle (0) <= 1 ;
--- factorielle (n) <=  factorielle (n-1) * n ;
\end{verbatim}

Attention, l'ordre des équations est important en Hope, car
l'interprèteur va essayer de les utiliser dans l'ordre o\`u elles ont
été déclarées. Le calcul de $factorielle(2)$ se déroule comme suit~:

\begin{itemize}
\item rejet de la première équation~: $2$ est différent de la constante $ 0 $, donc l'équation ne convient pas.
\item essai de la seconde équation~: on peut poser $n=2$. Donc on remplace
$ factorielle(2)$ par $factorielle(2-1)\times 2$ :
	$factorielle(2) = factorielle(2-1)\times 2 = factorielle(1)\times 2$
\end{itemize}

reste à résoudre $factorielle(1)$~:

\begin{itemize}
\item rejet de la première équation car $1$ est différent de la constante $0$.
\item essai de la seconde équation~: on peut poser $n=1$. Donc on remplace
$ factorielle(1)$ par $ factorielle(1-1)\times 2 $:

$$\begin{array}{rl}
factorielle(2) &= factorielle(1)\times 2 \\
		&= factorielle(1-1)\times 1\times 2 \\\
	& = factorielle(0)\times 1\times 2
\end{array}$$
\end{itemize}
reste à résoudre factorielle(0)~:
\begin{itemize}
\item la première équation ``colle''~: on remplace$ factorielle(0)$ par$ 1$, et on obtient~:

$$\begin{array}{rl}
factorielle(2)  &= factorielle(0)\times 1\times 2 \\
&= 1\times 1\times 2 \\
&= 2
\end{array}$$
\end{itemize}

Il est clair que si nous avions posé les équations dans l'ordre
inverse, nous n'aurions jamais pu détecter le cas de base, et le
calcul ne se serait jamais terminé.  En règle générale donc, il faut
écrire les équations de base en t\^ete, et ensuite les équations de
récurrence.


\paragraph{Remarque~: }on pouvait aussi écrire~:

\begin{verbatim}
dec factorielle : num -> num ;
--- factorielle (n) <= if n=0 
                       then 1 
                       else factorielle (n-1) * n ;
\end{verbatim}

Mais, sous cette forme, on voit moins bien les équations
sous-jacentes. Quand c'est possible, il faut éviter l'emploi de la
structure conditionnelle.


Exercices~:
\begin{exercice}
 Écrire une fonction \texttt{som} qui calcule la somme des $n$ premiers entiers strictement positifs, c'est-à-dire~:	$som(n) =   1+2+3+...+n $.

Indications~:
\begin{verbatim}
som(0) =
som(1) =
som(2) =
som(3) =
som(4) =
...
\end{verbatim}
\end{exercice}






\begin{exercice}
 Écrire une fonction \texttt{somcarre} qui calcule la somme des carrés des $n$ premiers entiers strictement positifs, c'est-à-dire~:	
$$ somcarre(n) =   1^2+2^2+3^2+\ldots.+n^2$$

\end{exercice}
\begin{exercice}
 Écrire une fonction \texttt{somcube} qui calcule la somme des cubes des n premiers entiers strictement positifs, c'est-à-dire~:	
$somcube(n) =   1^3+2^3+3^3+....+n^3$.
\end{exercice}	

\begin{exercice}	
 En utilisant le type prédéfini \texttt{truval} (\emph{truth value} =
 booléen) qui possède deux valeurs \texttt{true} et \texttt{false}, écrire une fonction
 \texttt{pair} qui détermine si un entier (positif) est pair ou non.
	
\begin{verbatim}
pair(0) =
pair(1) =
pair(2) =
pair(3) = 
...
\end{verbatim}
\end{exercice}



\begin{exercice}
Voici la suite de Fibonnacci  (Léonard de Pise 1170 ?- 1250)~:
$$ 1, 1, 2, 3, 5, 8, 13, 34, 55, 89 \ldots $$ Comme vous pouvez le remarquer,
chaque élément de cette suite est la somme des deux précédents (sauf
les deux premiers, qui sont les cas de base)~: il en sera évidemment
également ainsi pour les termes suivants.
\begin{verbatim}
dec fib : num -> num ;

--- fib (1) 	<=

--- fib (2) 	<=	

--- fib (n)	<=
\end{verbatim}
\end{exercice}

\begin{exercice}
Et vous ne pouvez ignorer le triangle de Pascal, les fameux
coefficients binomiaux~:
\begin{tabular}{lllll}
c(0,0)=1 \\
c(1,0)=1 &	c(1,1)=1\\
c(2,0)=1&	c(2,1)=2	&c(2,2)=1 \\
c(3,0)=1	&c(3,1)=3	&c(3,2)=3	&c(3,3)=1 \\
c(4,0)=1 	&c(4,1)=4 	&c(4,2)=6 	&c(4,3)=4	&c(4,4)=1\\
\end{tabular}


\end{exercice}


\section{Raisonnement par récurrence}

L'écriture d'une spécification récursive est une activité
intellectuelle très proche du raisonnement par récurrence. Soit à
démontrer par exemple la proposition~:

\paragraph*{Proposition}
	Pour tout entier $n$ positif ou nul,  on a  
$$som(n)= \frac{n\times(n+1)}{2}$$


\subsection{Preuve formelle}

Nous allons d'abord présenter une preuve formelle de cette
proposition~: la démonstration sera une suite d'étapes dont nous
montrerons systématiquement les justifications. Ainsi nous serons
certains que notre preuve est inattaquable.


\textbf{Début de la Preuve.}

Nous allons montrer que pour tout entier $n$ possède la
propriété  $P(n)$  définie par~: 
$$P(n) \equiv  \left( som(n)=\frac{n\times(n+1)}{2} \right)$$

\begin{itemize}
\item A. P(0) est vrai en effet~:

\begin{tabular}{rll}
1.&	$P(0) \equiv (som(0)=0)$	&	par définition de $P$	 \\
2.&	$som(0)=0$ &				par définition de $som(0)$ \\
3.&	$P(0)$ est vrai		&		conséquence de 1 et 2
\end{tabular}

\item B. pour tout entier $n$, $P(n) \Rightarrow  P(n+1)$, car

\begin{tabular}{rll}
4. &	$P(n)$ vrai		 & Hypothèse \\
5. &	$som(n)=\frac{n \times(n+1)}{2}$ &	conséquence de 4 et définition de P \\
6. &	$P(n+1) \equiv (som(n+1) = \frac{(n+1)\times(n+2)}{2})$ &par définition de P \\
7. &	$som(n+1) = som(n) + (n+1)$		& par définition de som \\
8. &	$som(n+1) = \frac{n\times(n+1)}{2}  + (n+1)$	&conséquence de 7 et 5 \\
9. &	$som(n+1) = \frac{n\times(n+1)}{2}  + \frac{2(n+1)}{2}$	&petit calcul \\
10.& 	$som(n+1) = \frac{(n+2)\times(n+1)}{2}$		&idem. \\
11.& 	$P(n+1)$ vrai				&conséquence de 6 et 10 \\
\hline 
12.&	$P(n) \Rightarrow P(n+1)$			&car l'hypothèse 4 entraîne 11 
\end{tabular}
\item C. D'après A et B, et en vertu du principe de récurrence, nous concluons~:
la propriété $P(n)$ est  vraie pour tout entier $n$ positif ou nul.
\end{itemize}
\textbf{Fin de la Preuve.}

On peut difficilement mettre en doute la rigueur d'un raisonnement
présenté sous cette forme !


\subsection{Preuve par transformation de programme}

Prouver qu'une fonction possède une certaine propriété revient à
démontrer qu'une autre fonction est constante. C'est évident~: sur
notre exemple, la propriété~:

\paragraph*{Propriété}	Pour tout entier positif ou nul, on a  $som(n)= \frac{n\times(n+1)}{2}$.

est vraie si et seulement si la fonction~:

\begin{verbatim}
dec P : num -> truval;
--- P(n) <= ( som(n) = n*(n+1)/2 );
\end{verbatim}
vaut \texttt{true} pour tout entier positif ou nul.

Nous allons modifier cette définition de \texttt{P}, pas à pas, jusqu'à ce que
nous puissions conclure que \texttt{P} vaut toujours \texttt{true}.

\begin{itemize}
\item Séparation en deux cas, selon que $n$ est nul ou non. On obtient une
définition équivalente~:

\begin{verbatim}
dec P : num -> truval
--- P(0) <= ( som(0) = 0*(0+1)/2 );
--- P(n) <= ( som(n) = n*(n+1)/2 );
\end{verbatim}

\item Comme $som(0)=0$ et, quand $n$ est différent de $0$, on a 
$som(n)=som(n-1)+n$, on simplifie~:
\begin{verbatim}
dec P : num -> truval;
--- P(0) <= true;
--- P(n) <= ( som(n-1)+n  = n*(n+1)/2 );
\end{verbatim}
\item Mais nous avons les équivalences~:
	
$$\begin{array}{llcl}
	& som(n-1) + n	&=&   \frac{n \times (n+1)}{2}    \\
\Leftrightarrow &   	som(n-1) 	&=&   \frac{n\times(n+1)}{2} - n    \\
\Leftrightarrow &	som(n-1) 	&=&   \frac{n\times(n -1)}{2}
\end{array}$$
\item  Mais, d'après la définition initiale de P, on a~:
$$\begin{array}{rl}
	P(n-1)  	&= (  som(n-1) = \frac{(n-1)\times(n+1-1)}{2})  \\
		&= (  som(n-1) = \frac{(n-1)\times n}{2} )
\end{array}$$

Ce qui amène à redéfinir une dernière fois \texttt{P}~:

\begin{verbatim}
dec P : num -> truval;
--- P(0) <= true ;
--- P(n) <= P(n-1) ;
\end{verbatim}
\end{itemize}

L'axiome d'induction nous permet de conclure que pour tout entier
positif ou nul n, \texttt{P(n) = true}.




\begin{exercice} Démontrez que 
$$\begin{array}{rl} 
somcarre(n) &= 1^2+2^2+3^2+\ldots+n^2) \\
	&= \frac{n(n+1)(2n+1)}{6}
\end{array}$$
\end{exercice}
\begin{exercice} $somcube(n) = (1^3+2^3+3^3+....+n^3) = som(n)^2$
  	(AL-KARAGI, fin du \siecle{X} siècle).\end{exercice}
\begin{exercice} $ 1\times factorielle(1)+2\times factorielle(2)+...+n\times factorielle(n) = factorielle(n+1)-1 $
\end{exercice}

\section{Fonctions d'ordre supérieur}

Vous avez sans doute remarqué une certaine ressemblance d'écriture
entre les fonctions \texttt{som}, \texttt{somcarre}, et
\texttt{somcube}. Si l'on vous demandait d'écrire une fonction \texttt{tresor}
qui calcule, pour tout entier $n$, la somme~:

\[	tresor(n) = azor(1) + azor(2) + .... azor(n) \]

dans laquelle \texttt{azor} est une fonction de type 
``\verb+num -> num+'', 
vous trouveriez que l'enseignant abuse de votre bonne volonté,
- ou manque sérieusement d'imagination - en vous posant toujours le
m\^eme genre d'exercices, que vous savez très bien faire.


% \citation{Programmer, c'est toujours la m\^eme chose (enfin presque).}

Dans la vie quotidienne du programmeur, cette situation se produit
fréquemment~: il vous faudra écrire un programme P2 qui fait presque
la m\^eme chose qu'un programme P1 que vous avez déjà écrit
auparavant~: t\^ache d'intér\^et uniquement alimentaire~! Dans un tel
cas bien s\^ur on ne repart pas de zéro~: on récupère le texte de P1,
et on le bricole un peu.

Par exemple, on ``bidouillerait'' la définition de \texttt{som} en la rebaptisant
\texttt{tresor}, et en rempla\c{c}ant \texttt{n+som(n-1)} par 
\texttt{azor(n)+tresor(n-1)}. Et le
tour est joué !


La programmation fonctionnelle offre une alternative plus séduisante
que ce bricolage indigne~: c'est l'utilisation de \emph{fonctions
d'ordre supérieur}. Expliquons-nous~: dans tous les cas il s'agit
d'additionner les valeurs $f(1)+f(2)+ ... + f(n)$, pour une certaine
fonction $f$ qui est soit l'identité, soit la fonction $carre$, la
fonction $cube$, $azor$ ou ce qu'on voudra. Alors nous allons écrire
une \emph{fonctionnelle} qui fera le calcul voulu pour tout entier $n$
et toute fonction $f$~:


\begin{verbatim}
dec somfonc : num X ( num -> num ) -> num ;
--- somfonc ( 0 , f )  <= 0 ;
--- somfonc ( n , f )  <= f(n) + somfonc (n-1 , f );
\end{verbatim}

\textbf{Vocabulaire}~: en mathématiques, on appelle
\emph{fonctionnelle} ou \emph{fonction d'ordre supérieur} une fonction
dont un (au moins) des paramètres est lui-m\^eme une fonction.

Maintenant \texttt{tresor} s'écrit plus simplement~:
\begin{verbatim}
dec tresor : num -> num ;
--- tresor ( n) <= somfonc ( n , azor ) ;
\end{verbatim}

Pour réécrire \texttt{somcarre} à l'aide de \texttt{trésor}, on peut~:

\begin{itemize}
\item soit se donner la peine d'écrire une fonction \texttt{carre}~:
\begin{verbatim}
dec carre : num -> num
--- carre(n) <= n*n ;

dec somcarre2 : num -> num ;
--- sommcarre2(n) <= somfonc( n , carre ) ;
\end{verbatim}

\item soit utiliser une \emph{fonction anonyme}~:
\begin{verbatim}
dec somcarre3 : num -> num ;
--- somcarre3(n) <= somfonc( n , lambda (n) => n*n  ) ;
\end{verbatim}
\end{itemize}

En HOPE la forme~:
\begin{alltt}
lambda ( \textrm{x, y, z} \ldots ) => \textrm{expression}
\end{alltt}
sert à désigner la fonction anonyme qui à $x, y, z \ldots$ fait correspondre
une certaine valeur décrite par l'expression. On peut utiliser les
lambda-expressions à volonté, tapez par exemple~:
\begin{alltt}
	(lambda (a,b) => 2*a - 3*b) (10,4) ;
\end{alltt}
et vous verrez appara\^{\i}tre ? 


\section{À propos des fonctions auxiliaires}

Pour résoudre un problème compliqué, on le décompose en problèmes plus
simples. En programmation fonctionnelle, on aura souvent besoin
d'écrire des fonctions auxiliaires, pour résoudre une partie d'un
problème. Pensez-y car~:

\begin{itemize}
\item Souvent on ne peut pas faire autrement ! Par exemple il semble
difficile décrire la fonction $f(n,k) = 1^k + 2^k + \ldots + n^k$ sans
écrire une fonction auxiliaire {\em a puissance b}
\item Vous aurez peut-\^etre besoin de la m\^eme fonction auxiliaire
ailleurs.
\item L'emploi d'une ``bonne'' fonction auxiliaire peut conduire à un
gain d'efficacité énorme.
\end{itemize}


Par exemple, avec la définition suivante :

\begin{verbatim}
dec fib : num -> num ;
--- fib(1) <= 1;
--- fib(2) <= 1 ;
--- fib(n) <= fib(n-1) + fib(n-2);
\end{verbatim}

Que vaut $mystere(1,n)$ ?

\begin{exercice}
Prouvez-le !
\end{exercice}


\begin{exercice}
Exercice 2 ( ``variable tampon'' )
\begin{verbatim}
dec bizarre : num X num X num -> num ;
--- bizarre ( a, 0, c) <= c;
--- bizarre ( a, b, c) <= bizarre(a-1, b, a*c);
\end{verbatim}
Que vaut $bizarre( n, p, 1)$ ? 
 
\end{exercice}


\begin{exercice}
\textbf{Test de primalité} : 
Un nombre (entier positif) est premier s'il admet exactement deux
diviseurs~: l'unité et lui m\^eme. Ècrire une fonction qui indique si
un nombre est premier.

Indications~:
\begin{itemize}
\item contrairement à une opinion très répandue, 1 n'est pas premier !
\item un nombre $ n \geq 2 $ n'est pas premier s'il a un diviseur dans
l'intervalle entier $\{ 2 \ldots n-1 \}$
\item $ \{ a \ldots b \} = \{ a \ldots b-1 \} \cup \{ b \}$ si  $a<b$.
\item $p$ est divisible par $q$ si et seulement si $p\ modulo\ q = 0$.
\end{itemize}

\end{exercice}

