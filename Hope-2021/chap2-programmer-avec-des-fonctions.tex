
\chapter{Programmer avec des Fonctions}

%	2.1 Quelques rappels mathématiques 
%	2.2 Les fonctions vues comme des boîtes noires
%	2.3 Transparence référentielle
%	2.4 Une session avec HOPE 
%	2.5 Un programme fonctionnel
%
%	Annexe : Quelques éléments du langage HOPE





\section{Quelques rappels mathématiques}

Soient $ A$ et $B$ deux ensembles quelconques. Le {\em produit cartésien}
\index{produit cartésien (définition}
$ A \times B$ est l'ensemble de tous les couples $(a,b)$ dont le premier 
élément appartient \`a $A$ et le second \`a $B$. 
Une {\em relation} \index{relation (définition)} 
entre $A$ et $B$ est un sous-ensemble de $A \times B$.

Une fonction  $f$ de $A$ dans $B$ (notée  $f : A \rightarrow B$ 
est une relation telle que, pour tout élément a de $A$, il existe 
{\em au plus} un élément $b$ de $B$ qui soit en correspondance 
avec lui. Si un tel élément existe, on dit que $f$ est {\em définie}
 pour $a$ et on appelle $b$ l'{\em image} de $a$ par $f$, ce que l'on note
$ b = f(a)$. 

Les ensembles $A$ et $B$ sont alors appelés respectivement {\em ensemble
 de départ} et {\em ensemble d'arrivée} de la fonction $f$.

Le {\em domaine de définition} de $f$ est la partie de $A$ pour laquelle $f$
 est définie, c'est-à-dire les éléments qui possèdent une image.

Une fonction est {\em totale} si elle est définie sur 
tout l'ensemble de départ 
(on dit également que c'est une {\em application} de $A$ dans $B$, et on
 le note $f : A \mapsto B$). Dans le cas contraire, 
c'est une fonction {\em partielle}.

Prenons un exemple simple, la fonction \textsl{signe} qui à tout
nombre entier fait correspondre $positif, négatif$ ou $nul$ selon le
signe du nombre~: nous appellerons naturellement cette fonction
signe. L'ensemble de départ est $Z$, l'ensemble d'arrivée est
$\{positif, négatif, nul\}$.

On peut caractériser cette fonction en écrivant la liste des couples
qui constituent la relation~:
$$signe = \{ \ldots (-2,négatif), (-1,négatif), (0,nul), (1,positif),
(2,positif), \ldots \} $$
ou bien par une suite d'équations~:
\begin{eqnarray*}
& \vdots & \\
signe ( - 2 ) &=& négatif \\
signe ( - 1 ) &=& négatif \\
signe ( 0 ) &=& nul \\
signe ( 1 ) &=& positif \\
signe ( 2 ) &=& positif \\
 & \vdots & 
\end {eqnarray*}
ou encore par un diagramme qui montre la correspondance entre les
éléments des ensembles de départ et d'arrivée.

\begin{center}

\setlength{\unitlength}{1mm}
\begin{picture}(60,80)


\put(15,40){\oval(20,70)}

\put(10,65){\makebox(10,10){...}}
\put(10,55){\makebox(10,10){-2}}
\put(10,45){\makebox(10,10){-1}}
\put(10,35){\makebox(10,10){-0}}
\put(10,25){\makebox(10,10){1}}
\put(10,15){\makebox(10,10){2}}
\put(10,5){\makebox(10,10){...}}

\put(60,40){\oval(30,40)}

\put(55,45){\makebox(10,10)[l]{négatif}}
\put(55,35){\makebox(10,10)[l]{nul}}
\put(55,25){\makebox(10,10)[l]{positif}}

\put(20,70){\vector(3,-2){29}}
\put(20,60){\vector(3,-1){29}}
\put(20,50){\vector(1,0){29}}
\put(20,40){\vector(1,0){29}}
\put(20,30){\vector(1,0){29}}
\put(20,20){\vector(3,1){29}}
\put(20,10){\vector(3,2){29}}

\put(20,60){\circle*{1}}
\put(20,50){\circle*{1}}
\put(20,40){\circle*{1}}
\put(20,30){\circle*{1}}
\put(20,20){\circle*{1}}


\put(50,50){\circle*{1}}
\put(50,40){\circle*{1}}
\put(50,30){\circle*{1}}

\end{picture}


\end{center}

Dans tous les cas, il y a un problème de notation évident, dans la
mesure où l'ensemble de départ est infini.

On peut aussi décrire \textsl{signe} par une simple règle~:
\[ 	signe ( x ) = \left\{
	\begin{array}{ll}
		négatif & \mbox{si $x<0$} \\
		nul	  & \mbox{si $x=0$} \\
		positif	  & \mbox{si $x>0$}
	\end{array}
	\right.  \]

    
$x$ est appelé le {\em paramètre formel} de $signe$, et 
représente n'importe quel élément de l'ensemble de 
départ de la fonction. Le \emph{corps de la règle} (la partie droite) 
indique l'élément de l'ensemble d'arrivée  
qui est en correspondance avec cet $x.$

Lorsqu'on applique la fonction $signe$ à un paramètre {\em effectif}
  (par exemple le nombre $6$), le corps de la règle est évalué et 
renvoie la valeur $positif$.

{\bf Remarque}~: Pendant longtemps le concept de fonction ne
recouvrait que les fonctions qui possèdent une expression,
c'est-à-dire dont on sait déterminer la valeur en tout point. Au
\siecle{XIX} siècle cette notion s'est généralisée aux correspondances
arbitraires, que l'on s'est mis alors à classer selon leurs propriétés
(en s'intéressant surtout aux fonctions numériques)~: fonctions
continues, discontinues, différentiables, intégrables, etc., en
faisant apparaître des cas pathologiques défiant l'intuition, comme la
fonction de Weierstrass, qui est continue sur un intervalle mais n'est
dérivable en aucun de ses points.


\section{Les fonctions vues comme des boîtes noires}

La plupart des sciences de l'ingénieur (électronique, automatique, mécanique, etc.) utilisent la notion de {\em  boîte noire}
\footnote{qui n'a pas grand chose à voir avec l'aéronautique, o\`u les fameuse boîtes noires sont de couleur orange}. 

Une boîte noire est un dispositif dont on connaît le comportement (c'est-à-dire les sorties qu'il fournira en réponse à certaines entrées) sans toutefois en connaître (ou en préférant oublier) les détails internes. 

On peut voir la fonction $signe$ comme une boîte noire avec 
des entrées (paramètres effectifs) et une sortie (la valeur renvoyée).

\begin{center}

\setlength{\unitlength}{1mm}
\begin{picture}(40,50)


\put(5,20){\framebox(30,10){signe}}
\put(15,40){\makebox(10,10){-3}}
\put(15,0){\makebox(10,10)[t]{négatif}}

\put(20,40){\vector(0,-1){9}}
\put(20,20){\vector(0,-1){9}}

\end{picture}


\end{center}

La {\em composition de fonctions} correspond à un agencement de plusieurs boîtes. Voici une fonction qui calcule le maximum de deux nombres~:


\[	max2 ( a, b ) = \left\{
	\begin{array}{ll}
		a	& \mbox{si $a > b$} \\
		b	& \mbox{sinon} 
	\end{array} \right. \]

Pour fabriquer une fonction qui calcule le maximum de trois nombres, on peut écrire~:

\[	max3 ( a, b, c ) = \left\{
	\begin{array}{ll}
 	a	& \mbox{si $a \geq b$ et $a \geq c$} \\
	b	& \mbox{si $b \geq a$ et $b \geq c$} \\
	c	& \mbox{sinon} 
	\end{array} \right. \]

mais c'est assez maladroit, car il est clairement préférable d'écrire~:

\[ max3 ( a, b, c ) =	max2 ( a, max2 ( b, c ) ) \]

Nous obtenons un assemblage de boites que nous pouvons voir, avec un
petit effort d'abstraction, comme une nouvelle boîte noire~:

\begin{center}
\setlength{\unitlength}{1mm}
\begin{picture}(50,50)


\put(10,45){\line(0,-1){5}}
\put(20,45){\line(0,-1){5}}
\put(25,10){\line(0,-1){5}}

\put(15,30){\line(1,-2){5}}
\put(30,45){\line(0,-1){25}}

\put(15,10){\framebox(20,10){max2}}
\put(5,30){\framebox(20,10){max2}}

\put(5,45){\makebox(10,5){12}}
\put(15,45){\makebox(10,5){-17}}
\put(25,45){\makebox(10,5){42}}

\put(10,25){\makebox(10,5)[l]{12}}
\put(20,0){\makebox(10,5)[]{42}}
\end{picture}


\end{center}

\begin{exercice} Représenter la fonction qui renvoie le signe 
du maximum de 4 nombres.
\end{exercice}

A partir de fonctions de base prédéfinies, qui réalisent des
opérations simples, nous pouvons donc construire des fonctions de plus
en plus élaborées, qui représentent des traitements complexes.


\section{Transparence référentielle}

Si nous pouvons ``brancher'' à volonté ces boîtes noires entre
elles, c'est grâce à une propriété fondamentale des fonctions
mathématiques~: la {\em transparence
référentielle}. \index{transparence référentielle} Cette expression
\footnote{qui peut sembler curieuse à propos de boîtes noires.}
signifie simplement que la valeur d'une expression (l'application
d'une fonction à des paramètres effectifs) ne dépend pas du tout
du référent (le contexte général) qui est donc imperceptible, mais
seulement de la valeur des données fournies en entrée.

Pour éclaircir ces propos quelque peu obscurs, 
considérons l'exemple de la figure \ref{ExempleTrompeur}

\begin{figure}[htb]
\barre
\begin{verbatim}
program exemple(output);
var flag : boolean;

    function f (n:integer) : integer;
    begin
      flag := not flag;
      if flag then f := n
              else f := 2*n;
    end;

begin
  flag := true;
  writeln( f(1) + f(2) );
  writeln( f(2) + f(1) );
end.
\end{verbatim}
\caption{L'addition est-elle commutative~?}
\barre
\label{ExempleTrompeur}
\end{figure}

Si vous avez la curiosité de faire exécuter ce programme, 
vous ferez une découverte  importante~: l'addition n'est plus une opération commutative !

Bien entendu, ce n'est pas l'arithmétique qu'il faut remettre en cause,
 mais plut\^ot l'assimilation un peu h\^ative que l'on peut faire 
entre les fonctions Pascal et les fonctions mathématiques. Ici la ``fonction'' {\tt f}
 dépend d'un ``paramètre caché''{ \tt flag}, qu'elle  modifie 
par {\em effet de bord}. \index{effet de bord}.

Cet exemple peut vous sembler artificiel, il est cependant typique de
 l'esprit des langages impératifs~: les procédures consultent et
 modifient les variables globales. Autrement dit, le résultat d'une
 évaluation dépend du contexte dans lequel cette évaluation se produit
~: c'est le contraire de la transparence référentielle.

Un ``module logiciel'' ne peut être réutilisé a priori que dans un
contexte semblable à celui pour lequel il a été con\c{c}u, ce qui ne
facilite pas la réutilisation de modules déjà écrits~: il faudra les
adapter, ce qui co\^ute du temps.

Le programmeur professionnel adoptera donc un style applicatif en
évitant les effets de bords ; s'il ne peut pas les éviter, il doit les
signaler avec soin dans la documentation. En Génie Logiciel on appelle
langages applicatifs les langages de programmation qui interdisent (ou
restreignent) l'utilisation d'effets de bords. Les langages
fonctionnels purs sont des langages applicatifs~: les modules déjà
écrits sont immédiatement réutilisables.

\begin{exercice} La logique la plus élémentaire nous dit que 
$$A \Rightarrow B$$ et $$A \Rightarrow C$$ entraînent 
 $$ A \Rightarrow B \mbox{ et } C$$ 

Soient maintenant $ A =$ ``J'ai 10 Fr'', 
$B =$``Je peux me payer un sandwitch à 9 Fr'', 
C = ``Je peux me payer un demi à 7 Fr''. Discutez le paradoxe.
\end{exercice}

\section{Une session avec HOPE}

Il existe plusieurs versions du langage Hope, qui ont chacune fait
l'objet de plusieurs réalisations. L'interpréteur dont nous disposons
sur le HP9000 a été écrit en Pascal (environ 8000 lignes) par un
étudiant de Hong-Kong dans une université britannique, et a fait
l'objet de portages successifs, ainsi que de diverses corrections. Le
produit final est loin d'être parfait, mais il fonctionne
suffisamment pour démontrer les concepts de la programmation
fonctionnelle.

Dans tout ce qui suit, ce que vous tapez apparaît en caractères
soulignés, les réponses de l'ordinateur en maigre, ainsi que nos
commentaires qui seront précédés d'un point d'exclamation.

Pour utiliser l'interpréteur Hope sur le HP9000, connectez-vous et
tapez la commande~:

\begin{alltt}
% \tape{hope}
\end{alltt}
Sur compatibles PC, tapez :
\begin{alltt}
A> \tape{ichope}
\end{alltt}
Il apparaît  une bannière, le dialogue peut alors commencer~:

\begin{alltt}
>: \tape{5+4;}                         ! une question comme ça ...
>:  9 : num                     ! le type du résultat est num (nombre)
>: \tape{(6-4)*2;}
>:  4 : num                     ! sans surprises ...
>: \tape{7 div 3;}
>:  2 : num 
\end{alltt}

Définissons une fonction, et appliquons-la~:
\begin{alltt}
>: \tape{dec double : num -> num;}     
>: \tape{--- double(n) <= 2*n;}
>:
>: \tape{double(4);}                   
>:  8 : num
>: \tape{double (double (3));}
>:  12 : num
>: \tape{1 + double(5-17);}
>:  -23 : num
\end{alltt}

On peut revoir les définitions~:
\begin{alltt}
>: \tape{display;}
dec double  : num  -> num  ;
--- double ( n )
       <=( 2 *  n ) ;
\end{alltt}

Essayons de nouvelles fonctions~:
\begin{alltt}
>: \tape{dec max2 : num X num -> num;}      ! X = produit cartésien
>: \tape{--- max2 (a, b) <= if a > b then a else b;}
>: \tape{max2 (4,2);}
>:  4 : num

>: \tape{dec max3 : num X num X num -> num;}	! de plus en plus fort
>: \tape{--- max3(x, y, z) <= max2(x, max2(y, z));}
>: \tape{max3(4, 12, 2);}
>:  12 : num
\end{alltt}

Nous pouvons bien sûr charger un fichier de fonctions (\verb+ex1.hope+ sous
Unix, \verb+ex1.hop+ sous DOS)~:
\begin{alltt}
>: \tape{load ex1;}                    
data signes == positif ++ negatif ++ nul;	! un ensemble

dec signe : num -> signes ;
--- signe (n) <=     if n<0 then negatif
                else if n>0 then positif
                            else nul;

dec fac : num -> num ;
--- fac (0) <= 1;                               ! cas particulier n=0
--- fac (n) <= n * fac(n - 1);                  ! cas général
\end{alltt}

Essayons ces fonctions~:
\begin{alltt}
>: \tape{signe(12);}
>:  positif  : signes
>: \tape{signe(6-9);}
>:  negatif  : signes
>: \tape{fac (2+1) ;}
>: 6
>: \tape{trace on;}                                    ! pour suivre les calculs
>: \tape{trace fac;}                                   ! de la fonction fac
>: \tape{fac(3);}
 fac ( 3)                                       ! les appels à fac
 fac ( 2)                                       ! ...
 fac ( 1)
 fac ( 0)
 1                                              ! les réponses
 1                                              ! ...
 2
 6
>:  6 : num
>: \tape{trace off;}
>: \tape{fac(5);}
>:  120 : num
>: \tape{exit;}                                        ! sortie de HOPE
\end{alltt}


\section{Un exemple de programme}

Nous allons écrire maintenant un (petit) programme en HOPE. La
programmation fonctionnelle va certainement vous sembler étrange car
il n'y aura pas de variables, pas d'affectations, pas de boucles et,
à proprement parler, même pas d'instructions~!

Au lieu de cela, un programme fonctionnel décrit comment, à
 certaines données (entrées), on peut associer des résultats
 (sorties), au moyen de fonctions.


\subsection{Exemple~: Un distributeur de boissons}

Pour faire marcher un distributeur de boissons, il faut mettre
quelques pièces dans la fente et appuyer sur un bouton.  Le breuvage
de votre choix s'écoule alors d'un réservoir dans un gobelet.

\bigskip
\begin{center}
\begin{tabular}{llcc}
\hline
	Bouton &	Boisson & Prix &  Réservoir	\\
\hline 

bleu &	café &		300		&	1 \\
rouge &	chocolat & 	300 &		2\\
		vert &	pepsi & 		250 &		3\\
\hline
\end{tabular}
\end{center}
\bigskip

Nous allons écrire un programme fonctionnel pour~:
\begin{itemize}
 \item définir le prix des boissons ;
\item  indiquer la correspondance entre les boutons, les réservoirs et les boissons ;
 \item indiquer ce qui sort quand on appuie sur un bouton après avoir mis une somme suffisante ;
 \item indiquer ce qui sort de la machine qui rend la monnaie.
\end{itemize}

Tout d'abord, précisons les valeurs que peuvent prendre les entrées et sorties~:

\begin{alltt}
	data  bouton	== bleu ++ rouge ++ vert ;
	data  reservoir	== r1 ++ r2 ++ r3 ;
	data  boisson	== cafe ++ chocolat ++ pepsi ++ rien ;
\end{alltt}
Nous venons de définir trois ensembles, de cardinaux respectifs 3, 3 et 4).
\begin{alltt}
	type  argent 	== num ;
\end{alltt}

Le type \texttt{argent} est maintenant synonyme de \texttt{num} qui signifie
``entier''~: cette version de
Hope ne connaît pas les nombres en virgule flottante ... Qu'à
cela ne tienne, nous compterons en centimes.


\subsection{Le prix des boissons}

Indiquons le nom de la fonction et sa ``signature'', c'est-à-dire
les ensembles de départ et d'arrivée~:
\begin{alltt}
	dec prix : boisson -> argent;
\end{alltt}
et la valeur des différentes boissons :
\begin{alltt}
--- prix(cafe)		<= 300;
--- prix(pepsi)		<= 250;
--- prix(chocolat)	<= 300;
\end{alltt}
Les trois tirets ``{\tt ---}'' ordonnent à l'interpréteur d'ajouter
 la ligne à la définition de la fonction.


\subsection{Correspondance entre boutons, réservoirs et boissons}

Indiquez maintenant à quel réservoir correspond chaque bouton~:
\begin{alltt}
dec reserve : 		 -> 		;

--- reserve (bleu)  	<= 	r1;

---

---
\end{alltt}

Et le contenu des réservoirs~:
\begin{alltt}
dec contenu : 	reservoir -> 		;

---

---

---
\end{alltt}

\subsection{Le distributeur}

Comment marche le distributeur ? C'est une ``boîte noire'' que l'on
fait fonctionner en mettant de l'argent et en appuyant sur un bouton ;
il en ressort alors une boisson. On représente donc le distributeur
par une fonction \texttt{distrib}~:
\begin{alltt}
dec distrib : 	argent X bouton	->	boisson ;
\end{alltt}
Quand on appuie sur le bouton \texttt{bou}, cela correspond au
réservoir \texttt{reserve(bou)} qui délivre la boisson
\texttt{contenu(reserve(bou))}~: le bouton rouge correspond au
chocolat par composition des fonctions \texttt{reserve} et
\texttt{contenu}. Mais, attention, nous n'aurons à boire que si nous
avons mis assez d'argent, c'est-à-dire une somme au moins égale à
\texttt{prix(contenu(reserve(bou)))} !
\begin{alltt}
--- distrib ( arg , bou ) <= 
          if  arg >= prix(contenu(reserve(bou)))
              then contenu(reserve(bou))
              else rien ;
\end{alltt}

\paragraph{Remarques~: }
\begin{itemize}
\item L'ensemble de départ de \texttt{distrib} est 
le produit cartésien de deux ensembles, dont l'un est infini.

\item Pour les fonctions \texttt{prix}, 
\texttt{reserve} et \texttt{contenu} nous avions 
donné une équation pour chaque élément de l'ensemble de départ. Ici
nous écrivons une seule équation, avec deux paramètres formels \texttt{arg}
et \texttt{bou} ; cette équation résume une infinité de cas.

\end{itemize}

\subsection{Un distributeur qui rend la monnaie}

On met de l'argent et on appuie sur un bouton, il ressort la monnaie
et la boisson choisie~:
\begin{alltt}
dec distrib2 : 	argent X bouton	-> argent X boisson ;

--- distrib2 (arg , bou ) <= 
        if distrib( arg , bou ) = rien
        then ( arg , rien )
        else ( arg - prix(distrib(arg,bou)) , distrib(arg,bou) );
\end{alltt}

\paragraph*{Remarque~:} Ici l'ensemble d'arrivée est également un produit cartésien~: les résultats sont des couples  notés entre parenthèses.



\section{Annexe~: Quelques éléments du langage HOPE}


\subsection*{Domaines prédéfinis}
\begin{center}
\begin{tabular}{lll}
\hline
Nom & Signification & Exemples\\
\hline
\texttt{num} & nombres entiers & \texttt{0,1,2,12654} \\ 
\texttt{char} & caractères
	&'a','5','!' \\ 
\texttt{truval} & valeurs logiques & \texttt{true, false} \\
	\texttt{alpha,beta} & types variables (génériques) &voir plus loin \\
\hline
\end{tabular}
\end{center}

\subsection*{Déclaration de domaine}

\begin{alltt}
data  nom	==  ... ++ ... ++ ... ;
\end{alltt}

Hope ne possède que peu de fonctions prédéfinies~: en voici la liste
(presque) exhaustive, chacune avec sa signature, et son niveau de
priorité quand elle est déclarée infixe.

\subsection*{Opérations arithmétiques }
\begin{alltt}
infix + , -  : 5 ;
infix * , div , mod : 6 ;
dec + , - , * , div , mod  : (num X num) -> num ;
\end{alltt}

\subsection*{Comparaisons}
	Ces fonctions permettent de comparer deux objets, à
	condition qu'ils soient comparables, c'est-à-dire qu'ils
	appartiennent à un même domaine.
	
\begin{alltt}
infix = , < , > , /= , =< , >= : 6 ;
	dec  = , < , > , /= , =< , >= : (alpha X alpha) -> truval ;
\end{alltt}
\subsection*{Opérateurs logiques}
\begin{alltt}
infix and : 5 ;
infix or  : 4 ;
dec and, or : (truval X truval) -> truval ;
dec not : truval -> truval ;
\end{alltt}

\subsection*{Expressions prédéfinies}

Expression conditionnelle~:
\begin{alltt}
if \textit{condition}               
    then \textit{expression}  
    else \textit{expression} 	
\end{alltt}

Expressions qualifiées :
\begin{alltt}
let \textit{variable} == \textit{expression}  
    in \textit{expression}	

\textit{expression}
  where \textit{variable} == \textit{expression}
\end{alltt}

\subsection*{Définition de fonction}
\begin{alltt}
dec nom : \textit{dépar}t -> \textit{arrivée}  ;
--- \textit{nom} ( .... ) <= \textit{expression} ;
...
--- \textit{nom} ( .... ) <= \textit{expression} ;
\end{alltt}
