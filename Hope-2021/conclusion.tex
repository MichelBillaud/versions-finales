

\chapter*{Conclusion}

La programmation s'apprend en général par t\^atonnements, ce qui fait
parfois conclure un peu h\^ativement qu'il s'agit d'un art (ou d'un
artisanat, voire un bricolage) plut\^ot que d'une technique. Il
convenait donc de montrer que l'édifice repose sur de robustes
fondations mathématiques~: composition de fonctions et calcul par
récurrence. Qu'en conclure ?

\begin{itemize}
\item La notion d'environnement est fondamentale pour la compréhension
  de la programmation impérative. Cette notion ne fait pas partie du
  bagage mathématique usuel. D'o\`u les réticences initiales à
  l'acceptation d'instructions du type~:\verb/i := i + 1/, qui contredisent
  l'aspect déclaratid.
 
\item pour comprendre  -et expliquer- une procédure
non triviale (comportant par exemple une boucle), on  utilise
nécessairement le raisonnement par récurrence, sous une forme plus ou
moins consciente. C'est donc une technique qu'il convient de
ma\^{\i}triser.

\item Ceci justifie l'apprentissage de la programmation fonctionnelle,
comme approche qui familiarise le programmeur avec
le raisonnement explicite par récurrence.  

\end{itemize}
