\chapter{Les listes}

%	4.1 Axiomatique des listes
%	4.2 Les listes en Hope
%	4.3 Induction naturelle sur les listes
%	4.4 Quelques méthodes de tri
%	4.5 Fonctionnelles usuelles
%	4.6 Exercices

\section{Axiomatique des listes}

Les \emph{listes} ou \emph{séquences} sont des groupements d'objets
qu'on ne peut accéder que dans un certain ordre préétabli.

Par exemple le troisième enregistrement d'un fichier séquentiel ne
peut \^etre consulté qu'après avoir lu les deux premiers
successivement.

Pour construire l'ensemble $ListeNombres$ (les listes de
nombres) par exemple, il nous suffit de deux choses~:
\begin{itemize}
\item la liste vide,
\item un moyen de rajouter un nombre quelconque à une liste.
De proche en proche, nous aurons donc la liste vide, puis les listes à un élément, puis à deux éléments, etc. 
\end{itemize}

Ce qui conduit à l'axiomatique suivante~:
\begin{enumerate}
\item $vide$ est une liste de nombres
\item si $n$ est un nombre et $l$ une liste de nombres, 
	alors $cons(n , l)$ est aussi une liste de nombres.

Ainsi, un élément de base (la liste vide) et une fonction 
\begin{verbatim}
cons : num X ListeNombres - > ListeNombres
\end{verbatim}
 nous permettent-ils de construire
des listes de plus en plus grandes~:

\begin{verbatimtab}
vide
cons(732,vide)
cons(1789, cons(732 , vide))
cons(1515, cons(1789 , cons(732 , vide))) 
\end{verbatimtab}
\item Il n'existe pas de couple $(n,l)$ tel que $cons(n , l) = vide$.
\item  cons est injective~: si $cons(n,l) = cons(n',l')$, c'est que 
forcément $n=n'$ et $l=l'$.
\item Si une propriété $P$ est vraie pour $vide$
	 et que $P(l) \Rightarrow  P( cons(n,l) )$ pour tout entier $n$ et toute liste $l$  
	alors $P$ est vraie pour toute liste de nombres.	(Principe d'induction)
\end{enumerate}
On retrouve là une axiomatique semblable à celle de PEANO pour les entiers. C'est bien normal, il suffit de faire correspondre à toute liste sa longueur~: 
\begin{itemize}
\item la liste vide a pour longueur zéro, 
\item si $l$ est de longueur $n$,  $cons( a,l )$ est de longueur $n+1 = succ(n)$.
\end{itemize}

\section{Les listes en Hope}

En Hope, il suffirait d'écrire (si les listes n'étaient pas déjà prédéfinies)~:

\begin{verbatimtab}
data ListeNombres == vide ++ cons( num X ListeNombres ) ;
\end{verbatimtab}

Ce qui définirait les listes de nombres (\texttt{num}), mais pas les
listes de caractères (\texttt{char}) ou de booléens (\texttt{truval}),
ni les autres.


\subsection{Le type générique \texttt{list}}

Le langage HOPE nous donne les moyens de construire des listes
d'objets d'un type arbitraire, par exemple on peut définir des listes
de nombres, des listes de caractères, des listes de listes de
caractères:

\begin{verbatimtab}
type ListeNombres == list(num);
type Chaine == list(char);
type Texte == list(Chaine);
\end{verbatimtab}

En quelque sorte les listes sont d'un type paramétrable (c'est ce
qu'on appelle la \emph{généricité}). Les \emph{listes génériques} HOPE sont
prédéfinies sous la forme~:

\begin{verbatimtab}
typevar truc ;
data list (truc) == nil 
                 ++  truc ::  list (truc) ;
\end{verbatimtab}

Remarque sur les notations~: par commodité ``\verb+::+'' a été
prédéclarée comme étant une opération infixe, ce qui explique pourquoi
on écrit~:
\begin{verbatimtab}
data list (truc) == nil 
                 ++  truc :: list (truc)  ;
\end{verbatimtab}
au lieu de~:
\begin{verbatimtab}
data list (truc) == nil 
		 ++  ::( truc X list (truc)  );
\end{verbatimtab}
mais la signification est  la m\^eme.

Attention, les listes doivent \^etre homogènes !

\begin{itemize}
\item \verb+{(3 :: (true :: nil))+ n'est pas une liste convenable.
\item \verb+(  ('a',true) :: (('b',false) :: nil ))+ est homogène~: 
ses éléments sont de type \verb+(char X truval)+.
\end{itemize}

\subsection{Notations simplifiées}

La notation des listes par des assemblages de \verb+::+ et \verb+nil+
est peu commode, à cause de la quantité de parenthèses. Hope permet
d'écrire les listes entre crochets, les éléments étant séparés par des
virgules. Exemples~:

\begin{verbatim}
[1,2,3,4]  =   (1 :: (2 :: (3 :: (4 :: nil) ) ) ) 
[ ]        =   nil
\end{verbatim}

Dans le cas des listes de caractères, on les écrit entre guillemets~:

\begin{verbatim}
"abcd"  =  [ 'a', 'b', 'c', 'd' ]  =  ('a' :: ('b' :: ('c' :: ('d' :: nil ))))
\end{verbatim}

Attention, il ne faudra pas confondre~:

\begin{itemize}
\item \verb+'a'+, 		qui est de type \texttt{ char},
\item \verb+"a" = ['a']+ 	qui est de type \texttt{list(char)},
\item \verb+["a"] = [ ['a']]+ 	qui est de type \texttt{list(list(char))}.
\end{itemize}

\begin{exercice}
Quels est le type des listes suivantes ? Vérifiez vos réponses sur machine.
\begin{verbatimtab}
1-      (1 :: ( 2::nil))				
2-      ('a' :: ('b' :: nil))
3-      (1 :: nil)
4-      ('w' :: nil)
5-      nil
6-      (false :: (true :: nil))
7-      "berlingot"
8-      [ "pommes","frites"]
9-      [ [ "pommes","vapeur"],["carottes","sautees"]]
10-     ( 12  :: (true :: nil) )
11-     (1 ::2)
\end{verbatimtab}
\end{exercice}

\section{Induction naturelle sur les listes}
\label{fonclistes}
Les listes sont définies à partir d'un objet de base 
\texttt{nil} et d'un constructeur  \verb+::+, selon le
\emph{schéma d'induction}~:

\begin{itemize}
\item $nil$ est une liste
\item si $a$ est un objet et $l$ une liste, alors $(a::l)$ est une liste
\end{itemize}

La majeure partie des fonctions que vous aurez à écrire seront
définies selon le m\^eme schéma d'induction. Par exemple, la fonction
qui retourne la longueur d'une liste~:
\begin{verbatimtab}
dec long : list(alpha) -> num;   ! le type des éléments est indifférent
--- long ( nil )  <= 0
--- long ( a::l ) <= 1+long(l);
\end{verbatimtab}
Ici nous avons défini la fonction $long$ en indiquant~:

\begin{itemize}
\item sa valeur pour l'élément de base $nil$,
\item comment on calcule, à partir de la valeur pour une liste $l$, la valeur de la fonction pour ($a::l$).
\end{itemize}

Nous pouvons suivre les étapes du calcul de $long( [ 12, 7, 4 ] )$~:
$$\begin{array}{rll}
long( [ 12, 7, 4 ] ) &= 1 + long( [7, 4 ] )&	\mbox{en appliquant la seconde équation} \\
			&= 1 + 1 + long( [ 4 ] )&\mbox{idem.}\\
			&= 1 + 1 + 1 + long ( [ ] )&\mbox{idem.}\\
			&= 1 + 1 + 1 + 0 &\mbox{en appliquant la première.} \\
			&= 3 &
\end{array}$$



\paragraph*{Remarque~:} C'est une fonction \emph{polymorphe}, 
dans la mesure o\`u elle peut agir sur plusieurs types de données~:
listes de nombres, de caractères, etc. Le polymorphisme, ajouté à la
génericité, donnent aux langages fonctionnels une grande puissance
d'abstraction. Dans les langages impératifs, si l'on veut calculer la
longueur de listes d'objets de 12 types différents, il faut écrire 12
sous-programmes.


\begin{exercice}
 Ecrire la fonction ``somme des éléments d'une liste de nombres''~:
\begin{verbatimtab}
dec somme : list ( num) -> num ;

--- somme ( [ ] )       <= 				

--- somme ( n :: l )    <=   somme(l)                    ;
\end{verbatimtab}
et montrez les étapes du calcul de somme ( [ 2,5,13,3 ] )~:
\begin{verbatimtab}
somme ( [ 2,5,13,3 ] ) 	=





\end{verbatimtab}
\end{exercice}






\begin{exercice}
 Ecrire une fonction qui indique si un élément est présent ou non dans une liste. Exemples~:
\begin{verbatimtab}
element (  'h' , "la cucarracha" ) = true
element ( 12 , [ 2,3,5,7,11,13] ) = false

dec element :  alpha X list(alpha) ->

--- element ( e , [ ] )         <=

--- element ( e , (a::l) )      <=  


\end{verbatimtab}
\end{exercice}
\begin{exercice}
 Ecrire la fonction ``nombre d'éléments qui sont supérieurs à une
 certaine valeur''.
Exemple~: \verb+super([1,3,12,4,6,2] , 5) = 2+	puisque 2 éléments sont 
plus grands que 5.
\begin{verbatimtab}
dec super : list ( num ) X num  -> num ; 

--- super ( [ ]     , val )     <=

--- super ( n :: l , val )      <= 		


\end{verbatimtab}
\end{exercice}

\begin{exercice}
Ecrire la fonction ``liste des éléments qui sont supérieurs à une certaine valeur''.

Exemple~: \verb+suplis([1,3,12,4,6,2] , 5) = [12,6]+
\begin{verbatimtab}
dec suplis : list ( num ) X num  -> 

--- suplis ( [ ]     , val ) 	<=

--- suplis ( n :: l , val )	<= 		

\end{verbatimtab}
\end{exercice}

\begin{exercice}

Ecrire une fonction qui, à tout entier positif $n$, fait correspondre la liste \texttt{[$n$,$n-1$,$n-2$,\ldots,$2$,$1$]}.
Exemple~: \verb+{yop(5) = [5,4,3,2,1]+
\begin{verbatimtab}
dec yop : 

---

---

\end{verbatimtab}
\end{exercice}
\begin{exercice}
Démontrez que pour tout entier $n$ on a 	
$$long(yop(n)) = n$$  
\end{exercice}


\begin{exercice}





Démontrez que $$somme(yop(n))=som(n)$$

\end{exercice}








\begin{exercice}

 Ecrire une fonction qui concatène deux listes (c-à-d qui les met bout-à-bout).
Exemple~: \verb+conc("abra","cadabra") = "abracadabra"+

\begin{verbatimtab}
dec conc : list(alpha) X list(alpha) -> list(alpha) ;

--- conc ( [  ]   ,  l2 ) 	<=

--- conc ( (a::l) , l2 )	<= 

\end{verbatimtab}
\end{exercice}

\paragraph*{Remarque~: } Cette fonction de concaténation 
est très utilisée en pratique. Par souci d'efficacité, elle a été
intégrée à l'interprèteur sous forme d'une fonction prédéfinie ``\verb+<>+''
(en notation infixe)~:

\begin{verbatimtab}
"etoile " <> "des neiges" = "etoile des neiges"
\end{verbatimtab}

\begin{exercice}
Démontrez les propriétés suivantes de la concaténation~:
\begin{itemize}
\item $nil$ est élément neutre pour l'opération $conc$~: 
$$\begin{array}{rcl}
conc(nil,l2) &=& l2\\
conc(l1,nil) &=& l1
\end{array}$$
\item la concaténation est associative~: 
$$ conc(l1,conc(l2,l3)) = conc( conc(l1,l2) , l3) $$
\item $long(conc(l1,l2)) = long(l1) + long(l2)$
\item $ suplis(conc(l1,l2),val) = conc( suplis(l1,val) , suplis(l2,val) )$
\end{itemize}
\end{exercice}

\section{Quelques méthodes de tri}

A titre d'exemple de programmes fonctionnels sur les listes, nous allons maintenant voir quelques méthodes pour trier une liste de nombres.

\subsection{Tri par insertion}

\paragraph*{Le principe~:}  Pour trier la liste à 4 
éléments \verb+[8,5,12,3]+~:
\begin{itemize}
\item on enlève un élément (le premier = 8)
\item on trie
 (récursivement) le reste de la liste~: on obtient la liste ordonnée 
\verb+[3,5,12]+
\item on insère le premier élément (8) à sa place
\item ce qui donne [3,5,8,12]~: le résultat voulu.
\end{itemize}
	
Mais comment a-t-on trié la liste à 3 éléments \verb+[5,12,3]+~? Et bien, de la m\^eme fa\c{c}on~:
\begin{itemize}
\item on a enlevé le premier (5)
\item on a trié le reste~: ce qui donnait \verb+[3,12]+
\item on a inséré le premier élément (5) à sa place
\item et on a obtenu \verb+[3,5,12]+.
\end{itemize}

Mais comment a-t-on trié \verb+[12,3]+~?
\begin{itemize}
\item ...
\end{itemize}

\paragraph*{Mise en oeuvre~:} Tout d'abord il 
nous faut une fonction auxiliaire pour insérer un élément à sa place
dans une liste ordonnée.

\begin{verbatimtab}
dec insertion : num X list(num) -> num ;

--- insertion (element , [ ] ) 	<=			;

--- insertion (element, premier :: reste)  
                           <= if element < premier
                                 then

                                 else
 
\end{verbatimtab}

Ceci fait, nous pouvons écrire la fonction TriInsertion~:

\begin{verbatimtab}
dec TriInsertion : list(num) - > list(num);

--- TriInsertion ( [ ] )              <=

--- TriInsertion ( premier :: reste ) <= 

\end{verbatimtab}

Cette méthode est facile à programmer, mais elle n'est pas très
efficace~: en effet dans le pire des cas, par exemple la liste
\verb+[15,13,8,5,1]+ les insertions se font toujours à la fin. Pour insérer
15 dans la liste triée \verb+[1,5,8,13]+ il faut 5 étapes de calcul. Pour
insérer 13 dans \verb+[1,5,8]+ il a fallu 4 étapes, etc.

Donc, toujours dans le pire des cas (une liste ordonnée en sens
inverse de longueur n), il faudra effectuer $n+(n-1)+(n-2)+\ldots+2+1 =
\frac{n(n+1)}{2}$ étapes. Le temps du calcul est donc de l'ordre de $n^2$.

\subsection{Tri par partition (version na\"{\i}ve)}

\paragraph*{Le principe~:}  Pour trier la liste à 5 éléments 
\verb+[8,12,5,3,9]+
\begin{itemize}
\item on met de c\^oté le premier élement (8)n
\item on extrait du reste deux listes : 
\begin{itemize}
	\item les éléments plus petits que 8, 
	\item ceux qui sont plus grands~;
\end{itemize}
\item ce qui donne deux listes \verb+[5,3]+  et \verb+[12,9]+
\item on trie ces deux listes (récursivement) ; on trouve alors
\verb+[3,5]+ et \verb+[9,12]+
\item on regroupe~:  \verb+[3,5] <> (  [8] <>  [9,12] )+
\item ce qui donne la liste triée \verb+[3,5,8,9,12]+.
\end{itemize}

\paragraph*{Mise en oeuvre~:} Il nous faut d'abord 
deux fonctions, qui extraient respectivement d'une liste les éléments
qui sont plus petits (ou plus grands) qu'un certain nombre.

\begin{verbatimtab}
dec PlusPetits : num X list(num) -> list(num);

--- PlusPetits ( n , [ ] ) <=

--- PlusPetits ( n , p :: r ) <=


dec PlusGrands : num X list(num) -> list(num);

--- PlusGrands ( n , [ ] ) <=

--- PlusGrands ( n , p :: r ) <=

\end{verbatimtab}

Et la définition du tri par partition s'en suit facilement~:

\begin{verbatimtab}
dec TriPartition : list(num) -> list(num) ;
--- TriPartition ( [ ] )    <= [ ]  ;
--- TriPartition ( p :: r ) <=  
              let pp == PlusPetits(p, r)
           in let pg == PlusGrands(p, r)
           in TriPartition(pp) <>( [p] <> TriPartition(pg) ) ;
\end{verbatimtab}

\subsection{Tri par partition (version améliorée)}

Le co\^ut de calcul peut \^etre diminué par une technique relativement
simple. D'abord o\`u est le problème~? Il vient de ce que le co\^ut
d'une concaténation \verb+g<>d+ est proportionnel à la longueur de la liste
\verb+g+. Et donc le co\^ut de l'évaluation de l'expression~:
\begin{verbatim}
TriPartition(pp) <> ( [p] <> TriPartition(pg) ) 
\end{verbatim}
est la somme du co\^ut
du tri de \verb+pp+ et \verb+pg+, et d'un facteur proportionnel à la taille de
\verb+pp+. (C'e\^ut été encore pire en groupant les parenthèses différemment)

Voici la technique~: on définit une nouvelle fonction à partir de 
\verb+TPConc+, apparemment plus compliquée, en ajoutant un paramètre  supplémentaire  (appelé parfois \emph{paramètre tampon}):
\begin{verbatimtab}
dec TPConc : list(num) X list(num) -> list(num);
--- TPConc( premiere , seconde ) <= TriPartition( premiere ) <> seconde ;
\end{verbatimtab}
Remarquez que~: 	
\begin{verbatim}
TriPartition( liste ) = TPConc( liste , [ ] ) 
\end{verbatim}


Maintenant nous allons voir que nous pouvons redéfinir \texttt{TPConc} de
manière à n'utiliser, dans sa définition, ni la concaténation , ni
\texttt{TriPartition}.

Dédoublons \texttt{TPConc}, selon que son premier argument est la liste vide ou pas~:
\begin{verbatimtab}
dec TPConc : list(num) X list(num) -> list(num);
--- TPConc( [ ]  , seconde ) <= TriPartition( [ ] ) <> seconde ;
--- TPConc(  p::r , seconde ) <= TriPartition( p::r ) <> seconde ;
\end{verbatimtab}
En utilisant la définition de \texttt{TriPartition} ceci équivaut à~:
\begin{verbatimtab}
dec TPConc : list(num) X list(num) -> list(num);
--- TPConc( [ ]  , seconde ) <= [ ] <> seconde ;
--- TPConc ( p :: r , seconde) )  
           <=  let pp == PlusPetits (p , r)
            in let pg == PlusGrands(p, r)
               in TriPartition(pp) <> ([p] <> TriPartition(pg)) <> seconde) ;
\end{verbatimtab}
La concaténation étant associative, on va pouvoir introduire TPConc~:
\begin{verbatimtab}
TriPartition(pp) <> ([p] <> TriPartition(pg) )  <> seconde
        = TriPartition(pp) <> ([p] <> (TriPartition(pg)   <> seconde)) ;
        = TriPartition(pp) <> ([p] <> TPConc(pg, seconde) );
        = TriPartition(pp) <> ( p :: TPConc(pg, seconde) ;
        = TPConc( pp , p::TPConc(pg,seconde) )
\end{verbatimtab}
Ce qui nous mène à une version nettement améliorée du tri par partition~:
\begin{verbatimtab}
dec TPConc : list(num) X list(num) -> list(num);
--- TPConc( [ ]  , seconde ) <=  seconde ;
--- TPConc ( p :: r , seconde) )  
           <=  let  pp == PlusPetits (p , r)
             in let pg == PlusGrands(p, r)
             in TPConc( pp , p::TPConc(pg,seconde) );

dec TriPartition : list(num) -> list(num) ;
--- TriPartition ( liste ) <= TPConc( liste , [ ] );
\end{verbatimtab}
On peut montrer que le c\^out moyen du tri d'une liste de n éléments
est proportionnel à $n \times log(n)$, ce qui est bien meilleur que pour le
tri par insertion. Cependant le co\^ut maximal (dans le pire des cas,
qui est statistiquement très rare) reste de l'ordre de $n^2$.

\subsection{Le tri-fusion}

Pour terminer, une méthode à la fois élégante et efficace, puisque ses co\^uts moyens et maximaux sont proportionnels à $n \times log(n)$.

\paragraph*{Principe~:} Pour trier une liste de 7 éléments 
\verb+[3,5,2,9,7,1,0]+~:
\begin{itemize}
\item on partage cette liste en deux sous-listes, en prenant un
	 élément sur deux.  On obtient alors deux listes \verb+[3,2,7,0]+ et
	\verb+[5,9,1]+ 
\item on les trie, récursivement. On obtient deux listes
	ordonnées \verb+[0,2,3,7]+ et \verb+[1,5,9]+
\item on fusionne ces deux listes,
	 ce qui donne le résultat \verb+[0,1,2,3,5,7,9]+.
\end{itemize}

L'opération de \emph{fusion} ou \emph{interclassement} (algorithme
classique en informatique de gestion) consiste à construire une liste
ordonnée à partir de deux listes également ordonnées~:
\begin{verbatimtab}
dec fusion : list(num) X list(num) -> list(num) ;

--- fusion ( [ ] , l2) <=

--- fusion ( l1, [ ] ) <=

--- fusion ( p1::r1 , p2::r2 ) <=  if p1<p2     then

                                                else
\end{verbatimtab}

Il faut savoir extraire un élément sur deux~:
\begin{verbatimtab}
dec RangPair, RangImpair : list(num) -> list(num);

--- RangPair ( [ ] ) 	<= [ ]
--- RangPair ( p::r ) 	<= RangImpair(r);

--- RangImpair ( [ ] ) 	<= [ ] ;
--- RangImpair ( p::r ) <= p :: RangPair(r) ;	!  récursivité croisée
\end{verbatimtab}
et le tri-fusion s'écrit facilement~:
\begin{verbatimtab}
dec TriFusion : list(num) X list(num) -> list(num) ;
--- TriFusion ( [ ] ) <= [ ] ;
--- TriFusion ( [ seul ] ) <= [ seul ] ;
--- TriFusion ( liste ) <= let (l1,l2) == (RangImpair(liste),RangPair(liste))
				in Fusion( TriFusion(l1) , TriFusion(l2) );
\end{verbatimtab}
Question~: Pourquoi devons nous traiter séparément le cas des listes à
un seul élément ?



\section{Fonctionnelles usuelles sur les listes}

Les fonctions simples vues en \ref{fonclistes} se généralisent
facilement en fonctionnelles ``d'intér\^et général''. Par exemple la
fonction ``nombre des éléments supérieurs à une certaine valeur''
\begin{verbatimtab}
dec super : list(num) X num -> num ;
--- super ( [ ] , val ) <= 0 ;
--- super ( n::l , val) <= if n>val 	then 1+super(l) 
					else super(l);
\end{verbatimtab}
est un cas particulier de la fonction ``nombre des éléments qui
possèdent une certaine propriété P''. Il suffit de passer en paramètre
le prédicat (fonction à résultat booléen) qui indique si un certain $x$
possède ou non la propriété recherchée. Par exemple on passera le
prédicat ``\verb+lambda (x) => x>5+'' 
pour compter les éléments supérieurs à 5. Voici la fonctionnelle~:
\begin{verbatimtab}
dec combien : list(num) X (num -> truval) -> num ;
--- combien ( [ ] , P ) <= 0 ;
--- combien ( n::l , P ) <= if P(n) 	
                            then 1+combien(l) 
		  	     else combien(l);
\end{verbatimtab}
De plus nous n'avons aucune raison de nous limiter aux listes de
nombres~: cette fonctionnelle marche pour des listes de tous types, à
condition bien s\^ur que le domaine du prédicat soit du type
convenable. Nous obtenons la fonctionnelle~:
\begin{verbatimtab}
dec combien : list(alpha) X (alpha -> truval) -> num ;
--- combien ( [ ] , P ) <= 0 ;
--- combien ( n::l , P ) <= if P(n) 	
                            then 1+combien(l,P) 
			    else combien(l,P);
\end{verbatimtab}

\begin{exercice}

\begin{enumerate}
\item Généraliser \texttt{suplis} (liste des
éléments supérieurs à une valeur) pour obtenir une fonctionnelle
\texttt{selection} qui extrait d'une liste les éléments qui
  possèdent une certaine propriété (par exemple ceux qui sont pairs,
  ou ceux qui sont premiers, ou qui sont entre \texttt{"BERTHE"} et
  \texttt{"CHARLOTTE"}, etc.)
\item
 Montrez comment écrire \texttt{suplis} à partir de \texttt{sélection}.
\end{enumerate}
\end{exercice}


\section{Exercices}

\subsection{Sur les fonctionnelles}
\begin{exercice}
 Ecrire une fonction de tri à tout faire (il faudra passer en paramètre un prédicat qui représente la relation d'ordre choisie). Par exemple~:
\begin{verbatim}
TriGeneral ( [ 1,5,4,3,2 ]  , lambda(a,b) => a < b ) = [1,2,3,4,5]
TriGeneral ( [ 1,5,4,3,2 ]  , lambda(a,b) => a > b ) = [5,4,3,2,1]
\end{verbatim}
\end{exercice}

\begin{exercice}
\begin{enumerate}
\item Ecrire une fonction \texttt{liscar} qui prend comme paramètre une
liste de nombres, et renvoie la liste des carrés de ces
nombres. Exemple~:
\begin{verbatim}
liscar [2, 8, 3] = [4, 64, 9]
\end{verbatim}
\item
Genéraliser cette fonction pour obtenir en une fonctionnelle \texttt{map}~:
liste des images par une certaine fonction.
\item Exprimer \texttt{liscar} à partir de \texttt{map}.
\end{enumerate}
\end{exercice}

\begin{exercice}
\begin{itemize}
\item 
   Trouvez une fonctionnelle \texttt{reduction} qui généralise les
   deux fonctions "somme des éléments d'une liste", et "produit des
   éléments d'une liste".
\item
 Montrez que cette fonctionnelle \texttt{reduction} permet
 d'exprimer \texttt{map} aussi bien que \texttt{combien}.
\end{itemize}
\end{exercice}


\subsection{Sur les transformations de programmes}

(revoir méthode du paramètre supplémentaire)~:

\begin{exercice}
Ecrire une fonction \texttt{iota} qui à tout entier $n$ fait
correspondre la liste des $n$ premiers entiers positifs dans le sens
croissant.  Exemple \verb/iota(5) = [1,2,3,4,5]/.

Soit \texttt{iotaconc} la fonction définie par 
\verb+iotaconc( n , liste ) = iota(n) <> liste +
\begin{itemize}
\item donnez une définition récursive directe de \texttt{iotaconc}.
\item en déduire une définition plus efficace de \texttt{iota}.
\end{itemize}
\end{exercice}
	
\begin{exercice}
Par la m\^eme méthode, donnez une définition efficace (conduisant à un
	co\^ut linéaire) de l'inverse d'une liste. Exemple~:
	\verb+inverse("pomme" ) = "emmop"+
\end{exercice}

