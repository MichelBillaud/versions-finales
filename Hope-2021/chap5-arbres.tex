
\chapter{Structures arborescentes}



%	5.1 Les arbres binaires
%	5.2 Fonctions sur les arbres
%	5.3 Arbres binaires de recherche
%	5.4 Expressions arithmétiques
%	5.5 Calcul symbolique





\section{Les arbres binaires}

Nul besoin n'est de présenter ici la notion d'arbre~: répertoires et
fichiers, structure des programmes, tout ou presque est arborescence
en Informatique. Dans cette partie nous nous restreindrons à la
catégorie la plus simple~: les arbres binaires. Nous montrerons un peu
plus loin une utilisation (très classique) des arbres binaires pour le
tri.

Un arbre binaire est, en quelques mots, un arbre dont les noeuds
"portent" deux sous-arbres, à gauche et à droite.  Voici quelques
exemples (rappelons que, par convention, la racine de l'arbre est en
haut)~:

\begin{center}

\setlength{\unitlength}{1mm}
\begin{picture}(105,55)

\put(10,50){\line(-1,-3){5}}
\put(10,50){\line(1,-3){5}}

\put(15,35){\line(-1,-2){5}}
\put(15,35){\line(1,-2){5}}

\put(10,50){\circle*{1}}
\put(5,35){\circle*{1}}
\put(15,35){\circle*{1}}

\put(10,25){\circle*{1}}
\put(20,25){\circle*{1}}

\put(5,10){\makebox(10,10){1}}



\put(40,50){\line(-1,-3){5}}
\put(35,35){\line(-1,-2){5}}
\put(35,35){\line(1,-2){5}}
\put(15,35){\line(-1,-2){5}}
\put(40,24){\line(-1,-1){5}}

\put(40,50){\circle*{1}}
\put(35,35){\circle*{1}}
\put(30,25){\circle*{1}}
\put(40,25){\circle*{1}}
\put(35,20){\circle*{1}}

\put(35,10){\makebox(10,10){2}}

% \put(70,50){\line(1,-3){5}}
\put(70,50){\circle*{1}}
% \put(75,35){\circle*{1}}

\put(65,10){\makebox(10,10){3}}


\put(95,10){\makebox(10,10){4}}
\end{picture}


\end{center}

Les arbres binaires peuvent \^etre vides~: c'est le cas par exemple du
sous-arbre droit du second exemple, et aussi du quatrième exemple
(invisible). Le troisièmes est réduit à un simple racine.
On appelle \emph{feuilles de l'arbre}  les sommets dont les deux
sous-arbres sont vides.

\begin{exercice}
\begin{itemize}
\item
 Dessinez tous les arbres à  0, 1, 2, 3 et 4 sommets.
\item  En général, combien y a-t'il d'arbres à $n$ sommets ?
\end{itemize}
\end{exercice}


\subsection{Définition inductive des arbres binaires}

Le plus souvent, on associe une information (\emph{étiquette}) à
chaque noeud de l'arbre~.

\begin{center}

\setlength{\unitlength}{1mm}
\begin{picture}(60,50)

\put(25,40){\line(-1,-1){15}}
\put(25,40){\line(1,-1){15}}

\put(40,25){\line(-2,-3){10}}
\put(40,25){\line(2,-3){10}}

\put(25,40){\makebox(10,10)[b]{a}}
\put(10,25){\makebox(10,10)[b]{b}}
\put(40,25){\makebox(10,10)[b]{c}}
\put(30,10){\makebox(10,10)[b]{k}}
\put(50,10){\makebox(10,10)[b]{w}}

\end{picture}


\end{center}


Voici une définition inductive de l'ensemble $ArbreBin(E)$ arbres
binaires prenant leurs étiquettes dans un certain ensemble
(tout-à-fait arbitraire) $E$. 

Pour commencer, il nous faut un objet représentant l'arbre
vide, et une fonction $noeud~: ArbreBin(E) \times E \times ArbreBin(E) \rightarrow
ArbreBin(E)$ pour fabriquer un arbre à partir de deux arbres plus
petits et d'une étiquette.

On pose les axiomes suivants~:
\begin{enumerate}
\item L'arbre vide appartient à $ArbreBin(E)$
\item si $A1$ et $A2$ appartiennent à $ArbreBin(E)$ et $e$ appartient à $E$,
		alors $noeud(A1,e,A2)$ appartient à $ArbreBin(E)$
\item $vide$ n'est l'image par $noeud$ d'aucun triplet $(A1,e,A2)$
\item  $noeud$ est injective
\item soit $P$ une propriété sur $ArbreBin(E)$. Si
\begin{itemize}
\item $P(vide)$ est vraie, 
\item pour tous $A1,A2$ dans $ArbreBin(E)$ possédant la propriété P, 
et quelque soit $e$ dans $E$, on a  $P( noeud(A1,e,A2) )$;
\end{itemize}
		alors $P$ est vraie pour tout $a$ dans $ArbreBin(E)$
\end{enumerate}

\subsection{Définition en Hope}

En Hope, c'est une définition d'un type générique~:

\begin{verbatimtab}
data ArbreBin(alpha) == vide 
                     ++ noeud(ArbreBin(alpha) X alpha X ArbreBin(alpha));
\end{verbatimtab}

L'exemple précédent, qui est de type \texttt{ArbreBin(char)}, est représenté par l'expression~:
\begin{verbatimtab}
dec ex1 : ArbreBin(char);
--- ex1 <= noeud( noeud( vide,'b',vide ) , 
		'a' , 
		noeud( noeud(vide,'k',vide) , 
			'c', 
			noeud(vide,'v',vide) ) ):
\end{verbatimtab}

\section{Fonctions sur les arbres}

La plupart des fonctions sur les arbres sont construites à partir du
schéma d'induction naturelle. Par exemple la fonction ``Somme des
étiquettes d'un arbre de nombres''~:
\begin{verbatimtab}
dec somarbre : ArbreBin(num) -> num ;
	
--- somarbre ( vide ) 	<= 

--- somarbre ( noeud(a1,e,a2) )  <= 	somarbre(a1)		somarbre(a2)
\end{verbatimtab}

\begin{exercice}
Ecrire une fonction ``liste des étiquettes des feuilles d'un arbre''.
\begin{verbatimtab}
dec ListeFeuilles : ArbreBin(alpha) -> list(alpha) ;

--- ListeFeuilles ( vide )  <=

--- ListeFeuilles ( noeud (a1,e,a2) ) <= 

\end{verbatimtab}
\end{exercice}

\begin{exercice}

\begin{itemize}
\item Ecrire une fonction $ListePrefixe$ qui renvoie la liste des
étiquettes d'un arbre, dans l'ordre préfixe, c'est-à-dire d'abord
l'étiquette de la racine, puis celles des sous-arbres gauche et
droit. Par exemple~: \verb+ListePrefixe(ex1) = "abckw"+
\item
M\^eme question pour l'ordre infixe (d'abord le sous-arbre gauche,
puis la racine et enfin le sous-arbre droit). Par exemple~:
\verb+ListeInfixe(ex1) <= "bakcw"+
\item Montrez que, pour chacune de ces fonctions, le temps de calcul (évalué
en nombre d'étapes) est (dans le pire des cas) proportionnel au carré
du nombre de sommets de l'argument.
\item
Transformez ces définitions (méthode du ``param\^etre tampon'') pour obtenir des fonctions dont le temps de calcul soit linéaire.
\end{itemize}
\end{exercice}

\section{Arbres binaires de recherche}

Nous avons vu au chapitre précédent une version du tri par insertion
qui n'était guère efficace, parce que nous cherchions à insérer un
élément dans une liste. Au lieu de construire une liste, nous allons
fabriquer un arbre de recherche qui contiendra tous les objets à
trier.

\paragraph*{Définition~:} un \emph{arbre de recherche} 
est un arbre binaire dans lequel toutes les étiquettes du sous-arbre
gauche (resp. droit) sont inférieures (resp. supérieures ou égales) à
l'étiquette de la racine, et de m\^eme pour tous les sous-arbres de
cet arbre.

Exemple : 
\begin{center}
%
% Un arbre de recherche équilibré
%
\setlength{\unitlength}{1mm}
\begin{picture}(40,35)

\put(20,30){\line(-2,-3){10}}
\put(20,30){\line(2,-3){10}}
\put(10,15){\line(-1,-2){5}}
\put(10,15){\line(1,-2){5}}

\put(22,30){10}
\put(12,15){5}
\put(32,15){12}
\put(7,5){7}
\put(17,5){8}
\end{picture}


\end{center}

\begin{exercice}
Ecrire une fonction qui recherche la plus petite étiquette dans un
arbre binaire de recherche (non vide bien s\^ur).
\begin{verbatimtab}
dec PlusPetite : ArbreBin(num) -> num;

--- PlusPetite
\end{verbatimtab}

M\^eme question pour la plus grande étiquette.
\end{exercice}


\begin{exercice}
 Ecrire une fonction qui indique si un arbre binaire quelconque est oui ou non un arbre de recherche.
\begin{verbatimtab}
dec estOrdonne : ArbreBin(num) -> truval ;

--- estOrdonne( vide ) <= true ;

--- estOrdonne( noeud(a1,e,a2) ) <= 
\end{verbatimtab}

\end{exercice}

\begin{exercice}
Montrez que si l'arbre binaire A est un arbre de recherche, alors la
liste $ListeInfixe(A)$ est ordonnée.
\end{exercice}

\subsection{Insertion dans un arbre binaire de recherche}

Pour insérer 9 dans l'arbre de l'exemple, il faudra d'abord aller à
gauche car $9<10$, puis à droite puisque $5<9$, puis encore à droite car
$8<9$. La place étant libre, on peut alors y mettre 9.

\begin{verbatimtab}
dec Insertion : num  X ArbreBin(num) -> ArbreBin(num);

--- Insertion ( n , vide ) <= 

--- Insertion ( n , noeud(a1,e,a2) ) <= if n<e 

			then noeud( Insertion(n,a1) , e , a2 ) 

			else 

\end{verbatimtab}

\begin{exercice}

Montrez que si l'arbre A est ordonné, alors Insertion(n,A) est
également ordonné.
\end{exercice}

Rappelons le principe du tri par insertion. On dispose d'une liste de
nombres à trier. On prend les éléments de cette liste, et on les
insère un-à-un dans une structure de données qui était vide au
départ. Dans la première version la structure était une liste, ici la
structure de données c'est un arbre binaire de recherche. Voici
comment on fabrique un arbre à partir d'une liste~:
\begin{verbatimtab}
dec Arbre : list(num) -> ArbreBin(num) ;
--- Arbre ( [ ] ) <= vide ;
--- Arbre ( n::r ) <= Insertion ( n , Arbre( r ) );
\end{verbatimtab}
Et nous obtenons un nouveau tri par insertion~:
\begin{verbatimtab}
dec TriIns : list( num ) -> list( num ) ;
--- TriIns (liste) <= ListeInfixe (FabriquerArbre ( liste ) ) ;

\end{verbatimtab}

\subsection{ Expressions arithmétiques}

Les expressions arithmétiques, comme par exemple $3.a.x + b + 1$ sont
également des structures arborescentes familières. Les objets de base
en sont les nombres~: ici $1$ et $3$, les variables~: $a, x, b$, et ils sont
combinés par les opérateurs arithmétiques~: addition, multiplication,
etc.

En Hope on peut très facilement créer un ensemble \texttt{expr} semblable aux
expressions arithmétiques (nous nous limiterons aux 4 opérations de
base) en écrivant, dans un premier temps~:
\begin{verbatimtab}
type chaine == list(char);	! par commodité

data expr == 			! une expression peut être ...
   	nombre(num)		! 	- un nombre ayant une certaine valeur
++	var(chaine)		! 	- une variable avec un nom
++	plus( expr X expr )	! 	- la somme ...
++ 	moins ( expr X expr )	!	- la différence ...
++ 	mult ( expr X expr )	! 	- le produit ...
++ 	divis ( expr X expr ) ;	!	- ou le quotient de deux expressions
\end{verbatimtab}

L'expression ``$a.x + b$" sera représentée par~:  

\begin{verbatim}
plus( mult ( var("a") , var("x") ) , var("b") );
\end{verbatim}

\paragraph*{Remarque}

En Hope on ne peut pas réutiliser les symboles \verb/+/, \verb/-/,
etc. comme constructeurs pour de nouveaux types. En effet cette
\emph{surcharge} (\emph{overloading}) conduirait à des ambiguités sur les types
des objets~: la fonction ``\verb/+/'' serait à la fois de type 
``\verb/num X num -> num/'' et ``\verb+expr X expr -> expr+''.


Il est préférable de signaler préalablement que \verb+plus+,
\verb+moins+, etc. sont des opérateurs infixes avec des priorités
semblables à celles de ``\verb/+/'', ``\verb/-/'', etc. Ceci conduit à~:
\begin{verbatimtab}
infix plus, moins : 5 ;
infix mult, divis : 6 ;
type chaine == list(char);

data expr == nombre(num)
	  ++ var(chaine)
 	  ++ expr plus expr
	  ++ expr moins expr
 	  ++ expr mult expr
  	  ++ expr divis expr ;
\end{verbatimtab}
Déclarons maintenant quelques exemples, sous forme d'une fonction qui
renvoie une expression correspondant au numéro d'exemple que l'on
veut~:
\begin{verbatimtab}
dec exemple : num -> expr;
--- exemple(1) <= var "a" mult var "x"  plus var "b" ;
--- exemple(2) <= nombre 1 divis var "x";
--- exemple(3) <= nombre 1 divis exemple(1);
\end{verbatimtab}

\begin{exercice}

\begin{itemize}
\item  A quelles expressions correspondent ces trois exemples ?
\begin{verbatimtab}
exemple 1 -> 

exemple 2 ->

exemple 3 ->
\end{verbatimtab}
\item Ajouter un exemple 4 représentant $a.x^2 - b.x + c$.
\begin{verbatimtab}
exemple(4) <= 
\end{verbatimtab}
\end{itemize}
\end{exercice}
\section{Calcul symbolique}

Dans cette partie nous allons développer un petit exemple typique de ce qu'on appelle programmation symbolique  ou encore manipulation d'expressions symboliques.

Il s'agit d'écrire un programme capable d'effectuer, comme tout
honn\^ete bachelier, la différentiation (ou dérivation) d'une
expression par rapport à une variable. Pour un début, nous
nous contenterons des expressions simples, limitées aux 4 opérations,
que nous avons vues dans la partie précédente.

Vue de près, la différentiation est une opération \texttt{diff} qui, à partir
d'une expression $E$ et d'une variable $V$, permet de trouver une autre
expression $E'$ qui représente la dérivée de $E$ par rapport à $V$. Par
commodité, nous déclarons \texttt{diff} comme opération infixe, ce qui donne~:
\begin{verbatimtab}
infix diff : 4 ;
dec diff : expr X expr -> expr ;
\end{verbatimtab}
Il ne nous reste plus qu'à étudier les différents cas, qui
correspondent aux différentes manières de manières de fabriquer les
expressions~:
\begin{verbatimtab}
--- nombre n     	diff  v       <= nombre 0 ;

--- var x        	diff  v       <= if var x = v
					then 

					else

--- (f plus g)	diff v	<= let (ff,gg) == (f diff v, g diff v)

				in ff plus gg ;

--- (f moins g)	diff v	<= let (ff,gg) == (f diff v, g diff v)
                                  
				in

--- (f mult g)	diff v	<= let (ff,gg) == (f diff v, g diff v)

				in

--- (f divis g)	diff v	<= let (ff,gg) == (f diff v, g diff v)
                                  
				in 
\end{verbatimtab}

Et voilà tout. Ce programme de quelques lignes sait calculer la
dérivée d'une expression par rapport à une variable ! Quelques
exemples pour s'en convaincre~:
\begin{verbatimtab}
>: var "x" diff var "x";
>:  nombre ( 1) : expr

>: exemple 1 diff var "x";
>: ((( nombre ( 0) mult  var ("x")) plus ( nombre ( 1) mult  var ("a"))) 
plus  nombre ( 0)) : expr
\end{verbatimtab}
C'est-à-dire : $0.x + 1.a + 0$ 
\begin{verbatimtab}
>: exemple 2 diff var "x";
>: ((( nombre ( 0) mult  var ("x")) moins ( nombre ( 1) mult  nombre ( 1))) 
divis (var ("x") mult  var("x"))) : expr
\end{verbatimtab}
Autrement dit ~: $\frac{0.x - 1.1}{x.x}$
\begin{verbatimtab}
>: exemple 3 diff var "x" ;
>: ((( nombre ( 0) mult (( var ("a") mult  var ("x")) plus  var ("b"))) 
moins ((((nombre ( 0) mult  var("x")) plus ( nombre ( 1) mult  var ("a"))) 
plus  nombre ( 0)) mult nombre ( 1))) divis ((( var ("a") mult  var ("x")) 
plus  var ("b")) mult (( var("a") mult var ("x")) plus  var ("b")))) : expr
\end{verbatimtab}
En clair ?


Bien que surprenants, ces résultats sont tout-à-fait corrects~: il
suffit de faire quelques simplifications pour retrouver les
expressions attendues. Mais pourquoi la fonction n'a-t-elle pas fait
ces simplifications ? Tout simplement parce que nous ne l'avons pas
demandé !

La manière la plus simple de procéder est de faire les simplifications
au moment o\`u l'on construit les expressions. Par exemple nous
remplacerons l'équation ``dérivée d'une somme d'expressions'' par~:
\begin{verbatimtab}
--- f plus g	diff  v	<= let (ff,gg) == (f diff v, g diff v)
				in ff Plus gg ;
\end{verbatimtab}
o\`u \verb+Plus+ est une nouvelle fonction déclarée (préalablement) ainsi~:
\begin{verbatimtab}
infix Plus : 5 ;
dec Plus : expr X expr -> expr ;
\end{verbatimtab}
La fonction \verb+Plus+ construit une somme d'expressions, comme plus,
mais elle ``sait'' effectuer un certain nombre de simplifications~:
\begin{verbatimtab}
--- nombre n	Plus 	nombre p 	<= nombre (n+p);
--- nombre 0	Plus 	g      		<= g;
--- f 		Plus 	nombre 0 	<= f;
--- f          	Plus 	g      		<= f plus g;

\end{verbatimtab}

\begin{exercice}

\begin{itemize}
\item Modifier \texttt{diff} en introduisant de nouvelles fonctions
\texttt{Moins}, \texttt{Mult} et \texttt{Divis}. (Attention pour la
division, Hope ne connait que les nombres entiers.)
\item Testez sur machine~:
\begin{verbatimtab}
exemple 1 diff var "x" = 

exemple 2 diff var "x" = 

exemple 3 diff var "x" = 
\end{verbatimtab}
\end{itemize}
\end{exercice}
