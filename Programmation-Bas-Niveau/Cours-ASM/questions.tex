%rubber: module xelatex
\documentclass[10pt]{article}

\usepackage[a4paper,hmargin=2.5cm,vmargin=2.0cm]{geometry}

\usepackage{fontspec}

% \usepackage{lmodern}
\usepackage[french]{babel}
\usepackage{hyperref}
\usepackage{multicol}
\usepackage{listings}
\renewcommand{\familydefault}{\sfdefault}


\title{Questions - Cours assembleur}

\author{ASR2-Système 2012-2013}
\date{\today}



\begin{document}
\begin{multicols*}{2}
\maketitle

\section{Cours}

\subsection{SSEM}
\begin{enumerate}
\item En quelle année a-t-elle été développée, et où ?
\item Quel était son objectif ?
\item Quelle était sa capacité mémoire ?
\end{enumerate}

\subsection{Structure d'un ordinateur}

\begin{enumerate}
\item Principaux éléments d'un ordinateur
\item Principaux éléments d'un processeur
\item Qu'est-ce qu'un registre ?
\item Qu'est-ce que l'accumulateur ?
\item Qu'appelle-t'on \emph{compteur de programme} ? Quel est son rôle ?
\item Le premier micro-processeur commercialisé : modèle, fabriquant,
  année ? Combien comportait-il de transistors ?  Combien de
  transistors dans un processeur actuel ?
\end{enumerate}


\subsection{Instructions}
\begin{enumerate}
\item Expliquez ce qu'est le \emph{format d'instruction}, en vous appuyant
sur l'exemple du processeur fictif.
\item Les grandes classes d'instructions
\item Quel est l'effet d'un instruction de saut conditionnel ?
\item Citez les types d'opérandes disponibles sur le processeur fictif
\item Qu'est-ce qu'un opérande \emph{immédiat} ?
\end{enumerate}

\subsection{Assemblage}

\begin{enumerate}
\item sur le programme suivant

\begin{lstlisting}[frame=single, numbers=left]
debut
   loadi 3
   add   a
   store b
fin
   halt 0
a  word 23
b  word 54
\end{lstlisting}
Indiquez ce que sont 
\begin{itemize}
\item les mnémoniques
\item les symboles
\item les opérandes
\item les directives
\end{itemize}

\item qu'est-ce qu'un \emph{assembleur} ?
\end{enumerate}

\section{Programmation}

\subsection{Affectations}

\begin{enumerate}
\item Traduisez l'instruction \texttt{ A = B-C+1 }
\item Traduisez 
\begin{lstlisting}[frame=single]
  if (A < 0) {
     A = -A;
  }
\end{lstlisting}
\item Sur cet exemple
\begin{lstlisting}[frame=single]
debut
   load  a
   add   b
   store c
fin
   halt  0
a  word 10
b word  20
c word  42
\end{lstlisting}

\begin{enumerate}
  \item Quel est le contenu du mot d'adresse 5 avant l'exécution ?
\item
   Quelle est la valeur des \emph{symboles} a,b, c, debut, fin ?
\item 
   Quel est le contenu des \emph{variables} a, b, c après l'exécution.
\item même question si on remplace la première instruction par \texttt{loadi a}
\end{enumerate}

   
\item Écrivez le code pour l'échange de deux variables A et B

\item Étudiez l'effet de la séquence d'instruction suivante, en notant
dans la partie droite le contenu de l'accumulateur et des variables A et B
(au suppose qu'au départ les valeurs respectives sont 4, 2, et 5).

\begin{center}
\begin{tabular}{|l||ccc|}
\hline
  instructions  &   Acc & A & B \\ 
                &    4  & 2 & 5 \\
\hline
   load b       &       &   &   \\
   add b  &&& \\
 store a &&& \\
 sub b &&& \\
 store b &&& \\
 load a &&& \\
 sub b &&& \\
 store a &&& \\
\hline
\end{tabular}
\end{center}x
Quel est en général l'effet de cette séquence, pour des valeurs 
quelconques $x$ et $y$  de A et B ?
\end{enumerate}

\subsection{Décisions}

\begin{enumerate}
\item traduire les séquences
\begin{lstlisting}[frame=single]
si X < SEUIL 
alors  INF = INF + 1
\end{lstlisting}
\begin{lstlisting}[frame=single]
si X <= SEUIL 
alors  INF = INF + 1
\end{lstlisting}
\begin{lstlisting}[frame=single]
si X == VAL 
alors  NB = NB + 1
\end{lstlisting}
\begin{lstlisting}[frame=single]
si X != VAL 
alors  DIFF = DIFF + 1
\end{lstlisting}
\end{enumerate}

\subsection{Boucles}

\begin{enumerate}
\item Traduisez l'algorithme de  calcul du n-ième terme de la 
suite de Fibonacci
\begin{lstlisting}[frame=single, numbers=left]
int n = 5;
int p = 1, f = 0;
for (int i = 0; i != n; i++) {
             // ici f == fib(i-1)
             //  et p == fib(i-2)
    int x = p + f;
             // ici x == fib(i)
    p = f;
    f = x;
         //  ici f == fib(i)
         //    et p == fib(i-1)
}
         //     ici i == n 
       //  et donc f == fib(n)
\end{lstlisting}


\item Traduisez l'algorithme de division par soustractions successives
\begin{lstlisting}[frame=single]
int a = 123, b = 10;

int r = a, q = 0;
while(r > b) {
  r = r - b;
  q = q + 1;
}
\end{lstlisting}

\item Traduisez l'algorithme d'Euclide ci-dessous pour le calcul
du PGCD de a et b. La variable a est supposée strictement positive au
départ, et contiendra le PGCD à la fin.
\begin{lstlisting}[frame=single]
int a = 144, b = 84, c = 0;
while ( b != 0 ) {
    if (a >= b) {
       a = a - b ;
    } else {
       c = a;
       a = b;
       b = c;
    } 
}
\end{lstlisting}
\end{enumerate}

\subsection{Tableaux, pointeurs}

\begin{enumerate}
  \item traduire le fragment de programme C suivant
\begin{lstlisting}[frame=single]
int t[5];
for (int i = 0; i != 5; i++) {
   t[i] = i;
}
\end{lstlisting}

  \item traduire le fragment de programme C suivant
\begin{lstlisting}[frame=single]
int t[5] = {7, 4, 8, 1, 2};
int val = 4;
int nb = 0;
for (int i = 0; i != 5; i++) {
   if (t[i] >= val) {
      nb = nb + 1;
   }
}
\end{lstlisting}

 \item traduire le fragment de programme C suivant
\begin{lstlisting}[frame=single]
int a[5] = {7, 4, 8, 1, 2};
int b[5] = {5, 4, 9, 1, 0}
int nb = 0;

for (int i = 0; i != 5; i++) {
   if (a[i] == b[i]) {
      nb = nb + 1;
   }
}
\end{lstlisting}
  \item traduire le fragment de programme C suivant
(construction de la table des carrés)
% corrigé
\begin{lstlisting}[frame=single]
int carre[10];
carre[0] = 0;
int c = 0;
for (int i = 1; i != 10; i++) {
    c = c + 2*i - 1;
    carre[i] = c;
}   
\end{lstlisting}

\item Voici l'algorithme du tri à bulles
trouvé sur Wikipedia. Traduisez-le pour le tri d'un tableau de 10
éléments.
\begin{lstlisting}[frame=single]
void bubbleSort(int val[], 
                int size)
{
  int i, j, temp;
 
  for (i = size-1; i > 0; i--) {
    for (j = 1; j <= i; j++) {
      if (val[j-1] > val[j]) {
        temp     = val[j-1];
        val[j-1] = val[j];
        val[j]   = temp;
      }
    }
  }
}
\end{lstlisting}
\end{enumerate}

\end{multicols*}
 
\end{document}
