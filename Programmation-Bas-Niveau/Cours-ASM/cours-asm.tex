%rubber: module xelatex
\documentclass[10pt]{article}

\usepackage[a4paper,hmargin=2.5cm,vmargin=2.5cm]{geometry}

\usepackage{fontspec}
\usepackage[french]{babel}

% \usepackage{lmodern}

\usepackage{listings}
\usepackage{hyperref}
\usepackage{multicol}



\title{Programmation de bas niveau \\
le langage de la machine}

\author{S2 - dept info}
\date{\today}


\begin{document}
\maketitle

\tableofcontents

\section{Structure d'un ordinateur}

Un ordinateur est un appareil qui comporte essentiellement
un \emph{processeur},
de la \emph{mémoire}, des \emph{dispositifs d'entrée-sortie}.

Ces éléments interagissent : 
\begin{itemize}
\item le processeur exécute les \emph{instructions}
qui sont dans la mémoire ;
\item ces instructions effectuent des calculs, prennent et placent des
  données qui sont aussi en mémoire, les envoient ou les lisent sur
  les dispositifs d'entrée-sortie.
\end{itemize}

\section{Les processeurs}

De nos jours les processeurs sont très complexes, ils intègrent
plusieurs coeurs, des lignes de caches, des coprocesseurs etc.

Pour comprendre ce qu'ils font, on peut regarder les premiers
ordinateurs.  Construits en technologies discrètes (lampes, puis
transistors et circuits intégrés), ils étaient par force relativement
simples. Le premier prototype d'ordinateur\footnote{le calculateur
  expérimental Small Scale Experimental Machine, développé en 1948 à
  l'Université de Manchester, était la première machine à programme enregistré en mémoire vive}, ne comportait que 550 tubes
électroniques.

Depuis, les choses ont légèrement évolué

\begin{center}
\begin{tabular}{|l|r|l|}
\hline
année &  transistors & processeur\\
\hline
1971 & 2,300 & Intel 4004, premier microprocesseur \\
1978 & 29,000 & Intel 8086, premiers PC \\
1979 & 68,000 & Motorola 68000 \\
1989 &	1,180,000 & Intel 80486 \\
1993 & 	3,100,000 & Pentium \\
1997 & 	9,500,000  & Pentium III \\
2000 & 42,000,000 & Pentium 4 \\
2012 & 1,400,000,000 & Quad-Core + GPU Core i7 \\
2012 & 5,000,000,000  & 62-Core Xeon Phi \\
\hline
\end{tabular}
\end{center}

Source : \url{http://en.wikipedia.org/wiki/Transistor_count}


Pour comprendre le fonctionnement des ordinateurs, nous allons donc nous ramener 
à des machines très simples, du niveau de celles qui existaient au début des années 60.

Fondamentalement, dans un processeur il y a des registres (circuits capables de mémoriser
quelques bits d'information), des circuits combinatoires (additionneur, comparateurs, ...),
et des circuits qui assurent le déroulement des différentes phases d'exécution des instructions.

Le programmeur n'a pas forcément connaissance de tous ces
éléments. Pour programmer, il se réfère au \emph{modèle du
  programmeur}, c'est à dire la partie qui lui est directement accessible.

En général, il s'agit
\begin{itemize}
\item du registre \emph{compteur de programme}, qui indique ou se trouve la prochaine instruction
à exécuter
\item de registres de travail, où sont stockés les résultats intermédiaires,
\item d'indicateurs de condition, qui permettent de savoir si l'addition a eu une retenue,
si la comparaison demandée a réussi ou échoué, etc.
\end{itemize}


\section{Un exemple très simple}

Pour les besoins pédagogiques, nous allons considérer une machine extrêmement simplifiée
\emph{quoique plus élaborée que la SSEM}, avec les caractéristiques suivantes :
\begin{itemize}
\item machine à mots de 16 bits
\item nombres en binaire complément à deux
\item instructions sur 16 bits
\item architecture Von Neumann (les données et les instructions sont dans le même espace mémoire)
\item adresses sur 12 bits (capacité d'adressage : 4096 mots de 16 bits)
\item compteur ordinal 16 bits
\item accumulateur 16 bits
\item opérations arithmétique : addition et soustraction.
\item adressage direct et indirect.
\end{itemize}

\section{Jeu d'instructions}

Les instructions de cette machine sont codées sur 16 bits, dans un format unique : 
\begin{itemize} 
\item les 4 bits
de poids fort indiquent le code opération
\item les 12 bits de poids faible contiennent l'opérande.
\end{itemize}

\begin{center}
\begin{tabular}{|c|c|}
\hline
code & opérande \\
4 bits & 12 bits \\
\hline
- - - - & - - - - - - - - - - - - \\
\hline
\end{tabular}
\end{center}
Trois types d'opérandes sont utilisés, selon les instructions. Par exemple il y a 3 variantes de l'instruction de chargement dans l'accumulateur :

\begin{itemize}
\item \fbox{\texttt{loadi 42}} place la valeur 42 dans l'accumulateur. Le nombre 42 est une \emph{valeur immédiate}.

\item \fbox{\texttt{load 23}} met dans l'accumulateur le contenu du
  mot mémoire d'adresse 23. Le nombre 23 est une \emph{ adresse
    directe}.
\item \fbox{\texttt{loadx 123}} met dans l'accumulateur le contenu du
  mot mémoire dont l'adresse est contenue dans le mot d'adresse
  123. Le nombre 123 est une \texttt{adresse indirecte}. On peut
  considérer que le mot d'adresse 123 est un pointeur.
\end{itemize}

\subsection{Table des instructions}

Par défaut, toutes les instructions ont pour effet d'incrémenter le compteur de programme (\verb/Cp++/), à l'exception des
instructions de saut.

Pour \texttt{halt}, l'opérande est ignoré. 

\begin{center}
\begin{tabular}{|lllll|}
\hline
Code & Mnémonique &  Description & Action & \texttt{Cp = } \\
\hline
0 & \texttt{loadi} \emph{imm12} &  chargement immédiat &  \verb/Acc = ext(imm12)/ & \verb/Cp + 1/\\
1 & \texttt{load} \emph{adr12} &  chargement direct &  \verb/Acc = M[adr12]/ & \verb/Cp + 1/ \\
2 & \texttt{loadx} \emph{adr12} &  chargement indirect &  \verb/Acc = M[M[adr12]]/& \verb/Cp + 1/ \\
3 & \texttt{store} \emph{adr12} &  rangement direct &  \verb/M[adr12] = Acc/ & \verb/Cp + 1/ \\
4 & \texttt{storex} \emph{adr12} &  rangement indirect &  \verb/M[M[adr12]] = Acc/& \verb/Cp + 1/ \\
\hline
5 & \texttt{add} \emph{adr12} & addition & \verb/Acc += M[adr12]/ & \verb/Cp + 1/ \\
6 & \texttt{sub} \emph{adr12} & soustraction & \verb/Acc -= M[adr12]/ & \verb/Cp + 1/ \\
\hline
7 & \texttt{jmp} \emph{adr12}  & saut inconditionnel & &\verb/adr12/  \\
8 & \texttt{jneg} \emph{adr12}  & saut si négatif & & \verb/Acc<0  ? adr12 : Cp+1/ \\
9 & \texttt{jzero} \emph{adr12}  & saut si zéro & & \verb/Acc==0 ? adr12 : Cp+1/ \\
\hline
A & \texttt{jmpx} \emph{adr12}  & saut indirect & & \verb/M[adr12]/ \\
B & \texttt{call} \emph{adr12}  & appel & \verb/M[adr12] = Cp+1/ & \verb/M[adr12]+1 / \\
\hline
C & \texttt{halt 0}  & arrêt & &  \\
\hline
D & & op. illégale & erreur &   \\
E & & op. illégale & erreur &   \\
F & & op. illégale & erreur &  \\
\hline
\end{tabular}
\end{center}

\paragraph{Commentaires}
\begin{itemize}
\item \emph{adr12} (resp. \emph{imm12}) désigne l'adresse (resp. la valeur immédiate) encodée sur les 12 bits de l'instruction
\item l'instruction \texttt{loadi} procède à une \emph{extension de
  signe} de \emph{imm12} : le bit de poids fort de la valeur immédiate
  est copiée dans 4 bits de poids fort de l'accumulateur. Par exemple
  l'instruction \fbox{\texttt{loadi -1}} est codée \texttt{0000 1111
    1111 1111} en binaire. Lors de l'affectation dans l'accumulateur
  16 bits, le bit de signe de la valeur immédiate est propagé de façon
  à obtenir la valeur \texttt{1111 1111 1111 1111} (qui représente -1
  sur 16 bits) dans l'accumulateur.
\item lors de l'exécution des opérations indirectes 
(\texttt{loadx}, \texttt{storex},\texttt{jmpx})
le contenu de \verb/M[adr12]/  qui est sur 16 bits est interprété
comme une adresse sur 12 bits. Une erreur est détectée si les 4 bits de poids 
ne sont pas nuls, et entraîne l'arrêt du processeur.
\item 3 codes ne sont pas utilisés. Ils pourront servir à définir de nouvelles opérations.
\end{itemize}
\section{Notation des programmes}

Le chargement d'un programme consiste à donner un contenu initial à la mémoire.
Ce contenu peut être décrit en hexadécimal :

\begin{center}
\begin{lstlisting}
0009 5005 6006 3007 C000 0005 0003 0000
\end{lstlisting}
\end{center}

\subsection{Utilisation de mnémoniques}

En décodant les 5 premiers mots, on verrait qu'il s'agit d'un programme
\begin{center}
\begin{tabular}{ll|ll}
adresse & contenu & mnémonique & opérande \\
\hline
0  & 0009 & loadi & 9 \\
1 & 5005 & add & 5 \\
2 & 6006 & sub & 6 \\
3 &  3007 & store & 7 \\
4 &  C000 & halt & 0 
\end{tabular}
\end{center}
qui charge la valeur 9 dans l'accumulateur, lui ajoute le contenu du
mot d'adresse 5, retranche celui de l'adresse 6 et range le résultat à
l'adresse 7.

À ces adresses on trouve initialement les valeurs 5, 3 et 0, ce qu'on peut noter
\begin{center}
\begin{tabular}{ll|ll}
adresse & contenu & directive & opérande \\ \hline 5 & 0005 & word & 5
\\ 6 & 0003 & word & 3 \\ 7 & 0000 & word & 0
\end{tabular}
\end{center}
 
La \emph{directive} \texttt{word} indique qu'un mot mémoire est réservé,
avec une certaine valeur initiale sur 16 bits.

\subsection{Utilisation d'étiquettes}

Écrire des programmes avec la notation ci-dessus serait fastidieux :
ajouter/enlever des instructions décale les variables qui sont située
plus loin, et oblige à modifier le codage des instructions qui y font 
référence.

Pour cela on utilise des noms symboliques, les \emph{étiquettes}, pour
désigner les adresses, et un programme prend l'allure suivante
\begin{lstlisting}[frame=single]
#
# premier programme
#
         load  9
         add   premier
         sub   second
         store resultat
         halt  0
#
# réservation des variables
#
premier  word  5
second   word  3
resultat word  0
\end{lstlisting}

La traduction de ce texte source est faite par un programme appelé
\emph{assembleur} : il assemble entre elles les lignes rédigées par
le programmeur, écrites en \emph{langage d'assemblage}\footnote{par un
  abus de langage courant, on parle souvent de ``programmer en
  assembleur''}, et comportant un mélange d'instructions sous forme
mnémonique et de directives de réservation.



\paragraph{Remarques}
\begin{itemize}
\item 
L'étiquette est facultative et commence obligatoirement en colonne 1
si elle est présente. 
\item
Si elle est absente, il doit y avoir au moins
un espace avant le mnémonique ou la directive \texttt{word}.
\item l'étiquette peut aussi être seule sur la ligne. Elle se réfère alors
au prochain mot, donnée ou instruction. 
\item un texte source contient aussi des commentaires, d'autant plus
utiles que la programmation est laborieuse.
\end{itemize}
 
\paragraph{Exercice : } traduire les affectations
\begin{itemize}
\item A = B
\item A = A + 1
\item A = B + C - 1
\end{itemize}

\section{Tests et décisions}

Le processeur ne comporte que deux instructions pour tester
des conditions : regarder si le contenu de l'accumulateur est
nul (\texttt{jzero}) ou si il est négatif (\texttt{jneg}).
Ces instructions sont des \emph{sauts conditionnels} : si la condition
indiquée est vraie, le déroulement se poursuit à l'adresse désignée,
sinon on passe à l'instruction suivante.

En pseudo-code, cela correspondrait à 
\begin{center}
si \texttt{Acc} est nul, aller à \emph{adresse} \\
si \texttt{Acc} est négatif, aller à \emph{adresse}
\end{center}

C'est rudimentaire, mais suffisant pour exprimer des si-alors-sinon.
Voyons par exemple un algorithme qui calcule la valeur absolue d'un nombre.

Sous la forme habituelle d'algorithme structuré :
\begin{lstlisting}[frame=single]
Donnée:
   X nombre
Résultat
   R nombre 
début
   si X >= 0
   alors
     |  R = X
   sinon
     |  R = -X
fin
\end{lstlisting}

quand la condition est fausse, on va exécuter le bloc ``sinon''. 
Quand elle est vraie, on exécute le bloc ``alors'', et on contourne
le bloc ``alors''
Ceci peut être exprimé sous forme de séquence d'instructions 
avec quelques sauts et étiquettes :
\begin{lstlisting}[frame=single]
#
# Séquence de pseudo-instructions pour
# calcul de  R = abs(X)
#
    si X < 0 aller à OPPOSE
    calculer R = X
    aller à SUITE
OPPOSE:
    calculer R = - X    
SUITE:
\end{lstlisting}

\paragraph{Important :} l'étape d'écriture en séquence 
de pseudo-instructions comme ci-dessus est \texttt{indispensable}
pour espérer arriver à un programme à moindre effort.

Les programmeurs expérimentés, qui donnent l'impression d'écrire
directement en langage d'assemblage, font figurer le pseudo-code en
commentaires. En réalité, ils écrivent les commentaires d'abord, qui
leur servent de guide pour aboutir aux instructions machine.

Une première version

\lstinputlisting[frame=single,numbers=left]{Exemples/valeur-absolue.asm}

\paragraph{Exercice.} Remarquez que
\begin{itemize}
\item lors du  second chargement de X, sa valeur est
déjà dans l'accumulateur.
\item les deux alternatives se finissent par la même
instruction.
\end{itemize}
et profitez-en pour \emph{optimiser} ce programme. Précisez
le gain effectué en espace mémoire, et en nombre d'instructions
effectuées.

\paragraph{Exercice.} Écrire la séquence d'instructions en pseudo-code qui
qui calcule le maximum de deux
nombres A et B. Traduire ensuite en instructions machine.
Remarque : comparer, c'est étudier la différence.


\paragraph{Exercice.} Programme qui ordonne deux nombres A et B.
(Après l'exécution A et B contiendront respectivement le max et le min des
deux valeurs initiales). 

\section{Faire des boucles}
Soit à écrire un programme qui calcule la somme S des entiers de 1 à N.

Sous forme d'algorithme classique
\begin{lstlisting}[frame=single]
donnée    N nombre
résultat  S nombre
variable  K nombre
début
   S = O
   K = 1
   tant que K <= N
    faire
      |   S = S + K
      |   K = K + 1
fin
\end{lstlisting}

L'exécution de la boucle commence par un test. Si la condition est vraie,
le corps de boucle est exécuté, et on revient au test. Quand la condition
du test est fausse, le corps de boucle est contourné.


Sous forme de séquence de pseudo-instructions
\begin{lstlisting}[frame=single]
BOUCLE
    S = 0
    K = 1
    si K > N aller à SUITE
    S = S + K
    K = K + 1
    aller à BOUCLE
SUITE
    ...
\end{lstlisting}

Le test revient à étudier le signe de la différence N-K.

D'où le programme
\lstinputlisting[frame=single,numbers=left]{Exemples/somme-entiers.asm}

\paragraph{Exercice : } quand la borne de la boucle est fixée, par exemple
``pour K de 0 à 4'', il y a diverses façons de traduire la boucle
selon qu'on fait le test de fin de boucle en premier ou après la
première exécution du corps, et qu'on effectue la comparaison avec 4
ou avec 5. Comparez-les.


\paragraph{Exercice :} programme qui multiplie
deux valeurs (additions successives)

\paragraph{Exercice :} programme qui divise
deux valeurs (soustractions successives) et fournit le quotient et le
reste.

\paragraph{Exercice :} programme qui calcule la factorielle d'un nombre.

\paragraph{Exercice : } programme qui trouve le plus petit diviseur non 
trivial d'un nombre (plus grand que 1).


\section{Utiliser des tableaux}

Les instructions \texttt{loadx} et \texttt{storex} chargent ou rangent
un mot mémoire à une adresse qui est définie par une variable en
mémoire. Exemple, si le mot d'adresse 23 contient 42, l'instruction
\texttt{loadx 23} aura le même effet que \texttt{load 42}.  Ceci
revient à utiliser le mot d'adresse 23 comme un \texttt{pointeur} vers
la donnée effectivement chargée.


Ceci permet de travailler avec des \emph{tableaux}, qui sont constitués
de cases consécutives :
\begin{lstlisting}
T     word 0         # T[0]
      word 0         # T[1]
      word 0         # T[2]
      word 0         # T[3]
      ...
\end{lstlisting}

Un tableau de
mots (indicé à partir de 0) commence à l´adresse T, sa K-ième case est
à l´adresse T+K. Il suffit donc d´additionner l´\texttt{adresse de base} T
et la valeur de l´indice K pour obtenir l´adresse de la case
\begin{lstlisting}
       loadi  T
       add    K
\end{lstlisting}
que l´on peut ranger dans un pointeur
\begin{lstlisting}
      store  PTR
\end{lstlisting}
qui donne accès à la case \verb/T[K]/, par 
\texttt{loadx  PTR}
ou 
\texttt{storex  PTR}.

\paragraph{Exemple :} remplir un tableau de 5 cases avec les nombres de 0 à 4.

\begin{lstlisting}[frame=single]
résultat: tableau T[5]
variables:  
début
   pour K de 0 à 4
   faire 
     |  T[K] = K
fin
\end{lstlisting}

L'instruction \verb/T[K] = K/ peut se traduire ainsi
\begin{lstlisting}[frame=single]
      loadi  T       # PTR = & T[K]
      add    K
      store  PTR
   
      load   K       # *PTR = K
      storex PTR
\end{lstlisting}


\paragraph{Exercice} Écrire complètement le programme qui remplit
le tableau.



\paragraph{Exercice} Écrire un programme qui calcule la somme des
éléments d'un tableau.

\textbf{Solution}

\lstinputlisting[frame=single,numbers=left]{Exemples/somme-tab.asm}

\paragraph{Exercice} Écrire un programme qui détermine le minimum
des éléments d'un tableau.

\paragraph{Exercice} Écrire un programme qui trie les
des éléments d'un tableau dans l'ordre croissant (algorithme de
tri par sélection).

\textbf{Solution : }

\lstinputlisting[frame=single,numbers=left]{Exemples/tri-tableau.asm}

\paragraph{Remarque : } Une autre approche consiste à incrémenter
 à chaque étape le pointeur
pour qu'à chaque étape il avance sur l'élément suivant
\begin{lstlisting}

     loadi T
     store PTR       # PTR = T
boucle:
     ...
     loadi  1        #  PTR++
     add    PTR
     store  PTR
     ....
     jmp  boucle
\end{lstlisting}

\section{Sous-programmes}

Les instructions \texttt{call} et \texttt{jmpx} servent à réaliser
des appels et retours de sous-programme.

\fbox{\texttt{call} \emph{adr12}} sauve l'adresse de 
l'instruction suivante (\texttt{Cp+1}) à l'emplacement indiqué par 
 \emph{adr12}, et continue à l'adresse (\emph{adr12}+1).\footnote{Ce type 
d'instruction d'appel existait sur les machines PDP/1 et PDP/4 de DEC,
et HP 1000 de Hewlett-Packard}.

Le premier mot d'un sous-programme est donc réservé, et contiendra
l'adresse à laquelle le sous-programme devra revenir, par l'intermédiaire
de l'instruction \texttt{jumpx}

Exemple : un sous-programme qui calcule le plus grand de deux nombres

\begin{lstlisting}
#
# sous programme de calcul du max de 2 nombres
# données dans MaxA et MaxB
# résultat dans MaxRes
#
Max      word  0        # res. adresse de retour
         load  MaxA     # si MaxA < MaxB, aller à MaxL2 
         sub   MaxB
         jneg  MaxL1    

         load  MaxA     # Acc =  MaxA 
         jmp   MaxL2

MaxL1    load  MaxB     # Acc =  MaxB
MaxL2    store MaxRes   # résultat
         jmpx  max      # retour
#
# paramètres / résultats
#
MaxA     word  0        # paramètre
MaxB     word  0
MaxRes   word  0
\end{lstlisting}

Les conventions d'appel sont les suivantes
\begin{itemize}
\item placer les valeurs des deux paramètres dans MaxA et MaxB
\item effectuer l'appel \texttt{call Max}
\item le résultat est disponible dans MaxRes
\end{itemize}

Exemple : séquence de calcul pour \texttt{R = max(max(X,Y),Z)}
\begin{lstlisting}
     load   X
     store  MaxA
     load   Y
     store  MaxB
     call   Max   # calcul temp = max(X,Y)
     load   MaxRes
     store  MaxA
     load   Z
     store  MaxB
     call   Max   # calcul max(temp,Z)
     load   MaxRes
     store  R
\end{lstlisting}

\paragraph{Exercice} Écrire un sous programme de multiplication par addition successives

\paragraph{Exercice} Écrire un sous programme de calcul de factorielle.


\paragraph{Exercice} Écrire un sous programme de division par soustractions successives

\paragraph{Exercice} Écrire un sous programme de calcul de coefficients binomiaux  $C^p_n = \frac{n !}{p! (n-p)!}$


\section{Passage de pointeurs}

Une action qui doit modifier ses paramètres (par exemple
échanger le contenu de deux variables) sera écrite sous forme
d'un sous-programme qui reçoit les \emph{adresses} des
données à échanger.

Exemple
\begin{lstlisting}
Swap    word   0
        loadx  SwapA    # tmp = *A
        store  SwapTmp

        loadx  SwapB    # *A = *B
        storex SwapA

        load   SwapTmp  # *B = tmp
        storex SwapB
        callx  Swap
#
# paramètres reçus
#
SwapA   word 0
SwapB   word 0
#
# variable locale
#
SwapTmp word 0

\end{lstlisting}

\section{Utilisation d'une pile}

La technique précédente ne convient pas aux fonctions qui s'appellent
elles-mêmes, directement ou indirectement. Exemple, la fonction
récursive définie par 
\begin{verbatim}
fib(0) = 0
fib(1) = 1
fib(n) = fib(n-1)+fib(n-2)
\end{verbatim}
 
en effet, avec la technique précédente, on ne réserve en
mémoire que des emplacements (adresse de retour, paramètres, variables
locales) pour une seule invocation de chaque fonction.

Une solution classique est d'utiliser une pile, composée de ``cadres''
contenant les données propres à chaque invocation de la fonction.

Voici un exemple de programme utilisant cette technique. Les
conventions d'appels du sous-programme FIB(N) sont ici :

\begin{itemize}
\item le paramètre d'entrée N est dans l'accumulateur
\item la variable SP pointe sur un espace libre assez grand (environ 2N mots)
qui sert de pile d'exécution
\item au retour le résultat sera dans l'accumulateur
\end{itemize}

Le déroulement de la fonction est schématiquement le suivant
\begin{verbatim}
FIB(N)
début
    si N < 1
       alors retourner N
    sauvegarder N et l'adresse de retour dans la pile
    calculer f1 =  FIB(N-1)
    sauvegarder f1 dans la pile
    calculer f2 =  FIB(N-2)
    récupérer N, l'adresse de retour et f1
    retourner la valeur f1 + f2
fin
\end{verbatim}

\lstinputlisting[frame=single,numbers=left]{Exemples/fib.asm}
   
\end{document}
