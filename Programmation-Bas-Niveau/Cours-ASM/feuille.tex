%rubber: module xelatex

\documentclass[11pt]{article}

\usepackage[a4paper,hmargin=2.5cm,vmargin=2.5cm]{geometry}

\usepackage{fontspec}
% \usepackage{lmodern}
\usepackage[french]{babel}
\usepackage{hyperref}
\usepackage{multicol}
\usepackage{listings}
\renewcommand{\familydefault}{\sfdefault}


\title{Jeu d'instructions du simulateur}

\author{ASR2-Système 2012-2013}
\date{\today}


\begin{document}
\maketitle

Machine à mots de 16 bits, adresses sur 12 bits. 
Registres : accumulateur 16 bits, compteur de programme 12 bits.

Simulateur accessible sur
\url{http://www.labri.fr/perso/billaud/WebSim16/}
et  \url{~/Bibliotheque/ASR2-systeme/WebSim16/index.html} au département.


\section{Table des instructions}



\begin{center}
\begin{tabular}{|lllll|}
\hline
Code & Mnémonique &  Description & Action & \texttt{Cp = } \\
\hline
0 & \texttt{loadi} \emph{imm12} &  chargement immédiat &  \verb/Acc = ext(imm12)/ & \verb/Cp + 1/\\
1 & \texttt{load} \emph{adr12} &  chargement direct &  \verb/Acc = M[adr12]/ & \verb/Cp + 1/ \\
2 & \texttt{loadx} \emph{adr12} &  chargement indirect &  \verb/Acc = M[M[adr12]]/& \verb/Cp + 1/ \\
3 & \texttt{store} \emph{adr12} &  rangement direct &  \verb/M[adr12] = Acc/ & \verb/Cp + 1/ \\
4 & \texttt{storex} \emph{adr12} &  rangement indirect &  \verb/M[M[adr12]] = Acc/& \verb/Cp + 1/ \\
\hline
5 & \texttt{add} \emph{adr12} & addition & \verb/Acc += M[adr12]/ & \verb/Cp + 1/ \\
6 & \texttt{sub} \emph{adr12} & soustraction & \verb/Acc -= M[adr12]/ & \verb/Cp + 1/ \\
\hline
7 & \texttt{jmp} \emph{adr12}  & saut inconditionnel & &\verb/adr12/  \\
8 & \texttt{jneg} \emph{adr12}  & saut si négatif & & \verb/Acc<0  ? adr12 : Cp+1/ \\
9 & \texttt{jzero} \emph{adr12}  & saut si zéro & & \verb/Acc==0 ? adr12 : Cp+1/ \\
\hline
A & \texttt{jmpx} \emph{adr12}  & saut indirect & & \verb/M[adr12]/ \\
B & \texttt{call} \emph{adr12}  & appel & \verb/M[adr12] = Cp+1/ & \verb/M[adr12]+1 / \\
\hline
C & \texttt{halt 0}  & arrêt & &  \\
\hline
D & & op. illégale & erreur &   \\
E & & op. illégale & erreur &   \\
F & & op. illégale & erreur &  \\
\hline
\end{tabular}
\end{center}

\paragraph{Commentaires}
\begin{itemize}
\item \emph{adr12} (resp. \emph{imm12}) désigne l'adresse (resp. la valeur immédiate) encodée sur les 12 bits de l'instruction
\item l'instruction \texttt{loadi} procède à une \emph{extension de
  signe} de \emph{imm12} : le bit de poids fort de la valeur immédiate
  est copiée dans 4 bits de poids fort de l'accumulateur. Par exemple
  l'instruction \fbox{\texttt{loadi -1}} est codée \texttt{0000 1111
    1111 1111} en binaire. Lors de l'affectation dans l'accumulateur
  16 bits, le bit de signe de la valeur immédiate est propagé de façon
  à obtenir la valeur \texttt{1111 1111 1111 1111} (qui représente -1
  sur 16 bits) dans l'accumulateur.
\item lors de l'exécution des opérations indirectes 
(\texttt{loadx}, \texttt{storex},\texttt{jmpx})
le contenu de \verb/M[adr12]/  qui est sur 16 bits est interprété
comme une adresse sur 12 bits. Une erreur est détectée si les 4 bits de poids 
ne sont pas nuls, et entraîne l'arrêt du processeur.
\item le paramètre de \texttt{halt} est ignoré.
\item 3 codes ne sont pas utilisés.
\end{itemize}

\newpage

\section{Exemples de programme}

\lstinputlisting[frame=single,numbers=left]{Exemples/somme-entiers.asm}


\begin{lstlisting}[frame=single]
# somme des éléments d'un tableau

   loadi 0
   store S
   store K
BOUCLE
   loadi 10       / loadi T
   sub   K       /  add   K
   jzero FIN    /   store PTR
   ........... /    loadx PTR
   loadi 1      \   add   S
   add   K       \  store S
   store K
   jmp   BOUCLE
FIN
   halt  0
\end{lstlisting}
\end{document}
