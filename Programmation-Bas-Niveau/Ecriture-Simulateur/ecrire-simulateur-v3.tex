%rubber: module xelatex

\documentclass[10pt]{article}

\usepackage{fullpage} 
\usepackage{fontspec}
\usepackage[french]{babel} 


\usepackage{multicol}
\usepackage{listings}
\renewcommand{\familydefault}{\sfdefault}


\title{Langage C : écriture d'un simulateur}

\author{M. Billaud}
\date{4 avril 2013 - V2}
\lstdefinestyle{sourcec}{numbers=left,frame=single,language=C}
\lstdefinestyle{makefile}{numbers=left,frame=single,language=make}

\begin{document}
\maketitle

\begin{abstract}
Comment programmer en C un simulateur pour un processeur.
On veut écrire un simulateur qui 
\begin{itemize}
\item lit dans un fichier un ``programme'' écrit en hexadécimal, 
destiné à un processeur fictif,
\item exécute ce programme,
\item affiche le contenu de la mémoire à l'issue de ce programme.
\end{itemize}


\end{abstract}

\tableofcontents

\newpage

\section{Décomposition}

\subsection{Le programme principal}

Le programme principal est très court :

\lstinputlisting[style=sourcec]{v2/simulateur.c}

On y voit différents particularités de C
\begin{enumerate}
\item la déclaration de la variable \texttt{m}, qui est du type
  \texttt{struct Machine}, et qui représente l'état de la machine,
  c'est-à-dire le contenu de la mémoire et des différents
  registres. En C, il n'y a pas de classes, mais des \emph{structures} qui
  regroupent simplement des champs.
\item l'écriture sur le terminal, par la sortie standard, se fait par la fonction \texttt{printf()}
qui utilise comme premier paramètre une ``chaîne de format'', dans laquelle 
\verb+\n+ figure le caractère de saut de ligne.
\item on voit cette même notion de format dans l'écriture sur la sortie
d'erreur (\texttt{stderr}), qui se fait par
\begin{center}
\texttt{fprintf(stderr,} \emph{format} \texttt{,} \emph{expressions...}\texttt{);}
\end{center}
Dans le format,  \verb/%s/ indique où et comment il faut afficher la variable \texttt{chemin}. 
Ici c'est une chaine :  \texttt{s} = string.
\item \texttt{charger\_fichier()}, comme son nom l'indique, a pour
  vocation de charge le contenu d'un fichier ``exécutable'' dans la
  mémoire de la machine \texttt{m}. En C++, on ferait un passage par
  référence, qui n'existe pas en C, on passe donc comme paramètre
  l'adresse de la structure \texttt{m}. La fonction
  \texttt{charger\_fichier()} recevra donc un \emph{pointeur}.

\end{enumerate}

\subsection{Structures de données : la machine}

Une machine comporte 4 champs : un tableau de mots de 16 bits 
représentant la mémoire, l'accumulateur et le compteur de programme, 
et un booléen qui indique si elle est arrête.

\paragraph{Commentaires}
\begin{enumerate}
\item nous utilisons les types \texttt{int16\_t} et \texttt{uint16\_t}
  qui codent des entiers, respectivement avec et sans signe, sur 16
  bits. Ces types sont déclarés dans \texttt{stdint.h}.
\item de même les booléens sont définis dans \texttt{stdbool.h}
\item le langage C dispose par ailleurs d'une variété de types
  standard : \texttt{char}, \texttt{int}, \texttt{float},
  \texttt{double}, etc.  Mais pas de chaines de caractères, qui sont
  simplement des tableaux de \texttt{char}, la fin de la chaîne étant
  marquée par le caractère nul \verb/'\0'/.
\item comme en C, les \texttt{char} sont simplement de petits entiers, codés
sur un octet.
\end{enumerate}

\subsection{Fonctions}
Pour l'instant les fonctions de chargement du programme, d'exécution et d'affichage
de l'état de la machine se réduisent à des ``\emph{stubs}'' qui sont présents pour que la compilation se passe bien, 
mais ne font rien d'utile.

\subsection{Makefile}

Enfin, il n'est pas de projet décent sans un bon petit \texttt{Makefile}, qui
contient essentiellement

\begin{lstlisting}[style=makefile]
CFLAGS = -std=c99 -Wall -Wextra -pedantic

simulateur : simulateur.o
\end{lstlisting}

\section{Affichage de l'état de la machine}

Commençons par le plus facile, et nous en aurons besoin ensuite pour
vérifier que les données sont chargées correctement : l'affichage de
l'état de la machine.

\subsection{Objectif}
La fonction \texttt{afficher\_etat} affiche
\begin{itemize}
\item le contenu de la mémoire, sous forme de 16 lignes de 16 mots en hexadécimal (4 caractères)
\item  les contenus des registres et indicateurs ACC, PC, HLT
\end{itemize}

\subsection{Exemple}
Voici un fragment de l'affichage produit
{\small
\begin{verbatim}
ADR     0    1    2    3    4    5    6    7    8    9    a    b    c    d    e    f
   +--------------------------------------------------------------------------------
00 | 0160 43da 7fff 0000 34b0 43d7 7fff 0000 34b0 43d7 7fff 0000 34b0 43d7 7fff 0000 
10 | aaa8 ec3b 7f62 0000 0003 0000 0000 0000 4e2e f63d 0000 0000 782c ec1a 7f62 0000 
.....
e0 | 0000 0000 0000 0000 04c3 0040 0000 0000 0000 0000 0000 0000 08c5 0040 0000 0000 
f0 | bac0 ebe4 7f62 0000 0880 0040 0000 0000 0000 0000 0000 0000 0530 0040 0000 0000 
Registres    ACC [hex=3710 dec= 14096]     PC [43d7]
\end{verbatim}
}

Ce qui montre que l'adresse \texttt{0xe4} contient la valeur \texttt{0x04c3}.

Avec une machine non initialisée, c'est normal de voir un
peu n'importe quoi.


\subsection{Code}

\lstinputlisting[style=sourcec]{v2/afficher-etat.c}

\paragraph{Remarques}
\begin{enumerate}
\item le paramètre \texttt{m} est ici un \emph{pointeur} vers une
  structure, les champs sont donc désignés par \verb/m->ACC/,
  \verb/m->PC/, \verb/m->M/ qui sont des raccourcis pour 
\verb/(*m).ACC/, etc.

\item la spécification de format \texttt{\%x} fait apparaître un
  nombre en hexadécimal, sur autant de caractères que nécessaire. Ici
  le nombre à afficher est entre 0 et 15, donc il n'occupera qu'un
  seul caractère.

\item plus loin, \texttt{\%02x}  (et \texttt{\%04x}) demande un affichage sur 2 (ou 4) caractères au moins,
avec éventuellement des zéros en tête si c'est nécessaire. 
\item enfin, la spécification \texttt{\%6d} demande une représentation
  décimale sur 6 chiffres, des espaces occupant les emplacements des
  zéros non-significatifs
\end{enumerate}

\section{Lecture des données}

\subsection{Objectifs}
\begin{itemize}
\item remplir la structure de données qui représente la mémoire avec 
le contenu d'un fichier texte
\item le fichier contient une suite de ``mots'' de 4 chiffres hexadécimaux.
\end{itemize}

\subsection{Préliminaires : lecture dans un ficheir}
Avant d'aborder la lecture des données dans un fichier, revoyons comment
lire sur le terminal.

\subsubsection{Lecture sur le terminal}

Nous employons \texttt{scanf()}, qui joue un rôle
quasi-symétrique à \texttt{printf()}
\begin{center}
\texttt{scanf(} \emph{format} \texttt{, } \emph{adresses ...}\texttt{) ;}
\end{center}

\lstinputlisting[style=sourcec]{essais/hex.c}

\paragraph{Remarques} :
\begin{enumerate}
\item comme \texttt{printf()}, le premier paramètre est une spécification de format
\item par contre, les paramètres suivants sont les  \emph{adresses}
des données à remplir. C'est une conséquence du seul mode de
passage de paramètres disponible en C : le passage par valeur.
\item ici la lecture se fait en hexadécimal (format \texttt{\%x}), ce
  qui implique que la donnée lue soit un entier non-signé, d'où la
  déclaration :
  \begin{center}
\begin{verbatim}
unsigned int n;
\end{verbatim}
  \end{center}
Pour cette démonstration, une variable de type \texttt{uint8\_t, uint16\_t, uint32\_t, uint64\_t} aurait aussi pu faire l'affaire.

\end{enumerate}

\subsubsection{Lecture d'un fichier}

L'exemple suivant nous permettra de voir la lecture sur un fichier,
à l'aide de 
\begin{center}
\texttt{fscanf(} \emph{fichier}\texttt{,} \emph{format} \texttt{, } \emph{adresses ...}\texttt{) ;}
\end{center}

\lstinputlisting[style=sourcec]{essais/lecture.c}

\begin{enumerate}
\item le premier argument de \texttt{fscanf()} est un fichier, plus précisément
un ``\texttt{FILE *}'' qui sert à désigner un fichier ouvert.
\item la fonction \texttt{fopen()} ouvre un fichier, à partir de son \emph{chemin d'accès} et d'un chaine qui indique son \emph{mode} d'ouverture. Ici \texttt{"r"} indique une ouverture en lecture (ce serait \texttt{"w"} pour une écriture).
\texttt{fopen()} retourne le pointeur \texttt{NULL} si l'ouverture a échoué.
\item \texttt{scanf()} et \texttt{fscanf()} retournent un entier, le
  nombre d'éléments qu'elles ont réussi à lire. Dans ce programme,
  normalement c'est 1, sauf à la fin du fichier où c'est 0. Ce serait
  également 0 si le fichier contenait autre chose que des nombres.
Ce qui explique la boucle de lecture ``tant qu'on arrive à lire un nombre, faire ...''.
\end{enumerate}

En exécutant ce programme avec un fichier \texttt{donnees.txt} qui
contient ceci, 
\begin{verbatim}
12 34 5678
9 1011 121314
15
\end{verbatim}
le déroulement produit, sans surprise :
\begin{verbatim}
$ ./lecture donnees.txt
12 34 5678 9 1011 121314 15 
\end{verbatim}

\subsection{Lecture dans le simulateur}

Pour en revenir notre simulateur, il y a une petite complication : d'une part 
nous voulons lire des \texttt{uint16\_t}, et non des \texttt{unsigned int},
d'autre part toutes les machines ne sont pas semblables : certaines ont des
entiers de 16 bits, d'autres de 32, ou de 64.

Nous voulons du code \emph{portable} : c'est pour ça que nous avons précisé
\texttt{uint16\_t}, pour être sûrs de travailler avec des entiers codés
sur 2 octets.

Pour la lecture, il faut en principe indiquer un format qui correspond
à la nature des données. Sur une machine à mots de 32 bits, on lit un 
entier hexadécimal avec \texttt{\%hx}, le \texttt{h} indiquant un demi-mot.

Mais cela dépend des machines : on se repose donc sur un fichier
\texttt{inttypes.h} qui indique les bons formats pour les différents types
disponibles sur cette machine.

En ce qui nous concerne, nous lirons nos mots de 16 bits avec la spécification
\begin{center}
\texttt{"\%" SCNx16}
\end{center}
qui concatène le caractère ``\%'' et la chaine qui va bien pour les 
entiers 16 bits en représentation hexadécimale. Ce qui nous donne :

\subsection{Code}
\lstinputlisting[style=sourcec]{v2/charger-fichier.c} 


\section{L'exécution}

\subsection{Objectifs}
La fonction d'exécution fait tourner le simulateur, en exécutant les instructions 
depuis la première (PC=0) jusqu'à ce que l'indicateur HLT passe à 1.

\subsection{Rappel: les instructions}

Rappelons que les instructions du processeur fictif sont sur 16 bits,
dont les 4 premiers indiquent le code-opération.
Nous n'allons implémenter que quelques instructions du processeur fictif :

\begin{center}
\begin{tabular}{|lll|}
\hline
Code & Mnémonique &  Description \\
\hline
0 & \texttt{loadi} \emph{imm12} &  chargement immediat \\
1 & \texttt{load} \emph{adr12} &  chargement direct  \\
3 & \texttt{store} \emph{adr12} &  rangement direct  \\
5 & \texttt{add} \emph{adr12} & addition \\
C & \texttt{halt 0}  & arrêt \\
\hline
\end{tabular}
\end{center}

Le programme suivant nous permettra de faire des tests. La traduction 
hexa est en commentaire.
\begin{verbatim}
  loadi 7    # 0007
  add   A    # 5004
  store B    # 3005
  halt  0    # C000
a word  6    # 0006
b word  0    # 0000
\end{verbatim}

\subsection{Technique : Décomposer une instruction}

Nous allons avoir besoin d'extraire 
le code instruction et l'opérande codés respectivement codés sur 4 et 12 birs dans une instruction représentée
sur 16 bits.

Pour cela, on utilise les opérations C qui agissent sur des entiers
considérés comme des vecteurs de bits (\emph{bitwise operators}) :
\begin{itemize}
\item l'opération de \emph{décalage à droite}, noté \verb/>>/.
Le code opération est dans les 4 premiers bits (sur 16), on le récupère en décalant le mot de l'instruction de 12 positions vers la droite 
\item l'opération \emph{et} ``bit à bit'' : l'opérande  s'obtient en faisant un ``et'' avec un masque qui contient 4 bits à 0 et 12 à 1, soit \texttt{000 1111 111 1111}, qui s'écrit \texttt{0x0FFF} en hexadécimal.
\end{itemize}


Voici un programme interactif qui met en oeuvre ces opérations

\lstinputlisting[style=sourcec]{essais/bits.c}

et son déroulement

\begin{verbatim}
$ ./bits < prog2.txt 
mot = 0007, code = 0, operande = 7
mot = 5004, code = 5, operande = 4
mot = 3005, code = 3, operande = 5
mot = c000, code = c, operande = 0
mot = 0004, code = 0, operande = 4
mot = 0000, code = 0, operande = 0
\end{verbatim}

\subsection{Technique : extension de signe}

Avec l'instruction \texttt{loadi} se pose un petit souci. En effet, on
souhaite légitimement avoir la possibilité de charger dans l'accumulateur des constantes négatives
\begin{center}
load -1
\end{center}
or la constante -1 se code  \texttt{1111 1111 1111 1111} sur 16 bits
et ne rentre pas sur les 12 bits d'opérande d'une instruction.

Le parti pris est donc de coder les constantes sur 12 bits seulement,
et l'instruction ci-dessus sera représentée (en hexadécimal) \texttt{0x0fff}.

Lors de l'exécution du \texttt{loadi}, il faudra mettre cette 
valeur -1 (codée) sur 12 bits, dans l'accumulateur de 16 bits.
Pour cela on procède à une \emph{extension de signe}, en profitant du fait
que le décalage à droite d'un entier \emph{signé} propage le bit de signe
vers la droite.

De façon détaillée :
\begin{itemize}
\item la valeur non signée sur 16 bits est décalée de 4 bits vers la
  gauche et placée dans un entier signé de 16 bits,
\item l'entier signé est décalé de 4 bits vers la droite. 
\end{itemize}

Ceci tient en une ligne, en utilisant la conversion en entier signé :
\begin{verbatim}
m->ACC = ((int16_t)(operande << 4))  >> 4; 
\end{verbatim}
Pour convertir explicitement une variable (ou le résultat d'une
expression) dans un autre type, on la précède par le nom du nouveau
type entre parenthèses.
\begin{verbatim}
   int num, den;
   ...
   float r;
   r = ((float) num) / den.
\end{verbatim}
Si on écrivait simplement \texttt{r = num / den}, il y aurait d'abord
une division entière. L'affectation déclencherait alors une conversion 
implicite de ce quotient en flottant.

En procédant comme indiqué, \texttt{num} est d'abord converti, ce qui provoque 
une division flottante.

\subsection{Le code}

Pendant l'exécution, nous allons faire ``tracer'' le déroulement de la
simulation, et pour cela nous utilisons une table des mnémoniques.

Il convient également de tester la validité des accès à la mémoire :
une fonction s'en charge, et en cas d'erreur imprime un message circonstancié,
et met la machine à l'arrêt.

\lstinputlisting[style=sourcec]{v2/lancer-execution.c}

et voici le compte rendu d'exécution du programme ci-dessus :

{\small
\lstinputlisting[]{v2/trace2.txt}
}
\end{document}

