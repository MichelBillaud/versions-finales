%rubber module: xelatex
\documentclass[]{beamer}

\usepackage{../Cours-ASM/transparents-cours}
\usepackage{graphicx}

\usefonttheme{default}

\subtitle{ASR2 - Système}
\title{Programmation C \\
Écriture d'un simulateur}

\author{Semestre 2, année 2012-2013}
\institute{
  Département informatique\\
  IUT Bordeaux 1
}
\date{Avril 2013}


\AtBeginPart{
  \frame{\partpage} 
  \frame{
\frametitle{Contenu}
\begin{multicols}{2}
\tableofcontents
    \end{multicols}
  }
}


\begin{document}
\newcommand{\bashlisting}[0]{\lstset{language=bash,numbers=left,numberstyle=\tiny,frame=single}}
\lstdefinestyle{sourcec}{frame=single,language=C}
\lstdefinestyle{makefile}{frame=single,language=make}

\begin{frame}
  \titlepage
\end{frame}

\begin{frame}
\frametitle{Fil conducteur}

\begin{itemize}
\item
Pour apprendre C, on écrit un 
\alert{simulateur 
pour le processeur fictif}.

\item
\alert{Usage typique} du C :
\begin{itemize}
\item programmation à bas niveau
\begin{itemize}
\item utilisation d'entiers 16 bits (signés / non signés)
\item manipulation de bits (extraction du code opération etc).
\end{itemize}
\item portabilité
\end{itemize}
\end{itemize}

\end{frame}

\begin{frame}
\frametitle{Découpage du programme}
\begin{itemize}
\item \alert{lire} dans un fichier un ``programme'' écrit en hexadécimal, 
\item \alert{exécuter} ce programme
\item à la fin, \alert{afficher} l'état de la machine
\end{itemize}
\end{frame}

\begin{frame}[containsverbatim]
\frametitle{Allure générale du programme}
\begin{lstlisting}[style=sourcec]
struct Machine {
  .... M[256];  
  .... ACC;     
  .... PC;      
  bool HALT;
};

int main(int argc, char **argv)
{
     struct Machine m;

     charger_fichier  (& m, argv[1]);
     lancer_execution (& m);
     afficher_etat    (& m);

     return EXIT_SUCCESS;
}
\end{lstlisting}
\end{frame}

\begin{frame}[containsverbatim]
  \frametitle{Structures}
  \begin{itemize}
  \item Les \alert{structures} regroupent plusieurs \alert{champs}

\begin{lstlisting}[style=sourcec]
struct Date {
  int jour;
  int mois;
  int annee;
};

struct Personne {
  char nom[30];
  char prenom[30];
  struct Date naissance;
};
\end{lstlisting}
  \end{itemize}
\end{frame}

\begin{frame}[containsverbatim]
  \frametitle{Structures (suite)}
  \begin{itemize}
\item = \alert{classe} C++, \alert{sans méthodes}, 
et avec tous les \alert{champs publics}.
\item notations pointée ``\verb+.+'' et flèche ``\verb+->+''
  \end{itemize}
{\small
\begin{lstlisting}[style=sourcec]
struct Date bastille;

bastille.jour =   14;
bastille.mois =    7;
bastille.annee = 1789;

struct Personne * musicien;  // pointeur

strcpy (musicien -> prenom, "kevin");
strcpy (musicien -> nom   , "ayers");

(musicien -> naissance).jour   = 16;
(musicien -> naissance).mois   = 8;
(musicien -> naissance).annee  = 1944;
\end{lstlisting}
}
\end{frame}




\begin{frame}[containsverbatim]
\frametitle{Les types}
\begin{lstlisting}[style=sourcec]
#include <stdint.h>
#include <stdbool.h>

struct Machine {
  uint16_t M[256];  
  int16_t  ACC;     
  uint16_t PC;      
  bool     HALT;
};
\end{lstlisting}
\begin{tabular}{lll}
\texttt{HALT} & indique si la machine tourne & \alert{\texttt{bool}} \\
accumulateur & un entier de 16 bits & \alert{\texttt{int16\_t}} \\
compteur de programme & entier de 16 bits, positif ou nul &\alert{\texttt{uint16\_t}} \\
mémoire & 256 mots de 16 bits & 
\end{tabular}

\end{frame}


\begin{frame}
  \frametitle{Types entiers}
  \begin{itemize}
  \item les \alert{types de base}
    \begin{itemize}
    \item \texttt{int}, \texttt{char}
    \item modificateurs : \texttt{short},  \texttt{long}
    \item \texttt{signed}, \texttt{unsigned}
    \end{itemize}
    dont la taille dépend de l'implémentation
  \item Les \alert{types standards}, pour la portabilité :
    \begin{itemize}
    \item de \alert{taille exacte} : 
      signés \texttt{int8\_t, int16\_t, int32\_t, int64\_t}
      et non signés \texttt{uint8\_t ...} 
    \item de \alert{taille minimum} : 	\texttt{int\_least8\_t, ...}
    \item les \alert{représentations efficaces} (en temps) : \texttt{int\_fast8\_t, ...}
    \end{itemize}
  \end{itemize}
    La \emph{représentation exacte} n'existe pas forcément (exemple
    microcontrôleur 18 bits).

\end{frame}


\begin{frame}
  \frametitle{Bilan provisoire}
Vous connaissez
\begin{itemize}
\item les \alert{structures}
\begin{itemize}
\item définition
\item différence avec les classes
\item notations pointée et fléchée
\end{itemize}
\item les \alert{types entiers}
\begin{itemize}
\item \alert{types de base} : int / char / long / short / signed / unsigned
\item \alert{types standards}, taille exacte, taille minimum, représentation efficace.
\end{itemize}
\end{itemize}

Le langage C permet de préciser la représentation des données : langage de bas niveau.
\end{frame}


\begin{frame}[containsverbatim]
\frametitle{Retour à l'écriture du simulateur}

Développement incrémental : on procède par étapes 
\begin{enumerate}
\item \alert{ajout de ``stubs}'' (fonctions bouchon) pour que le source
soit accepté à la compilation
\item \alert{écriture et test} de chaque fonction, dans l'ordre de facilité :
\begin{enumerate}
\item affichage de l'état
\item chargement du programme
\item exécution
\end{enumerate}
\end{enumerate}

\alert{Fonctions bouchon} : ne font rien (à part afficher un message !)
\begin{lstlisting}[style=sourcec]
 int foo (float bar) 
 {
    printf("Appel de %s\n", __PRETTY_FUNCTION__);
}
\end{lstlisting}
\end{frame}


\begin{frame}[containsverbatim]
\frametitle{Travail : complétez le code par des stubs}
\footnotesize
\begin{multicols}{2}
\begin{lstlisting}[style=sourcec]
#include <stdio.h>
#include <stdint.h>
#include <stdbool.h>

struct Machine {
  uint16_t M[256];  
  int16_t  ACC;     
  uint16_t PC;      
  bool     HALT;
};
\end{lstlisting}
\break
\begin{lstlisting}[style=sourcec]
// ....         ....
// .... STUBS ? ....
///....         ....
\end{lstlisting}
\begin{lstlisting}[style=sourcec]
int main(int argc, char **argv)
{
  struct Machine m;

  charger_fichier  (& m, 
                    argv[1]);
  lancer_execution (& m);
  afficher_etat    (& m);

  return EXIT_SUCCESS;
}

\end{lstlisting}
\end{multicols}
\end{frame}

\begin{frame}[containsverbatim]
  \frametitle{Travail : programmer l'affichage de l'état de la machine}


\alert{Exemple de présentation } : 
{
\scriptsize
\begin{lstlisting}
ADR     0    1    2    3    4    5    6    7    8    9    a    b  
   +--------------------------------------------------------------
00 | 0160 43da 7fff 0000 34b0 43d7 7fff 0000 34b0 43d7 7fff 0000 3
10 | aaa8 ec3b 7f62 0000 0003 0000 0000 0000 4e2e f63d 0000 0000 7
.....
e0 | 0000 0000 0000 0000 04c3 0040 0000 0000 0000 0000 0000 0000 0
f0 | bac0 ebe4 7f62 0000 0880 0040 0000 0000 0000 0000 0000 0000 0
 
Registres    ACC [hex=3710 dec= 14096]     PC [43d7]    HALT [0]
\end{lstlisting}
}
\alert{Mémoire} : 16 lignes de 16 nombres de 4 chiffres en hexadécimal 
\begin{lstlisting}
   pour ligne de 0 à 15 faire
    | pour colonne de 0 à 15 faire
    |  | afficher  M[ 16*ligne + colonne ]
    | sauter à la ligne
\end{lstlisting}

\alert{Formats} à utiliser : \texttt{\%x}, \texttt{\%4x} 
\end{frame}



\begin{frame}[containsverbatim]
  \frametitle{Chargement du programme}


\begin{itemize}
\item Programme = suite de mots de 16 bits
\item 1 mot = 4 chiffres hexadécimaux
\item chargement à partir de l'adresse 0
\end{itemize}

Exemple :
\begin{lstlisting}[frame=single]
0007  5004 3005
C000 0006 
0000
\end{lstlisting}

Traduction de 
\begin{verbatim}
  loadi 7    # 0007
  add   A    # 5004
  store B    # 3005
  halt  0    # C000
a word  6    # 0006
b word  0    # 0000
\end{verbatim}
\end{frame}


\begin{frame}[containsverbatim]
  \frametitle{Lecture dans un fichier}
Vous connaissez déjà \texttt{scanf( format, adr....)} pour lire sur le terminal.

\begin{itemize}
\item équivalent : \texttt{fscanf( fichier, format, adr....)}
\item où \texttt{fichier} est un ``FILE *''.
\end{itemize}

\begin{lstlisting}[style=sourceC]
FILE * f;
int nombre, somme = 0;

f = fopen("monfichier.txt", "r");
while ( fscanf(f, "%d", & nombre) == 1) {
   somme += nombre;
};
fclose(f);
printf("La somme vaut %d\n", somme);
\end{lstlisting}

\end{frame}


\begin{frame}
\frametitle{Fichiers de haut niveau}
\begin{itemize}
\item 
\alert{ouverture} par \texttt{fopen(chemin, mode-d'accès)}
\begin{itemize}
\item retourne un  \texttt{FILE *}
\item modes : \texttt{"r"} en lecture, \texttt{"w"} en écriture, ...
\end{itemize}
\item \alert{lecture} par \texttt{fscanf(fichier, format, adr...)}
\begin{itemize}
\item retourne
le nombre d'éléments lus avec succès
\end{itemize}
\item \alert{écriture} par \texttt{fprintf(fichier, format, ....);}
\item \alert{fermeture} par \texttt{fclose(fichier);}
\end{itemize}
\end{frame}


\begin{frame}[containsverbatim]
  \frametitle{Travail : écrire la lecture d'un programme}

  \begin{block}{Fichier "\texttt{prog1.hex}"}
\begin{lstlisting}[frame=single]
0007  5004 3005  C000 0006 0000
\end{lstlisting}
  \end{block}

\alert{Portabilité} pour lire des mots de 16 bits en hexadécimal,
utiliser le format portable \fbox{\texttt{"\%" SCNx16}}, défini dans
\texttt{inttypes.h}

  \begin{block}{Exemple de programme de lecture}
\begin{lstlisting}[style=sourceC]
FILE * f = fopen("monfichier.txt", "r");
if ( f == NULL ) {
   ... // fichier absent ?
}
uint16_t nombre;
while ( fscanf(f, "%" SCNx16, & nombre) == 1) {
   ....
};
fclose(f);
\end{lstlisting}    
  \end{block}

\end{frame}


\begin{frame}
\frametitle{Bilan}
Vous savez maintenant 
\begin{itemize}
\item ouvrir un fichier
\item y lire des données 
\item utiliser les formats de lecture adaptés aux types standards
\end{itemize}

Ecrire : c'est pareil.
\end{frame}

\begin{frame}[containsverbatim]
  \frametitle{Exécution du programme}
Pour lancer l'exécution :
\begin{lstlisting}[frame=single]
initialiser les registres :  HALT = faux, PC = 0 ...

tant que non HALT 
  | aller chercher l'instruction en M[PC]
  | la décomposer en  code opération + opérande
  |
  | selon le code opération
  | si 1 (load)  =>   Acc = M [opérande], PC=PC+1
  | si 3 (store) =>   M [opérande] = ACC, PC=PC+1
  | ....
  | si 12 (halt) =>   HALT = vrai
\end{lstlisting}
\end{frame}

\begin{frame}[containsverbatim]
\frametitle{Énumérations}
\begin{multicols}{2}
Avec une  \alert{enumération}
\begin{lstlisting}[langage=C, frame=single] 
enum Code {
  LOADI, LOAD,  LOADX,   
  STORE, STOREX,
  ADD,   SUB,   JMP,     
  JNEG,  JZERO,
  JMPX,   CALL, HALT
};
\end{lstlisting}
\break
on pourra utiliser des constantes
\begin{lstlisting}[frame=single, language=C]
enum Code code;
...
switch (code} {
case LOADI:
    ....
case LOAD:
    ....
default:
    ...
};
\end{lstlisting}
\end{multicols}
\end{frame}

\begin{frame}[containsverbatim]
\frametitle{Décomposition d'une instruction : opérande}

\begin{itemize}
\item une instruction = 16 bits, non signé
\begin{tabular}{|*{16}{c|}}
\hline
0 & 0 & 1 & 1 &
0 & 0 & 0 & 0 &
0 & 0&  0&  0&
0 & 1 & 0 & 0 \\
\hline
\end{tabular}
\item pour obtenir l'\alert{opérande} (12 bits de droite) : masquer
\begin{lstlisting}
    uint_16  avant =  0x3005;
    uint_16  apres =  avant & 0x0FF;
\end{lstlisting}
{\small
\begin{tabular}{l|*{16}{c|}}
\hline
avant & \alert{0} & \alert{0} & \alert{1} & \alert{1} &
0 & 0 & 0 & 0 &
0 & 0&  0&  0&
0 & 1 & 0 & 1 \\
\hline
et 0x0FF & \alert{0} & \alert{0} & \alert{0} & \alert{0} &
1 & 1 & 1 & 1 &
1 & 1 & 1 & 1 &
1 & 1 & 1 & 1 \\
\hline\hline
après & \alert{0} & \alert{0} & \alert{0} & \alert{0} &
0 & 0 & 0 & 0 &
0 & 0&  0&  0&
0 & 1 & 0 & 1 \\
\hline
\end{tabular}
}
\item Opération : 
 \emph{expression entière} \texttt{\&} \emph{expression entière}

\item \alert{Bitwise operations} : 
\begin{itemize}
\item binaires : 
et (\verb+&+), 
ou  (\verb+|+), 
ou-exclusif  (\verb+^+), 
\item unaire : 
complément (\verb+~+).
\end{itemize}
\end{itemize}
\end{frame}


\begin{frame}[containsverbatim]
\frametitle{Décomposition d'une instruction : code opération}

\begin{itemize}
\item Pour obtenir le \alert{code opération} : décaler de 12 bits vers la droite
\begin{lstlisting}
    uint_16  avant =  0x3005;
    uint_16  apres =  avant >> 12;
\end{lstlisting}
\begin{tabular}{l|*{16}{c|}}
\hline
avant & \alert{0} & \alert{0} & \alert{1} & \alert{1} &
0 & 0 & 0 & 0 &
0 & 0&  0&  0&
0 & 1 & 0 & 1 \\
\hline
après & 
0 & 0 & 0 & 0 &
0 & 0 & 0 & 0 &
0 & 0 & 0 & 0 &
\alert{0} & \alert{1} & \alert{0} & \alert{1} \\
\hline
\end{tabular}
\item \alert{Décalages} :  \begin{itemize}
\item \emph{expression entière} \verb+>>+ \emph{nombre de positions vers la droite}
\item \emph{expression entière} \verb+<<+ \emph{nombre de positions vers la gauche}
\end{itemize}
\item \alert{Propagation de signe} pour le décalage à droite des
\alert{entiers signés}, exemple : 
\verb+  -1 >> 1 + donne  -1 \fbox{111...1111}
\end{itemize}
\end{frame}


\begin{frame}[containsverbatim]
\frametitle{Travail : décomposition}
Programme interactif qui lit des instructions  (nombres 16 bits hexadécimaux), et les décompose
en code opération + opérande :
\begin{lstlisting}[style=sourcec]
int main( ...) {
  for(;;) {
     uint16_t instruction;
     printf("Instruction : ");    // lecture
     if (scanf( ...... instruction) != 1) 
        break;
     uint16_t code     = .....    // décomposition
     uint16_t operande = .....
                                  // affichage
     printf("code = ..... operande = .....\n", 
              code, operande);
    }
  return 0;
}
\end{lstlisting}
\end{frame}

\begin{frame}[containsverbatim]
\frametitle{travail : petit supplément}
Ajoutez l'affichage des mnémoniques, en utilisant un tableau
\begin{lstlisting}[style=sourcec]

char * mnemoniques[16] = {
    "loadi", "load", "loadx",
    ...
    "halt", "illegal13", "illegal14", "illegal15"
};
\end{lstlisting}

\end{frame}



\begin{frame}[containsverbatim]
  \frametitle{Exécution du programme : écriture en C}
\footnotesize
\begin{lstlisting}[frame=single, language=C]
void lancer_execution(struct Machine * m) 
{
  // initialiser les registres
  while ( ! m -> HALT ) { 
     uint16_t instruction = m->M [ m->PC ];
     enum Code code       = ... 
     uint16_t operande    = ...
                                 // tracer instruction
     switch(code) {
     case LOAD :
               ...
               m->PC ++;
               break;
     case HALT :
               m -> HALT = true;
               break;
   }

}
\end{lstlisting}
\alert{Implémentez} \texttt{load}, \texttt{store}, \texttt{add}, et \texttt{halt}.

\end{frame}

\begin{frame}[containsverbatim]
\frametitle{Cas de l'instruction \texttt{loadi}}

\alert{Rôle} : charger la valeur de l'opérande
dans l'accumulateur.

Mais
\begin{itemize}
\item code opération sur 4 bits
\item l'opérande est codé sur 12 bits
\item l'accumulateur un mot de 16 bits.
\end{itemize}

\begin{block}{Exemples }
\begin{itemize}
\item 
 \texttt{loadi 42} = Ox0\alert{02A}, charge 0x\alert{002A}
\item 
 \texttt{loadi -1} = Ox0\alert{FFF}, charge 0x\alert{FFFF}
\end{itemize}
\end{block}
\end{frame}

\begin{frame}
\frametitle{Extension de signe}
Pour les nombres négatifs
\begin{itemize}
\item -1 codé \fbox{1111 1111 1111} en binaire
\item loadi -1 : \fbox{0000 1111 1111 1111} en binaire
\item pour amener -1 dans l'accumulateur, il faut donc
étendre le nombre signé 12 bits sur 16 bits.
\item propager le bit de signe (5 ième) dans les 4 premiers.
\begin{tabular}{r|cccc|}
\hline
avant & 0000 & Syyy & zzzz & tttt \\
\hline
après & SSSS & Syyy & zzzz & tttt \\
\hline

\end{tabular}
\end{itemize}
\end{frame}


\begin{frame}[containsverbatim]
\frametitle{Propagation de signe 12 à 16 bits : comment faire}

On utilise
\begin{itemize}
\item un \alert{décalage à gauche} pour amener le bit de signe à gauche
\item un \alert{changement de type} pour transformer en nombre signé
\item un \alert{décalage à droite} du nombre signé
\end{itemize}

{ \small
\begin{tabular}{r|c|l}
expression & type & contenu en binaire \\
\hline
\verb+opérande+ & uint16\_t & \texttt{0000 Syyy zzzz tttt} \\
\verb+opérande << 4+ & uint16\_t & \texttt{Syyy zzzz tttt 0000} \\
\hline
\verb+(int16_t) (opérande << 4)+ & \alert{int16\_t} & \texttt{Syyy zzzz tttt 0000} \\
\verb+((int16_t) (opérande << 4)) >> 4+ & int16\_t & \texttt{SSSS Syyy zzzz tttt} \\
\hline
\end{tabular}
}

\vspace{1cm}

\alert{Conversions de type} (typecast) 
  \begin{itemize}
\item \texttt{float quotient = (float) num / den;}
  \end{itemize}
\end{frame}


\begin{frame}
\frametitle{Bilan}
Vous connaissez maintenant
\begin{itemize}
\item les opérations ``bit-à-bit''
\item les décalages
\item les conversions de type
\end{itemize}
qui permettent à travailler ``à bas niveau'' sur les données
en mémoire
\end{frame}

\begin{frame}
\frametitle{Travail : complétez le simulateur}
\begin{enumerate}
\item \texttt{loadi}
\item \texttt{sub}
\item \texttt{jmp},  \texttt{jzero}, \texttt{jneg},
\item ...
\end{enumerate}
\end{frame}


\begin{frame}[containsverbatim]
\frametitle{Allocation dynamique}

\alert{Objectif} 
\begin{itemize}
\item simulateur avec taille de mémoire variable
\end{itemize}

\begin{verbatim}
struct Machine {

  uint16_t *M;
  int      memsize;

  int16    ACC;     
  uint16_t PC;      
  bool HALT;
};
\end{verbatim}

\begin{itemize}
\item l'\alert{allocation mémoire} se fait par \texttt{malloc(nombre d'octets)},
\item la \alert{libération}, par \texttt{free}
\end{itemize}
\end{frame}


\begin{frame}[containsverbatim]
\frametitle{Allocation dynamique (suite)}

\alert{Utilisation}
\begin{itemize}
\item l'\alert{allocation mémoire} par \texttt{malloc(nombre d'octets)},
\begin{verbatim}
#include <stdlib.h>

m->memsize = 256;
m->M       = malloc ( m->memsize * sizeof(uint16_t) );
\end{verbatim}
\alert{sizeof()} indique la taille d'un type / d'une variable
\item la \alert{libération} par \texttt{free(pointeur)}
\begin{verbatim}
free( m->M );
\end{verbatim}
\item il est possible \alert{redimensionner} (en recopiant)
\begin{verbatim}
void *realloc(void *ptr, size_t size);
\end{verbatim}
\end{itemize}
\end{frame}



\begin{frame}
  \frametitle{Bilan général}
A travers le développement d'un exemple typique :
\begin{multicols}{2}
\begin{itemize}
\item types, instructions
\item chaines
\item utilisation des pointeurs
\item structures, énumérations
\item programmation à bas niveau (manipulation de bits)
\item allocation dynamique
\end{itemize}

\break
Il reste quelques autres aspects à voir
\begin{itemize}
\item unions
\item utilisation du préprocesseur
\item fichiers à bas niveau
\item bibliothèque Unix
\item ...
\end{itemize}
\end{multicols}
\end{frame}
\end{document}


