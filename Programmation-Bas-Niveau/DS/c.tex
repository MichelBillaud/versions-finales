\section{Langage C}

\subsection{De C++ à C}
\begin{tabular}{ll}
\begin{minipage}[t]{0.45\textwidth}
\begin{lstlisting}[language=c]
// C source code (file s.c)

#include <stdlib.h>
#include <stdio.h>

void swap(int a, int b)
{
	int tmp = a;
	a = b;
	b = tmp;
}

int main(int argc, char *argv[])
{
   int a = atoi(argv[1]);
   int b = atoi(argv[2]);
   swap(a, b);
   printf("a = %d\n", a);
   printf("b = %d\n", b);
   return EXIT_SUCCESS;
}

// End of file
\end{lstlisting}
\end{minipage}
&
\begin{minipage}[t]{0.45\textwidth}
\begin{lstlisting}[language=c++]
// C++ source code (file spp.cpp)

#include <stdlib.h>
#include <iostream>

using namespace std;

void swap(int &a, int &b)
{
	int tmp = a;
	a = b;
	b = tmp;
}

int main(int argc, char *argv[])
{
   int a = atoi(argv[1]);
   int b = atoi(argv[2]);
   int &ra = a;
   int &rb = b;
   swap(ra, rb);
   cout << "a = " << a << endl;
   cout << "b = " << b << endl;
   return EXIT_SUCCESS;
}

// End of file
\end{lstlisting}
\end{minipage}
\end{tabular}

\cparagraph{On lance les deux programmes avec  
les nombres \texttt{4} et \texttt{2} en paramètres. Qu'affichent-ils ?}

\cparagraph{Donnez une version modifiée du code 
source \texttt{C} pour qu'il se comporte de la même manière que celui en \texttt{C++}.}

% -------------------------------------------------------------------

\subsection{Taille de fichier}
\cparagraph{Écrivez un programme qui affiche la taille en octet d'un fichier texte
dont le nom est passé en paramètre.}


\paragraph*{Exemple d'utilisation :}
\begin{verbatim}
$ ./count toto.txt
toto.txt : 42
\end{verbatim}

\textbf{Remarques :}
\begin{itemize}
\item 1 caractère = 1 octet.
% note : précision, pas d'affichage en cas d'erreur.
\item Une fonction annexe \texttt{taille\_fichier} doit ouvrir le fichier, calculer sa taille, 
refermer le fichier et retourner la taille trouvée. En cas d'erreur
la fonction n'affiche rien, et retourne -1.
\item La fonction \textit{main} doit afficher le résultat, ou un message
si le fichier n'a pas été trouvé.
\end{itemize}

\vspace{1cm}

\cparagraph{Modifiez votre fonction \texttt{main} pour qu'elle puisse opérer sur 
plusieurs fichiers, et afficher le total.}

\paragraph*{Exemple d'utilisation :}

\begin{verbatim}
$ ./count toto.txt tata.txt titi.txt
toto.txt : 42
tata.txt : error
titi.txt : 100
--------------------
TOTAL    : 142
\end{verbatim}

% ------------------------------------------------------------------------------------------

\subsection{Base de données}
\begin{lstlisting}[language=c]
struct Date{
	int day;
	int month;
	int year;
};

struct Personne{
	char first_name[20];
	char last_name[20];
	struct Date birth_date;
};

struct Personne *base[128];
int base_size;
\end{lstlisting}

\cparagraph{Écrivez une fonction qui devra afficher la base 
de données contenue dans les \texttt{base\_size} premiers
éléments du tableau \texttt{base}.}

\paragraph*{Exemple d'affichage :}
\begin{verbatim}
First Name            Last Name           Birth Date
--------------------------------------------------------
Bob                   Grieves             31 / 01 / 1979
Alice                 Smith               01 / 12 / 1978
Ruppert               Giles               16 / 02 / 1812
\end{verbatim}

\end{document}
