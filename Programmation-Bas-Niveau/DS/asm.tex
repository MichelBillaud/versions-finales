\section{Assembleur}

% ------------------------------------------------------------------------------------------

\subsection{Un programme de calcul}
On veut calculer $R = 2^N$, selon l'algorithme ci-dessous :
\begin{lstlisting}
     R = 1
     faire N fois
       |  R = R + R
\end{lstlisting}
(on suppose que N est positif ou nul)

\cparagraph{Écrivez le programme en assembleur 
pour le processeur étudié en cours.}

% ------------------------------------------------------------------------------------------

\subsection{Compréhension d'un code}
Soit le programme suivant (les adresses sont indiquées à gauche).

\lstset{frame=single}

\begin{tabular}{ll}
\begin{minipage}[t]{0.45\linewidth}
\begin{lstlisting}
 0 :        loadi 0
 1 :        store a
 2 :boucle  sub   taille
 3 :        jzero fin
 4 :        loadi c
 5 :        add   a
 6 :        store b
 7 :        load  a
 8 :        storex b
 9 :        load  a
10 :        add   un
11 :        store a
12 :        jmp   boucle
13 :fin     halt 0   
\end{lstlisting}
\end{minipage}
&
\begin{minipage}[t]{0.45\linewidth}
\begin{lstlisting}
14 :un      word 1
15 :taille  word 5
16 :a       word 0
17 :b       word 0
18 :c       word 0
19 :        word 0
20 :        word 0
21 :        word 0
22 :        word 0
23 :        word 0
24 :        word 0
\end{lstlisting}
\end{minipage}
\end{tabular}

\cparagraph{Que valent les mots d'adresses \texttt{16} à \texttt{24} au moment où le programme
se termine ?}

\cparagraph{Quel est le rôle respectif de \texttt{a}, \texttt{b} 
et \texttt{c} ?}

\cparagraph{Donnez l'algorithme du programme en pseudo-code.}

\cparagraph{Proposez des optimisations pour ce code, en les justifiant 
clairement}.

% ------------------------------------------------------------------------------------------

\subsection{De C à l'assembleur}
On profite du fait que $n^2 = (n-1 + 1)^2 = (n-1)^2 + 2(n-1) + 1 
= (n-1)^2 + 2n - 1$
pour construire une table des carrés des entiers de 0 à 9, 
sans faire de multiplications.
 
\begin{lstlisting}[language=C]
int carre [10];
carre [0] = 0;
int c = 0;
for (int i = 1 ; i != 10 ; i++) {
      c = c + 2*i - 1 ; 
      carre [ i ] = c ;
}
\end{lstlisting}

\cparagraph{Traduire le code ci-dessus en langage d'assemblage. Vous 
commenterez soigneusement.}

