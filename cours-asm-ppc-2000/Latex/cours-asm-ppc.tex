%rubber: module xelatex
\documentclass[10pt,twoside,a4paper]{article}


\usepackage{fontspec}

\usepackage{a4wide}

% \usepackage[french]{babel}
\usepackage{polyglossia}
\setmainlanguage{french}

\usepackage{moreverb}
\usepackage{multicol}


\title{Initiation à la programmation en langage d'assemblage}
\date{9 mai 2000}
\author{M Billaud}

\begin{document}

\maketitle

\begin{abstract}
  Un plan de cours pour l'initiation à la programmation en langage
  d'assemblage en utilisant un processeur PowerPC. Le cours a été
  inauguré sur Bull Escala sous AIX, puis transposé sur un Mac, et
  enfin sur une machine virtuelle simulant un PowerPC.
\end{abstract}

% \renewcommand{\epsfsize}[2]{0.5#1}

\newenvironment{tableinstr}{\begin{tabular}{p{4cm}l}}{\end{tabular}}
\newcommand{\titreinstr}[1]{\\ \multicolumn{2}{c}{\bf #1}}
\newcommand{\definstr}[2]{\texttt{#1} & $ #2 $}

\begin{multicols}{2}
\tableofcontents

\section{Objectifs du cours}

\begin{itemize}
\item
  Montrer les différents éléments : jeu d'instruction, registres, 
pointeurs, adressages, etc.
\item
  Réalisation des opérations de haut niveau (boucles, décisions,
sous-programmes) au moyen des  instructions
élémentaires de la machine. 
\item
  Conventions de passage des paramètres
\item
  Aperçu des techniques d'optimisation (déroulage de boucle, etc.).
Méthodologie de l'optimisation (recherche des parties coûteuses,
mesures et comparaisons).
\end{itemize}

Approche basée sur l'étude du code fabriqué par les compilateurs.


\section{Étude d'un processeur : le PowerPC 604}


\begin{itemize}
\item
Processeur RISC, 
\item 
32 registres banalisés de 32 bits (+ 32 registres flottants de 64 bits 
et registres spéciaux)
\item
Registre de condition de 32 bits, découpé en 8 sous-registres
de 4 bits.
\item 
La plupart des instructions arithmétiques et logiques se réfèrent 
à 3 registres. Par exemple \texttt{add 8,4,3} additionne les contenus de 
$R4$ et $R3$ et met le résultat dans $R8$.

\end{itemize}
Signification : $ R_8 \leftarrow (R_4) + (R_3)$

\section{Étude d'exemples simples}

Pour commencer, on ne regarde que des \emph{fonctions ``feuilles''} (leaf functions), qui n'appellent
pas d'autres fonctions.


\subsection{Un calcul simple}

\subsubsection{Le source C++}

\begin{center}
\begin{boxedverbatim}
int foo(int a,int b)
{
  return a-b+42;
}
\end{boxedverbatim}
\end{center}

\subsubsection{Traduction en langage d'assemblage}

On demande la traduction de ce programme \texttt{foo.cc} en langage
d'assemblage, grâce à l'option \texttt{-S} de \texttt{g++}, avec les
optimisations maximum.

\begin{center}
\begin{boxedverbatim}
$ g++ -S -O9 foo.cc 
$
\end{boxedverbatim}
\end{center}

La traduction est mise dans \texttt{foo.s}

\subsubsection{Le code en langage d'assemblage}

\begin{center}
\begin{boxedverbatim}
        .file   "foo.cc"
.toc
.csect .text[PR]
gcc2_compiled.:
__gnu_compiled_cplusplus:
        .align 2
        .globl foo__Fii
        .globl .foo__Fii
.csect foo__Fii[DS]
foo__Fii:
        .long .foo__Fii, TOC[tc0], 0
.csect .text[PR]
.foo__Fii:
        subf 4,4,3
        addi 3,4,42
        blr
LT..foo__Fii:
        .long 0
        .byte 0,9,32,64,0,0,2,0
        .long 0
        .long LT..foo__Fii-.foo__Fii
        .short 8
        .byte "foo__Fii"
_section_.text:
.csect .data[RW]
        .long _section_.text
        .file   "foo.cc"
.toc
\end{boxedverbatim}
\end{center}

\subsubsection{Analyse du code}

La partie intéressante se limite en fait aux trois instructions 
qui sont après l'étiquette d'entrée de la fonction~:

\begin{center}
\begin{boxedverbatim}
.foo__Fii:
        subf 4,4,3
        addi 3,4,42
        blr
\end{boxedverbatim}
\end{center}

Le reste se compose de \emph{directives} que nous n'étudierons pas.

Commençons par la fin
\begin{itemize}
\item L'instruction \texttt{blr} (\emph{Branch to link register}) fait revenir à
la fonction appelante. Celle-ci, lors de l'appel de \texttt{foo},
a mis dans l'\emph{adresse de retour} dans le \emph{registre de liens}.
\texttt{blr} copie cette adresse dans le registre pointeur de programme
(compteur ordinal).
\item L'instruction \texttt{addi 3,4,42} additionne la valeur 42 au
contenu du registre R4, et place le résultat dans R3. Par convention,
c'est dans le registre R3 qu'une fonction doit placer la valeur
retournée.

C'est une addition avec une ``valeur immédiate'' : en effet dans 
l'instruction 42 n'est pas le numéro d'un registre, mais une valeur qui
est utilisée telle quelle.

\item \texttt{subf} est une soustraction (\emph{subtract from})  entre registres
``$R4 \leftarrow (R3) - (R4)$'' (attention à l'ordre). À l'entrée de
la fonction \texttt{foo} les paramètres
\texttt{a} et \texttt{b} sont donc reçus respectivement dans R3 et R4.
\end{itemize}

\subsubsection{Exercices}

Essayez de traduire vous-même les deux fonctions qui suivent, puis 
comparez avec ce que donne le compilateur:

\begin{center}
\begin{boxedverbatim}
int plus  (int n) { return n+1; }
int moins (int n) { return n-1; }
\end{boxedverbatim}
\end{center}

\subsection{Si-alors-sinon}

\subsubsection{Source}
\begin{center}
\begin{boxedverbatim}
int distance(int a, int b)
{
  if(a > b)
    return a-b;
  else 
    return b-a;
}
\end{boxedverbatim}
\end{center}

\subsubsection{Traduction}
\begin{center}
\begin{listing}{1}
.distance__Fii:
        cmpw 1,3,4
        bc 12,5,L..2
        subf 3,3,4
        blr
L..2:
        subf 3,4,3
        blr
\end{listing}
\end{center}

\subsubsection{Explications}

Structure générale : la première instruction (\texttt{cmpw} =
\emph{Compare Word}, ligne 2) compare les contenus des registres R3 et
R4, c'est-à-dire les valeurs de \texttt{a} et \texttt{b}.  La seconde
(\texttt{bc}) est un branchement conditionnel~: dans certaines
circonstances (liées à la comparaison) l'exécution saute à l'adresse
\texttt{L..2}, sinon elle se poursuit en séquence à l'instruction
suivante.
 
On voit facilement que les lignes 7 et 8 correspondent à la partie ``alors'',
le ``sinon'' étant lignes 4 et 5.

Fonctionnement détaillé de la comparaison~: le registre de condition CR
contient 8 sous-registres de condition CR0 à CR7, de 4 bits chacun.
Le premier argument de \texttt{cmpw} indique que la comparaison des
registres R3 et R4 positionnera le sous-registre de condition CR1, qui
correspond aux bits 4 à 7 de CR (CR0 contient les bits 0 à 3, etc).
Le premier bit d'un sous-registre de condition indique ``inférieur'',
le second ``supérieur'', le troisième ``égal''. 

Si $a<b$, les bits 4, 5 et 6 de CR vaudront donc $100$, si $a=b$ on aura
$001$ et si $a>b$ $010$.

Le branchement conditionnel comporte trois éléments : un critère (ici
12), un numéro de bit (ici 5) et une destination (\texttt{L..2}). Le
critère est en réalité un masque binaire qui peut prendre des valeurs
entre 0 et 15. Nous ne retiendrons ici que deux valeurs : 12 qui
signifie ``si le bit désigné est à 1'' et 4 qui précise ``si le bit
désigné est à 0''\footnote{Les autres valeurs permettent de tenir compte
également du contenu d'un registre spécial (compteur)}.

La combinaison des deux instructions peut donc se lire ``si
R3 plus grand que R4, aller à L..2''.

\subsubsection{Exercices}

Écrire les fonctions ``maximum de deux nombres'' et ``valeur absolue''.
 
\subsection{Boucles}

\subsubsection{Le code}

\begin{center}
\begin{boxedverbatim}
int triangle(int n)
{
  int r; 
  int k;
  r = 0;
  for (k=1; k<=n; k++)
     r += k;
  return r;
}
\end{boxedverbatim}
\end{center}


\begin{listing}{1}
.triangle__Fi:
        mr 9,3
        li 0,1
        cmpw 1,0,9
        li 3,0
        bclr 12,5
L..5:
        add 3,3,0
        addic 0,0,1
        cmpw 1,0,9
        bc 4,5,L..5
        blr
\end{listing}

\subsubsection{Analyse}

\begin{itemize} 
\item On repère facilement le \emph{corps de la boucle}, délimité par l'étiquette
de la ligne 5 et le saut de la ligne 12.
\item Dans ce corps de boucle on doit trouver le cumul dans \texttt{r},
l'incrémentation de \texttt{k}, et la comparaison de \texttt{k} avec \texttt{n}.
On en déduit facilement que \texttt{r} est dans R3, \texttt{k} dans R0, et 
\texttt{n} dans R9.
\item Remarquez le tour de passe-passe de la ligne 2, qui transfère
\texttt{n} dans R9, ce qui permet d'employer le registre R3 pour \texttt{r}, qui sera le résultat.
\item La boucle \texttt{for} n'est pas traduite sous la forme ``naturelle''
d'une boucle tant-que
\begin{verbatim}
      k = 1
boucle:
      comparer k et n 
      si > aller à ...
      ...
      k++
      aller à boucle

\end{verbatim}
mais en ``dépliant'' le premier cas
\begin{verbatim}
      k = 1
      comparer k et n
      si > aller à ....
boucle:
      ....
      k++
      comparer k et n
      si <= aller à boucle
\end{verbatim}
Ceci économise l'exécution d'une instruction à chaque tour de boucle
(ici on aurait 5 instructions au lieu de 4, soit une perte de 25 \%).

\item L'instruction \texttt{bclr} est un retour conditionnel.

\end{itemize}

\section{Tableaux}

\begin{center}
\begin{boxedverbatim}
int element(int t[], int i)
{
  return (t[i]);
}
\end{boxedverbatim}
\end{center}

\begin{center}
\begin{boxedverbatim}
.element__FPii:
        slwi 4,4,2
        lwzx 3,4,3
        blr
\end{boxedverbatim}
\end{center}

\begin{center}
\begin{boxedverbatim}
int somtab(int t[], int n)
{
  /* somme des n premiers 
     éléments du tableau t */
  int k, s;
  s = 0;
  for (k=0; k<n; k++)
    s += t[k];
  return s;
}
\end{boxedverbatim}
\end{center}

\begin{center}
\begin{boxedverbatim}
.somtab__FPii:
        mr 10,3
        li 3,0
        cmpw 1,3,4
        mr 11,3
        bclr 4,4
        mr 9,3
L..5:
        lwzx 0,9,10
        addi 11,11,1
        cmpw 1,11,4
        addi 9,9,4
        add 3,3,0
        bc 12,4,L..5
        blr
\end{boxedverbatim}
\end{center}


\section{Exercices}

\section{Passage de paramètre par valeur (C++)}

\begin{center}
\begin{boxedverbatim}
void incrementer(int &n)
{
   n++;
}
\end{boxedverbatim}
\end{center}

\begin{center}
\begin{boxedverbatim}
.incrementer__FRi:
        lwz 0,0(3)
        addic 0,0,1
        stw 0,0(3)
        blr
\end{boxedverbatim}
\end{center}

\begin{center}
\begin{boxedverbatim}
void echanger(int &a, int &b)
// pont aux ânes
{
  int c;
  c=a;
  a=b;
  b=c;
}
\end{boxedverbatim}
\end{center}

\begin{center}
\begin{boxedverbatim}
.echanger__FRiT0:
        lwz 0,0(4)
        lwz 9,0(3)
        stw 0,0(3)
        stw 9,0(4)
        blr
\end{boxedverbatim}
\end{center}



\section{Conventions de passage de paramètres}

Les paramètres sont passés dans les registres R3, R4, R5 etc.
Si il y a un résultat il est transmis dans R3.

\section{Utilisation de la pile}



Pour voir comment se passe l'appel~:

\begin{center}
\begin{boxedverbatim}
extern int foo(int a, int b);

int bar()
{
  return(bar(123,456));
}
\end{boxedverbatim}
\end{center}

\begin{center}
\begin{boxedverbatim}
.bar__Fv:
        mflr 0
        stw 0,8(1)
        stwu 1,-56(1)
        li 3,123
        li 4,456
        bl .foo__Fii
        cror 31,31,31
        addi 1,1,56
        lwz 0,8(1)
        mtlr 0
        blr
\end{boxedverbatim}
\end{center}


Le coeur de la fonction \texttt{bar} consiste à appeler la fonction
\texttt{foo}, avec les paramètres 123 et 456. C'est ce qui est fait
par les 3 instructions
\begin{boxedverbatim}
        li 3,123
        li 4,456
        bl .foo__Fii
\end{boxedverbatim}

L'instruction \texttt{cror 31,31,31} est une ``non-opération'' qui ne fait 
rien. Sa présence est cependant obligatoire (contrainte de l'éditeur de liens).


En entrant dans \texttt{bar}, le registre de liens contient l'adresse
à laquelle il faudra revenir. Le contenu de ce registre sera modifié
par l'appel de \texttt{foo}, il faut donc en sauver le contenu avant
d'effectuer cet appel, et le restaurer ensuite.

Pour la sauvegarde, manoeuvre en 2 temps : on transfère le registre de liens
dans le registre R0, que l'on sauve ensuite dans la pile.  (La restauration
se fera de façon symétrique).

Sur la pile dont le sommet est pointé par R1, la fonction
\texttt{bar} se réserve ensuite un bloc de 56 octets~:
\begin{itemize}
\item le contenu courant de R1 est sauvé au sommet de ce nouveau bloc
(attention la pile croît vers le bas !)
\item R1 est mis à jour pour pointer sur ce nouveau bloc.
\end{itemize}


\subsection{Un autre exemple}

\begin{center}
\begin{boxedverbatim}
int pascal(int n, int p)
{
  if (n==0) return 1;
  if (p==0) return 1;
  return (pascal(n-1,p)
          + pascal(n-1,p-1));
}
\end{boxedverbatim}
\end{center}

\begin{center}
\begin{boxedverbatim}
.pascal__Fii:
        mflr 0
        stw 28,-16(1)
        stw 29,-12(1)
        stw 30,-8(1)
        stw 31,-4(1)
        stw 0,8(1)
        stwu 1,-72(1)
        mr. 3,3
        mr 31,4
        bc 12,2,L..3
        cmpwi 1,31,0
        bc 12,6,L..3
        addi 29,3,-1
        mr 3,29
        mr 4,31
        bl .pascal__Fii
        mr 28,3
        mr 3,29
        addi 4,31,-1
        bl .pascal__Fii
        add 3,28,3
        b L..5
L..3:
        li 3,1
L..5:
        addi 1,1,72
        lwz 0,8(1)
        mtlr 0
        lwz 28,-16(1)
        lwz 29,-12(1)
        lwz 30,-8(1)
        lwz 31,-4(1)
        blr
\end{boxedverbatim}
\end{center}

\end{multicols}

\newpage
\appendix
\section{Annexe : Jeu d'instructions}



\subsection{Notations}

\begin{tabular}{ll}
$D$ & le déplacement \\
$CR$              & le Registre de Condition \\
Rx              & registre entre r0 et r31 \\
& (RT pour "Target", RS pour "Source", RA et RB pour des opérandes)  \\
$(Rx)$            & le contenu du registre Rx \\
$(RA|0)$          & si RA=0, alors 0, sinon le contenu de RA \\
$V, SI$           & quantité immédiate signée sur 16 bits \\
$UI$              & quantité immédiate non signée sur 16 bits \\
$rep(N,B)$        & champ formé de $N$ répétitions du bit $B$ \\
$mem(A,N)$        & $N$ octets en mémoire à partir de l'adresse $A$ \\
$A || B$          & Le champ obtenu par concaténation de $A$ et $B$ \\
$exts(V)$         & la valeur $V$ avec extension de signe sur 32 bits \\
$A\{i \ldots j\}$         & les bits d'indice $i$ à $j$ de $A$ \\
$A \leftarrow B$          & affectation \\
\end{tabular}


\subsection{Mouvements de données}

\subsubsection{Par octet}

\begin{tableinstr}
\titreinstr{Chargement octet et zéro} \\
\definstr{lbz  RT,D(RA)}{RT \leftarrow  rep(24,0) || mem( (RA|0) + D, 1)} \\

\titreinstr{Chargement octet et zéro avec mise-à-jour }  \\
\definstr{lbzu  RT,D(RA)}{RT \leftarrow rep(24,0) || mem( (RA) + D, 1) ;} \\
\definstr{}{RA \leftarrow (RA) + D} \\

\titreinstr{Chargement indexé octet et zéro}   \\
\definstr{lbzx RT,RA,RB}{RT \leftarrow rep(24,0) || mem( (RA|0) + (RB), 1)} \\


\titreinstr{Chargement indexé octet et zéro avec mise-à-jour } \\

\definstr{lbzux RT,RA,RB     }{RT \leftarrow rep(24,0) || mem( (RA) + (RB), 1)} \\
\definstr{                   }{RA \leftarrow (RA) + (RB)} \\


\titreinstr{Rangement octet } \\

\definstr{stb  RS,D(RA)      }{mem( (RA|0) + D ,1)  \leftarrow (RS)\{24..31\}} \\


\titreinstr{Rangement octet avec mise-à-jour } \\

\definstr{stbu  RS,D(RA)     }{mem( (RA) + D ,1)  \leftarrow (RS)\{24..31\}} \\
\definstr{                   }{RA \leftarrow (RA) + D} \\


\titreinstr{Rangement indexé octet } \\

\definstr{stbx RS,RA,RB      }{mem( (RA|0) + (RB), 1) \leftarrow (RS)\{24..31\}} \\


\titreinstr{Rangement indexé octet avec mise-à-jour } \\

\definstr{stbux RS,RA,RB     }{mem( (RA) + (RB), 1) \leftarrow (RS)\{24..31\}} \\
\definstr{                   }{RA \leftarrow (RA) + (RB)} \\

\end{tableinstr}



\subsubsection{Par demi-mot}

\begin{tableinstr}
\titreinstr{Chargement demi-mot et zéro } \\
\definstr{lhz  RT,D(RA)          }{ RT \leftarrow rep(16,0)\ ||\ mem( (RA|0) + D, 2) } \\


\titreinstr{Chargement demi-mot et zéro avec mise-à-jour } \\

\definstr{lhzu  RT,D(RA)          }{RT \leftarrow rep(16,0)\ ||\ mem( (RA) + D, 2) } \\
\definstr{                        }{RA \leftarrow (RA) + D} \\


\titreinstr{Chargement indexé demi-mot et zéro } \\

\definstr{lhzx RT,RA,RB           }{RT \leftarrow rep(16,0) || mem( (RA|0) + (RB), 2)} \\


\titreinstr{Chargement indexé demi-mot et zéro avec mise-à-jour } \\

\definstr{lhzux RT,RA,RB          }{RT \leftarrow rep(16,0) || mem( (RA) + (RB), 2)} \\
\definstr{                        }{RA \leftarrow (RA) + (RB)} \\


\titreinstr{Chargement algébrique demi-mot } \\

\definstr{lha  RT,D(RA)           }{RT \leftarrow exts (mem( (RA|0) + D, 2)) } \\


\titreinstr{Chargement algébrique demi-mot avec mise-à-jour } \\

\definstr{lhau  RT,D(RA)          }{RT \leftarrow exts (mem( (RA) + D, 2)) } \\
\definstr{                        }{RA \leftarrow (RA) + D} \\


\titreinstr{Chargement algébrique indexé demi-mot } \\

\definstr{lhax  RT,RA,RB          }{RT \leftarrow exts (mem( (RA|0) + (RB), 2)) } \\


\titreinstr{Chargement algébrique indexé demi-mot avec mise-à-jour } \\

\definstr{lhaux  RT,RA,RB         }{RT \leftarrow exts (mem( (RA) + (RB), 2)) } \\
\definstr{                        }{RA \leftarrow (RA) + (RB)} \\


\titreinstr{Rangement demi-mot } \\

\definstr{sth  RS,D(RA)           }{mem( (RA|0) + D ),2)  \leftarrow (RS)\{16..31\}} \\


\titreinstr{Rangement demi-mot avec mise-à-jour } \\

\definstr{sthu  RS,D(RA)          }{mem( (RA) + D ),2)  \leftarrow (RS)\{16..31\}} \\
\definstr{                        }{RA \leftarrow (RA) + D} \\


\titreinstr{Rangement indexé demi-mot } \\

\definstr{sthx RS,RA,RB           }{mem( (RA|0) + (RB), 2) \leftarrow (RS)\{16..31\}} \\


\titreinstr{Rangement indexé demi-mot avec mise-à-jour } \\

\definstr{sthux RS,RA,RB          }{mem( (RA) + (RB), 2) \leftarrow (RS)\{16..31\}} \\
\definstr{                        }{RA \leftarrow (RA) + (RB)} \\

\end{tableinstr}


\subsubsection{Par mot}

\begin{tableinstr}

\titreinstr{Chargement mot } \\

\definstr{lwz RT,D(RA)            }{RT \leftarrow mem( (RA|0) + D, 4)} \\


\titreinstr{Chargement mot avec mise-à-jour } \\

\definstr{lwzu RT,D(RA)           }{RT \leftarrow mem( (RA) + D, 4)} \\
\definstr{                        }{RA \leftarrow (RA) + D} \\


\titreinstr{Chargement indexé mot } \\

\definstr{lwzx RT,RA,RB           }{RT \leftarrow mem( (RA|0) + (RB), 4)} \\


\titreinstr{Chargement indexé mot avec mise-à-jour } \\

\definstr{lwzux RT,RA,RB          }{RT \leftarrow mem( (RA) + (RB), 4)} \\
\definstr{                        }{RA \leftarrow (RA) + (RB)} \\


\titreinstr{Rangement mot } \\

\definstr{stw  RS,D(RA)           }{mem( (RA|0) + D, 4)  \leftarrow (RS)} \\

               
\titreinstr{Rangement mot avec mise-à-jour } \\

\definstr{stwu  RS,D(RA)          }{mem( (RA) + D, 4)  \leftarrow (RS)} \\
\definstr{}{                        RA \leftarrow (RA) + D} \\


\titreinstr{Rangement indexé mot } \\

\definstr{stwx  RS,RA,RB          }{mem( (RA|0) + (RB), 4)  \leftarrow (RS) } \\


\titreinstr{Rangement indexé mot avec mise-à-jour } \\

\definstr{stwux  RS,RA,RB         }{mem( (RA) + (RB), 4)  \leftarrow (RS) } \\
\definstr{                        }{RA \leftarrow (RA) + (RB)} \\

\end{tableinstr}


\subsubsection{Chargement de constantes}

\begin{tableinstr}
\titreinstr{Chargement immédiat } \\

\definstr{li      RT,V            }{RT \leftarrow exts(V)} \\


\titreinstr{Chargement immédiat en partie haute } \\

\definstr{lis     RT,V }{           RT(0..15) \leftarrow V}
\end{tableinstr}


En fait, ces deux instructions sont des mnémoniques étendus qui représentent respectivement
\texttt{addi    RT,0,V} et \texttt{addis   RT,0,V}



\subsubsection{Mouvement de registre à registre}

\begin{tableinstr}
\titreinstr{Chargement registre général } \\

\definstr{mr      RT,RS           }{RT \leftarrow (RS)} \\
\definstr{}{\mbox{C'est un mnémonique étendu pour \texttt{or      RT,RS,RS}}} \\


\titreinstr{Chargement à partir du registre de lien (move from link register) } \\

\definstr{mflr    RT              }{RT \leftarrow (LR)} \\


\titreinstr{Rangement dans registre de lien (move to link register) } \\

\definstr{mtlr    RS              }{LR \leftarrow (RS)} \\
\end{tableinstr}



\subsection{Branchements et conditions}


BI (Bit In the CR) = numéro du bit de condition à tester (entre 0 et 31). 

BO (Branch Option) = spécification du test à effectuer. Principales valeurs: 

\begin{itemize}
\item       4 : branchement si faux 
\item       12 : branchement si vrai 
\item       20 : branchement inconditionnel 
\end{itemize}

\begin{tableinstr}
\titreinstr{Branchement } \\

\definstr{b       adr             }{} \\


\titreinstr{Branchement avec lien (adresse suivante dans LR, Link Register) } \\

\definstr{bl      adr     }{} \\


\titreinstr{Branchement conditionnel } \\

\definstr{bc      BO,BI,adr}{} \\


\titreinstr{Branchement conditionnel avec lien } \\

\definstr{bcl     BO,BI,adr}{} \\


\titreinstr{Branchement conditionnel au registre de lien } \\

\definstr{bclr    BO,BI}{} \\


\titreinstr{Branchement conditionnel au registre de lien avec lien } \\

\definstr{bclrl   BO,BI}{} \\


\titreinstr{Branchement au registre de lien } \\

\definstr{blr}{\mbox{C'est une abréviation pour \texttt{bclr 20,0}}} \\

\titreinstr{Combinaisons de bits du registre de condition } \\

\definstr{crand   BT,BA,BB                }{CR_{BT} \leftarrow CR_{BA}\ and\ CR_{BB}} \\
\definstr{cror    BT,BA,BB                }{CR_{BT} \leftarrow CR_{BA}\ or\ CR_{BB}} \\
\definstr{crxor   BT,BA,BB                }{CR_{BT} \leftarrow CR_{BA}\ xor\ CR_{BB}} \\
\definstr{crnand  BT,BA,BB                }{CR_{BT} \leftarrow CR_{BA}\ nand\ CR_{BB}} \\
\definstr{crnor   BT,BA,BB                }{CR_{BT} \leftarrow CR_{BA}\ nor\ CR_{BB}} \\
\definstr{creqv   BT,BA,BB                }{CR_{BT} \leftarrow CR_{BA}\ eqv\ CR_{BB}\ (eqv=not\ xor)} \\
\definstr{crandc  BT,BA,BB                }{CR_{BT} \leftarrow CR_{BA}\ and\ not(CR_{BB} )} \\
\definstr{crorc   BT,BA,BB                }{CR_{BT} \leftarrow CR_{BA}\ or\ not(CR_{BB})} \\
\end{tableinstr}



\subsection{Opérations arithmétiques et logiques}

La notation \texttt{[.]} signifie que le mnémonique peut être
suivi d'un point. Dans ce cas, l'opération met à jour le sous-registre
de condition CR0.


\begin{tableinstr}
\titreinstr{Opérations arithmétiques} \\

\definstr{addi    RT,RA,SI                }{RT \leftarrow (RA|0)\ +\ exts(SI)} \\
\definstr{addis   RT,RA,SI                }{RT \leftarrow (RA|0)\ +\ (SI)||rep(16,0)} \\
\definstr{add[.]  RT,RA,RB                }{RT \leftarrow (RA)+(RB)} \\
\definstr{subf[.] RT,RA,RB               }{ RT \leftarrow (RB)-(RA)} \\
\definstr{neg[.]  RT,RA                  }{ RT \leftarrow -(RA)} \\
\definstr{mulli    RT,RA,SI              }{ RT \leftarrow (RA)\ *\ exts(SI)} \\
\definstr{mullw[.] RT,RA,RB               }{RT \leftarrow (RA)\ *\ (RB)       } \\
\definstr{divw[.]  RT,RA,RB              }{ RT \leftarrow (RA)\ /\ (RB)} \\
\end{tableinstr}

\paragraph*{Note~: } la séquence suivante permet de calculer le reste d'une division entière: 


\begin{verbatim}
        divw  RT,RA,RB
        mullw RT,RT,RB
        subf  RT,RT,RA
\end{verbatim}

Les comparaisons positionnent les indicateurs d'un sous-registre de condition. 

\begin{tableinstr}
\titreinstr{Comparaisons} \\
\definstr{cmpwi   CR,RA,SI}{\mbox{comparaison de $(RA)$ et $exts(SI)$,
                        résultat dans CR}} \\
\definstr{cmpw    CR,RA,RB}{\mbox{comparaison de $(RA)$ et $(RB)$}} \\

\end{tableinstr}



\begin{tableinstr}
\titreinstr{Opérations logiques} \\
\definstr{andi    RA,RS,UI }{       RA \leftarrow (RS)\ and\ rep(16,0)||UI} \\
\definstr{ori     RA,RS,UI }{       RA \leftarrow (RS)\ or\  rep(16,0)||UI} \\
\definstr{xori    RA,RS,UI }{       RA \leftarrow (RS)\ xor\ rep(16,0)||UI} \\

\definstr{andis   RA,RS,UI }{       RA \leftarrow (RS)\ and\ UI||rep(16,0)} \\
\definstr{oris    RA,RS,UI }{       RA \leftarrow (RS)\ or\  UI||rep(16,0)} \\
\definstr{xoris   RA,RS,UI }{       RA \leftarrow (RS)\ xor\ UI||rep(16,0)} \\

\definstr{and[.]  RA,RS,RB }{       RA \leftarrow (RS)\ and\  (RB)} \\
\definstr{or[.]   RA,RS,RB }{       RA \leftarrow (RS)\ or\   (RB)} \\
\definstr{xor[.]  RA,RS,RB }{       RA \leftarrow (RS)\ xor\  (RB)} \\
\definstr{nand[.] RA,RS,RB }{       RA \leftarrow (RS)\ nand\ (RB)} \\
\definstr{nor[.]  RA,RS,RB }{       RA \leftarrow (RS)\ nor\  (RB)} \\
\definstr{eqv[.]  RA,RS,RB }{       RA \leftarrow (RS)\ eqv\  (RB) (eqv=not xor)} \\
\definstr{andc[.] RA,RS,RB }{       RA \leftarrow (RS)\ and\ not(RB)} \\
\definstr{orc[.]  RA,RS,RB }{       RA \leftarrow (RS)\ or\  not(RB)} \\

\titreinstr{Extension de signe} \\

\definstr{extsb[.] RA,RS }{         RA \leftarrow exts( RS\{24..31\} )} \\
\definstr{extsh[.] RA,RS }{         RA \leftarrow exts( RS\{16..31\} )} \\

\titreinstr{Décalages} \\

\definstr{slw[.]  RA,RS,RB}{        RA \leftarrow (RS) << (RB)} \\
\definstr{srw[.]  RA,RS,RN}{        RA \leftarrow (RS) >> (RB)} \\
\definstr{sraw[.] RA,RS,RB}{        RA \leftarrow exts( RS\{0..31-(RB)\} )  } \\
\definstr{}{   \mbox{    (décalage algébrique)}} \\
\definstr{srawi[.] RA,RS,SI}{       RA \leftarrow exts( RS\{0..31-SI\} )  } \\
\end{tableinstr}




\end{document}

